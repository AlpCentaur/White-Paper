% !TEX root = paper.tex
\section{Einleitung}
\label{sec:einleitung}

\subsection{Identität}
\label{sec:einleitung_identitaet}

Zunächst einmal eine gänzlich kontextfreie Sicht auf den Begriff "Identität" quasi "ganz von Null". Es folgt die \href{https://de.wikipedia.org/wiki/Identit%C3%A4t}{Wikipedia-Definition}:

\vspace{0.3cm}

\begin{Business-Def}[Identität]\label{defIdentity}

\textbf{Identität} ist die Gesamtheit der Eigentümlichkeiten (\textbf{"Gesamtheit persönlicher Eigenheiten"}), die eine Entität, einen Gegenstand oder ein Objekt kennzeichnen und als \textbf{Individuum} von anderen unterscheiden. In ähnlichem Sinn wird der Begriff auch zur \textbf{Charakterisierung von Personen} verwendet. […] So folgt die rechtliche \href{https://de.wikipedia.org/wiki/Identit%C3%A4tsfeststellung}{Identitätsfeststellung} den für \textbf{Inklusion} und \textbf{Exklusion} relevanten Markern moderner bürgerlicher Gesellschaften.

\end{Business-Def}

\vspace{0.3cm}

Die nahezu philosophische Auseinandersetzung mit dem allgemeinen Verständnis der Identität wollen wir an dieser Stelle nicht weiter vertiefen und verweisen stattdessen u. a. folgende Quellen:

\vspace{0.3cm}

\begin{Quelle}

\begin{itemize}
  \item \href{https://blockchainopedia.atlassian.net/wiki/spaces/RESEARCH/pages/81264861/Digitale+Identit+t+-+eine+allgemeine+Sicht}{Confluence}
  \item \href{https://vsis-www.informatik.uni-hamburg.de/getDoc.php/publications/191/InfTage_CPK.pdf}{Diplomarbeit - Christian Philip Kunze}
\end{itemize}

\end{Quelle}

\vspace{0.3cm}

\todo{TODO: Oben zitierter Confluence-Artikel ist natürlich nicht öffentlich. Ggf. sollte man relevante Dinge daraus hier einarbeiten und den Link entfernen. Insbesondere der im Confluence thematisierte Begriff der \textbf{Identitäts-Feststellung} könnte für unsere Zwecke von Relevanz sein.}

\vspace{0.3cm}

Die Deutung des Begriffs der \textbf{Identität} ist also ungemein stark abhängig von der Perspektive, aus der die Deutung erfolgt. Die Schaffung eines übergeordneten Identitäts-Verständnis - insbesondere unter Einbeziehung der \textbf{"Identitäts-Digitalisierung} - ist eine riesengroße Herausforderung. Aber gleichzeitig eine eben so große Chance. Da die eben angesprochene Digitalisierung aktuell nach wie vor größtenteils in staatlicher Hand liegt, im Folgenden noch eine weitere wesentliche (Perspektive-abhängige und etwas überspitzt formulierte) Identitäts-Definition: 

\vspace{0.3cm}

\begin{Business-Def}[gesellschaftliches/staatliches Verständnis der Identität]\label{defStaatIdentity}

Ich bin genau der, von dem mein Ausweis behauptet, ich sei es.

\end{Business-Def}

\vspace{0.5cm}

\subsection{Verständnis der digitalen Identität}
\label{sec:einleitung_digitale_identitaet}

Mal von jeglicher philosophischen Sichtweise mal abgesehen, könnte der Begriff der Identität in der realen Welt kaum simpler zu verstehen sein. Ich bin ich! Ich trete stets mit derselben Identität auf - ob im Freundeskreis, bei der Arbeit oder beim Elternabend. Die Rollen und die relevanten Identitätsmerkmale mögen sich bei unterschiedlichen Anlässen unterscheiden, aber es bleibt dieselbe Person. Wenn man sich Geld von einem Kollegen auf Arbeit leiht, kann man es ihm auch dann zurückgeben, wenn man sich zufällig im Restaurant trifft. Weil kein Zweifel an den Identitäten der beiden Betroffenen besteht. \textbf{Dies ist in der digitalen Welt ganz anders.}

\vspace{0.2cm}

Die Definition von digitaler Identität erscheint auf den ersten Blick nahezu trivial:

\vspace{0.3cm}

\begin{Business-Def}[Digitale Identität]

Die digitale Identität ist nichts anderes als ein eindeutiger (technischer) Identifier/Username/Kundennummer - ein Primary Key in einer Datenbanktabelle, wo das vermeintliche Individuum zu einer "Entität" wird.

\vspace{0.2cm}

Angereichert wird der zum technischen Identifier gehörende Entitäts-Datensatz mit zusätzlichen Properties ganz im Sinne der obigen allgemeinen Identitäts-Definition \ref{defIdentity}. 

\end{Business-Def}

\vspace{0.3cm}

Mit ein wenig technischem Verständnis erkennt man sofort das aus der eben formulierten Definition resultierende Problem: \textbf{Diese ist nämlich in unserer aktuellen digitalen Welt alles andere als eindeutig}. Und zwar deshalb nicht, weil sie auf Datenmodellierungs-Ebene zu interpretieren ist, die jeder digitale Service-Provider für sich allein vornimmt. Der so simplen und unmissverständlich klaren Definition der \textit{digitalen Identität} fehlt also eine winzige Kleinigkeit, deren Fehlen das Verständnis der \textit{digitalen Identität} plötzlich von \textit{trivial} zu \textit{höchst komplex} hievt: \textbf{Der Forderung \textit{global eindeutig} zu sein.}

Dieser Umstand verstärkt konsequenterweise sogar das im vorigen Abschnitt bereits aufgegriffene Problem hinsichtlich des komplexen \textit{Identitäts-Verständnis}: \textbf{Das Identitäts-Verständnis ist stark Perspektive-abhängig}. Um dies zu verdeutlichen transformieren wir die (bereits unbefriedigende) Definition \ref{defStaatIdentity} in die digitale Welt und bekommen ein sehr sprechendes Analogon:

\vspace{0.3cm}

\begin{Business-Def}[Online-Account = (eine) digitale Identität]\label{defAccount}

Ich bin genau der, als den mich ein jeder Online-Provider in seinem Datenmodell modelliert.

\end{Business-Def}

\vspace{0.3cm}

Damit hat die Digitalisierung - gleichwohl sie die Mittel besäße, Abhilfe für viele Probleme im Kontext der \textit{Identität} beizusteuern - das \textbf{Identitäts-Verständnis} sogar noch komplexer gemacht, als es vorher schon war.

Zusammengefasst:

\vspace{0.3cm}

\begin{Fazit}

\begin{itemize}
  \item Eine \textbf{digitale Identität} entspricht einer (User-)Entität innerhalb der Datenmodells eines beliebigen digitalen Service-Provider.
  \item Es existiert keinerlei Forderung/Spezifikation/Konsens nach Einheitlichkeit oder gar Eindeutigkeit der \textbf{digitalen Identität}. Auf Grund dessen kann auch keinesfalls die Rede von \textbf{der} digitale Identität sein. Stattdessen besitzt ein Individuum zig - wenn nicht gar hunderte - digitale Identitäten. 
  \item Es existieren keinerlei "Querverweise" zwischen der Vielzahl der digitale Identitäten eines Einzelnen, die es erlauben würden, die vielen Online-Account (= digitale Identitäten) zu einer \textbf{einzigen digitalen Identität} zu konsolidieren.
\end{itemize}

\end{Fazit}

\vspace{0.3cm}

\todo{ab hier WIP}

\vspace{0.3cm}

\textbf{\textit{Anologie in die analoge Welt:}}

\vspace{0.3cm}

Aufgrund der gängigen Praxis nahezu aller Web-Service-Anbieter/Apps/Online-Shops existieren Zig - wenn nicht gar Hunderte - von digitalen Kopien meines Ichs. 

Das eine Ich darf nur in dem einen Laden einkaufen, das andere nur in dem anderen. Die unterschiedlichen Ichs haben unterschiedliche Kreditkarten dabei (hinterlegte oder akzeptierte Zahlungsmittel bei unterschiedlichen Anbietern) - manche gar keine (Zahlungsmittel wird nicht akzeptiert). Einige Ichs haben ihren Ausweis dabei (Ident-Verfahren durchgeführt), andere wieder nicht. Gleiches gilt für den Führerschein (Anmeldung bei unterschiedlichen CarSharings). Und während das eine Ich bereits einen frischen Führerschein dabei hat, hat das andere noch den abgelaufenen (lange nicht benutzt und Führerschein abgelaufen). Die Ichs haben teils unterschiedliche Telefonnummern oder unterschiedliche Email-Adressen. Manche Ichs haben ihr Telefon komplett vergessen. Manche Ichs sind bereits längt tot oder kurz davor (Account verstaubt oder vergessen, überhaupt einen zu besitzen). Die Ichs sind gut vernetzt (Telefonnummern, WhatsApp, Facebook, LinkedIn, Xing), aber die einen Ichs kennen manche Leute nicht, die die anderen Ich kennen und umgekehrt. Und wenn sie irgendwie doch von der letzten Party erkennen, wissen sie plötzlich den Namen des Gegenüber nicht mehr oder auch nicht, worüber man bei genannter Party gesprochen hat.

Das alles ist eine metaphorisch polemische Darstellung des digitalen Status quo den vorherrschenden schier unendlichen Multi-Accountings in der Web2.0-Welt. Jeder Account ist das Abbild meiner Identität in die digitale Welt. Es bin immer ich, der hinter jeder dieser Identitäten steht. Jede dieser digitalen Identitäten ist fraglos eine Identität im Sinne der Definition. Sie kann gar ein detailliertes und durchaus sehr vertrauenswürdigen Abbild sein - um Fake-Identitäten soll es hierbei gar nicht gehen - aber sie ist stets eine weitere Kopie. Ich lasse also Zig und Hunderte Kopien meines Selbst in die digitale Welt raus, ohne dass sie als die Kopie derselben echten Identität erkennbar sind.

Dies kann natürlich an vielen Stellen sogar von Vorteil sein.

Einige meiner Ichs sind auf so weit voneinander entfernten Kontinenten unterwegs, dass sie sich niemals treffen oder von dem gegenseitigen Geschehen beeinflusst werden (Amazon vs. CarSharing). Andere Ichs sind wiederum so schüchtern, dass sie sehr gerne unerkannt bleiben (Datenschutz/Privatsphäre). 

Alle meine Ichs, die aber stets ihre Brieftasche mit sich führen, werden gewissen Interesse daran haben, das dem einen dieser Ichs nicht das Bargeld, dem anderen die Kreditkarte und dem dritten der Ausweis fehlt. Sie würden gerne eine gemeinsame Brieftasche haben, in der ihre gemeinsame Identität für alle Zwecke bereitliegt.

\vspace{0.3cm}

\todo{TODO: Ggf. noch den Sign-Up/-In als Identifizierung einer Online-Identity einbeziehen und erklären.}
\vspace{0.5cm}

\subsection{Missstände der digitalen Identität}
\label{sec:einleitung_probleme_digitaler_identitaet}

\vspace{0.3cm}

\begin{Problem}[fehlende Eindeutigkeit]

\todo{TODO: ausformulieren}
dasselbe vs. das gleiche Individuum

\end{Problem}

\vspace{0.3cm}


\begin{Problem}[Redundanz und fehlerbehaftete Daten]

\todo{TODO: ausformulieren}
Kreditkarte abgelaufen $\rightarrow$ in einigen Account aktuell; in anderen nicht

\end{Problem}

\vspace{0.3cm}


\begin{Problem}[Datenschutz]

\todo{TODO: ausformulieren}
\begin{itemize}
  \item meine Daten liegen an zig/hunderten Stellen gespeichert
  \item Hacks sind an zig/hunderten Stellen möglich
  \item Daten werden an anderer Stelle erhoben als sie gebraucht werden --> Beispiel mit der Supermarkt-Kassiererin bzw. Fluggesellschaften
\end{itemize}

\end{Problem}

\vspace{0.3cm}


\begin{Problem}[Datenmissbrauch/Bereicherung]

\todo{TODO: ausformulieren}
Big Tech nutzt meine Daten, um daran Geld zu verdienen. Und ich werde nicht an der Wertschöpfung beteiligt.

\end{Problem}

\vspace{0.3cm}


\begin{Problem}[Abhängigkeit von Big Tech]

\todo{TODO: ausformulieren}
Derzeit dominieren zentrale ID-Provider wie Google und Facebook die Verwaltung von Identitätsdaten sehr vieler IT-Dienste weltweit, was zu einer großen Abhängigkeit unserer Gesellschaft in Bezug auf den Fortgang der Digitalisierung führt.

\end{Problem}

\vspace{0.3cm}


\begin{Problem}[Ungenutzte Möglichkeiten]

\todo{TODO: ausformulieren}
Daten-Querverweise $\rightarrow$ Beispiel anführen 

(zB aus \href{https://norbert-pohlmann.com/glossar-cyber-sicherheit/self-sovereign-identity-ssi/}{Vorlesung zu SSI})

\end{Problem}

\vspace{0.5cm}
