% !TEX root = paper.tex
\section{Project 'Pools'}
\label{sec:pools}

\subsection{Einleitung}
\label{sec:pools-einleitung}

\vspace{0.3cm}

Die Idee hinter den sogenannten \textit{Wunder-Pools} ist das Bündeln von Liquidität mehrerer User/Teilnehmer bzw. eine Art 'Treuehandverwahrung' in einem gemeinsamen Pool. Die Anwendungsfälle solche Pools sind sehr zahlreich. Um im Folgenden nur einige Beispiele zu nennen:  

\begin{itemize}
  \item Gemeinsame Invests in (Crypto-)Assets.
  \item Pool für ein gemeinsames (Geburtstags-)Geschenk.
  \item Kicktipp-Pool (der über die gesamte Saison verwahrt werden muss).
  \item Wetten unter Freunden.
  \item Ausgleichspool für Auslagen von Geld an Freunde (Splitwise).
\end{itemize}

\vspace{0.2cm}

Das besondere an dem in den folgenden Abschnitten genauer zu beschreibenden Modell, ist sein sehr allgemein gehaltener Ansatz, mit dem sich gleichzeitig Cases umsetzen lassen, die auf den ersten Blick sehr verschieden zu sein scheinen. Genauer genommen lassen sich solche Pools mit speziellen \textit{DAO-Strukturen} beschreiben.

Abgesehen von der den Pools zugrundeliegenden Geschäftslogik, besteht der zentrale Ansatz unserer \textit{Wunder-Pools} darin, dem User ein rundes Produkt anzubieten - und zwar ganz unabhängig dessen, welcher der oben genannten Cases nun tatsächlich umgesetzt wird. An dieser Stelle möchten wir uns daher ganz explizit von dem Status quo der heute gängigen UX in der Web3-Welt abgrenzen.

\vspace{0.2cm}

Ganz grob beschrieben, streben wir in etwa folgende Geschäftslogik an:

\begin{itemize}
  \item Ein User erstellt einen Pool (in einer dafür implementierten Wunder-Pool-Applikation).
  \item Derselbe User definiert die Pool-Art, ein etwaiges dazugehöriges Regelwerk und fordert andere User auf, dem Pool beizutreten. Idealerweise erfolgt die Einladung mittels Suche nach der Wunder-ID des einzuladenden Teilnehmers (und nicht etwa anhand seiner Ethereum-Adresse oder sonstigem).
  \item Die eingeladenen Teilnehmer erhalten die Einladung (in der WunderPass-App oder der Wunder-Pool-Applikation) und können entscheiden, ob sie dem Pool beitreten möchten oder nicht. 
  \item In der Regel ist sofort beim Beitritt des Pools der definierte Einsatz zu entrichten (der anschließend in die Pool-Treasury geht). In einigen Cases kann der Einsatz evtl. zu einem späteren Zeitpunkt erfolgen oder gar ganz entfallen (z.B. beim Case \textit{Splitwise}).
  \item Der Pool ist eingerichtet und "kommt zu Einsatz", wie in etwa
  \begin{itemize}
  	\item zum gemeinschaftlichen Investieren in (Crypto-)Assets,
  	\item zum Verwahren "in Treuhand" bei einer oder mehreren abgeschlossenen Wetten (oder auch z. B. Kicktipp)
  	\item etc.
  \end{itemize}
  \item Der Pool wird liquidiert und das Geld nach dem vorher festgelegten Regelwerk auf alle Pool-Teilnehmer (nach einem aus dem Regelwerk folgenden Verteilungsschlüssel) verteilt. Die Liquidierung selbst kann entweder ebenfalls durch das Regelwerk auf einen bestimmten Zeitpunkt und/oder Ereignis terminiert werden (z.B. Ende einer BuLi-Saison beim Case \textit{Kicktipp}) oder aber durch die Teilnehmer beschlossen werden (mittels einer DAO-Abstimmung). Die Errechnung des genannten Verteilungsschlüssels möchten wir möglichst allgemein halten und übertragen diese Verantwortlichkeit einem \textit{abstrakten Oracle}, welches es stets Case-spezifisch zu definieren (und zu implementieren) gilt.
\end{itemize}

\vspace{0.2cm}

\underline{\textbf{Product-Sicht}}

\vspace{0.2cm}

Abschließend sei noch einmal betont, dass wir das/die aus den Wunder-Pools hervorgehende(n) Product(s) (mittelfristig) alternativlos user-friendly sehen. Ohne notwendigen Bezug zur Crypto-Szene, ohne MetaMask und ohne kryptische hexadezimale Wallet-Adressen. Stattdessen clean und simpel.

\vspace{0.5cm}

\subsection{Formalisierungen}

\vspace{0.3cm}

Zunächst einmal benötigen wir einige formale Werkzeuge und bedienen uns dafür folgender Definition:

\vspace{0.2cm}

\begin{Def}\label{defPoolTeilnehmer}

Im folgenden setzen wir Voraus, die Nutzung der Pools seitens der User erfordert zwingend den Besitz eines WunderPass (bzw. Wunder-ID) und betrachten von daher auch nur solche User.

\begin{equation*}
  U := \left\{ u_1; u_2;...; u_{n} \text{ } | \text{ } u_i \text{ besitzt eine Wunder-ID} \right\}
\end{equation*}

\vspace{0.2cm}

Wir stellen zusätzlich, dass der vorausgesetzte Besitz einer Wunder-ID mit dem Besitz von unterschiedlichen Wallets bzw. anderen durch die Wunder-ID implizierten Dingen einhergeht. So hat jeder User $u_i$ z.B. eine Telefonnummer mit seiner Wunder-ID verknüpft (anhand derer er mittels Kontakte-Scan auf dem Smartphone als Inhaber einer Wunder-ID und damit potenzieller Pool-Teilnehmer erkannt werden kann und soll). Des weiteren kann $u_i$ einen NFT-Pass (siehe Kapitel \ref{sec:nft-pass}) besitzen und/oder unser ERC20-Utility-Token (im Folgenden als \textit{WPT} bezeichnet; siehe Kapitel \todo{TODO: verlinken}). 

\vspace{0.2cm}

Wir formalisieren den in Kapitel \ref{sec:nft-pass} definierten NFT-Pass als die (geordnete) Menge aller bisher geminteter NFT-Pässe:

\begin{align*}
  WPN &:= \left\{ wpn_1; wpn_2;... \right\} \text{ mit} \\
  wpn_i &:= (s_i, w_i, m_i)
\end{align*}

\vspace{0.2cm}

Dabei repräsentiert $s_i$ den Status des NFT-Passes, $m_i$ sein Muster und $w_i$ das sich auf ihm abgebildete Weltwunder.

\vspace{0.2cm}

Den Besitz eines Pass-NFTs beschreiben wir durch die Funktion 

\vspace{0.2cm}

\begin{equation*}
  \omega : U \rightarrow \mathcal{P} \left( WPN \right)  
\end{equation*}

\begin{equation*}
  \omega(u):= \left\{ wpn \in WPN \text{ } | \text{ User u besitzt den Pass-NFT } wpn \right\}. 
\end{equation*}

\vspace{0.2cm}

Analog dazu definieren wir auch den Besitz am \textit{WPT} eines Users - mit dem Unterschied, dass der Funktionsbereich dieser Funktion aufgrund der Fungibilität von einer Potenzmenge auf einen simplen numerischen Wert zusammenfällt:

\vspace{0.2cm}

\begin{equation*}
  \varphi : U \rightarrow \mathbb{Q}  
\end{equation*}

\begin{equation*}
  \varphi(u):= \text{ Balance des Users u am ERC20-Token WPT}. 
\end{equation*}

\end{Def}

\vspace{0.5cm}

\subsection{Pool-Erzeugung}

\vspace{0.3cm}

Die Erzeugung des Pools findet in zwei Phasen statt: Der \textit{Initialisierungs-Phase} und der \textit{Teilnahme-Phase}

\vspace{0.2cm}

\underline{\textbf{Initialisierungs-Phase}}

\vspace{0.2cm}

Die Initialisierungs-Phase läuft in etwa in folgenden Schritten ab:

\begin{itemize}
	\item Ein Initiator (Admin) $u_A \in U$ erzeugt den Pool in einer dafür vorgesehenen \textit{Pool-Applikation} (vergleichbar mit der Erstellung einer WhatsApp-Gruppe). Der Initiator $u_A$ ist dabei selbst ein Teilnehmer des Pools. Unsere klare Absicht hierbei ist jedoch keine "gesonderten" Pool-Teilnehmer zu haben bzw. mit besonderen Rechten auszustatten. Die Unterscheidung zwischen dem Admin $u_A$ und anderen Pool-Teilnehmern $u \in U$ ist idealerweise - sofern es denn der spezielle Case zulässt - nur für die Initialisierungs-Phase von Nöten und kann anschließend entfallen.
	\item Der Admin definiert das Regelwerk für den zu erstellenden Pool:
	\begin{itemize}
		\item Art des Pools (Invest-Pool, Wette, Spende, Kicktipp, Splitwise etc.)
		\item privater oder öffentlich zugänglicher Pool
		\item etwaige Obergrenze an Teilnehmern
		\item Einsatz (minimaler, maximaler oder exakter Einsatz pro Teilnehmer und Währung des Einsatzes)
		\item Auszahlungslogik (per Abstimmung oder Adresse eines Oracle-Smart-Contracts, der abhängig seiner Contract-Logik einen Auszahlungsschlüssel bereitstellt)
	\end{itemize}
\end{itemize}

\vspace{0.3cm}

\underline{\textbf{Teilnahme-Phase}}

\vspace{0.2cm}

Die Teilnahme-Phase besteht grob aus folgenden Schritten:

\begin{itemize}
	\item Der Admin verliert seine Sonderstellung und wird stattdessen zum ersten Teilnehmer seines eigens initiierten Pools. 
	\item Der (ursprüngliche) Admin lädt Teilnehmer ein, sich am Pool zu beteiligen. Die Beteiligung erfordert dabei einen WunderPass (= Wunder-ID) seitens des Teilnehmers. Idealerweise sind die Wunder-IDs mit Telefonnummern verknüpft, sodass sich die einzuladenden User in den Kontakten des Admins erkennbar als potenzielle Teilnehmer wiederfinden.
	\item Alternativ kann der (ursprüngliche) Admin (oder auch jeder andere bereits beigetreten Teilnehmer) einen Teilnahme-Link an (weitere) potenzielle Teilnehmer verschicken.
	\item Jeder adressierte User erhält die Einladung mit allen relevanten Informationen zum eingeladenen Pool (insbesondere auch dem zu entwendenden Einsatz) in seiner Wunder-Pool-Applikation, und muss diese lediglich entweder bestätigen oder ablehnen (Pull-Prinzip). Insbesondere braucht der User für den Beitritt zum Pool kein MetaMask oder Sonstiges (Push-Prinzip wie aktuell bei DAOs üblich). 
	\item Auch der Einsatz des neuen Teilnehmers muss nicht aktiv entrichtet (an eine Wallet) werden, sondern wird stattdessen im Zuge des vorigen Schritts nach Bestätigung der Teilnahme am Pool automatisch eingezogen.
\end{itemize}

\vspace{0.2cm}

Wir fassen die bisher erzielten Ergebnisse etwas formaler zusammen:

\vspace{0.2cm}

\begin{Def}\label{defPool}

Ein (jungfräulicher) Pool im Sinne der oben Aufgezählten Eigenschaften und Anforderungen lässt sich formal schreiben als

\begin{equation*}
  Pool := \left( \mathcal{U}, \mathcal{R}, \mathcal{T}, \mathcal{G} \right) \text{ mit}
\end{equation*}

\begin{align*}
  & \mathcal{U} = \left\{ u_1; u_2;...;u_n \right\} \subseteq U \text{ die Menge der n Pool-Teilnehmer, } \\
  & \mathcal{R} \text{ das Regelset des Pools, was es gesondert zu formalisieren gilt, } \\
  & \mathcal{T} = \left\{ s_1...;s_n \right\} \text{ mit } s_i \in \mathbb{Q} \text{ die Treasury des Pools und} \\
  & \mathcal{G} = \left\{ g_1...;g_n \right\} \text{ mit } g_i \in \mathbb{N} \text{ die Governance des Pools.}
\end{align*}

\vspace{0.2cm}

Dabei beschreibt jedes $s_i$ den Einsatz des Teilnehmers $u_i \in \mathcal{U}$ ($s$ für Stake). Dieser Einsatz liegt dabei in einem vom Regelset $\mathcal{R}$ definierten Intervall $\mathcal{I} \subseteq \mathbb{Q}$. Damit haben wir bereits an dieser Stelle einen kleinen Teil der noch fehlenden Formalisierung von $\mathcal{R}$ identifiziert. Bei genauer Betrachtung fehlt uns noch die Einheit der Einsätze $s_i$. Diese wird sehr wahrscheinlich \textit{USDT} sein oder ein anderer Stable-Coin.

Zudem beachte man an dieser Stelle zusätzlich, die Definition von $\mathcal{T}$ werde im Verlaufe der Lifetime eines Pools nicht so simpel bleiben können, als nur aus dem eingebrachten Einsätzen der Teilnehmer zu bestehen. Die Pool-Treasury bedeutet nämlich mehr als nur die Menge der initialen Stakes. Etwaige Invests aus der Treasury heraus würden nämlich ebenfalls in der Treasury landen.

\vspace{0.2cm} 

Die $g_i$ dagegen beschreiben ganz simpel die Anzahl der Governance-Tokens pro User $u_i \in \mathcal{U}$. Man kann diese auch als Gesellschaftsanteile einer GbR betrachten. Das Stammkapital dieser Gesellschaft würde sich in diesem Vergleich auf

\begin{equation*}
  \kappa := \sum_{i=1}^{n} g_i 
\end{equation*}

belaufen. Der prozentuale Stimmrecht-Anteil eines Users $u_i \in \mathcal{U}$ ergäbe sich damit als 

\begin{equation*}
  \rho_i = \frac{g_i}{\kappa}, \text{   } \forall i = 1, 2, ...,n. 
\end{equation*}

\end{Def}

\vspace{0.3cm}

Die zusammengetragenen Anforderungen für die Initialisierung eines WunderPools lassen sofort deutlich erkennen, \textbf{diese Pools könnten mittels DAO-ähnlicher Strukturen implementiert werden}. Dies erscheint insofern noch logischer, nachdem wir erkannt haben, die Pools stellen gesellschaftsrechtlich GbRs dar - also Gesellschaften und/oder Organisationen. Diese Erkenntnis wollen wir noch einmal als eine formale Annahme formulieren: 

\vspace{0.2cm}

\begin{Assumption}[Ein WunderPool stellt eine GbR dar]
\label{assumptionGbR} 

Sei $\mathcal{P} := \left( \mathcal{U}, \mathcal{R}, \mathcal{T}, \mathcal{G} \right)$ ein WunderPool wie in Definition \ref{defPool} beschrieben. Wir ziehen die Analogie zu einer \textbf{Gesellschaft} bürgerlichen Rechts:

\begin{itemize}
	\item Die Menge $\mathcal{U}$ der Pool-Teilnehmer ist der \textbf{Gesellschafterkreis der Gesellschaft}.
	\item $\mathcal{G}$ bildet den \textbf{Cap-Table der Gesellschaft} ab.
	\item Das Pool-Regelwerk $\mathcal{R}$ ist der \textbf{Gesellschaftervertrag zur Gesellschaft}.
	\item Die Pool-Treasury $\mathcal{S}$ ist das \textbf{Gesellschaftskonto und/oder -depot der Gesellschaft}.
\end{itemize}

\end{Assumption}
 
\vspace{0.5cm}

\subsection{Pool-Lifetime}

\vspace{0.3cm}

Eine (allgemeine) funktionale Beschreibung derjenigen WunderPool-Funktionalität, die der Überschrift der gegenständigen Sektion gerecht wird, ist insofern sehr schwierig, als dass sich diese deutlich schwerer auf unterschiedliche Pool-Cases verallgemeinern lässt. Wie anfangs in dem Einführungskapitel \ref{sec:pools-einleitung} ist die möglichste Verallgemeinerung aller Cases oberste Prämisse gewesen. Hier müssen wir versuchen zu verallgemeinern, was nur geht, und den Rest eben Case-spezifisch lösen. 

\vspace{0.1cm}

Wir schauen auf die Anfangs in Kapitel \ref{sec:pools-einleitung} hervorgehobenen Anwendungsfälle für die WunderPools an - nun mit kurzer Skizzierung ihrer Lifetime:

\begin{itemize}
  \item \textbf{\textit{Social Investing:}} Das ist mit der klarste Case für eine relevante Lifetime eines Pools. Während der Lifetime werden mögliche Invests vorgeschlagen, zur Abstimmung gestellt und im Erfolgsfall abgewickelt. Die Möglichkeiten zur Erweiterung von Investmöglichkeiten (Staking, Lending, Liquidity-Providing, Yield Farming, Aktien, ETFs etc.) scheinen schier unendlich. In diesem Case unterliegt \textbf{die Dauer der Lifetime auch keinerlei natürlicher Grenzen} - diese Art von Pool kann theoretisch ewig existieren.
  \item \textbf{\textit{Geschenk-Pool:}} In diesem Case besteht die Daseinsberechtigung des Pools eigentlich lediglich darin, bequem und einfach Geld einzusammeln und evtl. bis zum Kauf des Geschenks "in Treuhand" zu verwahren. Sind alle gewünschten Teilnehmer beigetreten (und somit ihren Beitrag zum Geschenk entrichtet), hat der Pool eigentlich bereits seinen Zweck erfüllt. Man kann zwar argumentieren, man könne die Auswahl des Geschenks mit DAO-Mitteln zur Abstimmung stellen, dies bleibt jedoch an den Haaren herbeigezogen, solange das Geschenk kein auf der Blockchain erwerbbares Asset ist. \textbf{Die Dauer der Lifetime der Pools in diesem Case sind also klar begrenzt}: Spätestens bis zu dem Moment des Kaufs des Geschenks.
  \item \textbf{\textit{Kicktipp-Pool:}} Das ist der Bilderbuch-Case für den Pool im Sinne der Treuhand-Verwahrung (eines Spieleinsatzes) über einen längeren Zeitraum. Hier wird eingezahlt, über einen Zeitraum (außerhalb des Pools) gespielt und am Ende - je nach Ergebnis - wieder ausgezahlt. Das Geld wird vom Pool also lediglich verwahrt und umverteilt. In der sogenannten \textit{Lifetime} des Pools passiert faktisch gar nichts. Man könnte sich sicherlich kreative Möglichkeiten zur Interaktion mit dem Pool überlegen (wie z.B. Abstimmungen über etwaige Regeländerungen oder über das Nachtragen von verspätet abgegebenen Tipps), dies beträfe aber nie die relevante Kernfunktionalität des Pools innerhalb dieses Cases. Die defacto \textit{'leere Lifetime'} des Pools endet in diesem Case mit Ablauf der Spielzeit, für die die Kicktipp-Runde eingerichtet wurde. Ihre \textbf{Dauer ist also begrenzt}.
  \item \textbf{\textit{Wetten:}} Dieser Case verhält sich sehr analog zum \textit{Kicktipp-Case}. Dazu muss jedoch klargestellt sein, dass wir den Case als eine einzige Wette (zwischen zwei oder mehr Leuten) verstehen, bei der der Pool der Treuhand-Verwahrung dient, und nicht etwa eine "Wett-Gruppe", wo immer mal wieder neue Wetten vorgeschlagen und umgesetzt werden. Der Pool dieses Cases bildet also eine einzige Wette ab und seine \textbf{Lifetime endet in dem Moment, wo das Ergebnis der Wette feststeht}.
  \item \textbf{\textit{Splitwise:}} Dies ist der außergewöhnlichste aller Cases. Hier existieren de facto weder eine echte Treasury noch eine Lifetime. Für Splitwise wird erst die Umverteilung interessant, wobei hier genau genommen der Betrag von 0 auf die Teilnehmer umverteilt wird. Da hier aber - im Gegensatz zu allen obigen Cases - auch negative Withdraws zulässig sind (also genau genommen eine Einzahlung von denjenigen Teilnehmern, die anderen Teilnehmern etwas schulden), klingt die Umverteilung des Betrags 0 plötzlich doch nicht mehr so abwegig. Die 0 signalisiert nur die Forderung, die verteilten Beträge (Schulden und Auslagen mit entsprechendem Vorzeichen) müssen sich auf 0 summieren. Da der Pool in diesem Case faktisch gar keine Lifetime besitzt, ist \textbf{die Dauer der Lifetime konsequenterweise begrenzt}.
\end{itemize}

\vspace{0.3cm}

Zusammenfassend halten wir fest, die Dauer der Pool-Lifetime ist nur für den \textit{Social-Investing-Case} theoretisch unbegrenzt. Bei allen anderen Cases wird der Pool nach einer bestimmten Zeit oder bei Eintreten eines bestimmten Ereignisses obsolet und muss/sollte anschließend aufgelöst werden. Und auch hinsichtlich relevanter Funktionalität während der \textit{Lifetime} scheint der \textit{Social-Investing-Case} ebenfalls der einzig interessante zu sein. 

\vspace{0.1cm}

Eine Verallgemeinerung erscheint also - zumindest für die zuletzt genannten vier Cases - evtl. doch im Rahmen des Möglichen. 

\vspace{0.3cm}

\todo{Etwaiges Austreten bestehender Teilnehmer oder Eintreten neuer Teilnehmer würde sich während der 'Lifetime' abspielen.}

\vspace{0.5cm}

\subsection{Pool-Liquidierung}

\vspace{0.3cm}

Für eine etwaige Pool-Liquidierung stellen sich exakt zwei Fragen: "\textbf{Wann} wird liquidiert?" und "\textbf{Wie} wird liquidiert?" Das \textbf{Wann} ist hierbei schnell geklärt. Es gibt grob folgende drei Möglichkeiten, von denen eine durch das in Definition \ref{defPool} definierte Regelset $\mathcal{R}$ zu spezifizieren ist:

\begin{itemize}
  \item $\mathcal{R}$ legt einen exakten Zeitpunkt fest, zu dem der Pool liquidiert werden soll.
  \item $\mathcal{R}$ definiert ein bestimmtes Ereignis, bei deren Eintreten der Pool liquidiert werden soll.
  \item $\mathcal{R}$ regelt, dass die Pool-Liquidierung per (DAO-)Abstimmung beschlossen werden muss.
\end{itemize}

\vspace{0.3cm}

Bullet 2 klingt hier leider noch nicht ausreichend abstrakt. Daher abstrahieren wir die genannten Forderungen in einer einzigen:

\vspace{0.2cm}

\begin{Fazit}[Liquidierungsentscheidung-Oracle]

Das in Definition \ref{defPool} definierte Regelset $\mathcal{R}$ definiert ein Oracle, welches zu jedem Zeitpunkt die Frage beantworten kann, ob der Pool zum jetzigen Zeitpunkt liquidiert werden soll oder nicht. 

Dieses Oracle kann beliebig einfach gestrickt sein (z.B. im Falle des obigen Bullet 1 einfach anhand $"SYSDATE <= T_{END}"$ über das Fortbestehen des Pools entscheidet) oder aber auch beliebig komplex. Dies braucht uns aber an dieser Stelle nicht weiter weiter interessieren. 

\end{Fazit}

\vspace{0.3cm}

Und da die Abstraktion mittels Oracle so bequem scheint, tun wir das Gleiche ebenfalls für das oben genannte \textbf{Wie}:

\vspace{0.2cm}

\begin{Fazit}[Auszahlungsschlüssel-Oracle]

Seien $\mathcal{P} = \left( \mathcal{U}, \mathcal{R}, \mathcal{T}, \mathcal{G} \right)$ 
der Pool und $\mathcal{U} = \left\{ u_1; u_2;...;u_n \right\}$ die Menge seiner $n$ Teilnehmer wie in Definition \ref{defPool} beschrieben und $v_{\mathcal{T}}$ der sich zum Liquidierungszeitpunkt in der Pool-Treasury $\mathcal{T}$ befindende Value. 

Falls der Pool lediglich als Treuhand-Verwahrung diente (also über die Zeit keine Veränderung der Treasury stattfand) ergibt sich $v_{\mathcal{T}}$ als  

\vspace{0.1cm}

\begin{equation*}
  v_{\mathcal{T}} = \sum_{i=1}^{n} s_i \text{ mit } s_i \text{ wie in Definition \ref{defPool}}
\end{equation*}

\vspace{0.2cm}

Wir definieren einen Auszahlungsvektor als

\begin{equation*}
  \varphi_{\mathcal{P}} = [\varphi_1, \varphi_1, ..., \varphi_n] \text{ mit } \sum_{i=1}^{n} \varphi_i = v_{\mathcal{T}} 
\end{equation*}

\vspace{0.2cm}

\todo{Ein mögliches $\varphi_{\mathcal{P}}$ ist tatsächlich $\mathcal{G}$}

\vspace{0.2cm}

\todo{$\mathcal{R}$ definiert das Oracle und sagt, ob die $\varphi_i$ negativ sein dürfen}

\vspace{0.2cm}

\todo{Oracle verteilt anhand von $\mathcal{T}$ und $\mathcal{G}$}

\end{Fazit}

\vspace{0.3cm}

\todo{TODO}

\vspace{0.5cm}

\subsection{Pool-Economics}

\vspace{0.3cm}

\todo{Staking von WUNDER}

\todo{Pool-Governance-Tokens}

\todo{Pass-NFT-Status-Rewards}

\todo{Pass-NFT-Wunder-Rewards}

\vspace{0.5cm}

\subsection{Pool-Vertrag}

\vspace{0.3cm}

\todo{Herausgearbeitete Dinge zu $\mathcal{R}$ zusammentragen}

\begin{itemize}
  \item Vorgabe zur Teilnehmer-Menge $\mathcal{U}$
  \item Vorgabe zur Pool-Treasury $\mathcal{S}$:
  \begin{itemize}
  	\item Währung (zB \textit{USDT})
  	\item Intervall $\mathcal{I}$ für $s_i \in \mathcal{I}$
  \end{itemize}
  \item Definition der \textit{Liquidierungsentscheidung-Oracle}
  \item Definition der \textit{Auszahlungsschlüssel-Oracle}
  \item Optionale Forderung $\varphi_i \geq 0$
  \item etc.
\end{itemize}

\vspace{0.5cm}

