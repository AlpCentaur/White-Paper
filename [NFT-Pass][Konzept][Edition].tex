% !TEX root = paper.tex

\subsubsection{Edition}

\vspace{0.3cm}

Die Edition unseres WunderPasses soll als Property auf die anfangs geforderte Möglichkeit einer gewissen Individualisierung des WunderPasses durch seinen Besitzer einzahlen. Zu individuell darf eine solche NFT-Property aber auch nicht sein, da der NFT zwingend seinen Eigentümer wechseln können soll, da das ganze Unterfangen mit dem NFT-Pass andernfalls ad absurdum führte.

Um die Edition-Property noch etwas interessanter zu gestalten, sollen Exemplare jeder Edition nicht endlos verfügbar sein, sondern stattdessen irgendwann einmal \textit{aufgebraucht}. In solch einem Fall soll sich der User aber nicht einfach irgendeine andere Edition auswählen müssen, sondern erhält die \textit{"Oberedition"} (Parent) seiner ursprünglich gewünschten Edition. Und falls auch diese \textit{aufgebraucht} sein sollte, die \textit{"Oberedition"} der \textit{"Oberedition"} usw. 

\vspace{0.2cm}

\begin{NFT-Prop}[Edition]

Als Ausprägung der WunderPass-NFT-\textbf{Edition} haben wir uns für \textbf{Städte} der Welt entschieden. Die \textbf{Parent-Edition} einer Stadt ist das dazugehörige \textbf{Land}, dessen 
Parent-Edition wiederum \textbf{Kontinent} und die \textbf{oberste Editions-Ebene} dann die \textbf{Welt-Edition}. Letztere unterliegt folglich auch keiner stückweisen Obergrenze mehr.

\vspace{0.2cm}

\underline{\textbf{\textit{Beispiel einer Edition-Kette:}}}

\vspace{0.2cm}

\begin{equation*}
\textrm{Berlin } \rightarrow \textrm{ Germany } \rightarrow \textrm{ Europe } \rightarrow \textrm{ World }
\end{equation*} 

\vspace{0.2cm}

Es gilt folgendes grobe Regel-Set, was jedoch explizit auch nach Launch modifizierbar bleiben soll:

\begin{itemize}
    \item Die möglichen Editionen werden von uns bestimmt. Diese müssen nicht zwingend beim Launch des NFT vollständig benannt werden, sondern können stattdessen auch nachträglich eingepflegt werden.
    \item Jede berücksichtigte \textit{Städte-Edition} ist genau \textbf{100} Mal verfügbar. Sind alle 100 Exemplare einer \textit{Städte-Edition} bereits gemintet (verbraucht), erhält die nächste Mint-Anfrage nach einem WunderPass derselben Edition automatisch die zu dieser Städte-Edition gehörende \textit{Landes-Edition}.
    \item Die \textit{Landes-Editionen} sind in einer maximalen Stückzahl von je \textbf{10.000} pro berücksichtigtem Land verfügbar. Sind auch diese aufgebraucht, wird die durch den User ausgewählte Stadt auf die ihrem Land übergeordnete \textit{Kontinent-Edition} gemappt.
    \item Die \textit{Kontinent-Editionen} sind in einer maximalen Stückzahl von je \textbf{1.000.000} für jeden Kontinent (außer der Antarktis) vorgesehen. Sollte auch diese Menge irgendwann erschöpfen, greifen wir zu der übergeordneten \textit{Welt-Edition}
\end{itemize} 

\end{NFT-Prop}

\vspace{0.24cm}

\underline{\textbf{Quantitative Daten zu den Editionen:}}

\begin{itemize}
    \item Nach aktuellem Stand sind mindestens 693 \textit{Städte-Edition} vorgesehen.
    \item Die genannten \textit{Städte-Edition} verteilen sich dabei aktuell auf 179 \textit{Landes-Edition}.
    \item Die unterschiedlichen \textit{Kontinent-Edition} belaufen sich auf 6 (Nord- und Südamerika, Europa, Afrika, Asien und Australien).
    \item Die übergeordnete \textit{Welt-Edition} ist in ihrer Stückzahl unbegrenzt. 
    \item 
    \item
\end{itemize}

\vspace{0.5cm}




