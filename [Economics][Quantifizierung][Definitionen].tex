% !TEX root = paper.tex
\subsubsection{Grundlegende Definitionen}
\label{sec:eco_zahlen_def}

Sei $t_0$ der initiale Zeitpunkt all unserer Messungen und Betrachtungen (vermutlich der Zeitpunkt des MVP-Launches).

Darauf aufbauend betrachten wir das künftige Zeitintervall $T$, welches einzig an Relevanz für unser Vorhaben und alle in diesem Kapitel getätigten Ausführungen besitzt:

\begin{equation*}
  T = [t_0; \infty[
\end{equation*}
Der Zeitstrahl muss nicht zwingend unendlich sein. Er muss ebenfalls nicht zwingend infinitesimal fortlaufend sein und kann stattdessen je nach Kontext endlich und/oder diskret betrachtet werden. Also z. B. auch wahlweise als 

\begin{equation*}
  T = [t_0; t_{ende}]
\end{equation*}

\begin{equation*}
  T = [t_0; t_1;...; t_{ende}]
\end{equation*}
definiert sein. In letzteren beiden Fällen wird jedoch $t_{ende}$ in aller Regel eine kontextbezogene (unverzichtbare) Bedeutung haben, die eine solche Definition des Zeitstrahls unverzichtbar macht. So könnte $t_{ende}$ z. B. für eine mathematisch quantifizierbare Erreichung unserer Vision stehen. \\

Sei $\mathbf{t \in T}$ fortan stets ein beliebiger Zeitpunkt, zu welchem wir eine Aussage treffen möchten. \\


Wir definieren die Anzahl aller zum Zeitpunkt $t$ potenziellen User $U^{(t)}$ überhaupt und ihre (maximale) Anzahl $n^{(t)}$ als \\

\begin{Def}\label{defU}
\begin{equation*}
  U^{(t)} = \left\{ u^{(t)}_1; u^{(t)}_2;...; u^{(t)}_{n} \right\}
\end{equation*}
\end{Def} 

\vspace{0.3cm}


Und ganz analog dazu ebenfalls die potenziellen Service-Provider $S^{(t)}$ und ihre (maximale) Anzahl $m^{(t)}$ als \\

\begin{Def}\label{defS}
\begin{equation*}
  S^{(t)} = \left\{ s^{(t)}_1; s^{(t)}_2;...; s^{(t)}_{m}\right\}
\end{equation*}
\end{Def}

\vspace{0.3cm}

Man beachte, dass die definierten Mengen $U^{(t)}$ und $S^{(t)}$ bzw. ihre Größe gewissermaßen den Fortschritt der Digitalisierung insgesamt beschreiben (potenzielle User brauchen einen Zugang zum digitalen Ökosystem und potenzielle Provider sind unabhängige Service-Dienstleister, die eigenmächtig darüber entscheiden, zu solchen zu werden) und in keiner Weise im Einfluss Wunderpasses stehen. Viel mehr beschreiben sie die "Umstände der Welt", mit denen WunderPass (wie alle anderen) "arbeiten" müssen.  

\vspace{0.6cm}


Nun definieren den \textbf{\textit{Connection-Koeffizienten}} zwischen den eben definierten potenziellen Usern $\mathbf{U^{(t)}}$ und den Service-Providern $\mathbf{S^{(t)}}$ zum Zeitpunkt $t$ als boolesche Funktion $\mathbf{\alpha^{(t)}}$, die über über die Tatsache \textit{"is connected"} bzw. \textit{"is not connected"} entscheidet: \\

\begin{Def}\label{defKoeff}

\begin{equation*}
  \alpha^{(t)} : U^{(t)} \times S^{(t)} \rightarrow \{0; 1\} 
\end{equation*}

\[
\alpha^{(t)}(u, s):=\left\{%
\begin{array}{ll}
    1, & \hbox{falls User $u \in U^{(t)}$ mit mit Provider $s \in S^{(t)}$ connectet ist} \\
    0, & \hbox{andernfalls} \\
\end{array}%
\right.
\]

\vspace{1cm}

Bzw. wenn man die diskreten Auslegungen der Pools $U^{(t)} = \left\{ u^{(t)}_1; u^{(t)}_2;...; u^{(t)}_{n} \right\}$ und $S^{(t)} = \left\{ s^{(t)}_1; s^{(t)}_2;...; s^{(t)}_{m} \right\}$ heranzieht, alternativ als

\[
\alpha^{(t)}_{ij}:=\left\{%
\begin{array}{ll}
    1, & \hbox{falls User $u^{(t)}_i \in U^{(t)}$ mit mit Provider $s^{(t)}_j \in S^{(t)}$ connectet ist} \\
    0, & \hbox{andernfalls} \\
\end{array}%
\right.
\]

\end{Def}

\vspace{0.6cm}

Man beachte, dass wir bei den diskreten/Aufzählungs-basierten Definitionen oben, der Übersicht halber etwas "geschlampt" haben, indem wir - klar zeitbedingte - Indizes stillschweigend als $n$ und $m$ bezeichnet haben, gleichwohl diese korrekterweise $n^{(t)}$ und $m^{(t)}$ lauten müssten. Nur verwirrt eben ein Ausdruck wie $u^{(t)}_{n^{(t)}}$ mehr, als dieser in seiner pedantischen Korrektheit einen Mehrwert generiert. Wir werden genannte Ungenauigkeit zudem im weiteren Verlauf in gleicher Weise fortführen und gehen davon aus, der Leser wisse damit umzugehen. 

\vspace{0.3cm}


