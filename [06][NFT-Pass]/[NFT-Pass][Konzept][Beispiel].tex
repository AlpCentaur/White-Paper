% !TEX root = paper.tex

Aus der in den vorigen Kapiteln formulierten Logik ergibt sich gleich vorweg folgende Implikation:

\vspace{0.3cm}

\begin{Fazit}[Vielfalt der Variationen]

Bei aktuellem Setting ergeben sich in Summe

\begin{equation*}
9 \cdot 8 \cdot 9 \cdot \left(693 + 179 + 6 + 1 \right) = \textrm{\textbf{569.592}}
\end{equation*} 

unterschiedliche Kombinationen an möglichen NFT-Pässen.
	
\end{Fazit}


\vspace{0.3cm}

Um ein besseres Gefühl über die formulierte Logik unseres NFTs zu bekommen, wollen wir ein Beispiel mit konkreten Zahlen rechnen und begeben uns dazu eine gute Weile in die Zukunft - zu einem Zeitpunkt, zu dem bereits genau 316.157 NFT-Pässe gemintet worden sind. Ich als potenzieller Interessent an einem Pass-NFT möchte verstehen, welchen Pass ich als nächsten in etwa zu erwarten hätte.

\vspace{0.2cm} 

Wir analysieren die 316.157 bereits geminteten Pässe.

\vspace{0.5cm} 

\underline{\textbf{Status:}}

\begin{itemize}
  \item Es wurden 200 Pässe des Status \textit{diamond} gemintet.
  \item Es wurden 1.600 Pässe des Status \textit{black} gemintet.
  \item Es wurden 12.800 Pässe des Status \textit{pearl} gemintet.
  \item Es wurden 102.400 Pässe des Status \textit{platin} gemintet.
  \item Es wurden 199.157 Pässe des Status \textit{ruby} gemintet.
  \item Von den insgesamt 819.200 vorgesehenen \textit{ruby} Pässen sind demnach noch 620.043 verfügbar.
\end{itemize}

\vspace{0.2cm}

\textit{\textbf{Unser Pass wird also definitiv den Status 'Ruby' haben!}}

\vspace{1.0cm}


\underline{\textbf{Hologramm:}}

\vspace{0.2cm}

Hinsichtlich der Hologramme können wir nur über die ersten 315.904 der 316.157 bisher geminteten Pässe eine definitive Aussage treffen. Die übrigen 253 folgen einer gewissen Wahrscheinlichkeitsverteilung. Zunächst zu den ersten 315.904:

\begin{itemize}
  \item Es wurden 159.186 Pässe mit dem Hologramm der \textit{Jesus-Statue} gemintet.
  \item Es wurden 78.976 Pässe mit dem Hologramm des \textit{Maj Mahal} gemintet.
  \item Es wurden 39.488 Pässe mit dem Hologramm des \textit{Machu Picchu} gemintet.
  \item Es wurden 19.744 Pässe mit dem Hologramm der \textit{Chichén Itzá} gemintet.
  \item Es wurden 9.872 Pässe mit dem Hologramm des \textit{Kolosseum} gemintet.
  \item Es wurden 4.936 Pässe mit dem Hologramm der \textit{Petra} gemintet.
  \item Es wurden 2.468 Pässe mit dem Hologramm der \textit{Chinesischen Mauer} gemintet.
  \item Es wurden immerhin stolze 1.234 Pässe mit dem seltensten Hologramm der \textit{Pyramiden von Gizeh} gemintet.
\end{itemize}

\vspace{0.3cm}

Die Evaluierung der übrigen 253 ist insofern recht dankbar, als dass die 253 schon sehr nah an der zyklischen 256 liegt ($= 2^{n}$, wobei $n=8$ für die acht verfügbaren Hologramme steht). Damit beschränkt sich die Analyse eigentlich lediglich auf die nächsten drei Pässe, von denen der erste unserer ist. 

Die Auswahl der möglichen Hologramme für die nächsten 3 Pässe bis zum Abschluss des aktuellen 256er-Zyklus ist recht begrenzt. Zum besseren Verständnis dessen vergegenwärtige man sich noch einmal die Veranschaulichung mit der Lostrommel aus Abschnitt \ref{sec:hologramm}. Stattdessen erscheint bei der Vergabe-Logik der Hologramme ein Szenario nicht ganz unwahrscheinlich, bei dem für die restlichen drei Pässe des Zyklus noch 1-2 \textit{Jesus-Statuen} verfügbar sind, im Falle nur einer \textit{Jesus-Statue} zusätzlich ein \textit{Taj-Mahal-Hologramm} und das dritte und letzte Hologramm mit etwas Glück auf ein etwas selteneres entfällt.

Wir setzen der Einfachheit halber voraus, im Besitz von Gottes Würfel gewesen zu sein und legen das eingetroffene Szenario auf folgendes fest:

\begin{itemize}
  \item Unter den 253 geminteten Hologrammen des aktuellen Zyklus sind: Eine Pyramide, 2 Chinesische Mauern, 4 Petras, 8 Kolosseen, \textbf{15 Chichén Itzás}, 32 Machu Picchus, 64 Taj Mahals und \textbf{127 Jesus-Statuen}.
  \item Für die letzten drei Pässe des aktuellen Zyklus sind als Hologramme also noch einmal die \textit{Chichén Itzá} und zweimal die \textit{Jesus-Statue} verfügbar.
\end{itemize}

\vspace{0.3cm}

\textbf{\textit{Die Wahrscheinlichkeit für unseren NFT-Pass, als Hologramm die Chichén Itzá zu erhalten, liegt also bei 33,33 \% und für die Jesus-Statue bei 66,67 \%.}}


\vspace{1.0cm}


\underline{\textbf{Pattern:}}

\vspace{0.2cm}

Da die \textit{Pattern}-Property einer trivialen Wahrscheinlichkeitsverteilung unterliegt, ist es bei dieser Property unmöglich, eine exakte Angabe zu der tatsächlichen Verteilung der \textit{Pattern} auf die 316.157 bisher geminteten Pässe machen zu können. Da die Zahl der geminteten Pässe jedoch sehr groß ist, liege es nahe, die tatsächliche Verteilung entspräche nahezu exakt der in Property \ref{pattern} angegebenen Wahrscheinlichkeitsverteilung. Für eine Prognose über etwaige Wahrscheinlichkeiten von möglichen Pattern für sowohl die nächsten 3 Pässe als auch den tatsächlich nächsten bleibt uns nichts anderes übrig, als mit denselben Wahrscheinlichkeits-Angaben aus Property \ref{pattern} zu kalkulieren.

\vspace{0.2cm}

\textbf{\textit{Wir gehen stillschweigend davon aus, hinsichtlich Pattern bei unserem zu mintenden NFT-Pass Pech zu haben, und eines der beiden häufigsten aller verfügbaren Pattern zu erwischen: Nämlich die 'Curves' oder das 'Linear'.}}


\vspace{1.0cm}


\underline{\textbf{Edition:}}

\vspace{0.2cm}

\textbf{\textit{Als Edition wählen wir 'Berlin', von dem wir ausgehen, es sei noch verfügbar.}}

\vspace{1.0cm}



\underline{\textbf{Unser NFT-Pass:}}

\vspace{0.2cm}

Unter Berücksichtigung der bisher zusammengetragenen Ergebnisse und Wahrscheinlichkeits-Annahmen, erhalten wir mit 75-prozentiger Wahrscheinlichkeit einen der folgenden 4 Pässe:

\newpage


\begin{figure}[h!]
  \centering
  \subfloat[][]{\includegraphics[width=0.4\linewidth]{{"[06][NFT-Pass]/images/pass 1"}}}%
  \qquad
  \subfloat[][]{\includegraphics[width=0.4\linewidth]{{"[06][NFT-Pass]/images/pass 2"}}}%
  \caption{mit einer Wahrscheinlichkeit von 50 \% bekommen wir einen dieser beiden Pässe (zu 33,33 \% den linken und zu 16,67 \% den rechten)}
\end{figure}

\begin{figure}[h!]
  \centering
  \subfloat[][]{\includegraphics[width=0.4\linewidth]{{"[06][NFT-Pass]/images/pass 3"}}}%
  \qquad
  \subfloat[][]{\includegraphics[width=0.4\linewidth]{{"[06][NFT-Pass]/images/pass 4"}}}%
  \caption{mit einer Wahrscheinlichkeit von 25 \% bekommen wir einen dieser beiden Pässe (zu 16,67 \% den linken und zu 8,33 \% den rechten)}
\end{figure}

\vspace{0.3cm}

Die restlichen 25 \% der hier nicht berücksichtigten Fälle unterscheiden sich von den oben dargestellten schlichtweg in den (selteneren) \textit{Pass-Pattern}.

\vspace{0.5cm}