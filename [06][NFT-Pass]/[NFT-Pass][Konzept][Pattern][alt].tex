% !TEX root = C:/Users/Slava/White-Paper/[06][NFT-Pass]/[NFT-Pass][Konzept].tex

\subsubsection{Pattern-Property}

\vspace{0.3cm}

\begin{sloppypar}
Diese NFT-Property soll ebenso wie die beiden vorigen einem abstufenden Raritätsprinzip zu Grunde liegen - und zwar ausschließlich dem Zufall folgend.
\end{sloppypar}

Im Gegensatz zu den beiden vorigen Properties obliegt die \textit{Pattern-Property} keiner absoluten Obergrenze - insbesondere auch dann nicht, falls einige Pattern zu einem Zeitpunkt verhältnismäßig unter- oder überrepräsentiert sind.

\vspace{0.3cm}

\begin{NFT-Prop}[Background (Pattern)]

Wir definieren folgende \textit{NFT-Pass-Background-Muster} mit den dazugehörenden Eigenschaften:

\begin{itemize}
    \item M1
    \begin{itemize}
    	\item Mögliche Ausprägung: \textbf{Safari} 
    	\item maximale Anzahl Pässe: 256
    	\item Wahrscheinlichkeit falls noch nicht aufgebraucht: 1,5625\% $\left( \frac{1}{64} \right)$
    \end{itemize}
    \item M2
    \begin{itemize}
    	\item Mögliche Ausprägung: \textbf{Bars} 
    	\item maximale Anzahl Pässe: 4.096
    	\item Wahrscheinlichkeit falls noch nicht aufgebraucht: 3,125\% $\left( \frac{1}{32} \right)$
    \end{itemize}
    \item M3
    \begin{itemize}
    	\item Mögliche Ausprägung: \textbf{Dots} 
    	\item maximale Anzahl Pässe: 65.536
    	\item Wahrscheinlichkeit falls noch nicht aufgebraucht: 6,25\% $\left( \frac{1}{16} \right)$
    \end{itemize}
    \item M4
    \begin{itemize}
    	\item Mögliche Ausprägung: \textbf{Muster festlegen} 
    	\item maximale Anzahl Pässe: 1.048.576
    	\item Wahrscheinlichkeit falls noch nicht aufgebraucht: 12,5\% $\left( \frac{1}{8} \right)$
    \end{itemize}
    \item M5
    \begin{itemize}
    	\item Mögliche Ausprägung: \textbf{Muster festlegen}  
    	\item maximale Anzahl Pässe: 16.777.216
    	\item Wahrscheinlichkeit falls noch nicht aufgebraucht: 25\% $\left( \frac{1}{4} \right)$
    \end{itemize}
    \item M6
    \begin{itemize}
    	\item Mögliche Ausprägung: \textbf{Muster festlegen}
    	\item maximale Anzahl Pässe: unbegrenzt  
    	\item Wahrscheinlichkeit: 50\% + x mit stets größer werdendem $x \in \left[ \frac{1}{64}; \frac{1}{2} \right]$
    \end{itemize}
\end{itemize}

\vspace{0.2cm}

Werden Pässe der Ausprägungen M1 bis M5 (aufgrund ihres Caps) im Laufe der Zeit aufgebraucht, geht deren Wahrscheinlichkeit auf die Property-Ausprägung M6 über. Für das x aus der Beschreibung der Ausprägung M6 gilt also:

\vspace{0.2cm}

\begin{equation*}
x = \frac{1}{64} + \sum_{aufgebrauchte \textrm{ } M \in \lbrace M1;...;M5 \rbrace} Wahrscheinlichkeit(M)
\end{equation*}

\end{NFT-Prop}

\vspace{0.3cm}

\begin{Algo}[Verlosungs-Mechanismus für Background-Property]

\begin{itemize}
    \item Zunächst bestimme man die Gesamtanzahl aller bisher geminteter Pässe $n$.
    \item Gleiches tue man nun für die Counts der geminteten Pässe pro Ausprägung der Background-Property M1 bis M6 als entsprechende Größen $n_1, n_2,...,n_6$.
    \item Seien $\Theta_i$ für $i \in \lbrace 1,...6 \rbrace$ die oben definierten \textbf{absoluten} Obergrenzen der \newline Ausprägungen der Background-Property M1 bis M6.
    \item Seien $\rho_i$ für $i \in \lbrace 1,...6 \rbrace$ die oben definierten Wahrscheinlichkeiten der \newline Ausprägungen der Background-Property M1 bis M6, deren Auswahl wir hier zusätzlich formalisieren wollen:
    
\[
\rho_i:=\left\{%
\begin{array}{ll}
    \hbox{\LARGE $\frac{1}{2^{7 - i}}$,} & \hbox{für $i = 1,...,5$} \\[0,3cm]
    \hbox{$\frac{1}{2} + \frac{1}{2^6}$,} & \hbox{für $i = 6$} \\
\end{array}%
\right.
\]    
    
    \item Alle Ausprägungen mit $n_i \geq \Theta_i$ sind aufgebraucht und können weder zum aktuellen Zeitpunkt noch in der Zukunft vergeben werden und damit fortan auch nicht beim Minting eines neuen Pass berücksichtigt werden.
    \item Wir unterteilen die Ausprägungen der Background-Property in \textit{"verbraucht"} und \textit{"verfügbar"}:
    
\begin{align*}
\overline{M} &:= \lbrace i \in \lbrace 1,...,6 \rbrace \textrm{ } | \textrm{ } n_i \geq \Theta_i \rbrace \textrm{     [verbraucht]} \\
M &:= \lbrace i \in \lbrace 1,...,6 \rbrace \textrm{ } | \textrm{ } n_i < \Theta_i \rbrace \textrm{     [verfügbar]}
\end{align*} 

Man vergewissere sich an dieser Stelle gedanklich, dass $M \cap \overline{M} = \emptyset$, $M \cup \overline{M} = \lbrace 1,...,6 \rbrace$ und $6 \in M$ gelten.
    
    \item Ist eine bestimmte Ausprägung $i \in \lbrace 1,...,5 \rbrace$ verbraucht, soll ihre Wahrscheinlichkeit $\rho_i$ auf die Ausprägung M6 übertragen werden (damit wir bei 100\% bleiben).

    \item Damit errechnen wir die aktuell vorliegenden neuen Wahrscheinlichkeiten $\widehat{\rho}_i$ für unsere Background-Ausprägungen als
\end{itemize} 

\[
\widehat{\rho}_i:=\left\{%
\begin{array}{ll}
	\hbox{0,} & \hbox{für $i \in \overline{M}$} \\[0,1cm] 
    \hbox{$\rho_i$,} & \hbox{für $i \in M$, $i \neq 6$} \\[0,3cm]
    \displaystyle \rho_1 + \sum_{i \in \overline{M}} \rho_i, & \hbox{für $i = 6$} \\
\end{array}%
\right.
\]

Man vergewissere sich an dieser Stelle gedanklich, dass auch für die neuen \newline Wahrscheinlichkeiten \[\sum_{i = 1}^6 \widehat{\rho}_i \textrm{ } = 1\] gilt.

\begin{itemize}
    \item Am Ende bestimme man mittels Randomisierung anhand der Wahrscheinlichkeiten $\widehat{\rho}_i$ für $i \in \lbrace 1,...6 \rbrace$ die zu vergebende Background-Ausprägung. 
\end{itemize}

\end{Algo}

\vspace{0.3cm}

