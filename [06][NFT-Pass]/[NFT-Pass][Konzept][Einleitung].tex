% !TEX root = paper.tex

Unser Anspruch an den zu modellierenden \textit{NFT-Pass} ist grob der folgende:

\vspace{0.2cm}

\begin{itemize}
  \item Der \textit{NFT-Pass} muss sich ganz klar von dem Großteil der heutigen - in größter Regel als Sammlerstück verstandenen - den Markt überflutenden NFTs abgrenzen. Er braucht einen klar ersichtlichen \textbf{intrinsischen Wert}. Man muss also "etwas mit dem \textit{NFT-Pass} anfangen/machen können" und diesen nicht "lediglich besitzen", um ihn ausschließlich mit einer gewissen Wahrscheinlichkeit gewinnbringend weiterverkaufen zu können ("Hot Potato"). Der Token bedarf also gewisser Eigenschaften eines \textit{Governance-Tokens} (DAO) oder Ähnlichem.
  \item 
  \begin{sloppypar}  
  Der \textit{NFT-Pass} braucht ungeachtet des vorigen Bullet-Points jedoch trotzdem zusätzlich ebenso eine ähnliche Beschaffenheit - wie solche der aktuell üblichen marktbeherrschenden NFTs - als Sammlerstück - gleichwohl nicht erstrangig. 
  \end{sloppypar}
  \item Anders als die aktuell gängigen NFTs soll unser \textit{NFT-Pass} \textbf{nicht begrenzt} in der Anzahl seiner Stücke sein. Stattdessen sollen theoretisch beliebig viele \textit{NFT-Pässe} existieren können. Nichtsdestotrotz soll unser \textit{NFT-Pass} ebenso die Eigenschaft der "nicht inflationären Begehrtheit" einverleibt bekommen. Dies möchten wir mittels einer ausgeklügelten Minting-Logik abbilden, die ein \textbf{endliches Sub-Set} an raren und begehrten \textit{NFT-Pässen} innerhalb des \textbf{unendlichen Gesamt-Sets} der \textit{NFT-Pässe} sicherstellt. Soll heißen: Einerseits werden \textit{NFT-Pässe} exis\-tieren, die den heutigen NFTs - im Sinne ihres Sammlerwertes - gleichkommen. Andererseits werden die restlichen mit ihrer steigenden Gesamtanzahl zunehmend entwertet, bis sie irgendwann (als NFT betrachtet) nahezu wertlos und lediglich "funktional" werden.
  \item Die Rarität und Begehrtheit unseres \textit{NFT-Pass} soll Gamification-Mechanismen folgen:
  \begin{itemize}
    \item Wir brauchen an etwaigen Stellen ein (wertbestimmendes) \textit{first-come-first-serve-Prinzip}.
    \item Wir brauchen an anderen Stellen ein (ebenso wertbestimmendes) Zufallsprinzip.
    \item Wir brauchen irgendwo ebenso ein (geringes) Maß an persönlicher Individua\-lisierung des \textit{NFT-Pass} - ausschließlich durch den User gesteuert.
    \item Abrundend könnte ein \textbf{gemeinnützig wertbestimmendes} (randomisiertes) Merkmal wirken. (Beispiel: Wenn die \textit{NFT-Pässe} irgendwann inflationär geworden sind, könnte der zehn-millionste plötzlich wieder richtig krass sein.)
  \end{itemize}
  \item Der \textit{NFT-Pass} muss gänzlich transparent und vor allem verständlich für den interessierten - gleichwohl vielleicht technisch nicht bewandertsten - User sein.
\end{itemize}

\vspace{0.3cm}

In den kommenden Abschnitten folgt ein initialer Abriss unserer Vorstellung des \textit{NFT-Pass}:

\vspace{0.3cm}
