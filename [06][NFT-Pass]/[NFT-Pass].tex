% !TEX root = paper.tex

\section{NFT-Pass}
\label{sec:nft-pass}

Ein exzellentes Mittel, um \textit{WunderPass} als Geschäftsmodell, Unternehmung und Unternehmen ein symbolisches - gewissermaßen plastisches - Sinnbild einzuverleiben, ist die Repräsentation von \textit{WunderPass} als Service/Protokoll mittels eines - eigens dafür kreierten - NFTs: \textbf{"Des WunderPass"} (im Folgenden auch \textit{NFT-Pass} genannt)

\vspace{0.3cm}

\begin{Fazit}[\textit{WunderPass} deabstrahiert durch \textbf{"den WunderPass"} als NFT]

"Ich nutze \textit{WunderPass}" wird symbolisiert durch "Ich besitze \textbf{meinen WunderPass}"!

\end{Fazit}

\vspace{0.3cm}

% !TEX root = paper.tex

\subsection{Konzeption}

\vspace{0.3cm}

Unser Anspruch an den zu modellierenden \textit{NFT-Pass} ist grob der folgende:

\vspace{0.2cm}

\begin{itemize}
  \item Der \textit{NFT-Pass} muss sich ganz klar von dem Großteil der heutigen - in größter Regel als Sammlerstück verstandenen - den Markt überflutenden NFTs abgrenzen. Er braucht einen klar ersichtlichen \textbf{intrinsischen Wert}. Man muss also "etwas mit dem \textit{NFT-Pass} anfangen/machen können" und diesen nicht "lediglich besitzen", um ihn ausschließlich mit einer gewissen Wahrscheinlichkeit gewinnbringend weiterverkaufen zu können ("Hot Potato"). Der Token bedarf also gewisser Eigenschaften eines \textit{Governance-Tokens} (DAO) oder Ähnlichem.
  \item 
  \begin{sloppypar}  
  Der \textit{NFT-Pass} braucht ungeachtet des vorigen Bullet-Points jedoch trotzdem zusätzlich ebenso eine ähnliche Beschaffenheit - wie solche der aktuell üblichen marktbeherrschenden NFTs - als Sammlerstück - gleichwohl nicht erstrangig. 
  \end{sloppypar}
  \item Anders als die aktuell gängigen NFTs soll unser \textit{NFT-Pass} \textbf{nicht begrenzt} in der Anzahl seiner Stücke sein. Stattdessen sollen theoretisch beliebig viele \textit{NFT-Pässe} existieren können. Nichtsdestotrotz soll unser \textit{NFT-Pass} ebenso die Eigenschaft der "nicht inflationären Begehrtheit" einverleibt bekommen. Dies möchten wir mittels einer ausgeklügelten Minting-Logik abbilden, die ein \textbf{endliches Sub-Set} an raren und begehrten \textit{NFT-Pässen} innerhalb des \textbf{unendlichen Gesamt-Sets} der \textit{NFT-Pässe} sicherstellt. Soll heißen: Es werden einerseits \textit{NFT-Pässe} exis\-tieren, die den heutigen NFTs - im Sinne ihres Sammlerwertes - gleichkommen, während die restlichen andererseits mit ihrer steigenden Gesamtanzahl zunehmend entwerten, bis sie irgendwann (als NFT betrachtet) nahezu wertlos und lediglich "funktional" werden.
  \item Die Rarität und Begehrtheit unseres \textit{NFT-Pass} soll Gamification-Mechanismen folgen:
  \begin{itemize}
    \item Wir brauchen an etwaigen Stellen ein (wertbestimmendes) \textit{first-come-first-serve-Prinzip}.
    \item Wir brauchen an anderen Stellen ein (ebenso wertbestimmendes) Zufallsprinzip.
    \item Wir brauchen irgendwo ebenso ein (geringes) Maß an persönlicher Individua\-lisierung des \textit{NFT-Pass} - ausschließlich durch den User gesteuert.
    \item Abrundend könnte ein \textbf{gemeinnützig wertbestimmendes} (randomisiertes) Merkmal wirken. (Beispiel: Wenn die \textit{NFT-Pässe} irgendwann inflationär geworden sind, könnte der zehn-millionste plötzlich wieder richtig krass sein.)
  \end{itemize}
  \item Der \textit{NFT-Pass} muss gänzlich transparent und vor allem verständlich für den interessierten - gleichwohl vielleicht technisch nicht bewandertsten - User sein.
\end{itemize}

\vspace{0.3cm}

In den kommenden Abschnitten folgt ein initialer Abriss unserer Vorstellung des \textit{NFT-Pass}:

\vspace{0.3cm}


% !TEX root = paper.tex

\subsubsection{Status-Property}

\vspace{0.2cm}

Diese NFT-Property - die wir gleichzeitig als die Main-Property unseres \textit{NFT-Pass} ansehen - soll der oben formulierten Anforderung nach einem first-come-first-serve-Prinzip Rechnung tragen. Zeitlich früher ausgestellte NFT-Pässe sollen einen rareren und begehrteren \textit{Pass-Status} inne haben als die späteren. Und vor allem sollen die \textit{NFT-Pässe} eines bestimmten ausgestellten Status in ihrer Anzahl begrenzt sein und nach Erreichen einer zu definierenden Höchstgrenze fortan nie wieder ausgestellt (ge\-mintet) werden können.

\vspace{0.3cm}

\begin{NFT-Prop}[Pass-Status]

Wir definieren folgende \textit{NFT-Pass-Status} mit den dazugehörenden Eigenschaften:

\begin{itemize}
    \item Status A (\textbf{Diamond})
    \begin{itemize}
    	\item Anzahl Pässe: 200
    	\item Gemintet an Nummer: 1 bis 200
    \end{itemize}
    \item Status B (\textbf{Black})
    \begin{itemize}
    	\item Anzahl Pässe: 1.600
    	\item Gemintet an Nummer: 201 bis 1800
    \end{itemize}
    \item Status C (\textbf{Pearl})
    \begin{itemize}
    	\item Anzahl Pässe: 12.800
    	\item Gemintet an Nummer: 1801 bis 14.600
    \end{itemize}
    \item Status D (\textbf{Platin})
    \begin{itemize}
    	\item Anzahl Pässe: 102.400
    	\item Gemintet an Nummer: 14.601 bis 117.000
    \end{itemize}
    \item Status E (\textbf{Ruby})
    \begin{itemize}
    	\item Anzahl Pässe: 819.200
    	\item Gemintet an Nummer: 117.001 bis 936.200
    \end{itemize}
    \item Status F (\textbf{Gold})
    \begin{itemize}
    	\item Anzahl Pässe: 6.553.600
    	\item Gemintet an Nummer: 936.201 bis 7.489.800
    \end{itemize}
    \item Status G (\textbf{Silver})
    \begin{itemize}
    	\item Anzahl Pässe: 52.428.800
    	\item Gemintet an Nummer: 7.489.801 bis 59.918.600
    \end{itemize}
    \item Status H (\textbf{Bronze})
    \begin{itemize}
    	\item Anzahl Pässe: 419.430.400
    	\item Gemintet an Nummer: 59.918.601 bis 479.349.000
    \end{itemize}
    \item Status I (\textbf{White})
    \begin{itemize}
    	\item Anzahl Pässe: $\infty$
    	\item Gemintet an Nummer: 479.349.001 bis $\infty$
    \end{itemize}
\end{itemize}

\end{NFT-Prop}

\vspace{0.3cm}

Diese NFT-Property ist per Definition trivialerweise \textbf{deterministisch}: Es ist stets zweifellos klar, welchen Status ein an x-ter Stelle geminteter \textit{NFT-Pass} haben wird. Die hinzugezogene "Reverse-Halving-Logik" \textbf{belohnt die Early-Adopter} mit einem begehrten NFT, dessen Rarität per Protokoll mit der Zeit stets abnimmt.

\vspace{0.1cm}


    % binde die Datei '[NFT-Pass][Konzept][Status].tex' ein

Die Beschaffenheit dieser \textit{first-come-first-serve-Property} soll jedoch einzigartig bleiben. Die folgenden Properties werden nicht mehr deterministisch sein, um unserem \textit{NFT-Pass} ein \textbf{unvorherbestimmbares "Eigenleben"} einzuverleiben. 

% !TEX root = paper.tex

\subsubsection{Wunder-Property}

\vspace{0.3cm}

Diese NFT-Property soll zwar einem ähnlichen abstufenden Raritätsprinzip zu Grunde liegen wie die Main-Property, dies jedoch nicht mehr einem first-come-first-serve- sondern stattdessen einem Zufallsprinzip folgend.

Ebenfalls abweichend von der Beschaffenheit der Main-Property soll bei dieser Pro\-perty die Rarität nicht mittels einer absoluten Obergrenze abgebildet werden, sondern mittels einer relativen. (Dies zahlt auf die oben formulierte Anforderung nach einem \textbf{gemeinnützig gewinnbringendem Value} unseres \textit{NFT-Pass} ein.

\vspace{0.3cm}

\begin{NFT-Prop}[Hologramm (Welt-Wunder)]

Wir definieren folgende \textit{NFT-Pass-Hologramme} mit den dazugehörenden Eigenschaften:

\begin{itemize}
    \item WW1
    \begin{itemize}
    	\item Mögliche Ausprägung: \textbf{Pyramids of Giza}
    	\item Anteil Pässe: 0,390625\% $\left( \frac{1}{256} \right)$
    \end{itemize}
    \item WW2
    \begin{itemize}
    	\item Mögliche Ausprägung: \textbf{Great Wall of China}
    	\item Anteil Pässe: 0,78125\% $\left( \frac{1}{128} \right)$
    \end{itemize}
    \item WW3
    \begin{itemize}
    	\item Mögliche Ausprägung: \textbf{Petra} 
    	\item Anteil Pässe: 1,5625\% $\left( \frac{1}{64} \right)$
    \end{itemize}
    \item WW4
    \begin{itemize}
    	\item Mögliche Ausprägung: \textbf{Colosseum} 
    	\item Anteil Pässe: 3,125\% $\left( \frac{1}{32} \right)$
    \end{itemize}
    \item WW5
    \begin{itemize}
    	\item Mögliche Ausprägung: \textbf{Chichén Itzá} 
    	\item Anteil Pässe: 6,25\% $\left( \frac{1}{16} \right)$
    \end{itemize}
    \item WW6
    \begin{itemize}
    	\item Mögliche Ausprägung: \textbf{Machu Picchu} 
    	\item Anteil Pässe: 12,5\% $\left( \frac{1}{8} \right)$
    \end{itemize}
    \item WW7
    \begin{itemize}
    	\item Mögliche Ausprägung: \textbf{Taj Mahal} 
    	\item Anteil Pässe: 25\% $\left( \frac{1}{4} \right)$
    \end{itemize}
    \item WW8
    \begin{itemize}
    	\item Mögliche Ausprägung: \textbf{Christ the Redeemer} 
    	\item Anteil Pässe: 50\% + x $\left( \frac{1}{2} + \frac{1}{256} \right)$
    \end{itemize}
\end{itemize}

\end{NFT-Prop}

\vspace{0.3cm}

Das Besondere an dieser Property spiegelt sich in der Tatsache wider, gewisse rar beschaffene Ausprägungen seien nur "zeitweise" ausgeschöpft, da sich ihre (rare) Anzahl lediglich \textbf{relativ} an der Gesamtzahl der aktuell \textit{ausgestellten NFT-Pässe} bemisst und nicht wie die Main-Property einer absoluten Obergrenze obliegt, deren Erreichung unumkehrbar ist. Soll heißen: Ist die prozentuale Obergrenze an Pässen mit einer beChrist the Redeemerstimmten Ausprägung der gegenwärtigen Property zu einem be\-stimmten Zeitpunkt erreicht, kann zwar für einen gewissen Zeitraum kein Pass mit dieser Ausprägung mehr ausgestellt werden. Sobald jedoch die Gesamtanzahl der \textit{ausgestellten NFT-Pässe} wieder groß genug ist - sodass die Anzahl der vorhandenen \textit{NFT-Pässe} mit der betroffenen Ausprägung wieder die prozentuale Obergrenze unterschreitet - werden Pässe der besagten Ausprägung "wieder verfügbar".

\vspace{0.3cm}

\begin{Algo}[Verlosungs-Mechanismus für Hologramm-Property]

\begin{itemize}
    \item Zunächst bestimme man die Gesamtanzahl aller bisher geminteter Pässe $n$.
    \item Gleiches tue man nun für die Counts der geminteten Pässe pro Ausprägung der Hologramm-Property WW1 bis WW8 als entsprechende Größen $n_1, n_2,...,n_8$.
    \item Und damit anschließend die aktuelle prozentuale Verteilung der Ausprägung auf die aktuell geminteten Pässe als $\sigma_i:= \frac{n_i}{n}$ für $i \in \lbrace 1,...,8 \rbrace$ berechnen.
    \item Seien $\Theta_i$ für $i \in \lbrace 1,...,8 \rbrace$ die oben definierten \textbf{relativen} Obergrenzen der \newline Ausprägungen der Hologramm-Property WW1 bis WW8.
    \item Alle Ausprägungen mit $\sigma_i \geq \Theta_i$ können zum aktuellen Zeitpunkt nicht vergeben werden und damit auch nicht beim Minting eines neuen Pass berücksichtigt werden.
    \item Für die Ausprägungen mit $\sigma_i < \Theta_i$ berechnen wir den Normierungsfaktor
\end{itemize} 

\begin{equation*}
\omega := \sum_{\sigma_i < \Theta_i} \Theta_i \textrm{ } \leq 1
\end{equation*} 

\begin{itemize}
    \item Damit errechnen wir die aktuell vorliegenden Wahrscheinlichkeiten $\rho_i$ für unsere Hologramm-Ausprägungen als
\end{itemize} 

\[
\rho_i:=\left\{%
\begin{array}{ll}
    0, & \hbox{falls $\sigma_i \geq \Theta_i$} \\[0,3cm]
    \hbox{\LARGE $\frac{\Theta_i}{\omega}$,} & \hbox{falls $\sigma_i < \Theta_i$}. \\
\end{array}%
\right.
\] 

Man vergewissere sich an dieser Stelle gedanklich, auch für die neuen \newline Wahrscheinlichkeiten gelte \[\sum_{i = 1}^7 \rho_i \textrm{ } = 1.\]

\begin{itemize}
    \item Am Ende bestimme man mittels Randomisierung anhand der Wahrscheinlichkeiten $\rho_i$ für $i \in \lbrace 1,...7 \rbrace$ die zu vergebende Hologramm-Ausprägung. 
\end{itemize}

\end{Algo}

\vspace{0.3cm}

Was hier so kompliziert klingt, lässt sich aber super simpel veranschaulichen:

Die \textit{Verlosung} der Wunder erfolgt in einem periodischen 256er-Turnus ($256 = 2^{n}$ mit $n=8$ für die acht bereitgestellten Hologramme). Nach jedem 256. geminteten Pass schmeißt man 256 Lose in eine Lostrommel: Ein Los für die \textit{Pyramiden}, zwei für die \textit{Chinesische Mauer}, vier für \textit{Petra} etc. Die \textit{Jesus-Statue} kommt letztendlich mit 129 Losen in die Trommel.

Nun ziehen wir blind ein Los und vergeben das gezogenen Hologramm an den nächsten zu mintenden NFT-Pass. Wir tun dies solange, bis die Trommel leer ist. Anschließend fangen wir wieder von Vorne an und befüllen die Trommel erneut mit denselben 256 Losen.

\textbf{Achtung:} Wir befüllen die Trommel ausschließlich nachdem sie komplett leer geworden ist und nicht etwa zwischendurch mal.

\vspace{0.3cm}

    % binde die Datei '[NFT-Pass][Konzept][Wunder].tex' ein
%% !TEX root = paper.tex

\subsubsection{Wunder-Property}

\vspace{0.3cm}

Diese NFT-Property soll zwar einem ähnlichen abstufenden Raritätsprinzip zu Grunde liegen wie die Main-Property, dies jedoch nicht mehr einem first-come-first-serve- sondern stattdessen einem Zufallsprinzip folgend.

Ebenfalls abweichend von der Beschaffenheit der Main-Property soll bei dieser Pro\-perty die Rarität nicht mittels einer absoluten Obergrenze abgebildet werden, sondern mittels einer relativen. (Dies zahlt auf die oben formulierte Anforderung nach einem \textbf{gemeinnützig gewinnbringendem Value} unseres \textit{NFT-Pass} ein.

\vspace{0.3cm}

\begin{NFT-Prop}[Hologramm (Welt-Wunder)]

Wir definieren folgende \textit{NFT-Pass-Hologramme} mit den dazugehörenden Eigenschaften:

\begin{itemize}
    \item WW1
    \begin{itemize}
    	\item Mögliche Ausprägung: \textbf{Pyramids of Giza}
    	\item Anteil Pässe: 0,390625\% $\left( \frac{1}{256} \right)$
    \end{itemize}
    \item WW2
    \begin{itemize}
    	\item Mögliche Ausprägung: \textbf{Great Wall of China}
    	\item Anteil Pässe: 0,78125\% $\left( \frac{1}{128} \right)$
    \end{itemize}
    \item WW3
    \begin{itemize}
    	\item Mögliche Ausprägung: \textbf{Petra} 
    	\item Anteil Pässe: 1,5625\% $\left( \frac{1}{64} \right)$
    \end{itemize}
    \item WW4
    \begin{itemize}
    	\item Mögliche Ausprägung: \textbf{Colosseum} 
    	\item Anteil Pässe: 3,125\% $\left( \frac{1}{32} \right)$
    \end{itemize}
    \item WW5
    \begin{itemize}
    	\item Mögliche Ausprägung: \textbf{Chichén Itzá} 
    	\item Anteil Pässe: 6,25\% $\left( \frac{1}{16} \right)$
    \end{itemize}
    \item WW6
    \begin{itemize}
    	\item Mögliche Ausprägung: \textbf{Machu Picchu} 
    	\item Anteil Pässe: 12,5\% $\left( \frac{1}{8} \right)$
    \end{itemize}
    \item WW7
    \begin{itemize}
    	\item Mögliche Ausprägung: \textbf{Taj Mahal} 
    	\item Anteil Pässe: 25\% $\left( \frac{1}{4} \right)$
    \end{itemize}
    \item WW8
    \begin{itemize}
    	\item Mögliche Ausprägung: \textbf{Christ the Redeemer} 
    	\item Anteil Pässe: 50\% + x $\left( \frac{1}{2} + \frac{1}{256} \right)$
    \end{itemize}
\end{itemize}

\end{NFT-Prop}

\vspace{0.3cm}

Das Besondere an dieser Property spiegelt sich in der Tatsache wider, gewisse rar beschaffene Ausprägungen seien nur "zeitweise" ausgeschöpft, da sich ihre (rare) Anzahl lediglich \textbf{relativ} an der Gesamtzahl der aktuell \textit{ausgestellten NFT-Pässe} bemisst und nicht wie die Main-Property einer absoluten Obergrenze obliegt, deren Erreichung unumkehrbar ist. Soll heißen: Ist die prozentuale Obergrenze an Pässen mit einer beChrist the Redeemerstimmten Ausprägung der gegenwärtigen Property zu einem be\-stimmten Zeitpunkt erreicht, kann zwar für einen gewissen Zeitraum kein Pass mit dieser Ausprägung mehr ausgestellt werden. Sobald jedoch die Gesamtanzahl der \textit{ausgestellten NFT-Pässe} wieder groß genug ist - sodass die Anzahl der vorhandenen \textit{NFT-Pässe} mit der betroffenen Ausprägung wieder die prozentuale Obergrenze unterschreitet - werden Pässe der besagten Ausprägung "wieder verfügbar".

\vspace{0.3cm}

\begin{Algo}[Verlosungs-Mechanismus für Hologramm-Property]

\begin{itemize}
    \item Zunächst bestimme man die Gesamtanzahl aller bisher geminteter Pässe $n$.
    \item Gleiches tue man nun für die Counts der geminteten Pässe pro Ausprägung der Hologramm-Property WW1 bis WW8 als entsprechende Größen $n_1, n_2,...,n_8$.
    \item Und damit anschließend die aktuelle prozentuale Verteilung der Ausprägung auf die aktuell geminteten Pässe als $\sigma_i:= \frac{n_i}{n}$ für $i \in \lbrace 1,...,8 \rbrace$ berechnen.
    \item Seien $\Theta_i$ für $i \in \lbrace 1,...,8 \rbrace$ die oben definierten \textbf{relativen} Obergrenzen der \newline Ausprägungen der Hologramm-Property WW1 bis WW8.
    \item Alle Ausprägungen mit $\sigma_i \geq \Theta_i$ können zum aktuellen Zeitpunkt nicht vergeben werden und damit auch nicht beim Minting eines neuen Pass berücksichtigt werden.
    \item Für die Ausprägungen mit $\sigma_i < \Theta_i$ berechnen wir den Normierungsfaktor
\end{itemize} 

\begin{equation*}
\omega := \sum_{\sigma_i < \Theta_i} \Theta_i \textrm{ } \leq 1
\end{equation*} 

\begin{itemize}
    \item Damit errechnen wir die aktuell vorliegenden Wahrscheinlichkeiten $\rho_i$ für unsere Hologramm-Ausprägungen als
\end{itemize} 

\[
\rho_i:=\left\{%
\begin{array}{ll}
    0, & \hbox{falls $\sigma_i \geq \Theta_i$} \\[0,3cm]
    \hbox{\LARGE $\frac{\Theta_i}{\omega}$,} & \hbox{falls $\sigma_i < \Theta_i$}. \\
\end{array}%
\right.
\] 

Man vergewissere sich an dieser Stelle gedanklich, auch für die neuen \newline Wahrscheinlichkeiten gelte \[\sum_{i = 1}^7 \rho_i \textrm{ } = 1.\]

\begin{itemize}
    \item Am Ende bestimme man mittels Randomisierung anhand der Wahrscheinlichkeiten $\rho_i$ für $i \in \lbrace 1,...7 \rbrace$ die zu vergebende Hologramm-Ausprägung. 
\end{itemize}

\end{Algo}

\vspace{0.3cm}

Was hier so kompliziert klingt, lässt sich aber super simpel veranschaulichen:

Die \textit{Verlosung} der Wunder erfolgt in einem periodischen 256er-Turnus ($256 = 2^{n}$ mit $n=8$ für die acht bereitgestellten Hologramme). Nach jedem 256. geminteten Pass schmeißt man 256 Lose in eine Lostrommel: Ein Los für die \textit{Pyramiden}, zwei für die \textit{Chinesische Mauer}, vier für \textit{Petra} etc. Die \textit{Jesus-Statue} kommt letztendlich mit 129 Losen in die Trommel.

Nun ziehen wir blind ein Los und vergeben das gezogenen Hologramm an den nächsten zu mintenden NFT-Pass. Wir tun dies solange, bis die Trommel leer ist. Anschließend fangen wir wieder von Vorne an und befüllen die Trommel erneut mit denselben 256 Losen.

\textbf{Achtung:} Wir befüllen die Trommel ausschließlich nachdem sie komplett leer geworden ist und nicht etwa zwischendurch mal.

\vspace{0.3cm}



% !TEX root = paper.tex

\begin{sloppypar}
Diese NFT-Property soll ebenso wie die beiden vorigen einem abstufenden Raritätsprinzip zu Grunde liegen - und zwar ausschließlich dem Zufall folgend.
\end{sloppypar}

Im Gegensatz zu den beiden vorigen Properties obliegt die \textit{Pattern-Property} keiner absoluten Obergrenze - insbesondere auch dann nicht, falls einige Pattern zu einem Zeitpunkt verhältnismäßig unter- oder überrepräsentiert sind.

\vspace{0.3cm}

\begin{NFT-Prop}[Background (Pattern)]
\label{pattern}

Wir definieren folgende \textit{NFT-Pass-Background-Muster} mit den dazugehörenden Eigenschaften:

\begin{itemize}
    \item P1
    \begin{itemize}
    	\item Mögliche Ausprägung: \textbf{Safari Fun} 
    	\item Wahrscheinlichkeit: 0,1953125\% $\left( \frac{1}{512} \right)$
    \end{itemize}
    \item P2
    \begin{itemize}
    	\item Mögliche Ausprägung: \textbf{Triangular Bars} 
    	\item Wahrscheinlichkeit: 0,390625\% $\left( \frac{1}{256} \right)$
    \end{itemize}
    \item P3
    \begin{itemize}
    	\item Mögliche Ausprägung: \textbf{Pointillism} 
    	\item Wahrscheinlichkeit: 0,78125\% $\left( \frac{1}{128} \right)$
    \end{itemize}
    \item P4
    \begin{itemize}
    	\item Mögliche Ausprägung: \textbf{Wavy waves} 
    	\item Wahrscheinlichkeit: 1,5625\% $\left( \frac{1}{64} \right)$
    \end{itemize}
    \item P5
    \begin{itemize}
    	\item Mögliche Ausprägung: \textbf{Stony desert} 
    	\item Wahrscheinlichkeit: 3,125\% $\left( \frac{1}{32} \right)$
    \end{itemize}
    \item P6
    \begin{itemize}
    	\item Mögliche Ausprägung: \textbf{WunderPass} 
    	\item Wahrscheinlichkeit: 6,25\% $\left( \frac{1}{16} \right)$
    \end{itemize}
    \item P7
    \begin{itemize}
    	\item Mögliche Ausprägung: \textbf{Zigzag} 
    		\item Wahrscheinlichkeit: 12,5\% $\left( \frac{1}{8} \right)$
    \end{itemize}
    \item P8
    \begin{itemize}
    	\item Mögliche Ausprägung: \textbf{Linear}  
    	\item Wahrscheinlichkeit: 25\% $\left( \frac{1}{4} \right)$
    \end{itemize}
    \item P9
    \begin{itemize}
    	\item Mögliche Ausprägung: \textbf{Curves}
    	\item Wahrscheinlichkeit: 50,1953125\% $\left( \frac{257}{512} \right)$
    \end{itemize}
\end{itemize}

\end{NFT-Prop}

\vspace{0.3cm}

    % binde die Datei '[NFT-Pass][Konzept][Pattern].tex' ein

% !TEX root = paper.tex

\subsubsection{Edition}

\vspace{0.3cm}

Die Edition unseres WunderPasses soll als Property auf die anfangs geforderte Möglichkeit einer gewissen Individualisierung des WunderPasses durch seinen Besitzer einzahlen. Zu individuell darf eine solche NFT-Property aber auch nicht sein, da der NFT zwingend seinen Eigentümer wechseln können soll, da das ganze Unterfangen mit dem NFT-Pass andernfalls ad absurdum führte.

Um die Edition-Property noch etwas interessanter zu gestalten, sollen Exemplare jeder Edition nicht endlos verfügbar sein, sondern stattdessen irgendwann einmal \textit{aufgebraucht}. In solch einem Fall soll sich der User aber nicht einfach irgendeine andere Edition auswählen müssen, sondern erhält die \textit{"Oberedition"} (Parent) seiner ursprünglich gewünschten Edition. Und falls auch diese \textit{aufgebraucht} sein sollte, die \textit{"Oberedition"} der \textit{"Oberedition"} usw. 

\vspace{0.2cm}

\begin{NFT-Prop}[Edition]

Als Ausprägung der WunderPass-NFT-\textbf{Edition} haben wir uns für \textbf{Städte} der Welt entschieden. Die \textbf{Parent-Edition} einer Stadt ist das dazugehörige \textbf{Land}, dessen 
Parent-Edition wiederum \textbf{Kontinent} und die \textbf{oberste Editions-Ebene} dann die \textbf{Welt-Edition}. Letztere unterliegt folglich auch keiner stückweisen Obergrenze mehr.

\vspace{0.2cm}

\underline{\textbf{\textit{Beispiel einer Edition-Kette:}}}

\vspace{0.2cm}

\begin{equation*}
\textrm{Berlin } \rightarrow \textrm{ Germany } \rightarrow \textrm{ Europe } \rightarrow \textrm{ World }
\end{equation*} 

\vspace{0.2cm}

Es gilt folgendes grobe Regel-Set, was jedoch explizit auch nach Launch modifizierbar bleiben soll:

\begin{itemize}
    \item Die möglichen Editionen werden von uns bestimmt. Diese müssen nicht zwingend beim Launch des NFT vollständig benannt werden, sondern können stattdessen auch nachträglich eingepflegt werden.
    \item Jede berücksichtigte \textit{Städte-Edition} ist genau \textbf{100} Mal verfügbar. Sind alle 100 Exemplare einer \textit{Städte-Edition} bereits gemintet (verbraucht), erhält die nächste Mint-Anfrage nach einem WunderPass derselben Edition automatisch die zu dieser Städte-Edition gehörende \textit{Landes-Edition}.
    \item Die \textit{Landes-Editionen} sind in einer maximalen Stückzahl von je \textbf{10.000} pro berücksichtigtem Land verfügbar. Sind auch diese aufgebraucht, wird die durch den User ausgewählte Stadt auf die ihrem Land übergeordnete \textit{Kontinent-Edition} gemappt.
    \item Die \textit{Kontinent-Editionen} sind in einer maximalen Stückzahl von je \textbf{1.000.000} für jeden Kontinent (außer der Antarktis) vorgesehen. Sollte auch diese Menge irgendwann erschöpfen, greifen wir zu der übergeordneten \textit{Welt-Edition}
\end{itemize} 

\end{NFT-Prop}

\vspace{0.24cm}

\underline{\textbf{Quantitative Daten zu den Editionen:}}

\begin{itemize}
    \item Nach aktuellem Stand sind mindestens 693 \textit{Städte-Edition} vorgesehen.
    \item Die genannten \textit{Städte-Edition} verteilen sich dabei aktuell auf 179 \textit{Landes-Edition}.
    \item Die unterschiedlichen \textit{Kontinent-Edition} belaufen sich auf 6 (Nord- und Südamerika, Europa, Afrika, Asien und Australien).
    \item Die übergeordnete \textit{Welt-Edition} ist in ihrer Stückzahl unbegrenzt. 
    \item Die Auswahl der angebotenen \textit{Städte-Editionen} folgt (mit Augenmaß) in etwa folgender Logik:
    \begin{itemize}
    	\item Die Hauptstadt eines jeden mit einer \textit{Landes-Edition} versehenen Landes ist gleichzeitig auch eine verfügbare \textit{Städte-Edition}.
    	\item Mit Ausnahme der Hauptstädte erfordert die Größe einer Stadt (nach Einwohnern) ein Mindestmaß $m_1$, um als \textit{Städte-Edition} aufgenommen zu werden.
    	\item Sofern es das vorige Kriterium hergibt, sollen nach Möglichkeit für jedes Land mit einer eigenen \textit{Landes-Edition} mindestens seine 5 größten Städte mit einer eigenen \textit{Städte-Edition} versehen werden.
    	\item Überschreiten die $n$ größten Städte eines in die \textit{Landes-Editionen} aufgenommenen Landes eine bestimmte Mindestgröße (nach Einwohnern) $m_2$, werden alle $n$ Städte in die verfügbaren \textit{Städte-Editionen} aufgenommen. Dieses Kriterium wird aufgrund des vorigen ausschließlich für $n > 5$ relevant.
    	\item Städte der G7-Länder werden (ungeachtet etwaiger Mindestgröße) vermehrt in die \textit{Städte-Editionen} aufgenommen (bis zu 25 \textit{Städte-Editionen} pro Land).
    \end{itemize} 
    \item Einzelne Städte können bei Bedarf auch bei Missachtung aller vorigen Kriterien aufgenommen werden.
\end{itemize}

\vspace{0.5cm}




    % binde die Datei '[NFT-Pass][Konzept][Edition].tex' ein

% !TEX root = paper.tex

% https://de.wikibooks.org/wiki/LaTeX-Kompendium:_Schnellkurs:_Grafiken

Dem teils trockenen Text der vorigen Kapitel sollen hier einfach wortlos einige denkbare Ausprägungen unseres WunderPasses in Bild folgen:

\begin{figure}[h]
  \centering
  \subfloat[][]{\includegraphics[width=0.4\linewidth]{{"[06][NFT-Pass]/images/diamand 1"}}}%
  \qquad
  \subfloat[][]{\includegraphics[width=0.4\linewidth]{{"[06][NFT-Pass]/images/diamand 2"}}}%
  \caption{zwei \textit{diamond} Pässe mit je unterschiedlichen Hologrammen und Pattern}%
\end{figure}

\begin{figure}[h]
  \centering
  \subfloat[][]{\includegraphics[width=0.4\linewidth]{{"[06][NFT-Pass]/images/black"}}}%
  \qquad
  \subfloat[][]{\includegraphics[width=0.4\linewidth]{{"[06][NFT-Pass]/images/pearl"}}}%
  \caption{Pässe des Status \textit{black} und \textit{pearl}}%
\end{figure}

\begin{figure}[h]
  \centering
  \subfloat[][]{\includegraphics[width=0.4\linewidth]{{"[06][NFT-Pass]/images/gold"}}}%
  \qquad
  \subfloat[][]{\includegraphics[width=0.4\linewidth]{{"[06][NFT-Pass]/images/bronze"}}}%
  \caption{rechts ein bronzener Pass mit den sehr sehr seltenen \textit{Pyramiden von Gizeh} als Hologramm}%
\end{figure}
    % binde die Datei '[NFT-Pass][Konzept][Design].tex' ein
%% !TEX root = paper.tex

% https://de.wikibooks.org/wiki/LaTeX-Kompendium:_Schnellkurs:_Grafiken

Dem teils trockenen Text der vorigen Kapitel sollen hier einfach wortlos einige denkbare Ausprägungen unseres WunderPasses in Bild folgen:

\begin{figure}[h]
  \centering
  \subfloat[][]{\includegraphics[width=0.4\linewidth]{{"[06][NFT-Pass]/images/diamand 1"}}}%
  \qquad
  \subfloat[][]{\includegraphics[width=0.4\linewidth]{{"[06][NFT-Pass]/images/diamand 2"}}}%
  \caption{zwei \textit{diamond} Pässe mit je unterschiedlichen Hologrammen und Pattern}%
\end{figure}

\begin{figure}[h]
  \centering
  \subfloat[][]{\includegraphics[width=0.4\linewidth]{{"[06][NFT-Pass]/images/black"}}}%
  \qquad
  \subfloat[][]{\includegraphics[width=0.4\linewidth]{{"[06][NFT-Pass]/images/pearl"}}}%
  \caption{Pässe des Status \textit{black} und \textit{pearl}}%
\end{figure}

\begin{figure}[h]
  \centering
  \subfloat[][]{\includegraphics[width=0.4\linewidth]{{"[06][NFT-Pass]/images/gold"}}}%
  \qquad
  \subfloat[][]{\includegraphics[width=0.4\linewidth]{{"[06][NFT-Pass]/images/bronze"}}}%
  \caption{rechts ein bronzener Pass mit den sehr sehr seltenen \textit{Pyramiden von Gizeh} als Hologramm}%
\end{figure}


% !TEX root = paper.tex

\subsubsection{Beispiel}

\vspace{0.2cm}

\todo{TODO: Beispielrechnung für geminteten NFT-Pass mit der Nummer x}

Angenommen x sei 1.005.965.

\begin{itemize}
  \item vorrechnet, welche ersten 1.005.964 NFT-Pässe schon weggemintet sein könnten und Wahrscheinlichkeiten für den neu zu mintenden NFT-Pass erklären.
  \item neuen NFT-Pass unter Einbindung der Wahrscheinlichkeiten und vorgegaukelten Zufalls errechnet.
  \item geminteten neuen NFT-Pass als exakte Grafik in unserem Design hier abbilden.
\end{itemize}

\vspace{0.3cm}    % binde die Datei '[NFT-Pass][Konzept][Beispiel].tex' ein
%% !TEX root = paper.tex

\subsubsection{Beispielhafte Analyse der Collection}

\vspace{0.2cm}

Um ein besseres Gefühl über die formulierte Logik unseres NFTs zu bekommen, wollen wir ein Beispiel mit konkreten Zahlen rechnen und begeben uns dazu eine gute Weile in die Zukunft - zu einem Zeitpunkt, zu dem bereits genau 316.157 NFT-Pässe gemintet wurden. Ich als potenzieller Interessent an einem Pass-NFT möchte verstehen, welchen Pass ich als nächsten in etwa zu erwarten hätte.

Wir analysieren die 316.157 bereits geminteten Pässe.

\vspace{0.2cm} 

\underline{\textbf{Status:}}

\begin{itemize}
  \item Es wurden 200 Pässe des Status \textit{diamond} gemintet.
  \item Es wurden 1.600 Pässe des Status \textit{black} gemintet.
  \item Es wurden 12.800 Pässe des Status \textit{pearl} gemintet.
  \item Es wurden 102.400 Pässe des Status \textit{platin} gemintet.
  \item Es wurden 199.157 Pässe des Status \textit{ruby} gemintet.
  \item Von den insgesamt 819.200 vorgesehenen \textit{ruby} Pässen sind demnach noch 620.043 noch verfügbar.
\end{itemize}

\vspace{0.2cm}

\textit{\textbf{Unser Pass wird also definitiv den Status 'Ruby' haben!}}

\vspace{0.3cm}


\underline{\textbf{Hologramm:}}

\vspace{0.2cm}

Hinsichtlich der Hologramme können wir nur über die ersten 315.904 der 316.157 bisher geminteten Pässe eine definitive Aussage treffen. Die übrigen 253 folgen einer gewissen Wahrscheinlichkeitsverteilung. Zunächst zu den ersten 315.904:

\begin{itemize}
  \item Es wurden 159.186 Pässe mit dem Hologramm der \textit{Jesus-Statue} gemintet.
  \item Es wurden 78.976 Pässe mit dem Hologramm des \textit{Maj Mahal} gemintet.
  \item Es wurden 39.488 Pässe mit dem Hologramm des \textit{Machu Picchu} gemintet.
  \item Es wurden 19.744 Pässe mit dem Hologramm der \textit{Chichén Itzá} gemintet.
  \item Es wurden 9.872 Pässe mit dem Hologramm des \textit{Kolosseum} gemintet.
  \item Es wurden 4.936 Pässe mit dem Hologramm der \textit{Petra} gemintet.
  \item Es wurden 2.468 Pässe mit dem Hologramm der \textit{Chinesischen Mauer} gemintet.
  \item Es wurden 1.234 Pässe mit dem Hologramm den \textit{Pyramiden von Gizeh} gemintet. noch verfügbar.
\end{itemize}

\vspace{0.3cm}

Die Evaluierung der übrigen 253 ist insofern recht dankbar, als dass die 253 schon sehr nah an der zyklischen 256 liegt ($= 2^{n}$, wobei $n=8$ für die acht verfügbaren Hologramme steht). Damit beschränkt sich die Analyse eigentlich lediglich auf die nächsten drei Pässe, von denen der erste unserer ist. 











\vspace{0.5cm}

\begin{itemize}
  \item vorrechnet, welche ersten 316.157 NFT-Pässe schon weggemintet sein könnten und Wahrscheinlichkeiten für den neu zu mintenden NFT-Pass erklären.
  \item neuen NFT-Pass unter Einbindung der Wahrscheinlichkeiten und vorgegaukelten Zufalls errechnet.
  \item geminteten neuen NFT-Pass als exakte Grafik in unserem Design hier abbilden.
\end{itemize}

\vspace{0.3cm} 

% !TEX root = paper.tex

Abschließend möchten wir an dieser Stelle an die anfangs formulierte zentrale Forderung nach einem \textbf{intrinsischen Wert unseres Pass-NFTs} anknüpfen und im folgenden einen Abriss zu denkbaren Einsatzmöglichkeiten des WunderPass-NFTs und seinen potenziellen Vorteilen für 
seinen Besitzer darlegen.  

\vspace{0.1cm}

Konsequenterweise folgen wir bei der Erarbeitung solcher User-Nutzen \& -Vorteile der Devise, ein seltenerer NFT-Pass solle als \textit{gut} gelten und damit auch mit einem größeren \textbf{intrinsischen Wert} einhergehen. Die Seltenheit ist bei unserem NFT einerseits durch Zufall (\textit{Hologramm} \& \textit{Pattern}) aber auch durch First-Mover-Sein (\textit{Status}) gesteuert. 

Wenn es um das Beimessung von Vorzügen und Einsatzmöglichkeiten eines \textit{guten} Pass-NFTs geht, erscheint es irgendwie sinnvoll, eher First-Mover zu begünstigen als etwaige Glückspilze, weshalb wir in der folgenden \textit{"Vorteile-Tabelle"} das Augenmerk tendenziell auf den \textit{Status} des Pass-NFTs legen möchten. Bei einigen der gleich \mbox
zu listenden Vorteilen erscheint jedoch auch der ausschließliche oder zusätzliche Glücksfaktor als ebenfalls sehr charmant, weshalb an solchen Stellen das \textit{Hologramm} des Pass-NFTs zur Beimessung der Vorzüge hinzugezogen wird. Insbesondere möchten wir dem \textit{Pyramiden-Hologramm} explizit und per default dasselbe Statussymbol im Sinne der zu definierenden Vorzüge bei\-messen wie dem \textit{Diamond-Status}.

\textbf{Das \textit{Pattern} findet nach aktuellem Stand jedoch nirgends Berücksichtigung hinsichtlich des intrinsischen Werts eines Pass-NFTs}. Dieses bleibt also zunächst pure Spielerei im Sinne des Pass-NFTs als reines Sammlerstück.

\vspace{0.2cm}

Bei der folgenden Auflistung der Vorzüge und Einsatzmöglichkeiten kategorisieren wir nach 
\textcolor{orange}{\textbf{Status-Vorteilen}}, \textcolor{brown}{\textbf{finanziellen Anreizen}} sowie \textcolor{purple}{\textbf{Mitgestaltungsspielraum}} innerhalb der WunderPass-Company, die an den zugehörigen Farben zu erkennen sind.



\vspace{1.0cm}


\begin{tabular}[c]{|c|c|c|c|c|c|c|c|}
\hline
 & \textbf{\textit{diamond}} & \textbf{\textit{black}} & \textbf{\textit{pearl}} & & \textbf{\textit{pyramid}} & \textbf{\textit{wall}} & \textbf{\textit{petra}} \\
\hline
\textcolor{orange}{Weihnachtsfeier} & $\color{green} \surd\surd\surd$ &  &  &  & $\color{green} \surd\surd\surd$ &  & \\
\hline
\textcolor{orange}{\parbox{2cm}{Workshops \\ Hackathons}} & $\color{green} \surd\surd\surd$ & $\color{green} \surd$ &  &  & $\color{green} \surd\surd\surd$ &  & \\
\hline
\textcolor{orange}{private Discord} & $\color{green} \surd\surd\surd$ & $\color{green} \surd\surd\surd$ &  &  & $\color{green} \surd\surd\surd$ &  & \\
\hline
\textcolor{orange}{Metallkarte} & $\color{green} \surd\surd\surd$ & $\color{green} \surd\surd$ &   $\color{green} \surd$ &  & $\color{green} \surd\surd\surd$ &  & \\
\hline
\textcolor{orange}{Goodies} &  &  &  &  & $\color{green} \surd\surd\surd$ & $\color{green} \surd\surd$ & $\color{green} \surd$ \\
\hline
\textcolor{orange}{Priorisierung} & $\color{green} \surd\surd\surd$ & $\color{green} \surd\surd$ &   $\color{green} \surd$ &  & $\color{green} \surd\surd\surd$ &  & \\
\hline
 &  &  &  &  &  &  & \\
\hline
\textcolor{brown}{\parbox{2.8cm}{Airdrops \\ (Utility Token)}} &  &  &  &  & $\color{green} \surd\surd\surd$ & $\color{green} \surd\surd$ & $\color{green} \surd$ \\
\hline
\textcolor{brown}{Rewards} & $\color{green} \surd\surd\surd$ & $\color{green} \surd\surd$ &   $\color{green} \surd$ &  & $\color{green} \surd\surd\surd$ &  & \\
\hline
\textcolor{brown}{Staking-Zinsen} & $\color{green} +++$ & $\color{green} ++$ &   $\color{green} +$ &  & $\color{green} +++$ &  & \\
\hline
\textcolor{brown}{\parbox{2.8cm}{Beteiligung an \\ NFT-Verkäufen}} & $\color{green} \surd\surd\surd$ &  &  &  & $\color{green} \surd\surd\surd$ &  & \\
\hline
\textcolor{brown}{Dividende} & $\color{green} \surd\surd\surd$ &  &  &  & $\color{green} \surd\surd\surd$ &  & \\
\hline
 &  &  &  &  &  &  & \\
\hline
\textcolor{purple}{early Access} & $\color{green} \surd\surd\surd$ & $\color{green} \surd\surd\surd$ &  &  & $\color{green} \surd\surd\surd$ &  & \\
\hline
\textcolor{purple}{\parbox{2.8cm}{Voting for \\ Product/Features}} & $\color{green} \surd$ & $\color{green} \surd$ & $\color{green} \surd$ &  & $\color{green} \surd$ & $\color{green} \surd$ & $\color{green} \surd$ \\
\hline
\textcolor{purple}{\parbox{2.8cm}{Govern. Tokens \\ (DAO-Membership)}} & $\color{green} \surd\surd\surd$ & $\color{green} \surd\surd$ &   $\color{green} \surd$ &  & $\color{green} \surd\surd\surd$ &  & \\
\hline
 &  &  &  &  &  &  & \\
\hline
\textcolor{red}{\textbf{Backlog}} &  &  &  &  &  &  & \\
\hline
exklusiveres Naming &  &  &  &  &  &  & \\
\hline
Zugang zu Interna &  &  &  &  &  &  & \\
\hline
\parbox{3.0cm}{Vergünstigung für \\ Wunder-Dienste} &  &  &  &  &  &  & \\
\hline
\end{tabular}\vspace*{0.3cm}\\

\vspace{0.5cm}    % binde die Datei '[NFT-Pass][Konzept][Intrinsischer Wert].tex' ein


\vspace{0.5cm}




    % binde die Datei '[NFT-Pass][Konzept].tex' ein
% !TEX root = C:/Users/Slava/White-Paper/[06][NFT-Pass]/[NFT-Pass].tex

\subsection{Technische Umsetzung}

\vspace{0.3cm}

\todo{TODO: technische Implementierung}

\vspace{0.3cm}

\begin{itemize}
  \item Abwandlung des ERC721-Standard, um unsere Metadaten-Logik zu bändigen.
  \item Die Metadaten werden wohl auch einem ähnlichen Konstrukt wie IPFS (off-chain) gespeichert werden und lediglich deren Hash als Datenfeld im Smart-Contract (on-chain), damit die Metadaten nicht nachträglich verändern werden können (dieses Vorgehen wird der absolute Standard sein).
  \item Unsere Metadaten sind jedoch so komplex, das deren Erzeugung (beim Minten) wohl einen zweiten Smart-Contract erfordern wird. Wir haben also quasi einen "Metadaten-Hybriden":
  \begin{itemize}
  	\item Erzeugung on-chain
  	\item Storing off-chain
  \end{itemize}
  \item Der Metadaten-Smart-Contract wird die oben skizzierte Logik implementieren
  \begin{itemize}
  	\item Wie viele Pässe gibts es bereits und welche (hinsichtlich Properties)?
  	\item Wie sind die aktuellen Verteilungen der Properties und deren Contstraints
  	\item Einbindung von Randomisierungs-Orakeln
  	\item Sicherstellung, dass die erzeugten Metadaten auch tatsächlich vom Caller (ERC721-Contract) verwendet wurden und keine nachträgliche Manipulation stattgefunden hat.
  \end{itemize}
  \item Es muss geklärt werden, ob hinsichtlich des Gedanken an den besagten "zweiten Smart-Contract" Standards/Best-Practices existieren, damit wir hier nicht das Rad neu erfinden.
  \item Es bleibt noch nicht ganz klar, wie die Metadaten nach ihrer Erzeugung nach IPFS gelangen, da dies laut meinem Verständnis ein Smart-Contract nicht selbst gewährleisten kann. Moritz Idee war grob die Folgende 
  \begin{itemize}
    \item Der Minting-Contract erzeugt den NFT, lässt seine Metadaten-Referenz jedoch zunächst ungesetzt (der NFT ist damit in gewisser Weise noch "unfertig"; kann in dem Zustand auch noch Constraints unterstellt sein).
    \item Der Minting-Contract callt den Metadaten-Contract mit dem Anliegen, Metadaten zu dem "unfertigen" NFT mit der zugehörigen ID zu erzeugen.
  	\item Der Metadaten-Contract erzeugt die Metadaten, hasht diese und gibt den Hash zurück an den Minting-Contract. Gleichzeitig publisht er ein Create-Event mit der Token-ID und den zugehörigen erzeugten Metadaten.
  	\item Der Minting-Contract speichert den erhaltenen Metadaten-Hash und wartet auf "approvement".
  	\item Das forcierte Event wird von einem dafür bestimmten (off-chain) Web3-Service vernommen und weiterverarbeitet: Die Metadaten werden geparst und nach IPFS gepusht. Als Ergebnis bekommen wir eine entsprechende IPFS-URI.
  	\item Unser Web3-Service stößt anschließend eine "Set-URI"-Transaktion mit den entsprechenden Input-Daten (Token-ID; IPFS-URI) beim Minting-Contract an, um den gesamten Minting-Prozess für den neuen Token abzuschließen.
  	\item Der Minting-Contract verifiziert die Metadaten mittels des gespeicherten Meta-Daten-Hashs (\todo{Hier ist nicht nicht ganz klar, wie. Ich weiß nicht, ob der Contract einfach die Daten von IPFS laden kann, um den Hash abzugleichen oder ob er vorher die URI implizit vorgeben muss, die irgendwie im Hash berücksichtigt werden muss, oder wie auch immer hier die Best-Practise aussieht}) und updatet die NFT-URI auf den Wert der übergebenen IPFS-URI. 
  	\item Hiermit ist der Minting-Prozess abgeschlossen, der NFT "fertig" gemintet und kann von etwaigen "Temporary-Locked-Constraint" entbunden werden und vom neuen Besitzer frei verfügt werden.
  \end{itemize}
  \item \textbf{Ein etwaiger Crypto-Freelancer muss auf die skizzierten Herausforderungen gechallenget werden.}
\end{itemize}    % binde die Datei '[NFT-Pass][Tech].tex' ein

\vspace{0.5cm}