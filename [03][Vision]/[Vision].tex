% !TEX root = paper.tex

\section{Vision}
\label{sec:vision}

\todo{TODO: Einleitung ausformulieren}

\vspace{0.3cm}

\textit{Wenn Personen auch im virtuellen Raum mehr sein wollen als Warenempfänger,
Zahlende oder "Nicknames", nämlich individuelle und facettenreiche Kommunikationspartner, dann muss sich die Komplexität des digitalen Identitätskonzeptes derjenigen
des realen annähern.}

\textit{Im Bereich der realen Identität ist aber weder ein fest abgegrenzter Raum von zu berücksichtigenden Bereichen oder Themen, welche einer Identität zuzuweisen wären, benennbar, noch sind Standards definiert, auf denen der Informationsaustausch zwischen
Individuen stattfindet. Dies macht ein umfassendes Konzept erforderlich mit den Möglichkeiten, die notwendige Flexibilität einerseits und eine Vereinheitlichung oder einen Abgleich des Informationsflusses andererseits zu gewährleisten. Dieses Konzept muss
dem durch die Individualität der Identitäten gegebenen Mangel an Kompatibilität entgegentreten.} 

\textit{Die Umsetzung des realen Identitätskonzeptes in ein digitales Identitätskonzept muss [...] Umfassende \textbf{Vernetzung mit direkten Verbindungen} und
die Möglichkeit zur Nutzung von Daten in maschinenlesbarer Form erscheinen für einen möglichen Einsatz als vorteilhaft. Um dem Anwender keine Barriere in den Weg
zu legen, ist es notwendig, möglichst viele Aspekte – gerade solche von struktureller und
organisatorischer Natur – \textbf{so weit wie möglich transparent} zu halten. \textbf{Wenn der Anwender auf der einen Seite durch keinen oder nur einen minimalen Mehraufwand, aber auf der anderen Seite verschiedene Vorteile, Erleichterungen oder neue Dienste erlangt, die auf dem Konzept digitaler Identitäten aufsetzen, wäre dies eine Bewältigung eines ansonsten sicherlich auftretenden Akzeptanzproblems.}} [\href{https://vsis-www.informatik.uni-hamburg.de/getDoc.php/thesis/47/DA_Gordian_Kaulbarsch.pdf}{Auszug aus der Arbeit "Identitäten und ihre Schnittstellen auf Basis von Ontologien in einer dezentralen Umgebung"}]

\vspace{0.3cm}

\todo{TODO: Aufgreifend aus vorigen Kapitel (verlinken)}

\vspace{0.3cm}

\begin{Solution}[fehlende Eindeutigkeit]\label{sol1}

Um Eindeutigkeit der Identität bzw. Identifizierung herzustellen, bedarf es eines übereinstimmenden Konsens, welches ausreichend der relevanten digitalen Akteure mittragen.

Der Status quo könnte in diesem Sinne gar nicht schlimmer sein. Meine digitale wird heutzutage gleichermaßen durch meine Email-Adresse, meine Telefonnummer, einen etwaigen frei wählbaren Nickname oder aber meinen Google- oder Facebook-Account repräsentiert. Wo es grad wem besser passt. Weniger eindeutig geht es also quasi kaum.

Die "Großen" haben das Problem bereits erkannt und forcieren die Eindeutigkeit gen Google-, Apple- oder Facebook-Account als eindeutige Identität. Nur ist es so, dass die sich unwahrscheinlich an einen Tisch setzen und ein gemeinsames Standard beschließen. Und wenn es am Ende nur die 4 großen GAFA-Identitäten gibt, sind es im Sinne der Eindeutigkeit 3 zu viel.

\vspace{0.2cm}

Und da eine kollaborative Festlegung auf eindeutige Identität völlig utopisch erscheint, muss eine von den Naturgesetzen vorgegebene gewählt werden. Eine, die jeder unmissverständlich und gleich interpretieren kann, ohne eine Kollaboration mit irgendwem eingehen zu müssen.

\vspace{0.2cm}

\textbf{Das kryptografische Private-Public-Key-Pair scheint der perfekte Kandidat für diese Anforderung zu sein!}

\end{Solution}

\vspace{0.3cm}


\begin{Solution}[Redundanz und fehlerbehaftete Daten] \label{sol2}

Wir möchten diesem Problem entgegnen, indem wir persönliche Daten nur an einer einzigen globalen Stelle speichern - einem Identity-Management-Service in der \newline Blockchain.

Der User muss seine Daten dann nur an einer einzigen Stelle aktuell und sauber halten und die Datennutzer (z. B. Online-Services) können stets von Aktualität und Korrektheit ausgehen.

\end{Solution}

\vspace{0.3cm}


\begin{Solution}[mangelhafte UX]

Wie auch bei obiger Lösung \ref{sol1} sehen wir auch für dieses Problem \textbf{das kryptografische Private-Public-Key-Pair} als sehr aussichtsreiches Allheilmittel an.

Denn es ist geeignet als 

\begin{itemize}
  \item	technischer Identifier mittels Public-Key,
  \item universelles und einheitliches Erkennungsmerkmal (\textbf{Identifizierung}) mittels Public-Key,
  \item sicherster "Identity-Proof" (\textbf{Autorisierung}) mittels kryptografischer Signatur (die zudem auch noch "zero-knowledge" ist, und nicht mittels Pishing gehackt werden kann).
\end{itemize}

Dies sichert uns schon einmal einen "Zero-Input-Sign-in" ("zero input" aus Usersicht; die Workflows laufen im Hintergrund ohne Zutuns des Users ab).

\vspace{0.1cm}

Gepaart mit obiger Lösung \ref{sol2} kann zudem ebenso der Sign-up abgeschafft (bzw. auf eine einzige universelle und übergeordnete Registrierung für sämtliche Online-Dienste reduziert) werden - nämlich auf die einmalige Erstellung seines WunderPasses.

\end{Solution}

\vspace{0.3cm}


\begin{Solution}[Datenschutz]

\todo{TODO: ausformulieren}
\begin{itemize}
  \item meine Daten liegen an einer einzigen Stelle gespeichert
  \item Daten sind verschlüsselt und unhackbar
  \item Derart ließen sich auch Identitätsdaten auf automatisierte Weise kontrolliert weitergeben. 'Kontrolliert' in diesem Zusammenhang bedeutet die Möglichkeit für den Anwender, selbst zu entscheiden, an wen er welche Daten wann und zu welchen Bedingungen übermittelt.
\end{itemize}

\end{Solution}

\vspace{0.3cm}


\begin{Solution}[Daten werden nicht dort erfasst, wo sie gebraucht werden]

\todo{TODO: ausformulieren}
\begin{itemize}
  \item Beispiel mit der Supermarkt-Kassiererin
  \item Beispiel mit Amazon und der Post
\end{itemize}

\end{Solution}

\vspace{0.3cm}


\begin{Solution}[Datenmissbrauch/Bereicherung]

\todo{TODO: ausformulieren}
Ich werde an der Verwendung meiner Daten monetär beteiligt (Token-Economics)

\end{Solution}

\vspace{0.3cm}


\begin{Solution}[Abhängigkeit von Big Tech]

\todo{TODO: ausformulieren}
Bei Self-Sovereign Identity (SSI) oder selbstbestimmter Identität kontrollieren und besitzen Nutzer ihre digitalen Identitäten und weitere verifizierbare digitale Nachweise (Verifiable Credentials (VC)), ohne hierfür auf eine zentrale Stelle, wie etwa Facebook oder Google, angewiesen zu sein. Sie sind somit komplett unabhängig von Dritt-Instanzen und entscheiden vollkommen eigenständig, wer welche Identitätsdaten zur Verfügung gestellt bekommt, da alle Identitätsdaten ausschließlich bei ihnen gespeichert werden. Dadurch ist ein einfacher, flexibler, sicherer und vertrauenswürdiger Austausch von manipulationssicheren digitalen Nachweisen zwischen Nutzer und Anwendungen möglich.

\end{Solution}

\vspace{0.3cm}


\begin{Solution}[Ungenutzte Möglichkeiten]

\todo{TODO: ausformulieren}
Daten-Querverweise $\rightarrow$ Beispiel anführen 

(zB aus \href{https://norbert-pohlmann.com/glossar-cyber-sicherheit/self-sovereign-identity-ssi/}{Vorlesung zu SSI})

\end{Solution}

\vspace{0.5cm}