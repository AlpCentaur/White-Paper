% !TEX root = paper.tex
\subsection{Einleitung}
\label{sec:eco_einleitung}


\begin{Fazit}[unser Ökosystem generiert Value]

\begin{itemize}
  \item Wir schöpfen Mehrwert, indem wir Datenerfassung ermöglichen (die ja einen nachgewiesenen Value besitzen. \todo{Beispiele für Value durch Querverweise}
  \item Besitzer der Daten werden entlohnt.
  \item Nutzer der Daten zahlen für Daten, generieren damit aber Value, der wiederum entlohnt wird.
  \item Am Ende haben alle Teilnehmer entweder Value generiert oder aber im Wert des values verkonsumiert
  \item Wir partizipieren am extrinsischen Wert des Tokens (Kurs-Entwicklung durch positive Wertschöpfung des gesamten Ökosystems).
  \item Incentives sind nötig, um das Henne-Ei-Problem zu lösen
  \item Incentives sollten nachträglich mit der dadurch geschaffenen Wertschöpfung verrechtet werden. 
\end{itemize}

\end{Fazit}

\vspace{0.3cm}

\todo{TODO}