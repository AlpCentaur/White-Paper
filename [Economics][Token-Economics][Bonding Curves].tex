% !TEX root = paper.tex

\vspace{0.2cm}

\begin{itemize}
  \item \href{https://medium.com/@simondlr/tokens-2-0-curved-token-bonding-in-curation-markets-1764a2e0bee5}{Gute Einführung}
  \item \href{https://www.newsbtc.com/sponsored/zap-fun-profit/}{Anwendungsbeispiel}
  \item \href{https://docs.google.com/document/d/1VNkBjjGhcZUV9CyC0ccWYbqeOoVKT2maqX0rK3yXB20/edit}{White-Paper}
  \item \href{https://github.com/ConsenSysMesh/curationmarkets/blob/master/CurationMarkets.sol}{Smart-Contract}
\end{itemize}  



\vspace{0.5cm}

Zunächst einmal eine Formalisierung einer Token-Distribution mittels Bonding-Curves:

\vspace{0.2cm}

\begin{Def}[Token-Distribution mittels Bonding-Curves]
\label{defBC}

Angenommen man möchte einen Projekt-Token \textbf{TKN} herausgeben und dieses im Markt distribuieren. Der Mechanismus der \textit{Bonding-Curves} stellt hierbei ein alternatives Modell zu gängigen Tokensales (z.B. ICO) dar und folgt dabei einigen wesentlich Merkmalen, die ihn teils grundlegend von herkömmlichen Tokensales abgrenzen.

\begin{itemize}
  \item TKN wird von einem Smart-Contract verwaltet, der wesentlich mehr Logik implementiert als ein herkömmlicher ERC20-Contract.
  \item TKN kann jederzeit und von jedem gemintet werden. Dies geschieht gegen eine Einlage/Bezahlung in einer dafür definierten Währung (z.B. ETH oder USDT). Der Token wird quasi von dem Verwalter-Contract verkauft.
  \item Es existiert damit keine initiale bevorzugte Token-Ausgabe beim Contact-Launch an etwaige bevorzugte Parteien (Contract-Owner, Herausgeber, Investoren etc.). Die Ausgabe erfolgt ausschließlich gegen Einlage und kennt keine Bevorzugung entgegen der Contract-Logik. 
  \item Die Einnahmen aus der Tokenausgabe kommen ausschließlich der Token-Cont\-ract-Treasury $\mathbf{\mathcal{T}}$ zugute anstatt wie bei herkömmlichen Tokensales bestimmten Begünstigten (wie z.B. der Herausgeber-Company oder deren Gründern).
  \item TKN unterliegt keinem maximalen Gesamt-Supply. Es können stets neue Tokens ausgegeben werden, solange Interessenten existieren, die die Einlage für die Token-Ausgabe erbringen.
  \item Tokens können jederzeit von ihren Besitzern gegen einen - einer bestimmten Contract-Logik folgenden - Rückkaufpreis an den Token-Contract zurückgegeben werden. Zurückgegebene Tokens werden dabei sofort von dem Verwalter-Contract geburnt und somit aus der Zirkulation genommen.
  \item TKN kann selbstverständlich auch am Sekundärmarkt gehandelt werden (falls dieser bessere Konditionen hergibt als die Aussage bzw. Rücknahme durch den Token-Contract selbst).
\end{itemize}

\vspace{0.2cm}

Sei $i \in \mathbb{N}$ der aktuelle Gesamt-Supply von TKR (wir nehmen hier mal an, TKR sei atomar) und \textbf{\$} die Tausch- bzw. Einlage-Währung (\textbf{\$} ist hier abstrakt und nicht als US-Dollar zu verstehen).

Dann werden die durch die Token-Contract vorgegebenen Ausgabepreis (Kaufpreis) $\mathcal{K}$, Rücknahmepreis (Verkaufspreis) $\mathcal{V}$ und Contract-Treasury-Inhalt $\mathbf{\mathcal{T}}$ für den zuletzt ausgegebenen Token $i$ - jeweils in der Einheit \textbf{\$} - durch die jeweils supply-abhängigen Funktionen beschrieben:

\begin{align*}
\mathcal{K}, \mathcal{V}, \mathcal{T} &: \mathbb{N} \rightarrow \mathbb{Q}^{+} \\
\mathcal{K} \left( i \right) &:= \textrm{ Letzter Token-Ausgabepreis in \$ bei einem Gesamt-Supply von i} \\
\mathcal{V} \left( i \right) &:= \textrm{ Aktueller Token-Rückkaufkurs in \$ bei einem Gesamt-Supply von i} \\
\mathcal{T} \left( i \right) &:= \sum_{j = 1}^{i} \mathcal{K} \left( j \right)
\end{align*}

\vspace{0.2cm}

Die definierende Logik unseres Tokens lässt sich damit also formal als 

\begin{equation*}
TKR = \left( \mathcal{K}, \mathcal{V}, \$ \right)
\end{equation*}

schreiben, wobei hierbei $\mathbf{\mathcal{T}}$ ausgespart bleibt, da es implizit durch $\mathbf{\mathcal{K}}$ gegeben ist.

\vspace{0.2cm}

Sicherlich können und werden bei einer konkreten Implementierung eines mittels der durch $\mathbf{\mathcal{K}}$ und $\mathbf{\mathcal{V}}$ gegebenen \textit{Bonding-Curves} beschriebenen Tokens noch andere zu formalisierende Faktoren und Mechanismen eine Rolle spielen. Für die simpelste abstrakte Definition reichen die genannten Größen 
jedoch für den Moment aus.

\end{Def}

\vspace{0.3cm}

Die eben - auf diskrete Weise - definierten Funktionen $\mathcal{K} \left( i \right)$ und $\mathcal{V} \left( i \right)$ erfordern insofern noch eine zusätzliche Bemerkung, als dass diese explizit nur eine Token-weise Auskunft über Ausgabe- und Rücknahmepreis geben. Möchte man mehrere Token minten oder zurückkgeben, muss der Preis für jeden der Tokens separat ausgerechnet und anschließend addiert werden. Möchte man bei einem aktuellen Supply von $i \in \mathbb{N}$ nicht einen sondern $k \in \mathbb{N}$ mit $k \leq i$ Tokens minten bzw. zurückgeben, beläuft sich der gesamte Kauf- bzw- Verkaufspreis auf

\begin{align*}
\mathcal{K}_{k} \left( i \right) &:= \sum_{j = i + 1}^{i + k} \mathcal{K} \left( j \right) \\
\mathcal{V}_{k} \left( i \right) &:= \sum_{j = i - k + 1}^{i} \mathcal{V} \left( j \right).
\end{align*}

\vspace{0.3cm}

Das besonders Hervorhebenswerte an diesem Token-Ausgabemodell ist zweifelsfrei die gemeinschaftliche aus der Token-Ausgabe gefütterte Contract-Treasury aus echten Geldeinlagen, die nicht etwa einer dritten (Ausgabe-)Partei zugute kommt, sondern de facto den Tokeninhabern gehört. Die Existenz dieser Rücklagen gibt den ausgegebenen Tokens theoretisch einen realen Wert und ermöglicht einzig und allein die Implementierung des Rückkauf-Mechanismus. Der Gebrauch von dieser Möglichkeit und die Verankerung der Rückkauffunktion $\mathbf{\mathcal{V}}$ in der Logik des Token-Contracts realisiert den besagten realen Wert dann auch in der Praxis und nennt sich \textbf{\textit{"market maker of last resort"}}. Denn wenn ich eine definitive - in Contract-Logik verankerte - Sicherheit habe, ein Asset jederzeit verkaufen zu können, besitzt dieses Asset auch einen echten, intrinsischen Value, der nicht zwingend den marktwirtschaftlichen Mechanismen unterliegt - und somit auch keinen etwaigen Hypes um einen gut vermarkteten Tokensale ohne dahinterliegende Substanz. Vielmehr folgt der Tokenpreis der gemeinschaftlichen Projekt-Treasury $\mathbf{\mathcal{T}}$, die ihrerseits substanziell mit dem Projekterfolg einhergeht. 

\vspace{0.2cm}

Die letzte Einsicht lässt uns zu zwei wesentlichen Gedanken gelangen, die die entscheidenden Argumente für das \textit{Bonding-Curves}-Modell liefern könnten:

\begin{itemize}
  \item Der Rückgabepreis $\mathcal{V} \left( i \right)$ sollte eine direkte Abhängigkeit vom Treasury-Inhalt $\mathcal{T} \left( i \right)$ aufweisen.
  \item Nach bisheriger Definition hängt der Treasury-Inhalt $\mathcal{T} \left( i \right)$ ausschließlich vom aktuellen Supply $i \in \mathbb{N}$ (und den damit einhergehenden Ausgabepreisen $\mathcal{K} \left( j \right)$ für $j \leq i$) ab. Gepaart mit der ersten Forderung bedeutete dies implizit nichts anderes, als dass der aktuelle Rücknahmepreis ausschließlich von den Ausgabepreisen der bisherigen Tokens abhinge. Dies ist schlecht und ein sehr großes Problem der gängigen \textit{Bonding-Curves}-Implementierungen, da Koppelung des Rücknahmepreises - also des intrinsischen Werts des Tokens - ausschließlich an den Kaufpreis voriger Tokens - und die Wertentwicklung damit an den Kaufpreis etwaiger zukünftig ausgegebener Tokens, würde mathematisch alternativlos eine monoton steigende Ausgabepreis-Funktion $\mathcal{K} \left( i \right)$ erfordern. Ohne eine sehr stark fundierte projekt-bezogene Argumentation für ein monoton steigendes $\mathcal{K} \left( i \right)$ schrie das gesamte Modell nur so nach \textit{Pump \& Dump} und \textit{Hot Potatoe}. Und tatsächlich ist es so, dass nahezu alle \textit{Bonding-Curves}-Implementierungen Gebrauch von einer (streng) monoton steigenden Ausgabepreis-Kurve $\mathcal{K} \left( i \right)$ machen. Sie argumentieren mit anderem generierten Projekt-Value, der nicht durch die Contract-Treasury $\mathcal{T} \left( i \right)$ gemessen werden kann und diese Argumentation muss nicht zwingend falsch oder ungenügend sein. Uns reicht dies aber nicht - zumal es mehr als gute Gründe für ein monoton steigendes $\mathcal{K} \left( i \right)$ gibt (dazu später mehr). Somit bleibt uns nichts anderes als die - zumindest teilweise - Abkoppelung von $\mathcal{T} \left( i \right)$ und $\mathcal{K} \left( i \right)$, womit wir auch unsere obige Definition von $\mathcal{T} \left( i \right) = \sum_{j = 1}^{i} \mathcal{K} \left( j \right)$ wieder teils verwerfen müssen.
\end{itemize} 

\vspace{0.2cm}

Wir fassen zusammen:

\vspace{0.3cm}

\begin{Fazit}[Contract-Treasury als wichtigster Baustein zum Erfolg]

Wie sind der Überzeugung, der Contract-Treasury-Inhalt - gemessen als $\mathcal{T} \left( i \right)$ - sei der entscheidende Baustein für einen soliden \textit{Bonding-Curves}-Token. $\mathcal{T} \left( i \right)$ beeinflusst direkt den Rücknahmepreis $\mathcal{V} \left( i \right)$, verleiht dem Token damit einen echten geldwerten Value, was wiederum als Kaufargument für neue Tokens gilt und damit implizit auch zur Beurteilung des Ausgabepreises $\mathcal{K} \left( i \right)$ seitens etwaiger neuer Investoren hinzugezogen wird.

\vspace{0.2cm}

Konkret auf den Rücknahmepreis $\mathcal{V} \left( i \right)$ bezogen sehen wir kaum sinnvollere Alternativen als der gängigen Definition 

\begin{equation*}
\mathcal{V} \left( i \right) = \frac{\mathcal{T} \left( i \right)}{i}
\end{equation*}

zu folgen, was gleichbedeutend damit ist, dass der Besitz am Contract-Treasury-Inhalt pro rata auf die sich in Zirkulation befindenden Token verteilt wird, was sich konsequenterweise im Rücknahmepreis $\mathcal{V} \left( i \right)$ widerspiegelt. Mit der in Definition \ref{defBC} beschriebenen Treasury-Funktion $\mathcal{T} \left( i \right)$ ergibt sich damit für den Rücknahmepreis $\mathcal{V} \left( i \right)$

\begin{equation*}
\mathcal{V} \left( i \right) = \frac{\mathcal{T} \left( i \right)}{i} = \frac{\sum_{j = 1}^{i} \mathcal{K} \left( j \right)}{i}.
\end{equation*} 

\vspace{0.2cm}

Um einen Kauf bei aktuellem Supply von $i \in \mathbb{N}$ zu dem Preis von $\mathcal{K} \left( i \right)$ für sich als Investor zu rechtfertigen, muss man argumentieren, man könne den gekauften Token irgendwann zu einem höheren Preis wieder zurückgeben:

\begin{equation*}
\exists k > i, \textrm{  so dass } \mathcal{V} \left( k \right) > \mathcal{K} \left( i \right)
\end{equation*} 

\begin{equation*}
\Leftrightarrow \textrm{    } \sum_{j = 1}^{k} \mathcal{K} \left( j \right) > k \cdot \mathcal{K} \left( i \right)
\end{equation*} 

Gleichzeitig sollte $\mathcal{V} \left( i \right) \leq \mathcal{K} \left( i \right)$ für alle Supplies $i$ selbstverständlich angenommen werden können, da sonst Arbitrage entstünde, was sofort vom Markt aufgefressen werden würde.

\vspace{0.2cm}

Solche Anforderungen gehen alternativlos mit einem zwingend monoton steigenden $\mathcal{K} \left( i \right)$ einher (\todo{'monoton steigend' wäre hierbei noch zu präzisieren, da es nicht zu hundert Prozent dem mathematischen Verständnis gleicht und die Aussage wahrscheinlich noch zu beweisen; intuitiv ist es aber klar}).
Da eine zwingend monoton steigende Preiskurve als Investitionsmotivation - in einem abstrakt stehenden Kontext und ohne zusätzliche Argumente - schlichtweg unseriös ist, fordern wir für den \textit{Contract-Treasury-Inhalt}

\begin{equation*}
\mathcal{T} \left( i \right) > \sum_{j = 1}^{i} \mathcal{K} \left( j \right),
\end{equation*} 

bzw.

\begin{equation*}
\mathcal{T} \left( i \right) = \sum_{j = 1}^{i} \mathcal{K} \left( j \right) \textrm{  +  } f(t, i) \textrm{ mit } f(t, i) > 0,
\end{equation*} 

wobei $f(t, i)$ eine nicht genau präzisierte, positive \textit{Value-Funktion} unseres Projekts darstellt. Also so eine Art Gradmesser der (externen) Wertschöpfung/Wirtschaftlich\-keit unseres Projekts, die nicht in direktem Bezug zum Projekt-Kontrakt steht. Man könnte $f(t, i)$ vielleicht \textbf{\textit{Of-Contract-EBIT}} unseres Projekt nennen - wie auch immer dieses erwirtschaftet wird. 

Wie durch das $t$ angedeutet, bringt das $f(t, i)$ die zeitliche Dimension in unsere \textit{Contract-Treasury}-Rechnung, die externen (wirtschaftlichen) Einflüssen unterliegt und nicht mehr ausschließlich vom Gesamtsupply $i$ abhängt - gleichwohl der Gesamtsupply $i$ durchaus auch die genannten externen Einflüsse und damit auch $f(t, i)$ begünstigen kann.

\end{Fazit}

\vspace{0.3cm}

Der \textit{Of-Contract-EBIT} muss zwar nicht, kann aber durchaus auch aus anderen Smart-Contracts fließen. Solche Inter-Contract-Geldflüsse sind natürlich - falls möglich - eine stets bevorzugte Variante.  

Ist dies nicht möglich - und etwaige finanzielle Projekteinnahmen tatsächlich gänzlich \textit{off-chain}, bedarf es eines definierten und vertrauensvollen Mechanismus, der dafür sorge, dass die erzielten Projekterträge in die gemeinschaftliche Projekt-Treasury eingezahlt werden. Dies könnte entweder durch eine gesonderte Contract-Function dargestellt werden, oder aber durch den allgemeingültigen Weg mittels Tokenkäufen. Im zweiten Fall würden die gegen die extern erwirtschaftete Einlage ausgegebenen Tokens pro rata auf alle aktuellen Tokenhalter verteilt werden.

\vspace{0.2cm}

Mit einer nachweislich existenten \textit{externen projektbezogenen Profit-Funktion} $f(t, i)$ bekäme ein einem Projekt zugrunde liegendes \textit{Bonding-Curves-Token-Modell} das ihm noch fehlende - und aus unserer Sicht unverzichtbare - Merkmal: Nämlich eine gewisse Abkoppelung des realen Tokenwerts - gemessen durch $\mathcal{V} \left( i \right)$ - von der oft - zurecht als scheeball-artig kritisierten - Ausgabepreis-Kurve $\mathcal{K} \left( i \right)$.

Damit liefern wir aus unserer Sicht den bisher bei dem meisten \textit{Bonding-Curves-Implementierungen} fehlenden Baustein für eine solide und fundierte Token-Distribution mittels eines \textit{Bonding-Curves-Modells} und entgegnen damit - zumindest auf abstrakt Weise - dem bestehenden Hauptproblem der \textit{Bonding-Curves}: Nämlich \textit{Ponzi, Pump \& Dump oder Hot Potatoe} zu sein.

\vspace{0.2cm}

Wir behaupten hierbei keinesfalls, den heiligen Gral für das Funktionieren von \textit{Bonding-Curves-gestützten Token-Modellen} gefunden zu haben. Denn schließlich ist die Existenz des von uns eingeführten $f(t, i)$ alles andere als gegeben, äußerst projekt-abhängig und zudem in der Regel vermutlich nicht leicht zu formalisieren - geschweige denn ohne optimistische Annahmen zu beweisen. Denn nicht zuletzt argumentieren viele auf \textit{Bonding-Curves-Token-Modelle} gestützte Projekte genau wie wir hier an dieser Stelle: \textit{Es existiere ein externer Projekt-Value, von welchem die Tokeninhaber profitieren!} Nur geben diese Argumentationen diesem Value nicht den Namen $f(t, i)$ und verpassen damit eine formal abstrakte Verallgemeinerung.

Wir fassen zusammen:

\vspace{0.2cm}

\begin{Fazit}[Token-externer Projekt-Value]

Wir reduzieren das bestehende Problem von \textit{Bonding-Curves-Token-Modellen} hinsichtlich \textit{'Ponzi' und 'Pump \& Dump'} auf den Nachweis der Existenz einer \textbf{\textit{Of-Contract-EBIT-Funktion}} $f(t, i) > 0$ und deren Formalisierung anstatt von undefiniertem \textit{externen Projekt-Value} zu sprechen.

\vspace{0.2cm}

Idealerweise wird $f(t, i) > 0$ durch unbestreitbare Logik eines Smart-Contracts begründet, der in wertschöpfender Interaktion mit unserem Token-Contract steht (Profite in seine Treasury einzahlt). Andernfalls erschwert sich die Argumentation und muss solide begründet und formalisiert werden.

\end{Fazit}

\vspace{0.3cm}

Um das $f(t, i)$ nicht ganz als abstraktes Mysterium dastehen zu lassen und Zweifel an dessen Existenz zu zerstreuen: 

\vspace{0.2cm}

\begin{Example}[DeFi-Projekte besitzen stets ein $f(t, i) > 0$]

Jedes DeFi-Projekt besitzt ein $f(t, i) > 0$! Ob

\begin{itemize}
  \item DEX-Provider à la \textit{Balancer}, 
  \item Market-Places à la \textit{OpenSea},
  \item Lending-Platforms
\end{itemize}  

oder viele andere agieren wirtschaftlich, indem sie in der Regel Provisionen auf ihre Dienstleistung veranschlagen. Diese Gebühren sind unmissverständlich in ihrer Contract-Logik verankert und gänzlich transparent. 

Würden solche Projekte einen \textit{Bonding-Curves-gestützten} Utility-Token herausgeben und einen gewissen Abfluss ihrer Provisions-Profite in die Contract-Treasury dieses Tokens implementieren, würde solch ein Mechanismus auf ganz natürlich Weise das von uns geforderte $f(t, i) > 0$ definieren. Jeder Tokeninhaber würde damit jeder gebühren-pflichtigen Transaktion mit partizipieren.

\end{Example}

\vspace{0.3cm}

Wie nun bereits mehrfach betont, ist und bleibt das von uns eingeführte $f(t, i)$ projekt-bezogen. Wir wollen an dieser Stelle dagegen weiterhin möglichst abstrakt und allgemein bleiben und uns im folgenden der Ausgabepreis-Kurve $\mathcal{K} \left( i \right)$ widmen - insbesondere auch mit der Zielsetzung die Anforderungen an das $f(t, i)$ so gering zu gestalten, wie nur möglich.

Wie ebenfalls bereits herausgearbeitet, ist die verwendete Implementierung des $\mathcal{K} \left( i \right)$ ohne nachweisbare Existenz eines $f(t, i) > 0$ in vielen \textit{Bonding-Curves-Projekten} als ein teils unseriöser Preistreiber zu sehen. Dies resultierte aus der bereits oben formalisierten Kaufentscheidung eines jeden potenziellen Tokenkäufers:

\begin{equation*}
\textrm{Ich sollte kaufen   } \Leftrightarrow \exists k > i, \textrm{  so dass } \mathcal{V} \left( k \right) > \mathcal{K} \left( i \right)
\end{equation*} 

Und zwar \textbf{Preistreiber} deswegen, weil die Kaufentscheidung eines Tokens bei aktuellem Supply $i \in \mathbb{N}$ ohne fundiertes $f(t, i)$ von Kaufentscheidungen Anderer bei größerem Supply $i, i+1,...,k$ abhängt. Man sollte also genau dann kaufen, wenn man davon ausgeht, das andere später auch kaufen - und das noch zu einem höheren Preis. Diese müssten dann davon ausgehen, dass wiederum andere noch später zu einem noch höheren Kurs einsteigen und so weiter.

\vspace{0.2cm}

\todo{An bier WIP}

\vspace{0.5cm}

Und obwohl wir die Existenz eines $f(t, i) > 0$ nicht allgemein sicherstellen und es erst recht nicht konkret benennen können, so können wir zumindest eine Mindestanforderung herleiten und diese dabei zusätzlich ins Verhältnis zum Risiko eines potenziellen Investors setzen.

Dazu wollen wir das $\mathcal{V} \left( i \right)$ von etwaigen künftigen Käufen abkoppeln und seine Abhängigkeit von $\mathcal{K} \left( j \right)$ auf ausschließlich vergangene Käufe $\mathcal{K} \left( j \right)$ mit $j <= i$ reduzieren.

\vspace{0.3cm}

\begin{Praemisse}[Risiko eines Tokenkaufs]

Wir definieren eine zusätzliche Abhängigkeit zwischen Ausgabepreis-Kurve $\mathcal{K} \left( i \right)$ und Rücknahmepreis-Kurve $\mathcal{V} \left( i \right)$:

\begin{equation*}
\mathcal{K} \left( i \right) = \left( 1 + \rho \right) \cdot \mathcal{V} \left( i \right) = \left( 1 + \rho \right) \cdot \frac{\mathcal{T} \left( i \right)}{i} = \left( 1 + \rho \right) \cdot \frac{\sum_{j = 1}^{i} \mathcal{K} \left( j \right)}{i} \textrm{  mit } \rho > 0
\end{equation*}

Neu an dieser Forderung ist hierbei lediglich die erste Gleichheit. Die anderen sind Resultat unserer bereits weiter oben definierten Anforderungen an die Rücknahmepreis-Kurve $\mathcal{V} \left( i \right)$.

\vspace{0.3cm}

Das $\rho > 0$ beschreibt hierbei das Invest-Risiko eines Tokenkaufs bei Supply von $i \in \mathbb{N}$. Unsere Forderung ist also wie folgt zu interpretieren: $\rho$ beziffert den prozentualen Höchstverlust, den ein potenzieller Investor bei einem Kauf eines Tokens zum Preis von $\mathcal{K} \left( i \right)$ bei einem Supply von $i \in \mathbb{N}$ im schlimmsten Fall zu tragen hätte. Denn er könnte den gekauften Token jederzeit zum Preis von $\mathcal{V} \left( i \right) = \frac{\mathcal{K} \left( i \right)}{1 + \rho}$ wieder an den Contract zurückgeben und sich dafür auszahlen lassen.

\vspace{0.1cm}

Hinsichtlich dieser Interpretation wird das $\rho > 0$ in der Praxis also vermutlich tendenziell sehr klein zu wählen sein: $0 < \rho \ll 1$.

\vspace{0.1cm}

Die konkrete Wahl von $\rho$ wird hierbei in starker Wechselwirkung zum projekt-bezogenen $f(t, i) > 0$ stehen. Je vielversprechender der durch $f(t, i)$ beschriebe \textit{Of-Contract-EBIT} des Projekts ausfällt, desto größeres Risiko und damit $\rho$ ist einem Tokenkäufer zuzumuten und umgekehrt. 

\vspace{0.1cm}

Am Ende gilt wahrscheinlich sogar 

\begin{equation*}
\textrm{Es existiert kein  } f(t, i) > 0 \textrm{   } \Leftrightarrow \textrm{   } \rho = 0 \textrm{   } \Leftrightarrow \textrm{   } \mathcal{V} \left( i \right) = \mathcal{K} \left( i \right) \textrm{  } \forall i \in \mathbb{N}
\end{equation*}

\end{Praemisse}

\vspace{0.3cm}

Das Risiko ist mit der gemachten Vorgabe an die Rücknahmepreis-Kurve ist insofern mit gegebener Sicherheit gedeckelt als das $\mathcal{V} \left( i \right)$ für $\rho > 0$ stark monoton steigend ist

\vspace{0.3cm}

Zudem kann man bei gegebenem $\rho > 0$ seine durch

\begin{equation*}
\exists k > i, \textrm{  so dass } \mathcal{V} \left( k \right) > \mathcal{K} \left( i \right)
\end{equation*} 

gechallengte Kaufentscheidung validieren und nicht nur die Existenz eines solchen $k > i$ beweisen, sondern dieses $k$ auch konkret in Abhängigkeit des $\rho$ berechnen.

\vspace{0.3cm}

\todo{WIP}

\vspace{0.3cm}

\todo{Begünstigte formalisieren (mit zusätzlicher Funktion). Begünstigte erhalten einen Teil der Projekt-Treasury zu ihrer freien Verfügung.}

\vspace{0.5cm}
