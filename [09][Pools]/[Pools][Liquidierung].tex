% !TEX root = paper.tex

Für eine etwaige Pool-Liquidierung stellen sich exakt zwei Fragen: "\textbf{Wann} wird liquidiert?" und "\textbf{Wie} wird liquidiert?" Das \textbf{Wann} ist hierbei schnell geklärt. Es gibt grob folgende drei Möglichkeiten, von denen eine durch das in Definition \ref{defPool} definierte Regelset $\mathcal{R}$ zu spezifizieren ist:

\begin{itemize}
  \item $\mathcal{R}$ legt einen exakten Zeitpunkt fest, zu dem der Pool liquidiert werden soll.
  \item $\mathcal{R}$ definiert ein bestimmtes Ereignis, bei deren Eintreten der Pool liquidiert werden soll.
  \item $\mathcal{R}$ regelt, dass die Pool-Liquidierung per (DAO-)Abstimmung beschlossen werden muss.
\end{itemize}

\vspace{0.3cm}

Bullet 2 klingt hier leider noch nicht ausreichend abstrakt. Daher abstrahieren wir die genannten Forderungen in einer einzigen:

\vspace{0.2cm}

\begin{Fazit}[Liquidierungsentscheidung-Oracle]

Das in Definition \ref{defPool} definierte Regelset $\mathcal{R}$ definiert ein Oracle, welches zu jedem Zeitpunkt die Frage beantworten kann, ob der Pool zum jetzigen Zeitpunkt liquidiert werden soll oder nicht. 

Dieses Oracle kann beliebig einfach gestrickt sein (z.B. im Falle des obigen Bullet 1 einfach anhand $"SYSDATE <= T_{END}"$ über das Fortbestehen des Pools entscheidet) oder aber auch beliebig komplex. Dies braucht uns aber an dieser Stelle nicht weiter weiter interessieren. 

\end{Fazit}

\vspace{0.3cm}

Und da die Abstraktion mittels Oracle so bequem scheint, tun wir das Gleiche ebenfalls für das oben genannte \textbf{Wie}:

\vspace{0.2cm}

\begin{Fazit}[Auszahlungsschlüssel-Oracle]

Seien $\mathcal{P} = \left( \mathcal{U}, \mathcal{R}, \mathcal{T}, \mathcal{G} \right)$ 
der Pool und $\mathcal{U} = \left\{ u_1; u_2;...;u_n \right\}$ die Menge seiner $n$ Teilnehmer wie in Definition \ref{defPool} beschrieben und $v_{\mathcal{T}}$ der sich zum Liquidierungszeitpunkt in der Pool-Treasury $\mathcal{T}$ befindende Value. 

Falls der Pool lediglich als Treuhand-Verwahrung diente (also über die Zeit keine Veränderung der Treasury stattfand) ergibt sich $v_{\mathcal{T}}$ als  

\vspace{0.1cm}

\begin{equation*}
  v_{\mathcal{T}} = \sum_{i=1}^{n} s_i \text{ mit } s_i \text{ wie in Definition \ref{defPool}}
\end{equation*}

\vspace{0.2cm}

Wir definieren einen Auszahlungsvektor als

\begin{equation*}
  \varphi_{\mathcal{P}} = [\varphi_1, \varphi_1, ..., \varphi_n] \text{ mit } \sum_{i=1}^{n} \varphi_i = v_{\mathcal{T}} 
\end{equation*}

\vspace{0.2cm}

\todo{Ein mögliches $\varphi_{\mathcal{P}}$ ist tatsächlich $\mathcal{G}$}

\vspace{0.2cm}

\todo{$\mathcal{R}$ definiert das Oracle und sagt, ob die $\varphi_i$ negativ sein dürfen}

\vspace{0.2cm}

\todo{Oracle verteilt anhand von $\mathcal{T}$ und $\mathcal{G}$}

\end{Fazit}

\vspace{0.3cm}

\todo{TODO}

\vspace{0.5cm}