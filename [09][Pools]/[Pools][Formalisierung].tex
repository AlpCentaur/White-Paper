% !TEX root = paper.tex

Zunächst einmal benötigen wir einige formale Werkzeuge und bedienen uns dafür folgender Definition:

\vspace{0.2cm}

\begin{Def}\label{defPoolTeilnehmer}

Im folgenden setzen wir Voraus, die Nutzung der Pools seitens der User erfordert zwingend den Besitz eines WunderPass (bzw. Wunder-ID) und betrachten von daher auch nur solche User.

\begin{equation*}
  U := \left\{ u_1; u_2;...; u_{n} \text{ } | \text{ } u_i \text{ besitzt eine Wunder-ID} \right\}
\end{equation*}

\vspace{0.2cm}

Wir stellen zusätzlich, dass der vorausgesetzte Besitz einer Wunder-ID mit dem Besitz von unterschiedlichen Wallets bzw. anderen durch die Wunder-ID implizierten Dingen einhergeht. So hat jeder User $u_i$ z.B. eine Telefonnummer mit seiner Wunder-ID verknüpft (anhand derer er mittels Kontakte-Scan auf dem Smartphone als Inhaber einer Wunder-ID und damit potenzieller Pool-Teilnehmer erkannt werden kann und soll). Des weiteren kann $u_i$ einen NFT-Pass (siehe Kapitel \ref{sec:nft-pass}) besitzen und/oder unser ERC20-Utility-Token (im Folgenden als \textit{WPT} bezeichnet; siehe Kapitel \todo{TODO: verlinken}). 

\vspace{0.2cm}

Wir formalisieren den in Kapitel \ref{sec:nft-pass} definierten NFT-Pass als die (geordnete) Menge aller bisher geminteter NFT-Pässe:

\begin{align*}
  WPN &:= \left\{ wpn_1; wpn_2;... \right\} \text{ mit} \\
  wpn_i &:= (s_i, w_i, m_i)
\end{align*}

\vspace{0.2cm}

Dabei repräsentiert $s_i$ den Status des NFT-Passes, $m_i$ sein Muster und $w_i$ das sich auf ihm abgebildete Weltwunder.

\vspace{0.2cm}

Den Besitz eines Pass-NFTs beschreiben wir durch die Funktion 

\vspace{0.2cm}

\begin{equation*}
  \omega : U \rightarrow \mathcal{P} \left( WPN \right)  
\end{equation*}

\begin{equation*}
  \omega(u):= \left\{ wpn \in WPN \text{ } | \text{ User u besitzt den Pass-NFT } wpn \right\}. 
\end{equation*}

\vspace{0.2cm}

Analog dazu definieren wir auch den Besitz am \textit{WPT} eines Users - mit dem Unterschied, dass der Funktionsbereich dieser Funktion aufgrund der Fungibilität von einer Potenzmenge auf einen simplen numerischen Wert zusammenfällt:

\vspace{0.2cm}

\begin{equation*}
  \varphi : U \rightarrow \mathbb{Q}  
\end{equation*}

\begin{equation*}
  \varphi(u):= \text{ Balance des Users u am ERC20-Token WPT}. 
\end{equation*}

\end{Def}

\vspace{0.5cm}