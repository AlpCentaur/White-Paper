% !TEX root = paper.tex

\newpage

\section{NFT-Pass}
\label{sec:nft-pass}

Ein exzellentes Mittel, um \textit{WunderPass} als Geschäftsmodell, Unternehmung und Unternehmen ein symbolisches - gewissermaßen plastisches - Sinnbild einzuverleiben, ist die Repräsentation von \textit{WunderPass} als Service/Protokoll mittels eines - eigens dafür kreierten - NFTs: \textbf{"Des WunderPass"} (im Folgenden auch \textit{NFT-Pass})

\vspace{0.3cm}

\begin{Fazit}[\textit{WunderPass} deabstrahiert durch \textbf{"den WunderPass"} als NFT]

"Ich nutze \textit{WunderPass}" wird symbolisiert durch "Ich besitze \textbf{meinen WunderPass}"!

\end{Fazit}

\vspace{0.3cm}

\subsection{Konzeption}

\vspace{0.3cm}

Unser Anspruch an den zu modellierenden \textit{NFT-Pass} ist grob der folgende:

\vspace{0.2cm}

\begin{itemize}
  \item Der \textit{NFT-Pass} muss sich ganz klar von dem Großteil der heutigen - in größter Regel als Sammlerstück verstandenen - den Markt überflutenden NFTs abgrenzen. Er braucht einen klar ersichtlichen \textbf{intrinsischen Wert}. Man muss also "etwas mit dem \textit{NFT-Pass} anfangen können" und diesen nicht "nur besitzen", um ihn ausschließlich mit einer gewissen Wahrscheinlichkeit gewinnbringend weiterverkaufen zu können ("Hot Potato"). Der Token bedarf also gewisse Eigenschaften eines \textit{Governance-Tokens} (DAO) oder Ähnlichem.
  \item Der \textit{NFT-Pass} braucht ungeachtet des vorigen Bullets jedoch trotzdem zusätzlich ebenso eine ähnliche Beschaffenheit - wie solche der aktuell üblichen marktbeherrschenden NFTs - als Sammlerstück - gleichwohl nicht erstrangig. 
  \item Anders als die aktuell gängigen NFTs soll unser \textit{NFT-Pass} \textbf{nicht begrenzt} in der Anzahl seiner Stücke sein. Stattdessen sollen theoretisch beliebig viele \textit{NFT-Pässe} existieren können. Nichtsdestotrotz soll unser \textit{NFT-Pass} ebenso die Eigenschaft der nicht "inflationären Begehrtheit" einverleibt bekommen. Dies möchten wir mittels einer ausgeklügelten Minting-Logik abbilden, die ein \textbf{endliches Sub-Set} an raren und begehrten \textit{NFT-Pässen} innerhalb des \textbf{unendlichen Gesamt-Set} der \textit{NFT-Pässe} sicherstellt. Soll heißen: Es werden einerseits \textit{NFT-Pässe} existieren, die den heutigen NFTs - im Sinne ihres Sammlerwertes - gleichkommen, während die restlichen andererseits mit ihrer steigenden Gesamtanzahl zunehmend entwerten, bis sie irgendwann (als NFT betrachtet) nahezu wertlos und lediglich "funktional" werden.
  \item Die Rarität und Begehrtheit unseres \textit{NFT-Pass} soll Gamification-Mechanismen folgen:
  \begin{itemize}
    \item Wir brauchen an etwaigen Stellen das (wertbestimmendes) first-come-first-serve-Prinzip.
    \item Wir brauchen an anderen Stellen ein (ebenso wertbestimmendes) Zufallsprinzip.
    \item Wir brauchen irgendwo ebenso ein (geringes) Maß an persönlicher Individualisierung des \textit{NFT-Pass} - ausschließlich durch den User gesteuert.
    \item Abrundend könnte ein \textbf{gemeinnützig wertbestimmendes} (randomisiertes) Merkmal wirken. (Beispiel: Wenn die \textit{NFT-Pässe} irgendwann inflationär geworden sind, könnte der 10-Mio-ste plötzlich wieder richtig krass sein.)
  \end{itemize}
  \item Der \textit{NFT-Pass} muss gänzlich transparent und vor allem verständlich für den interessierten - gleichwohl vielleicht technisch nicht bewandertsten - User sein.
\end{itemize}

\vspace{0.3cm}

\todo{TODO: "Monalisa-Prinzip" ($\rightarrow$ NFT ganz neu gedacht $\rightarrow$ USP)}

\vspace{0.3cm}

Im folgenden ein initialer Abriss unserer Vorstellung des \textit{NFT-Pass}:

\vspace{0.3cm}

\begin{NFT-Prop}[Pass-Status]

Diese NFT-Property - die wir gleichzeitig als die Main-Property unseres \textit{NFT-Pass} ansehen - soll der oben formulierten Anforderung nach einem first-come-first-serve-Prinzip Rechnung tragen. Zeitlich früher \textit{ausgestellte NFT-Pässe} sollen einen rareren und begehrteren \textit{Pass-Status} inne haben als die späteren. Und vor allem sollen die \textit{NFT-Pässe} eines bestimmten ausgestellten Status in ihrer Anzahl begrenzt sein und nach Erreichen einer zu definierenden Höchstgrenze fortan nie wieder ausgestellt (gemintet) werden können.

\vspace{0.3cm}

Wir definieren folgende \textit{NFT-Pass-Status} mit den dazugehörenden Eigenschaften:

\begin{itemize}
    \item Status A (\textbf{Diamond})
    \begin{itemize}
    	\item Anzahl Pässe: 200
    	\item Gemintet an Nummer: 1 bis 200
    \end{itemize}
    \item Status B (\textbf{Black})
    \begin{itemize}
    	\item Anzahl Pässe: 1.600
    	\item Gemintet an Nummer: 201 bis 1800
    \end{itemize}
    \item Status C (\textbf{Pearl})
    \begin{itemize}
    	\item Anzahl Pässe: 12.800
    	\item Gemintet an Nummer: 1801 bis 14.600
    \end{itemize}
    \item Status D (\textbf{Platin})
    \begin{itemize}
    	\item Anzahl Pässe: 102.400
    	\item Gemintet an Nummer: 14.601 bis 117.000
    \end{itemize}
    \item Status E (\textbf{Rubin})
    \begin{itemize}
    	\item Anzahl Pässe: 819.200
    	\item Gemintet an Nummer: 117.001 bis 936.200
    \end{itemize}
    \item Status F (\textbf{Gold})
    \begin{itemize}
    	\item Anzahl Pässe: 6.553.600
    	\item Gemintet an Nummer: 936.201 bis 7.489.800
    \end{itemize}
    \item Status G (\textbf{Silver})
    \begin{itemize}
    	\item Anzahl Pässe: 52.428.800
    	\item Gemintet an Nummer: 7.489.801 bis 59.918.600
    \end{itemize}
    \item Status H (\textbf{Bronze})
    \begin{itemize}
    	\item Anzahl Pässe: 419.430.400
    	\item Gemintet an Nummer: 59.918.601 bis 479.349.000
    \end{itemize}
    \item Status I (\textbf{White})
    \begin{itemize}
    	\item Anzahl Pässe: $\infty$
    	\item Gemintet an Nummer: 479.349.001 bis $\infty$
    \end{itemize}
\end{itemize}

\end{NFT-Prop}

\vspace{0.3cm}

Diese NFT-Property ist per Definition trivialerweise \textbf{deterministisch}: Es ist stets zweifellos klar, welchen Status ein an x-ter Stelle geminteter \textit{NFT-Pass} haben wird. Die hinzugezogene "Reverse-Halving-Logik" \textbf{belohnt die Early-Adopter} mit einem begehrten NFT, dessen Rarität per Protokoll mit der Zeit stets abnimmt.

Die Beschaffenheit dieser first-come-first-serve-Property soll jedoch einzigartig bleiben. Die folgenden Properties werden nicht mehr deterministisch sein, um unserem \textit{NFT-Pass} ein \textbf{unvorherbestimmbares "Eigenleben"} einzuverleiben. 


\vspace{0.5cm}

\begin{NFT-Prop}[Hologramm (Welt-Wunder)]

Diese NFT-Property soll zwar einem ähnlichen abstufenden Raritätsprinzip zu Grunde liegen wie die Main-Property, dies jedoch nicht mehr einem first-come-first-serve- sondern stattdessen einem Zufallsprinzip folgend.

Ebenfalls abweichend von der Beschaffenheit der Main-Property soll bei dieser Property die Rarität nicht mittels einer absoluten Obergrenze abgebildet werden, sondern mittels einer relativen. (Dies zahlt auf die oben formulierte Anforderung nach einem \textbf{gemeinnützig gewinnbringendem Value} unseres \textit{NFT-Pass} ein.

\vspace{0.3cm}

Wir definieren folgende \textit{NFT-Pass-Hologramme} mit den dazugehörenden Eigenschaften:

\begin{itemize}
    \item WW1
    \begin{itemize}
    	\item Mögliche Ausprägung: \textbf{Pyramiden von Gizeh}
    	\item Anteil Pässe: 0,390625\% $\left( \frac{1}{256} \right)$
    \end{itemize}
    \item WW2
    \begin{itemize}
    	\item Mögliche Ausprägung: \textbf{Chinesische Mauer}
    	\item Anteil Pässe: 0,78125\% $\left( \frac{1}{128} \right)$
    \end{itemize}
    \item WW3
    \begin{itemize}
    	\item Mögliche Ausprägung: \textbf{Steinstadt Petra} 
    	\item Anteil Pässe: 1,5625\% $\left( \frac{1}{64} \right)$
    \end{itemize}
    \item WW4
    \begin{itemize}
    	\item Mögliche Ausprägung: \textbf{Kolosseum} 
    	\item Anteil Pässe: 3,125\% $\left( \frac{1}{32} \right)$
    \end{itemize}
    \item WW5
    \begin{itemize}
    	\item Mögliche Ausprägung: \textbf{Chichén Itzá} 
    	\item Anteil Pässe: 6,25\% $\left( \frac{1}{16} \right)$
    \end{itemize}
    \item WW6
    \begin{itemize}
    	\item Mögliche Ausprägung: \textbf{Machu Picchu} 
    	\item Anteil Pässe: 12,5\% $\left( \frac{1}{8} \right)$
    \end{itemize}
    \item WW7
    \begin{itemize}
    	\item Mögliche Ausprägung: \textbf{Taj Mahal} 
    	\item Anteil Pässe: 25\% $\left( \frac{1}{4} \right)$
    \end{itemize}
    \item WW8
    \begin{itemize}
    	\item Mögliche Ausprägung: \textbf{Christus-Statue Corcovado} 
    	\item Anteil Pässe: 50\% + x $\left( \frac{1}{2} + \frac{1}{256} \right)$
    \end{itemize}
\end{itemize}

\end{NFT-Prop}

\vspace{0.3cm}

Das Besondere an dieser Property spiegelt sich in der Tatsache wider, gewisse rar beschaffene Ausprägungen seien nur "zeitweise" ausgeschöpft, da sich ihre (rare) Anzahl lediglich \textbf{relativ} an der Gesamtzahl der aktuell \textit{ausgestellten NFT-Pässe} bemisst und nicht wie die Main-Property einer absoluten Obergrenze obliegt, deren Erreichung nicht wieder umkehrbar ist. Soll heißen: Ist die prozentuale Obergrenze an Pässen mit einer bestimmten Ausprägung der gegenwärtigen Property zu einem bestimmten Zeitpunkt erreicht, kann zwar für einen gewissen Zeitraum kein Pass mit dieser Ausprägung mehr ausgestellt werden. Sobald jedoch die Gesamtanzahl der \textit{ausgestellten NFT-Pässe} wieder groß genug ist - sodass die Anzahl der vorhandenen \textit{NFT-Pässe} mit der besagten Ausprägung wieder die prozentuale Obergrenze unterschreitet - werden Pässe der besagten Ausprägung "wieder verfügbar".

\vspace{0.3cm}

\begin{Algo}[Verlosungs-Mechanismus für Hologramm-Property]

\begin{itemize}
    \item Zunächst bestimme man die Gesamtanzahl aller bisher geminteter Pässe $n$.
    \item Gleiches tue man nun für die Counts der geminteten Pässe pro Ausprägung der Hologramm-Property WW1 bis WW7 als entsprechende Größen $n_1, n_2,...,n_7$.
    \item Und damit anschließend die aktuelle prozentuale Verteilung der Ausprägung auf die aktuell geminteten Pässe als $\sigma_i:= \frac{n_i}{n}$ für $i \in \lbrace 1,...7 \rbrace$ berechnen.
    \item Seien $\Theta_i$ für $i \in \lbrace 1,...7 \rbrace$ die oben definierten \textbf{relativen} Obergrenzen der \newline Ausprägungen der Hologramm-Property WW1 bis WW7.
    \item Alle Ausprägungen mit $\sigma_i \geq \Theta_i$ können zum aktuellen Zeitpunkt nicht vergeben werden und damit auch nicht beim Minting eines neuen Pass berücksichtigt werden.
    \item Für die Ausprägungen mit $\sigma_i < \Theta_i$ berechnen wir den Normierungsfaktor
\end{itemize} 

\begin{equation*}
\omega := \sum_{\sigma_i < \Theta_i} \Theta_i \textrm{ } \leq 1
\end{equation*} 

\begin{itemize}
    \item Damit errechnen wir die aktuell vorliegenden Wahrscheinlichkeiten $\rho_i$ für unsere Hologramm-Ausprägungen als
\end{itemize} 

\[
\rho_i:=\left\{%
\begin{array}{ll}
    0, & \hbox{falls $\sigma_i \geq \Theta_i$} \\[0,3cm]
    \hbox{\LARGE $\frac{\Theta_i}{\omega}$,} & \hbox{falls $\sigma_i < \Theta_i$} \\
\end{array}%
\right.
\] 

Man vergewissere sich an dieser Stelle gedanklich, dass auch für die neuen \newline Wahrscheinlichkeiten \[\sum_{i = 1}^7 \rho_i \textrm{ } = 1\] gilt.

\begin{itemize}
    \item Am Ende bestimme man mittels Randomisierung anhand der Wahrscheinlichkeiten $\rho_i$ für $i \in \lbrace 1,...7 \rbrace$ die zu vergebende Hologramm-Ausprägung. 
\end{itemize}

\end{Algo}


\vspace{0.5cm}

\begin{NFT-Prop}[Background (Muster)]

Diese NFT-Property soll ebenso wie die vorige einem ähnlichen abstufenden \newline Raritätsprinzip zu Grunde liegen wie die Main-Property, dies jedoch ebenso nicht einem first-come-first-serve- sondern stattdessen einem Zufallsprinzip folgend.

Genauso wie bei der Main-Property soll die Rarität dieser Property einer absoluten Obergrenze obliegen, bei deren Erreichung fortan keine \textit{NFT-Pässe} mit der erschöpften Property-Ausprägung mehr ausgestellt/gemintet werden können.

\vspace{0.3cm}

Wir definieren folgende \textit{NFT-Pass-Background-Muster} mit den dazugehörenden Eigenschaften:

\begin{itemize}
    \item M1
    \begin{itemize}
    	\item Mögliche Ausprägung: \textbf{Safari} 
    	\item maximale Anzahl Pässe: 256
    	\item Wahrscheinlichkeit falls noch nicht aufgebraucht: 1,5625\% $\left( \frac{1}{64} \right)$
    \end{itemize}
    \item M2
    \begin{itemize}
    	\item Mögliche Ausprägung: \textbf{Bars} 
    	\item maximale Anzahl Pässe: 4.096
    	\item Wahrscheinlichkeit falls noch nicht aufgebraucht: 3,125\% $\left( \frac{1}{32} \right)$
    \end{itemize}
    \item M3
    \begin{itemize}
    	\item Mögliche Ausprägung: \textbf{Dots} 
    	\item maximale Anzahl Pässe: 65.536
    	\item Wahrscheinlichkeit falls noch nicht aufgebraucht: 6,25\% $\left( \frac{1}{16} \right)$
    \end{itemize}
    \item M4
    \begin{itemize}
    	\item Mögliche Ausprägung: \textbf{Muster festlegen} 
    	\item maximale Anzahl Pässe: 1.048.576
    	\item Wahrscheinlichkeit falls noch nicht aufgebraucht: 12,5\% $\left( \frac{1}{8} \right)$
    \end{itemize}
    \item M5
    \begin{itemize}
    	\item Mögliche Ausprägung: \textbf{Muster festlegen}  
    	\item maximale Anzahl Pässe: 16.777.216
    	\item Wahrscheinlichkeit falls noch nicht aufgebraucht: 25\% $\left( \frac{1}{4} \right)$
    \end{itemize}
    \item M6
    \begin{itemize}
    	\item Mögliche Ausprägung: \textbf{Muster festlegen}
    	\item maximale Anzahl Pässe: unbegrenzt  
    	\item Wahrscheinlichkeit: 50\% + x mit stets größer werdendem $x \in \left[ \frac{1}{64}; \frac{1}{2} \right]$
    \end{itemize}
\end{itemize}

\vspace{0.2cm}

Werden Pässe der Ausprägungen M1 bis M5 (aufgrund ihres Caps) im Laufe der Zeit aufgebraucht, geht deren Wahrscheinlichkeit auf die Property-Ausprägung M6 über. Für das x aus der Beschreibung der Ausprägung M6 gilt also:

\vspace{0.2cm}

\begin{equation*}
x = \frac{1}{64} + \sum_{aufgebrauchte \textrm{ } M \in \lbrace M1;...;M5 \rbrace} Wahrscheinlichkeit(M)
\end{equation*}

\end{NFT-Prop}

\vspace{0.3cm}

\begin{Algo}[Verlosungs-Mechanismus für Background-Property]

\begin{itemize}
    \item Zunächst bestimme man die Gesamtanzahl aller bisher geminteter Pässe $n$.
    \item Gleiches tue man nun für die Counts der geminteten Pässe pro Ausprägung der Background-Property M1 bis M6 als entsprechende Größen $n_1, n_2,...,n_6$.
    \item Seien $\Theta_i$ für $i \in \lbrace 1,...6 \rbrace$ die oben definierten \textbf{absoluten} Obergrenzen der \newline Ausprägungen der Background-Property M1 bis M6.
    \item Seien $\rho_i$ für $i \in \lbrace 1,...6 \rbrace$ die oben definierten Wahrscheinlichkeiten der \newline Ausprägungen der Background-Property M1 bis M6, deren Auswahl wir hier zusätzlich formalisieren wollen:
    
\[
\rho_i:=\left\{%
\begin{array}{ll}
    \hbox{\LARGE $\frac{1}{2^{7 - i}}$,} & \hbox{für $i = 1,...,5$} \\[0,3cm]
    \hbox{$\frac{1}{2} + \frac{1}{2^6}$,} & \hbox{für $i = 6$} \\
\end{array}%
\right.
\]    
    
    \item Alle Ausprägungen mit $n_i \geq \Theta_i$ sind aufgebraucht und können weder zum aktuellen Zeitpunkt noch in der Zukunft vergeben werden und damit fortan auch nicht beim Minting eines neuen Pass berücksichtigt werden.
    \item Wir unterteilen die Ausprägungen der Background-Property in \textit{"verbraucht"} und \textit{"verfügbar"}:
    
\begin{align*}
\overline{M} &:= \lbrace i \in \lbrace 1,...,6 \rbrace \textrm{ } | \textrm{ } n_i \geq \Theta_i \rbrace \textrm{     [verbraucht]} \\
M &:= \lbrace i \in \lbrace 1,...,6 \rbrace \textrm{ } | \textrm{ } n_i < \Theta_i \rbrace \textrm{     [verfügbar]}
\end{align*} 

Man vergewissere sich an dieser Stelle gedanklich, dass $M \cap \overline{M} = \emptyset$, $M \cup \overline{M} = \lbrace 1,...,6 \rbrace$ und $6 \in M$ gelten.
    
    \item Ist eine bestimmte Ausprägung $i \in \lbrace 1,...,5 \rbrace$ verbraucht, soll ihre Wahrscheinlichkeit $\rho_i$ auf die Ausprägung M6 übertragen werden (damit wir bei 100\% bleiben).

    \item Damit errechnen wir die aktuell vorliegenden neuen Wahrscheinlichkeiten $\widehat{\rho}_i$ für unsere Background-Ausprägungen als
\end{itemize} 

\[
\widehat{\rho}_i:=\left\{%
\begin{array}{ll}
	\hbox{0,} & \hbox{für $i \in \overline{M}$} \\[0,1cm] 
    \hbox{$\rho_i$,} & \hbox{für $i \in M$, $i \neq 6$} \\[0,3cm]
    \displaystyle \rho_1 + \sum_{i \in \overline{M}} \rho_i, & \hbox{für $i = 6$} \\
\end{array}%
\right.
\]

Man vergewissere sich an dieser Stelle gedanklich, dass auch für die neuen \newline Wahrscheinlichkeiten \[\sum_{i = 1}^6 \widehat{\rho}_i \textrm{ } = 1\] gilt.

\begin{itemize}
    \item Am Ende bestimme man mittels Randomisierung anhand der Wahrscheinlichkeiten $\widehat{\rho}_i$ für $i \in \lbrace 1,...6 \rbrace$ die zu vergebende Background-Ausprägung. 
\end{itemize}

\end{Algo}



\vspace{0.5cm}

\begin{NFT-Prop}[Edition]

\todo{TODO}

\vspace{0.2cm}

Wrangel-Kiez $\rightarrow$ Berlin $\rightarrow$ Germany $\rightarrow$ Europe $\rightarrow$ WunderWorld $\rightarrow$ Sonnensystem $\rightarrow$ Milchstraße

\end{NFT-Prop}



\vspace{0.5cm}

\todo{TODO: Beispielrechnung für geminteten NFT-Pass mit der Nummer x}

\vspace{0.2cm}

Angenommen x sei 1.005.965.

\begin{itemize}
  \item vorrechnet, welche ersten 1.005.964 NFT-Pässe schon weggemintet sein könnten und Wahrscheinlichkeiten für den neu zu mintenden NFT-Pass erklären.
  \item neuen NFT-Pass unter Einbindung der Wahrscheinlichkeiten und vorgegaukelten Zufalls errechnet.
  \item geminteten neuen NFT-Pass als exakte Grafik in unserem Design hier abbilden.
\end{itemize}


\vspace{0.3cm}

\todo{TODO: Design}

\todo{TODO: intrinsischer Wert mittels Berechtigungen als Governance-Token}

\todo{TODO: Strategie des Minting und der Vergabe (Exploit-Prävention)}

\vspace{0.3cm}

TODOs könnten als Teil der Tech-Deep-Dive-Termine erarbeitet werden.

\vspace{0.3cm}

\subsection{Technische Umsetzung}

\vspace{0.3cm}

\todo{TODO: technische Implementierung}

\vspace{0.3cm}

\begin{itemize}
  \item Abwandlung des ERC721-Standard, um unsere Metadaten-Logik zu bändigen.
  \item Die Metadaten werden wohl auch einem ähnlichen Konstrukt wie IPFS (off-chain) gespeichert werden und lediglich deren Hash als Datenfeld im Smart-Contract (on-chain), damit die Metadaten nicht nachträglich verändern werden können (dieses Vorgehen wird der absolute Standard sein).
  \item Unsere Metadaten sind jedoch so komplex, das deren Erzeugung (beim Minten) wohl einen zweiten Smart-Contract erfordern wird. Wir haben also quasi einen "Metadaten-Hybriden":
  \begin{itemize}
  	\item Erzeugung on-chain
  	\item Storing off-chain
  \end{itemize}
  \item Der Metadaten-Smart-Contract wird die oben skizzierte Logik implementieren
  \begin{itemize}
  	\item Wie viele Pässe gibts es bereits und welche (hinsichtlich Properties)?
  	\item Wie sind die aktuellen Verteilungen der Properties und deren Contstraints
  	\item Einbindung von Randomisierungs-Orakeln
  	\item Sicherstellung, dass die erzeugten Metadaten auch tatsächlich vom Caller (ERC721-Contract) verwendet wurden und keine nachträgliche Manipulation stattgefunden hat.
  \end{itemize}
  \item Es muss geklärt werden, ob hinsichtlich des Gedanken an den besagten "zweiten Smart-Contract" Standards/Best-Practices existieren, damit wir hier nicht das Rad neu erfinden.
  \item Es bleibt noch nicht ganz klar, wie die Metadaten nach ihrer Erzeugung nach IPFS gelangen, da dies laut meinem Verständnis ein Smart-Contract nicht selbst gewährleisten kann. Moritz Idee war grob die Folgende 
  \begin{itemize}
    \item Der Minting-Contract erzeugt den NFT, lässt seine Metadaten-Referenz jedoch zunächst ungesetzt (der NFT ist damit in gewisser Weise noch "unfertig"; kann in dem Zustand auch noch Constraints unterstellt sein).
    \item Der Minting-Contract callt den Metadaten-Contract mit dem Anliegen, Metadaten zu dem "unfertigen" NFT mit der zugehörigen ID zu erzeugen.
  	\item Der Metadaten-Contract erzeugt die Metadaten, hasht diese und gibt den Hash zurück an den Minting-Contract. Gleichzeitig publisht er ein Create-Event mit der Token-ID und den zugehörigen erzeugten Metadaten.
  	\item Der Minting-Contract speichert den erhaltenen Metadaten-Hash und wartet auf "approvement".
  	\item Das forcierte Event wird von einem dafür bestimmten (off-chain) Web3-Service vernommen und weiterverarbeitet: Die Metadaten werden geparst und nach IPFS gepusht. Als Ergebnis bekommen wir eine entsprechende IPFS-URI.
  	\item Unser Web3-Service stößt anschließend eine "Set-URI"-Transaktion mit den entsprechenden Input-Daten (Token-ID; IPFS-URI) beim Minting-Contract an, um den gesamten Minting-Prozess für den neuen Token abzuschließen.
  	\item Der Minting-Contract verifiziert die Metadaten mittels des gespeicherten Meta-Daten-Hashs (\todo{Hier ist nicht nicht ganz klar, wie. Ich weiß nicht, ob der Contract einfach die Daten von IPFS laden kann, um den Hash abzugleichen oder ob er vorher die URI implizit vorgeben muss, die irgendwie im Hash berücksichtigt werden muss, oder wie auch immer hier die Best-Practise aussieht}) und updatet die NFT-URI auf den Wert der übergebenen IPFS-URI. 
  	\item Hiermit ist der Minting-Prozess abgeschlossen, der NFT "fertig" gemintet und kann von etwaigen "Temporary-Locked-Constraint" entbunden werden und vom neuen Besitzer frei verfügt werden.
  \end{itemize}
  \item \textbf{Ein etwaiger Crypto-Freelancer muss auf die skizzierten Herausforderungen gechallenget werden.}
\end{itemize}


\newpage