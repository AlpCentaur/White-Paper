% !TEX root = paper.tex

\subsection{Identität}
\label{sec:einleitung_identitaet}

Zunächst einmal eine gänzlich kontextfreie Sicht auf den Begriff "Identität" quasi "ganz von Null". Es folgt die \href{https://de.wikipedia.org/wiki/Identit%C3%A4t}{Wikipedia-Definition}:

\vspace{0.3cm}

\begin{Business-Def}[Identität]\label{defIdentity}

\textbf{Identität} ist die Gesamtheit der Eigentümlichkeiten (\textbf{"Gesamtheit persönlicher Eigenheiten"}), die eine Entität, einen Gegenstand oder ein Objekt kennzeichnen und als \textbf{Individuum} von anderen unterscheiden. In ähnlichem Sinn wird der Begriff auch zur \textbf{Charakterisierung von Personen} verwendet. […] So folgt die rechtliche \href{https://de.wikipedia.org/wiki/Identit%C3%A4tsfeststellung}{Identitätsfeststellung} den für \textbf{Inklusion} und \textbf{Exklusion} relevanten Markern moderner bürgerlicher Gesellschaften.

\end{Business-Def}

\vspace{0.3cm}

Die nahezu philosophische Auseinandersetzung mit dem allgemeinen Verständnis der Identität wollen wir an dieser Stelle nicht weiter vertiefen und verweisen stattdessen u. a. folgende Quellen:

\vspace{0.3cm}

\begin{Quelle}

\begin{itemize}
  \item \href{https://blockchainopedia.atlassian.net/wiki/spaces/RESEARCH/pages/81264861/Digitale+Identit+t+-+eine+allgemeine+Sicht}{Confluence}
  \item \href{https://vsis-www.informatik.uni-hamburg.de/getDoc.php/publications/191/InfTage_CPK.pdf}{Diplomarbeit - Christian Philip Kunze}
\end{itemize}

\end{Quelle}

\vspace{0.3cm}

\todo{TODO: Oben zitierter Confluence-Artikel ist natürlich nicht öffentlich. Ggf. sollte man relevante Dinge daraus hier einarbeiten und den Link entfernen. Insbesondere der im Confluence thematisierte Begriff der \textbf{Identitäts-Feststellung} könnte für unsere Zwecke von Relevanz sein.}

\vspace{0.3cm}

Die Deutung des Begriffs der \textbf{Identität} ist also ungemein stark abhängig von der Perspektive, aus der die Deutung erfolgt. Die Schaffung eines übergeordneten Identitäts-Verständnis - insbesondere unter Einbeziehung der \textbf{"Identitäts-Digitalisierung} - ist eine riesengroße Herausforderung. Aber gleichzeitig eine eben so große Chance. Da die eben angesprochene Digitalisierung aktuell nach wie vor größtenteils in staatlicher Hand liegt, im Folgenden noch eine weitere wesentliche (Perspektive-abhängige und etwas überspitzt formulierte) Identitäts-Definition: 

\vspace{0.3cm}

\begin{Business-Def}[gesellschaftliches/staatliches Verständnis der Identität]\label{defStaatIdentity}

Ich bin genau der, von dem mein Ausweis behauptet, ich sei es.

\end{Business-Def}

\vspace{0.5cm}
