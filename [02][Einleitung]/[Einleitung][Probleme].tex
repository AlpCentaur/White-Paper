% !TEX root = C:/Users/Slava/White-Paper/[02][Einleitung]/[Einleitung].tex

\subsection{Missstände der digitalen Identität}
\label{sec:einleitung_probleme_digitaler_identitaet}

\vspace{0.3cm}

Das im letzte Kapitel beleuchtete Verständnis der \textit{digitalen Identität} lässt bereits erahnen, dieses sei alles andere als optimal. Nicht aus technischer Sicht, nicht aus gesetzlicher Sicht und schon gar nicht aus Sicht des Anwenders. Profitierende Akteure des Status quo in diesem Kontext, sind bestenfalls diejenigen, die sich aufgrund einer etwaigen Vormachtstellung an Ineffizienzen des Gesamtsystems bereichern können, weil sie eben weniger Nachteile durch besagte Ineffizienzen erfahren als der restliche Markt. Also Google, Apple, Amazon, Facebook etc. Nur darf der Umstand, die größten Player da draußen, haben gar kein eigenes Interesse daran, das aktuelle Verständnis der \textit{digitalen Identität} (öffentlich) zu hinterfragen, nicht darüber hinwegtäuschen, das dieses tatsächlich alles andere als optimal und sehr wohl zu hinterfragen sei.

Dies liegt in erster Linie daran, dass die Einsicht zur Notwendigkeit einer sauberen Spezifikation der digitalen Identität erst viel später reifte, als ihre praktische Notwendigkeit. Spätestens mit dem massentauglichen Vormarsch des Web 2.0, mussten von so gut wie jedem Online-Dienst Userdaten modelliert werden. Da wären Gedanken, wie wir diese hier anstellen, hellseherisch gewesen. Die heutigen Definitionen \ref{defDigIdentity} und \ref{defAccount} entstanden also aus damaliger Sicht "by doing" und nicht etwa aus (dummen) Überlegungen.

\vspace{0.1cm}

Denn für Anbieter von Online-Diensten ist es schier unabdingbar, Daten des Users - also zumindest einen Teil der \textit{Identität} - zu erfassen: Sei es 

\begin{itemize}
  \item im Falle eines Versandhandels: \textbf{die Lieferadresse}
  \item im Falle der Absicherung gegenüber Jugendlichen: \textbf{die Altersfreigabe}
  \item im Falle von Entgeltforderungen: \textbf{Konto- oder Kreditkartendaten}
\end{itemize}
Auch die für das Marketing Verantwortlichen eines solchen Anbieters sind vielmals an einem \textbf{registrierten und wiedererkennbaren Kunden} und an dessen Kaufverhalten interessiert. [Der letzte Absatz folgte vielen Formulierungen der \href{https://vsis-www.informatik.uni-hamburg.de/getDoc.php/thesis/47/DA_Gordian_Kaulbarsch.pdf}{Diplomarbeit "Identitäten und ihre Schnittstellen auf Basis von Ontologien in einer dezentralen Umgebung"}]. 

\vspace{0.3cm}

Aber die ebenso suboptimale Fortentwicklung der eher ungesteuert geborenen \textit{digitalen Identität} blieb fortan nicht nur dem "Ist-Eben-So-Gewachsen" geschuldet.

Im Gegensatz zum User war es für den Dienstanbieter meist interessanter, \textbf{Informationen über die Nutzer an zentraler Stelle vorzuhalten}, deren Kontrolle ihm selbst oblag. Denn besagte Datenerfassung - gegeben durch freiwilligen oder gar erzwungen durch verpflichtend eingeforderten Daten-Input seitens des Anwenders - ermöglichte dem Dienstanbieter die Wiedererkennung und Verfolgung des Users, bzw. das Speichern und Auslesen von identifizierenden Dateien – sogenannten Cookies – und die Vergabe von zusätzlich identifizierenden Session-IDs. Auf diese Weise ließen und lassen sich heute noch extrem große Mengen an Daten erfassen, verknüpfen und systematisch auswerten. [Der letzte Absatz folgte vielen Formulierungen der \href{https://vsis-www.informatik.uni-hamburg.de/getDoc.php/thesis/47/DA_Gordian_Kaulbarsch.pdf}{Diplomarbeit "Identitäten und ihre Schnittstellen auf Basis von Ontologien in einer dezentralen Umgebung"}].

\vspace{0.3cm}

Ungeachtet dessen, wem oder was die besagte suboptimale "Geburt" und Fortentwicklung der digitalen Identität geschuldet sei, wollen wir im folgenden die konkreten Probleme und Missstände dieser aufarbeiten.

\vspace{0.3cm}

\begin{Problem}[fehlende Eindeutigkeit]

Das Problem der fehlenden Eindeutigkeit der Identität in der digitalen Welt wird am besten deutlich an dem Vergleich des sprachlichen Unterschieds zwischen den beiden Begriffen \textit{"dasselbe"} und \textit{"das Gleiche"}. Während ich im REWE-Supermarkt und am EasyJet-Terminal am Flughafen dieselbe Person darstelle, bin ich beim (online) REWE-Lieferdienst und beim Buchen eines Flugtickets auf der EasyJet-Homepage - aus Sicht der beiden Dienstleister - nur der gleiche Online-Konsument. Bestenfalls ist dies überhaupt erkennbar...

Ich bin mit \textit{denselben} Personen befreundet, mit denen ich auch gleichzeitig auf WhatsApp, Facebook, LinkedIn etc. connectet, ohne dass die sichere - geschweige denn zweifellos logisch implizierte - Gewissheit besteht, dass es sich tatsächlich stets um dieselbe Person handelt. Es könnte theoretisch ja auch ein Fake-Account sein (Facebook) oder längst veraltete Telefonnummer (WhatsApp), die sich hinter der geglaubten Identität verbirgt.

Die Sicherstellung der Eindeutigkeit erfolgt stets analog: Z. B. aus einem  (plausiblen) Chat-Verlauf bei WhatsApp oder einem Foto auf Instagram, wo man selbst drauf ist, was die geglaubte Identität beweist.

\vspace{0.2cm}

Dass diese \textit{analoge Verifizierung} aber nichts taugt, zeigt spätestens das Beispiel, dass ich sowohl eine KFZ-Führerschein- als auch eine Motorboots-Führerschein-Identität habend, bei einem Alkohol-Vergehen - was gesetzlich beide Identitäten beträfe - nur an derjenigen Identität belangt werde, die im direkten Zusammenhang mit dem Vergehen stand. Weil es eben oft bürokratisch und schwierig ist zwei \textit{gleiche} Datensätze aus unterschiedlichen digitalen Systemen zu \textit{derselben} Person zusammenzuführen. Weil eben Gleichheit keine Eindeutigkeit garantiert.

\vspace{0.2cm}

Verkörpert wird das Problem der fehlenden Eindeutigkeit in der digitalen Welt durch den sogenannten "Sign-Up", wo ich mich mal mit meiner Email-Adresse, mal mit meiner Telefonnummer, mal mit einem frei wählbaren Nickname und mal mit Google oder Facebook registrieren kann.

\end{Problem}

\vspace{0.3cm}


\begin{Problem}[Redundanz und fehlerbehaftete Daten]

Kann heutzutage noch irgendeiner zählen, wie oft er schon sein Email-Adresse eingeben musste, um sich irgendwo zu registrieren? Und das trotz sämtlicher Browser-Autovervollständigung. Wie oft seine Adresse bei Versandhandeln? Seine Kreditkarten-Nummer oder zumindest -CVC? Ebenso werden die meisten die Konsequenzen von Umzügen in eine neue Wohnung, den Wechsel der Telefonnummer oder den Verlust oder Ablauf einer Kreditkarte im Hinblick auf die bürokratischen Konsequenzen bei etwaigen Online-Diensten einzuordnen wissen. \textbf{Fuckup pur}.

Und das alles nur, weil unsere Daten abermals und abermals redundant von jedem Online-Service separat gespeichert werden. Ich ziehe nur einmal um, muss diese Info aber zig Mal mit Anderen teilen. ich verliere nur einmal meine Kreditkarte - und bekomme eine neue - muss dies aber an zig Stellen manuell aktualisieren. Ich wechsele meine Telefonnummer und es wird von 100 Kontakten trotzdem 10 geben, die mich deswegen nicht mehr erreichen können werden. Es wird Stellen geben, wo sich Typos in meine persönlichen Daten, meine Email-Adresse oder meine Telefonnummer einschleichen, von denen ich nichts ahne und andere Stellen, von denen ich noch nicht einmal mehr weiß, sie besäßen noch Daten von mir, die zu aktualisieren sind.

Dies ist nicht nur ein Problem beim User (Aufwand) sondern ebenso großes Problem beim Dienstleister (falsche Daten).

\end{Problem}

\vspace{0.3cm}


\begin{Problem}[mangelhafte UX]

Die heutige Existenz von zig, wenn nicht gar hunderten von Online-Accounts (= digitale Identitäten) pro User sehen wir nicht nur aufgrund der beiden eben formulierten Probleme der Uneindeutigkeit und Redundanz, sondern insbesondere auch aus User-Sicht als gänzlich unzeitgemäß und unzumutbar. Weil es eben nicht der Anspruch des digitalen Fortschritts unserer Zeit sein kann, zig und hunderte von Accounts und Passwörtern verwalten zu müssen, und diesen Missstand mit Verweis auf etwaige unterstützende Passwort-Manager auszublenden. \textbf{Was wir brauchen, sind keine Passwort-Manager oder Auto-Completion-Browser-Extensions, sondern ein grundlegendes Neudenken des digitalen Identifizierungs-Managements (Sign-up / Sign-in).}

\vspace{0.2cm}

Um den hiesigen Appell besser nachvollziehen zu können, brauchen wir einen etwas technischeren Blick auf den heute gängigen Sign-up-/Sign-in-Prozess. Für die Nutzung eines Online-Dienstes bedarf es folgender (simpler) Elemente:

\begin{itemize}
  \item Anlegen einer neuen Online-Identität beim zugehörigen Online-Dienst (\textbf{Sign-up}) als Mapping zwischen
  \begin{itemize}
  	\item technischem Identifier beim zugehörigen Online-Dienst (Kundennummer/
  	\newline Nickname/Telefonnummer/Emailadresse).
  	\item Userdaten
  \end{itemize}
  \item Identifizierung mittels Eingabe des technischen Identifier, um dem zugehörigen Online-Dienst mitzuteilen, wer man ist (Teil des \textbf{Sing-in}s).
  \item Autorisierung mittels Eingabe des persönlichen Passworts (oder auch 2FA), um dem zugehörigen Online-Dienst zu beweisen, man sei auch tatsächlich derjenige, als den man sich ausgibt (Teil des \textbf{Sing-in}s).
\end{itemize}

Alle dieser drei Elemente sind fraglos nötig (gleichwohl das erste genau genommen nur einmal universell für alle existierenden Online-Dienste nötig wäre; siehe auch die beiden oben adressierten Probleme), das Problem hier ist nur, dass hier viel zu viel manuelles Zutun vom User eingefordert wird und damit die besagte UX ruiniert. 

Dabei ist der Status quo hinsichtlich der Autorisierung bereits auf ganz gutem (UX-)Weg. Der Stand hinsichtlich der Identifizierung ist dagegen weiterhin katastrophal! Und katastrophal heterogen und uneinheitlich noch dazu.

\end{Problem}

\vspace{0.3cm}



\begin{Problem}[Datenschutz]

\todo{TODO: ausformulieren}
\begin{itemize}
  \item meine Daten liegen an zig/hunderten Stellen gespeichert
  \item Hacks sind an zig/hunderten Stellen möglich
\end{itemize}

\end{Problem}

\vspace{0.3cm}



\begin{Problem}[Daten werden nicht dort erfasst, wo sie gebraucht werden]

\todo{TODO: ausformulieren}
\begin{itemize}
  \item Daten werden an anderer Stelle erhoben als sie gebraucht werden --> Beispiel mit der Supermarkt-Kassiererin bzw. Fluggesellschaften
\end{itemize}

\end{Problem}

\vspace{0.3cm}


\begin{Problem}[Datenmissbrauch/Bereicherung]

\todo{TODO: ausformulieren}
Big Tech nutzt meine Daten, um daran Geld zu verdienen. Und ich werde nicht an der Wertschöpfung beteiligt.

\end{Problem}

\vspace{0.3cm}


\begin{Problem}[Abhängigkeit von Big Tech]

\todo{TODO: ausformulieren}
Derzeit dominieren zentrale ID-Provider wie Google und Facebook die Verwaltung von Identitätsdaten sehr vieler IT-Dienste weltweit, was zu einer großen Abhängigkeit unserer Gesellschaft in Bezug auf den Fortgang der Digitalisierung führt.

\end{Problem}

\vspace{0.3cm}


\begin{Problem}[Ungenutzte Möglichkeiten]

\todo{TODO: ausformulieren}
Daten-Querverweise $\rightarrow$ Beispiel anführen 

(zB aus \href{https://norbert-pohlmann.com/glossar-cyber-sicherheit/self-sovereign-identity-ssi/}{Vorlesung zu SSI})

\end{Problem}

\vspace{0.5cm}
