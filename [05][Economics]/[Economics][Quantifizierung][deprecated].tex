% !TEX root = paper.tex

\todo{[TODO6][deprecated Inhalt verarbeiten]}
\vspace{0.3cm}

Mit diesen geschaffenen Formalisierungs-Werkzeugen lässt sich nun auch die übergeordnete WunderPass-Vision formal erfassen - und zwar indem man den Zeitpunkt $t_{*} \in T$ ihrer Erreichung benennt:

\begin{Def}\label{defVision}

Wir betrachten die WunderPass-Vision zu einem Zeitpunkt $t_{*} \in T$ als erreicht, falls

\vspace{0.3cm}

\begin{equation}
\label{eq:1}
  \alpha^{(t_{*})}_{ij} = 1 \textrm{ für alle } i \in \{1,...,n\} \textrm{ und } j \in \{1,...,m\}
\end{equation}\\
erfüllt ist. Darüber hinaus ist es noch nicht ganz klar, welche Aussage für die Zeitpunkte $t > t_{*}$ hinsichtlich der Visions-Erreichung zu treffen sei. Grundsätzlich ist es ja durchaus denkbar, die obige Voraussetzung gelte für $t > t_{*}$ nicht mehr. Bleibt die Vision in diesem Fall trotzdem als 'erreicht' zu betrachten?

\end{Def}

\vspace{1cm}

Die gelungene Formalisierung unserer Vision mittels Definition \ref{defVision} mag einen Fortschritt hinsichtlich unserer "Business-Mathematics" darstellen, bleibt jedoch losgelöst zunächst einmal ziemlich wertlos. Zum einen ist das Erreichen der Vision im formellen Sinne der Definition \ref{defVision} weder praxistauglich noch akribisch erforderlich. Zudem bleibt zum anderen der resultierende (intrinsische) Business-Value der Visions-Erreichung bisher weiterhin nicht ohne Weiteres erkennbar.
Vielmehr sollten wir die Anforderung von Gleichung \eqref{eq:1} als eine Messlatte unseres Fortschritts heranziehen, und eher als (unerreichbare) 100\%-Zielerreichungs-Marke betrachten. Zudem müssen wir zeitnah - obgleich die vollständige oder nur fortschreitend partielle - Zielerreichung unserer Vision in klaren, quantifizierbaren Business-Value übersetzen.

Dazu definieren wir als erstes ein intuitives Maß der Zielerreichung:

\vspace{0.3cm}

\begin{Def}\label{defGamma2}

\begin{equation*}
  \Gamma : T \rightarrow \mathbb{N} 
\end{equation*}

\begin{equation*}
  \Gamma(t):= \sum_{i=1}^n \sum_{j=1}^m \alpha^{(t)}_{ij} 
\end{equation*}

\end{Def}

\vspace{1cm}

Damit liefert uns die definierte $\Gamma$-Funktion aber auch ein extrem greifbares und intuitiv nachvollziehbares Fortschrittsmaß unseres Vorhabens. Zudem fügt sich dieses perfekt in unsere mittels Definition \ref{defVision} quantifizierte Unternehmens-Vision und unterliegt einer fundamentalen (bezifferbaren) Obergrenze. Dies zeigt folgendes Lemma:

\vspace{0.3cm}

\begin{Lemma}

Es gelten folgende Aussagen:

\begin{align}
\Gamma(t) &\leq n^{(t)} * m^{(t)} \textrm{ für alle } t \in T \tag{i} \label{eq:l1_erste} \\ 
  \text{es gilt Gleichheit bei }  \eqref{eq:l1_erste} &\Leftrightarrow \text{ es gilt Gleichung } \eqref{eq:1} \text{ aus Def } \ref{defVision} \tag{ii} \label{eq:l1_zweite}
\end{align}

\end{Lemma}

\vspace{0.3cm}

Gleichung \eqref{eq:l1_zweite} ermöglicht uns die Definition \ref{defGamma} auf ein relatives Zielereichungs-Maß auszuweiten:

\vspace{0.3cm}

\begin{Def}\label{defKleinGamma}
\begin{equation*}
  \gamma : T \rightarrow [0; 1] 
\end{equation*}

\begin{equation*}
  \gamma(t):= \frac{\Gamma(t)}{n^{(t)} * m^{(t)}}
\end{equation*}

\end{Def}

\todo{[ende TODO6]}
\vspace{1.5cm}