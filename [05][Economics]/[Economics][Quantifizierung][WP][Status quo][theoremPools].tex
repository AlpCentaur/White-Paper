% !TEX root = paper.tex

\begin{Theorem}\label{theoremPools}

Seien $n = |U^{(t)}|$ und $m = |S^{(t)}|$ bzw. $(n, m) = dP^{(t)}$. Dann gelten folgende Aussagen:

\begin{align}
&\widehat{n} \leq n \tag{i.u} \\ 
&\widehat{m} \leq m \tag{i.s} \\
&\widehat{n} = n \Leftrightarrow \widehat{U}^{(t)} = U^{(t)} \tag{ii.u} \\
&\widehat{m} = m \Leftrightarrow \widehat{S}^{(t)} = S^{(t)} \tag{ii.s} \\
&\widehat{n} * \widehat{m} > 0 \Leftrightarrow \widehat{U}^{(t)} \neq \emptyset \Leftrightarrow \widehat{S}^{(t)} \neq \emptyset \tag{iii} \\
&\widehat{n} * \widehat{m} = 0 \Leftrightarrow \widehat{U}^{(t)} = \emptyset = \widehat{S}^{(t)} \tag{iv} \\
&\widehat{u}^{(t)} \in \widehat{U}^{(t)} \Leftrightarrow \exists \widehat{s} \in \widehat{S}^{(t)} \textrm{ mit } \alpha^{(t)}\left(\widehat{u}, \widehat{s}\right) = 1 \tag{v} \\
&\widehat{s}^{(t)} \in \widehat{S}^{(t)} \Leftrightarrow \exists \widehat{u} \in \widehat{U}^{(t)} \textrm{ mit } \alpha^{(t)}\left(\widehat{u}, \widehat{s}\right) = 1 \tag{vi} \label{theoremPools_6}
\end{align}

\end{Theorem}

\vspace{0.3cm}

\begin{proof}[Beweis] \textrm{ }

\vspace{0.3cm}

(i) und (ii) sind (in jeweils beiden Varianten) trivial!

\vspace{0.3cm}

zu (iii): Zunächst einmal ist

\begin{align*}
\widehat{n} * \widehat{m} > 0 &\Leftrightarrow \widehat{n}, \widehat{m} > 0 \\
&\Leftrightarrow |\widehat{U}^{(t)}|, |\widehat{S}^{(t)}| > 0 \\
&\Leftrightarrow \widehat{U}^{(t)}, \widehat{S}^{(t)} \neq \emptyset
\end{align*}

\vspace{0.3cm}

Es bleibt also nur noch $\widehat{U}^{(t)} \neq \emptyset \Leftrightarrow \widehat{S}^{(t)} \neq \emptyset$ zu beweisen. Wir beschränken uns hierbei lediglich auf "$\Rightarrow$". Die Rückrichtung erfolgt gänzlich analog. Sei also $\widehat{U}^{(t)} \neq \emptyset$.

\begin{align*}
\widehat{U}^{(t)} \neq \emptyset &\Rightarrow \exists u^{*} \in \widehat{U}^{(t)} \\
&\xRightarrow{Def \ref{defPools}} \exists s^{*} \in S^{(t)} \textrm{ mit } \alpha^{(t)}(u^{*}, s^{*}) = 1 \\
&\Rightarrow s^{*} \in \widehat{S}^{(t)} \\
&\Rightarrow \widehat{S}^{(t)} \neq \emptyset
\end{align*}

\vspace{0.3cm}

zu (iv): 
"$\Leftarrow$" ist gänzlich trivial. Die Richtung "$\Rightarrow$" folgt dagegen aus

\begin{align*}
\widehat{n} * \widehat{m} = 0 &\Rightarrow \textrm{ mindestens eine der Mengen } \widehat{U}^{(t)}, \widehat{S}^{(t)} \textrm{ ist leer } \\
&\xRightarrow{(iii)} \widehat{U}^{(t)}, \widehat{S}^{(t)} = \emptyset
\end{align*}

\vspace{0.3cm}

zu (v): Die Richtung "$\Leftarrow$" folgt trivial aus Def \ref{defPools} und $\widehat{s} \in \widehat{S}^{(t)} \subseteq {S}^{(t)}$.

Für "$\Rightarrow$" mögen wir annehmen 

\begin{equation*}
  \exists u^{*} \in \widehat{U}^{(t)} \textrm{ mit } \forall \widehat{s} \in \widehat{S}^{(t)} \textrm{ gilt } \alpha^{(t)}(u^{*}, \widehat{s}) = 0
\end{equation*}
Da jedoch laut Annahme $u^{*} \in \widehat{U}^{(t)}$, muss aufgrund von Def \ref{defPools} ein $s^{*} \in S^{(t)} \setminus \widehat{S}^{(t)}$ mit $\alpha^{(t)}(u^{*}, s^{*}) = 1$ existieren. Da $u^{*} \in \widehat{U}^{(t)} \subseteq {U}^{(t)}$, muss $s^{*}$ jedoch laut Def \ref{defPools} auch in $\widehat{S}^{(t)}$ liegen. Im Widerspruch zu $s^{*} \in S^{(t)} \setminus \widehat{S}^{(t)}$.

\vspace{0.3cm}

Aussage (vi) ergibt sich ganz analog zu (v)!
  
\end{proof}
\vspace{0.3cm}

