% !TEX root = paper.tex

\paragraph{Messbarkeit Status quo} 
\label{sec:eco_zahlen_zustand_wp_nowVlaue}
\textrm{ }

\vspace{0.3cm}

Kommend von der intuitiven Annahme, die Größe der definierten "connected Pools" $\widehat{U}^{(t)}$ und $\widehat{S}^{(t)}$ sei irgendwie erstrebenswert in unserem Sinne, definierten wir im vorangehenden Abschnitt das - aus unserer Sicht fundierteres und geeigneteres - Maß $\Gamma(t)$, um dem Verständnis von "erstrebenswerter Zustand" besser gerecht zu werden.

In diesem Abschnitt wollen wir die - bisher eher wertfrei/objektiv formulierten -  Ergebnisse des vorigen Abschnitts in den Kontext der "Erstrebenswertigkeit" stellen. Also eine formale und quantifizierbare Vergleichbarkeit unserer - ohnehin beim Lesen des letzten Abschnitts mitschwingender - Intuition schaffen, die Werte 
\begin{itemize}
  \item $\widehat{n} = |\widehat{U}^{(t)}|$, 
  \item $\widehat{m} = |\widehat{S}^{(t)}|$ und vor allem 
  \item $\Gamma(t)$ 
\end{itemize}
seien umso besser je größer sie seien. Alle der eben genannten Größen, denen wir hier eine intuitiv spürbare "Erstrebenswertigkeit" beimessen, besitzen einen klaren Zeitbezug. Daher überrascht es kaum, wir strebten die genannte quantifizierbare Vergleichbarkeit für je zwei beliebige Zeitpunkte $t_1, t_2 \in T$ an. Formale Vergleichbarkeit schreit nur so nach der mathematisch verstandenen "Ordnungsrelation":

\vspace{0.3cm}

\begin{Def}\label{defRelation}

Wir bedienen uns der in Definition \ref{defGamma} beschriebenen Funktion $\Gamma(t)$, um damit eine \href{https://de.wikipedia.org/wiki/Ordnungsrelation}{Ordnungsrelation} 
auf unserem Zeitstrahl $T$ für je zwei beliebige Zeitpunkte $t_1, t_2 \in T$ zu erhalten: 

\vspace{0.3cm}

\begin{equation*}
  R_{\preceq} \subseteq T \times T \textrm{ mit}
\end{equation*}

\begin{equation*}
  R_{\preceq}:= \left\{ (t_1, t_2) \in T \times T \mid \Gamma(t_1) \leq \Gamma(t_2) \right\}
\end{equation*}
\vspace{1cm}
Mittels $R_{\preceq}$ erhalten wir eine Ordnung unseres Zeitstahls $T$ und erklären zudem insbesondere, was "erstrebenswert" bedeutet. Ein beliebiger Zeitpunkt $t_1 \in T$ ist nämlich verbal genau dann "nicht weniger erstrebenswert" in Sinne unserer Vision als ein beliebiger anderer Zeitpunkt $t_2 \in T$, falls $(t_1, t_2) \in R_{\preceq}$ gilt.

\vspace{0.3cm}

\begin{equation*}
  \textrm{Wir schreiben fortan statt } (t_1, t_2) \in R_{\preceq} \textrm{ lieber } t_1 \preceq t_2 
\end{equation*}

\end{Def}

\vspace{1cm}

Man beachte, dass es sich bei der definierten Ordnungsrelation gar um eine \href{https://de.wikipedia.org/wiki/Ordnungsrelation#Totalordnung}{Totalordnung} handelt!
Der Form halber ergänzen wir an der Stelle noch um zwei weitere - schematisch induzierte - Relationen auf unserem Zeitstrahl $T$:

\vspace{0.3cm}

\begin{Def}\label{defRelationen}

Um zusätzlich zur in Def \ref{defRelation} definierten Ordnungsrelation "$\preceq$", auch dem Verständnis von "echt besser" und "gleich gut" Rechnung zu tragen, definieren wir die beiden Relationen "$\prec$" und "$\simeq$"

\vspace{0.3cm}

\begin{equation*}
  R_{\prec}:= \left\{ (t_1, t_2) \in T \times T \mid \Gamma(t_1) < \Gamma(t_2) \right\}
\end{equation*}

\begin{equation*}
  R_{\simeq}:= \left\{ (t_1, t_2) \in T \times T \mid \Gamma(t_1) = \Gamma(t_2) \right\}
\end{equation*}

\vspace{1cm}

Bei $R_{\prec}$ handelt es sich im Übrigen wieder um eine Ordnungsrelation. Bei $R_{\simeq}$ dagegen nicht.

\end{Def}

\vspace{0.3cm}

Auch für die letzten beiden Relationen wollen wir fortan die vereinfachte Schreibweise $t_1 \prec t_2$ und $t_1 \simeq t_2$ nutzen.

\vspace{0.6cm}

