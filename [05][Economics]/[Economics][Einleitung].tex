% !TEX root = paper.tex

\subsection{Einleitung}
\label{sec:eco_einleitung}

Wir beginnen mit einer gewagten Behauptung:

\vspace{0.2cm}

\begin{Hypothese}[Daten haben einen Wert]
\textbf{Digitale Daten besitzen einen realen} (nicht leicht zu beziffernden) \textbf{Value} - zumindest für die von ihnen direkt oder indirekt adressierten digitalen Individuen (User und Service-Provider). Dieser Value existiert bereits in Isolation des einzelnen Datensatzes, wird aber mit Zunahme der verfügbaren Gesamtdaten innerhalb eines Netzwerks durch entstehende Synergien nicht nur in Summe sondern zudem ebenfalls pro einzelnem Datensatz stets größer (Netzwerkeffekt). Eine einzelne Information - ein Datensatz - besitzt also bereits einen isolierten Mehrwert - anfangs vielleicht nur für sehr wenige Teilnehmer des Netzwerks - und gewinnt zudem zunehmend weiter an Wert mit Wachstum des "Wissens" des Gesamtnetzwerks und verhilft auch anderen Dateninformationen zu deren Wertsteigerung.

Wir behaupten damit also, jeder digitale Datensatz habe sogar einen Value über die von ihm adressierten digitalen Individuen hinaus. Und zwar für die Gesamtheit des Netzwerks und all seiner Teilnehmer. Dies jedoch natürlich nicht im gleichen Maße für alle.

\vspace{0.1cm}

Damit ist \textbf{digitale Datenerfassung und -auswertung wertschöpfend} für die gesamte digitale Welt und wünschenswert. Lediglich die \textbf{Verteilung des geschöpften Values muss hinterfragt werden}. Wir möchten ein Okösystem definieren, der genau dies gerecht und transparent tut.
 
\end{Hypothese}

\vspace{0.3cm}
Rein formal mathematisch betrachtet, ist die formulierte Behauptung ziemlich einleuchtend - zumindest wenn man auf die Forderung, dieser "Value" (sei er mit $v_{data}$ bezeichnet) habe ein positives Vorzeichen, verzichtet. Als "Value" also zunächst lediglich abstrakt einen "Impact" annimmt (der auch einen "Schaden" mit $v_{data} < 0$ darstellen könnte, wenn die Daten in irgendeiner Weise missbraucht werden). Von 

\begin{equation*}
\vert v_{data} \vert > 0
\end{equation*}

können wir also ziemlich bedenkenlos ausgehen.

\vspace{0.2cm}

Auf der gesellschaftlich/sozialen Ebene wird man dagegen deutlich mehr Widerspruch zur getätigten Hypothese ernten (bzw. von abstrakt-denken-könnenden Menschen zumindest das negative Vorzeichen von $v_{data}$ vorgehalten bekommen). Denn leider ist die elektronische Datenverarbeitung - in ihrem wahren Sinne des Wortes - ziemlich in Verruf geraten. Man hört den Tadel von mit Datenschutz in Verbindung gebrachten Missständen deutlich lauter als die Anerkennung des Nutzens von Datenerfassung und ihrer Verarbeitung - auf die im Übrigen so gut wie niemand mehr verzichten können würde. Weil Menschen eben schnell vergessen und zudem in der Regel kein ausreichendes technisches Verständnis besitzen. 

Kein Mensch würde doch heute wieder bei Taxizentralen anrufen wollen und seine Abholadresse durchgeben, weil GPS-Lokalisierungen unterbunden werden sollen. Gleiches bei Food-Lieferanten. Und auch nur die Wenigsten auf die Intelligenz von Google, weil Google nur das für uns tun kann, was sie für uns tun, weil sie das über uns wissen, was sie eben über uns wissen. Der Zug in diesem Kontext ist bereits unumkehrbar abgefahren. Weil eben selbst so gut wie in jeder Lebenssituation von dem besagten Daten-Value $v_{data}$ profitieren. Und zwar mit unbestreitbarem positiven Vorzeichen.

Was die besagten Datenschutz-Skeptiker da beanstanden, ist tatsächlich etwas ganz anderes als sie glauben: Es ist nicht die Datenerfassung, -verarbeitung und -monetarisierung, sondern die teils unfaire Verteilung von $v_{data}$ an die Netzwerkteilnehmer (insbesondere die beteiligten). Das ist auch genau das Problem, was wir mit WunderPass zu lösen versuchen.

\vspace{0.2cm}

Aber zunächst einmal zurück zu unserer einleitenden Hypothese. Sie formal zu beweisen ist äußerst schwer - wenn nicht gar unmöglich. Sie ist aber - laut unserer festen Überzeugung - trotzdem wahr, was wir an folgendem vielschichtigem (zugegeben ziemlich konstruiertem) Beispiel - bestehend aus "Journeys" mehrerer User - veranschaulichen möchten.

\vspace{0.3cm}

\begin{Example}[Ökosystem von Datensätzen] 

\vspace{0.2cm}

\underline{\textbf{Setup:}}

\begin{itemize}
  \item User A plant im Zeitraum x eine Reise von Berlin nach London.
  \begin{itemize}
  	\item User A hat bereits seinen Flug bei einem Flug-Provider gebucht (z. B. EasyJet).
  	\item User A hat ebenso ein Hotel gebucht (z. B. über HRS).
  	\item User A besitzt einen Airbnb-Account und hat in dem letzten Jahr bereits häufig Wohnungen im Ausland angemietet.
  \end{itemize}
  \item User B plant in demselben (oder zumindest stark überlappenden) Zeitraum x eine Reise von London nach Berlin.
  \begin{itemize}
  	\item User B hat ebenso seinen Flug gebucht - und zwar bei demselben Flug-Provider  wie User A.
  	\item User B hat für den Zeitraum seiner Abwesenheit seine Wohnung in London bei Airbnb zur Vermietung eingestellt, jedoch bisher kein Angebot erhalten.
  	\item User B hat für seinen Aufenthalt in Berlin ein Auto bei einem 
  	\newline Autovermietung-Provider (z. B. Sixt) reserviert.
  	\item User B scheint keinen Account bei Providern privaten Car-Sharings (wie Drivy) zu besitzen.
  \end{itemize}
  \item User C wohnt in Berlin, besitzt ein Auto, welches er im Zeitraum x (oder einem überlappenden Zeitraum) nicht benötigt, und es deshalb bei einem Provider von privatem Car-Sharing (z. B. Drivy) zur Vermietung angeboten, ohne jedoch bisher ein Angebot erhalten zu haben.
\end{itemize}

\vspace{0.3cm}

\underline{\textbf{Informationsgehalt \& -value:}}

\vspace{0.2cm}

Wir wollen an dieser Stelle den gänzlich offensichtlichen (und tendenziell isolierten) Informationsgehalt/Datenverarbeitung - wie z. B. Reservierungsbestätigungen, Rechnungen oder schlichtweg zusammenfassende "Reminder" - der im obigen Setup-Kontext stehenden Daten ignorieren und diese stattdessen in einem deutlich stärker "rausgezoomten" und übergeordnetem Kontext betrachten und auf mögliche Synergien auswerten.

\vspace{0.1cm} 

Im Folgenden eine punktuelle Zusammenfassung der relevanten Informationen bzw. vorliegenden Datensätzen unseres Beispiel-Szenarios - teils samt erfolgter Interpretation:

\vspace{0.2cm}

\begin{tabular}[h]{|c|c|c|c|c}
\hline
\textbf{insight} & \textbf{Information} & \textbf{time} & \textbf{data owner} \\
\hline
\textbf{info 1} & [Berlin $\rightarrow$ London] zu Zeitraum x & $x$ & EasyJet und User A \\
\hline
\textbf{info 2} & [London $\rightarrow$ Berlin] zu Zeitraum x & $x$ & EasyJet und User B \\
\hline
\textbf{info 3} & User A benötigt Unterkunft in London & $x$ & \parbox{3.5cm}{1st: User A \\ 2nd: EasyJet \& HRS} \\
\hline
\textbf{info 4} & User A hat Unterkunft in London & $x$ & HRS und User A \\
\hline
\textbf{info 5} & \parbox{5.7cm}{User A hätte theoretisch Interesse \\ an Airbnb- Wohnung in London} & $x$ & Airbnb und User A \\
\hline
\textbf{info 6} & \parbox{5.5cm}{User B sucht einen Mieter \\ für Wohnung in London} & $\approx x$ & Airbnb und User B \\
\hline
\textbf{info 7} & User B benötigt ein Auto in Berlin & $\approx x$ & Sixt und User B \\
\hline
\textbf{info 8} & \parbox{5.5cm}{User C möchte sein Auto \\ in Berlin vermieten} & $\approx x$ & Drivy und User C \\
\hline
\end{tabular}\vspace*{0.3cm}\\*

\vspace{0.3cm}

Aus obiger Auflistung wird bereits ersichtlich, worauf wir hier eigentlich hinauswollen: Nämlich die offensichtliche Tatsache, die tatsächliche "Journey" weiche möglicherweise stark von der optimalen "Journey" (optimal im Sinne der Gesamtheit aller betroffenen Teilnehmer unseres Beispiel-Cases) ab, weil kein \textit{vollumfängliches Wissen aller beteiligten Teilnehmer über alle Gegebenheiten besteht}. Das Problem hierbei ist schlichtweg die Tatsache, dass oben aufgezählte \textit{Insights} nur einigen der Teilnehmer bekennt sind, jedoch auch andere Teilnehmer betreffen. Wir können hierbei von \textit{Informations-Vor- und nachteilen} bestimmter Teilnehmer sprechen.

\vspace{0.2cm}

Angenommen obige \textit{Insights} lägen allen Teilnehmern vor. Dann ergäben sich folgende zusätzliche \textit{Insights}:

\vspace{0.2cm}

\begin{tabular}[h]{|c|c|c|c}
\hline
\textbf{insight} & \textbf{Information} & \textbf{owner} \\
\hline
\textbf{info 9} & User A könnte Wohnung von User B in London mieten & "Gott" \\
\hline
\textbf{info 10} & \parbox{10cm}{\textbf{gegeben Info 9:} \\ (1) Stornierungsrisiko der HRS-Buchung seitens User A  \\ (2) HRS könnte u. U. das Zimmer von User A gewinnbringender weitervermieten (bei großer Nachfrage)} & "Gott" \\
\hline
\textbf{info 11} & User B könnte den Wagen von User C in Berlin anmieten & "Gott" \\
\hline
\textbf{info 12} & \parbox{10cm}{\textbf{gegeben Info 11:} \\ (1) Stornierungsrisiko der Sixt-Reservierung seitens User B  \\ (2) Sixt könnte u. U. den reservierten Wagen von User B gewinnbringender vermieten (bei großer Nachfrage)} & "Gott" \\
\hline
\end{tabular}\vspace*{0.3cm}\\*

\vspace{0.2cm}

Zusammenfassend stellen wir den bisher aufgearbeiteten Informationsgehalt pro betroffenen Teilnehmer auf.

\vspace{0.1cm}

Legende: 

User A sei abgekürzt mit \textcolor{red}{\textbf{A}}, User B mit \textcolor{green}{\textbf{B}} und User C mit \textcolor{blue}{\textbf{C}}.

EasyJet sei mit \textcolor{violet}{\textbf{EJ}}, HRS mit \textcolor{dunkelgruen}{\textbf{HRS}}, Airbnb mit \textcolor{cyan}{\textbf{ABN}}, Sixt mit \textcolor{gray}{\textbf{SX}} und Drivy mit \textcolor{orange}{\textbf{DRV}}.

\vspace{0.2cm}

\begin{tabular}[h]{|c|c|c|c|c|c|c|c|c|c|c}
\hline
\textbf{info} & \textbf{owner} & \textcolor{red}{\textbf{A}} & \textcolor{green}{\textbf{B}} & \textcolor{blue}{\textbf{C}} & \textcolor{violet}{\textbf{EJ}} & \textcolor{dunkelgruen}{\textbf{HRS}} & \textcolor{cyan}{\textbf{ABN}} & \textcolor{gray}{\textbf{SX}} & \textcolor{orange}{\textbf{DRV}} \\
\hline
\textbf{1} & \textcolor{red}{\textbf{A}} + \textcolor{violet}{\textbf{EJ}} & $\surd$ & \textcolor{dunkelgruen}{+} & (o) & $\surd$ & \textcolor{dunkelgruen}{+} & \textcolor{dunkelgruen}{+} & \textcolor{dunkelgruen}{+} & \textcolor{dunkelgruen}{+} \\
\hline
\textbf{2} & \textcolor{green}{\textbf{B}} + \textcolor{violet}{\textbf{EJ}} & (o) & $\surd$ & \textcolor{dunkelgruen}{+} & $\surd$ & \textcolor{dunkelgruen}{+} & \textcolor{dunkelgruen}{+} & \textcolor{dunkelgruen}{+} & \textcolor{dunkelgruen}{+} \\
\hline
\textbf{3} & \parbox{1.8cm}{\textcolor{red}{\textbf{A}} + evtl. \\ (\textcolor{violet}{\textbf{EJ}}+\textcolor{dunkelgruen}{\textbf{HRS}})} & $\surd$ & \textcolor{dunkelgruen}{++} & (o) & (o) & \textcolor{dunkelgruen}{++} & \textcolor{dunkelgruen}{++} & (o) & (o) \\
\hline
\textbf{4} & \textcolor{red}{\textbf{A}} + \textcolor{dunkelgruen}{\textbf{HRS}} & $\surd$ & ? & (o) & (o) & $\surd$ & ? & (o) & (o) \\
\hline
\textbf{5} & \textcolor{red}{\textbf{A}} + \textcolor{cyan}{\textbf{ABN}} & $\surd$ & \textcolor{dunkelgruen}{++} & (o) & (o) & ? & $\surd$ & (o) & (o) \\
\hline
\textbf{6} & \textcolor{green}{\textbf{B}} + \textcolor{cyan}{\textbf{ABN}} & \textcolor{dunkelgruen}{++} & $\surd$ & (o) & (o) & (o) & $\surd$ & (o) & (o) \\
\hline
\textbf{7} & \textcolor{green}{\textbf{B}} + \textcolor{gray}{\textbf{SX}} & (o) & $\surd$ & \textcolor{dunkelgruen}{++} & (o) & (o) & (o) & $\surd$ & \textcolor{dunkelgruen}{++} \\
\hline
\textbf{8} & \textcolor{blue}{\textbf{C}} + \textcolor{orange}{\textbf{DRV}} & (o) & \textcolor{dunkelgruen}{++} & $\surd$ & (o) & (o) & (o) & (o) & $\surd$ \\
\hline
\textbf{9} & ----- & \textcolor{dunkelgruen}{+++} & \textcolor{dunkelgruen}{+++} & (o) & (o) & \textcolor{red}{- -} & \textcolor{dunkelgruen}{++} & (o) & (o) \\
\hline
\textbf{10} & ----- & $\surd$ & $\surd$ & (o) & (o) & \textcolor{dunkelgruen}{+++} & (o) & (o) & (o) \\
\hline
\textbf{11} & ----- & (o) & \textcolor{dunkelgruen}{+++} & \textcolor{dunkelgruen}{+++} & (o) & (o) & (o) & \textcolor{red}{- -} & \textcolor{dunkelgruen}{++} \\
\hline
\textbf{12} & ----- & (o) & $\surd$ & $\surd$ & (o) & (o) & (o) & \textcolor{dunkelgruen}{+++} & (o) \\
\hline
\end{tabular}\vspace*{0.3cm}\\*

\vspace{0.2cm}

\todo{Interpretation Verlierer/Gewinner der zusätzlichen Insights}:

\begin{itemize}
  \item Info 9 ist zwar absolut nicht im Sinne von HRS, kann HRS jedoch das Bekanntwerden dieser Info nicht verhindern, bekommt Info 10 für sie an signifikanter Relevanz (Value).
  \item Gleiches gilt für die Infos 11 und 12 aus Sicht von Sixt. 
\end{itemize}

\vspace{0.2cm}
\todo{Gesamt-Value mit zusätzlichen Insights vs. ohne}

\end{Example}

\vspace{0.3cm}

\todo{WIP}

\vspace{0.3cm}

\begin{Fazit}[unser Ökosystem generiert Value]

\begin{itemize}
  \item Wir schöpfen Mehrwert, indem wir Datenerfassung ermöglichen (die ja einen nachgewiesenen Value besitzen. \todo{Beispiele für Value durch Querverweise}
  \item Besitzer der Daten werden entlohnt.
  \item Nutzer der Daten zahlen für Daten, generieren damit aber Value, der wiederum entlohnt wird.
  \item Am Ende haben alle Teilnehmer entweder Value generiert oder aber im Wert des values verkonsumiert
  \item Wir partizipieren am extrinsischen Wert des Tokens (Kurs-Entwicklung durch positive Wertschöpfung des gesamten Ökosystems).
  \item Incentives sind nötig, um das Henne-Ei-Problem zu lösen
  \item Incentives sollten nachträglich mit der dadurch geschaffenen Wertschöpfung verrechtet werden. 
\end{itemize}

\end{Fazit}

\vspace{0.3cm}

\todo{TODO}