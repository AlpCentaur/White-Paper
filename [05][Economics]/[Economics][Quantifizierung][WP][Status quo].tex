% !TEX root = paper.tex

\paragraph{Status quo} 
\label{sec:eco_zahlen_zustand_wp_now}
\textrm{ }

\vspace{0.3cm}

Aufbauend auf die bisher erzielten Ergebnisse, wollen wir nun auch dem Stand von WunderPass für einen beliebigen Zeitpunkt $t \in T$ einen formalisierten Charakter verleihen und definieren zunächst einmal mittels der in Def \ref{defKoeff} beschriebenen Koeffizienten $\alpha^{(t)}_{ij}$ die sogenannten "connected Pools" von Usern und Service-Providern zum Zeitpunkt $t \in T$:

\vspace{0.3cm}

\begin{Def}\label{defPools}

Wir definieren den "connected User-Pool" $\widehat{U}^{(t)} \subseteq U^{(t)}$ und den "connected Service-Provider-Pool" $\widehat{S}^{(t)} \subseteq S^{(t)}$ als

\begin{align}
\widehat{U}^{(t)}:&= \left\{u \in U^{(t)} \mid \exists s^{*} \in S^{(t)} \textrm{ mit } \alpha^{(t)}(u, s^{*}) = 1 \right\} \tag{i} \\ 
\widehat{S}^{(t)}:&= \left\{s \in S^{(t)} \mid \exists u^{*} \in U^{(t)} \textrm{ mit } \alpha^{(t)}(u^{*}, s) = 1 \right\} \tag{ii}
\end{align}

\vspace{0.3cm}

Für die diskrete/sortierte Variante ist dies wieder gleichbedeutend mit
\begin{align}
\widehat{U}^{(t)} &= \left\{ \widehat{u}^{(t)}_1; \widehat{u}^{(t)}_2;...; \widehat{u}^{(t)}_{\widehat{n}} \right\} \tag{iii} \\ 
\widehat{S}^{(t)} &= \left\{ \widehat{s}^{(t)}_1; \widehat{s}^{(t)}_2;...; \widehat{s}^{(t)}_{\widehat{m}}\right\} \tag{iv}
\end{align}

\vspace{0.6cm}

Der Wert $\widehat{n} \leq n$ beschreibt die Größe des connecteten User-Pools - also die Anzahl $\widehat{n}$ der tatsächlich mit WunderPass connecteten User unter den $n$ potenziellen Usern. Analog steht $\widehat{m} \leq m$ für die Anzahl der tatsächlich mit WunderPass connecteten Providern. Der Vollständigkeit halber übertragen wir das aus Def \ref{defKoeff} stammende Verständnis der Connection-Koeffizienten auch auf die eben definierten "connected Pools"

\[
\widehat{\alpha}^{(t)}_{ij}:=\left\{%
\begin{array}{ll}
    1, & \hbox{falls User $\widehat{u}^{(t)}_i \in \widehat{U}^{(t)}$ mit mit Provider $\widehat{s}^{(t)}_j \in \widehat{S}^{(t)}$ connectet ist} \\
    0, & \hbox{andernfalls} \\
\end{array}%
\right. \tag{v}
\] 

\end{Def}

\vspace{0.6cm}

Man beachte bei den diskreten/sortierten Schreibweisen der definierten Mengen $U^{(t)}$, $\widehat{U}^{(t)}$, $S^{(t)}$ und $\widehat{S}^{(t)}$, dass in aller Regel $u^{(t)}_i \neq \widehat{u}^{(t)}_i$ und $s^{(t)}_j \neq \widehat{s}^{(t)}_j$ gelten. Die sich teils trivial aus den letzten Definitionen ergebenden Zusammenhänge fallen wir in Form eines Theorems zusammen:

\vspace{0.3cm}

% !TEX root = [Economics][Quantifizierung][WP][Status quo][theoremPools].tex

\begin{Theorem}\label{theoremPools}

Seien $n = |U^{(t)}|$ und $m = |S^{(t)}|$ bzw. $(n, m) = dP^{(t)}$. Dann gelten folgende Aussagen:

\begin{align}
&\widehat{n} \leq n \tag{i.u} \\ 
&\widehat{m} \leq m \tag{i.s} \\
&\widehat{n} = n \Leftrightarrow \widehat{U}^{(t)} = U^{(t)} \tag{ii.u} \\
&\widehat{m} = m \Leftrightarrow \widehat{S}^{(t)} = S^{(t)} \tag{ii.s} \\
&\widehat{n} * \widehat{m} > 0 \Leftrightarrow \widehat{U}^{(t)} \neq \emptyset \Leftrightarrow \widehat{S}^{(t)} \neq \emptyset \tag{iii} \\
&\widehat{n} * \widehat{m} = 0 \Leftrightarrow \widehat{U}^{(t)} = \emptyset = \widehat{S}^{(t)} \tag{iv} \\
&\widehat{u}^{(t)} \in \widehat{U}^{(t)} \Leftrightarrow \exists \widehat{s} \in \widehat{S}^{(t)} \textrm{ mit } \alpha^{(t)}\left(\widehat{u}, \widehat{s}\right) = 1 \tag{v} \\
&\widehat{s}^{(t)} \in \widehat{S}^{(t)} \Leftrightarrow \exists \widehat{u} \in \widehat{U}^{(t)} \textrm{ mit } \alpha^{(t)}\left(\widehat{u}, \widehat{s}\right) = 1 \tag{vi} \label{theoremPools_6}
\end{align}

\end{Theorem}

\vspace{0.3cm}

\begin{proof}[Beweis] \textrm{ }

\vspace{0.3cm}

(i) und (ii) sind (in jeweils beiden Varianten) trivial!

\vspace{0.3cm}

zu (iii): Zunächst einmal ist

\begin{align*}
\widehat{n} * \widehat{m} > 0 &\Leftrightarrow \widehat{n}, \widehat{m} > 0 \\
&\Leftrightarrow |\widehat{U}^{(t)}|, |\widehat{S}^{(t)}| > 0 \\
&\Leftrightarrow \widehat{U}^{(t)}, \widehat{S}^{(t)} \neq \emptyset
\end{align*}

\vspace{0.3cm}

Es bleibt also nur noch $\widehat{U}^{(t)} \neq \emptyset \Leftrightarrow \widehat{S}^{(t)} \neq \emptyset$ zu beweisen. Wir beschränken uns hierbei lediglich auf "$\Rightarrow$". Die Rückrichtung erfolgt gänzlich analog. Sei also $\widehat{U}^{(t)} \neq \emptyset$.

\begin{align*}
\widehat{U}^{(t)} \neq \emptyset &\Rightarrow \exists u^{*} \in \widehat{U}^{(t)} \\
&\xRightarrow{Def \ref{defPools}} \exists s^{*} \in S^{(t)} \textrm{ mit } \alpha^{(t)}(u^{*}, s^{*}) = 1 \\
&\Rightarrow s^{*} \in \widehat{S}^{(t)} \\
&\Rightarrow \widehat{S}^{(t)} \neq \emptyset
\end{align*}

\vspace{0.3cm}

zu (iv): 
"$\Leftarrow$" ist gänzlich trivial. Die Richtung "$\Rightarrow$" folgt dagegen aus

\begin{align*}
\widehat{n} * \widehat{m} = 0 &\Rightarrow \textrm{ mindestens eine der Mengen } \widehat{U}^{(t)}, \widehat{S}^{(t)} \textrm{ ist leer } \\
&\xRightarrow{(iii)} \widehat{U}^{(t)}, \widehat{S}^{(t)} = \emptyset
\end{align*}

\vspace{0.3cm}

zu (v): Die Richtung "$\Leftarrow$" folgt trivial aus Def \ref{defPools} und $\widehat{s} \in \widehat{S}^{(t)} \subseteq {S}^{(t)}$.

Für "$\Rightarrow$" mögen wir annehmen 

\begin{equation*}
  \exists u^{*} \in \widehat{U}^{(t)} \textrm{ mit } \forall \widehat{s} \in \widehat{S}^{(t)} \textrm{ gilt } \alpha^{(t)}(u^{*}, \widehat{s}) = 0
\end{equation*}
Da jedoch laut Annahme $u^{*} \in \widehat{U}^{(t)}$, muss aufgrund von Def \ref{defPools} ein $s^{*} \in S^{(t)} \setminus \widehat{S}^{(t)}$ mit $\alpha^{(t)}(u^{*}, s^{*}) = 1$ existieren. Da $u^{*} \in \widehat{U}^{(t)} \subseteq {U}^{(t)}$, muss $s^{*}$ jedoch laut Def \ref{defPools} auch in $\widehat{S}^{(t)}$ liegen. Im Widerspruch zu $s^{*} \in S^{(t)} \setminus \widehat{S}^{(t)}$.

\vspace{0.3cm}

Aussage (vi) ergibt sich ganz analog zu (v)!
  
\end{proof}
\vspace{0.3cm}

    % binde die Datei '[Economics][Quantifizierung][WP][Status quo][theoremPools].tex' ein


Es ist klar, dass WunderPass sich in gewisser Weise an den definierten numerischen Messgrößen ihrer angebundenen Teilnehmer $\widehat{n}$ und $\widehat{m}$ messen können wird. Zusätzlich dazu möchten wir ein - womöglich deutlich relevanteres - numerisches Maß formalisieren. Nämlich die intuitive und sehr simple KPI "Gesamtzahl bestehender User-to-Provider-Connections" zum Zeitpunkt $t \in T$.

\vspace{0.3cm}

\begin{Def}\label{defGamma}

\begin{equation*}
  \Gamma : T \rightarrow \mathbb{N} 
\end{equation*}

\begin{equation*}
  \Gamma(t):= \sum_{i=1}^n \sum_{j=1}^m \alpha^{(t)}_{ij} \textrm{ mit } (n, m) = \left(n^{(t)}, m^{(t)}\right) = dP^{(t)}
\end{equation*}

\end{Def}

\vspace{0.6cm}

Nun beweisen wir folgende sich ergebende Zusammenhänge:

\begin{Theorem}\label{theremConnectionsCount}

Sei $t \in T$ ein beliebiger Zeitpunkt, $(n, m) = dP^{(t)}$ und $\widehat{U}^{(t)} = \left\{ \widehat{u}^{(t)}_1; \widehat{u}^{(t)}_2;...; \widehat{u}^{(t)}_{\widehat{n}} \right\}$ und $\widehat{S}^{(t)} = \left\{ \widehat{s}^{(t)}_1; \widehat{s}^{(t)}_2;...; \widehat{s}^{(t)}_{\widehat{m}} \right\}$ die connecteten Teilnehmer-Pools mit $\widehat{n} < n$ sowie $\widehat{m} < m$.

Zudem soll in Anlehnung an Annahme \ref{assumptionRatio} $\widehat{m} << \widehat{n}$ gelten. Dann gelten zusätzlich auch folgende Aussagen:

\vspace{0.3cm}

\begin{align*}
\Gamma(t)&= \sum_{i=1}^{\widehat{n}} \sum_{j=1}^{\widehat{m}} \widehat{\alpha}^{(t)}_{ij} \tag{i} \label{theremConnectionsCount_1} \\ 
\widehat{n} &\leq \Gamma(t) \leq \widehat{n} * \widehat{m} \tag{ii} \\
\widehat{n} = \Gamma(t) &\Leftrightarrow \forall i \in \left\{1;...; \widehat{n} \right\}
\textrm{ gilt } \sum_{j=1}^{\widehat{m}} \widehat{\alpha}^{(t)}_{ij} = 1 \tag{iii} \\
\Gamma(t) = \widehat{n} * \widehat{m} &\Leftrightarrow \widehat{\alpha}^{(t)}_{ij} = 1 \textrm{  } \forall i \in \left\{1;...; \widehat{n} \right\} \textrm{ und } \forall j \in \left\{1;...; \widehat{m} \right\} \tag{iv}
\end{align*}

\vspace{0.3cm}

Man beachte, Aussage (ii) impliziert insbesondere

\begin{equation*}
\Gamma(t) = 0 \Leftrightarrow \widehat{n} = 0 \Leftrightarrow \widehat{U}^{(t)} = \emptyset  \Leftrightarrow \widehat{S}^{(t)} = \emptyset
\end{equation*}

\vspace{0.3cm}

Aussage (iv) beschreibt dagegen quasi eine "\textbf{\textit{Voll-Vernetzung}}" der aktuell connecteten Teilnehmer!

\end{Theorem}

\vspace{0.3cm}


\begin{proof}[Beweis] \textrm{ }

\vspace{0.3cm}

Die Aussage (i) ist intuitiv nahezu trivial. Das explizite Vorrechnen dagegen etwas aufwendig, erfolgt aber in Grunde sehr ähnlich wie der Beweis der Aussagen (v) und (vi) des Theorems \ref{theoremPools}.

\vspace{0.4cm}

zu (ii): 

$\Gamma(t) \leq \widehat{n} * \widehat{m}$ ergibt sich aus
\begin{equation*}
  \Gamma(t) = \sum_{i=1}^{\widehat{n}} \sum_{j=1}^{\widehat{m}} \widehat{\alpha}^{(t)}_{ij} \leq \sum_{i=1}^{\widehat{n}} \sum_{j=1}^{\widehat{m}} 1 = \widehat{n} * \widehat{m}
\end{equation*}

\vspace{0.3cm}

Nun zeigen wir $\widehat{n} \leq \Gamma(t)$. Für $n = 0$ ergibt sich die Aussage aus Punkt (iv) aus Theorem \ref{theoremPools}. Sei also $n > 0$. Dann ist

\begin{align*}
\Gamma(t) &\overset{\text{(i)}}{=} \sum_{i=1}^{\widehat{n}} \sum_{j=1}^{\widehat{m}} \widehat{\alpha}^{(t)}_{ij} \\
&\overset{(Def \ref{defPools})}{\geq} \sum_{i=1}^{\widehat{n}} 1 = \widehat{n}
\end{align*}

\vspace{0.4cm}

zu (iii): 
"$\Leftarrow$" ist trivial. 

\vspace{0.4cm}

Zu "$\Rightarrow$": Sei $\widehat{n} = \Gamma(t)$. Angenommen es gäbe ein $i^{*} \in \left\{1;...; \widehat{n} \right\}$ mit $\sum_{j=1}^{\widehat{m}} \widehat{\alpha}^{(t)}_{i^{*}j} > 1$. Dann müsste es aufgrund der Annahme aber auch ein $i^{**} \in \left\{1;...; \widehat{n} \right\}$ mit $\sum_{j=1}^{\widehat{m}} \widehat{\alpha}^{(t)}_{i^{**}j} < 1$ also $\sum_{j=1}^{\widehat{m}} \widehat{\alpha}^{(t)}_{i^{**}j} = 0$ geben. In diesem Fall wäre aber $\widehat{u}^{(t)}_{i^{**}} \notin \widehat{U}^{(t)}$ und somit auch $i^{**} \notin \left\{1;...; \widehat{n} \right\}$. Widerspruch!

\vspace{0.4cm}

zu (iv): 
"$\Leftarrow$" ist wieder trivial.

\vspace{0.4cm} 

Zu "$\Rightarrow$": Es gelte also $\Gamma(t) = \widehat{n} * \widehat{m}$. Angenommen es gäbe ein $i^{*} \in \{1,...,\widehat{n}\}$ und ein $j^{*} \in \{1,...,\widehat{m}\}$, sodass $\alpha^{(t)}_{i^{*}j^{*}} = 0$. Dann wäre unter Gültigkeit der Aussage (i)

\begin{align*}
\widehat{n} * \widehat{m} &= \Gamma(t) = \sum_{i=1}^{\widehat{n}} \sum_{j=1}^{\widehat{m}} \widehat{\alpha}^{(t)}_{ij} \\
&\leq \left(\sum_{i=1}^{\widehat{n}} \sum_{j=1}^{\widehat{m}} 1 \right) - 1 = \widehat{n} * \widehat{m} - 1 < \widehat{n} * \widehat{m} \\
\end{align*}
Widerspruch!
  
\end{proof}
\vspace{0.6cm}


