\documentclass[11pt]{scrartcl}


%---------Konfiguration------


%-----Sprache und Zeichen----
\usepackage[utf8]{inputenc}
% \usepackage{ucs}
% \usepackage[T1]{fontenc}

% Zeilenumbrüche der deutschen Sprache
% \usepackage[ngerman]{babel}


%---------Farben-------------
\usepackage[RGB]{xcolor}
\definecolor{dunkelgruen}{RGB}{0 136 0}
\newcommand\todo[1]{\textcolor{red}{#1}}


%---------Links/Refs---------
\usepackage[colorlinks=true, urlcolor=blue]{hyperref}


%---------Grafiken-----------
%\usepackage{graphicx}



%---------Sonstiges----------
\usepackage{parcolumns}
\usepackage{enumitem}



%---------Mathe--------------
\usepackage{amsthm, amssymb, mathtools}
% \usepackage{amsmath, amssymb, amstext ,lipsum}



%---------Umgebungen---------
\usepackage[framemethod=tikz]{mdframed}

%---------Business---------
\mdtheorem[
  linecolor=dunkelgruen,
  frametitlefont=\sffamily\bfseries\color{white},
  frametitlebackgroundcolor=dunkelgruen,
]{Business-Def}{Definition}

\mdtheorem[
  linecolor=dunkelgruen,
  frametitlefont=\sffamily\bfseries\color{white},
  frametitlebackgroundcolor=dunkelgruen,
]{Hypothese}{Hypothese}

\mdtheorem[
  linecolor=gray,
  frametitlefont=\sffamily\bfseries\color{white},
  frametitlebackgroundcolor=gray,
]{Praemisse}{Prämisse}

\mdtheorem[
  linecolor=violet,
  frametitlefont=\sffamily\bfseries\color{white},
  frametitlebackgroundcolor=violet,
]{Quelle}{Quellen}

\mdtheorem[
  linecolor=violet,
  frametitlefont=\sffamily\bfseries\color{white},
  frametitlebackgroundcolor=violet,
]{Zitat}{Zitat}

\mdtheorem[
  linecolor=blue,
  frametitlefont=\sffamily\bfseries\color{white},
  frametitlebackgroundcolor=blue,
]{Fazit}{Conclusion}

\mdtheorem[
  linecolor=red,
  frametitlefont=\sffamily\bfseries\color{white},
  frametitlebackgroundcolor=red,
]{Problem}{Problem}

\mdtheorem[
  linecolor=cyan,
  frametitlefont=\sffamily\bfseries\color{black},
  frametitlebackgroundcolor=cyan,
]{Solution}{Lösung}


\mdtheorem[
  linecolor=dunkelgruen,
  frametitlefont=\sffamily\bfseries\color{white},
  frametitlebackgroundcolor=dunkelgruen,
]{NFT-Prop}{NFT-Property}



%---------Mathe------------
\mdtheorem[
  linecolor=gray,
  frametitlefont=\sffamily\bfseries\color{white},
  frametitlebackgroundcolor=gray,
]{Def}{Definition}

\mdtheorem[
  linecolor=dunkelgruen,
  frametitlefont=\sffamily\bfseries\color{white},
  frametitlebackgroundcolor=dunkelgruen,
]{Theorem}{Theorem}

\mdtheorem[
  linecolor=blue,
  frametitlefont=\sffamily\bfseries\color{white},
  frametitlebackgroundcolor=blue,
]{Lemma}{Lemma}

\mdtheorem[
  linecolor=red,
  frametitlefont=\sffamily\bfseries\color{white},
  frametitlebackgroundcolor=red,
]{Assumption}{Annahme}

\mdtheorem[
  linecolor=violet,
  frametitlefont=\sffamily\bfseries\color{white},
  frametitlebackgroundcolor=violet,
]{Example}{Beispiel}

\mdtheorem[
  linecolor=cyan,
  frametitlefont=\sffamily\bfseries\color{black},
  frametitlebackgroundcolor=cyan,
]{Algo}{Algorithmus}




%---------Backup-------------



% Zähler:

% Eigenen Zähler erzeugen
%\newcounter{counter}
%\newcounter{sub_counter}[counter]

%Zähler initialisieren
%\setcounter{counter}{0}
%\setcounter{sub_counter}{0}

% Equationumgebung auf den Zähler umdefinieren
% \arabic sorgt für die Nummerierung mit arabischen Zahlen. 
% Alternativ wäre auch \roman für kleine römische Zahlen oder 
% \Roman für große römische Zahlen denkbar. 
% Auch Buchstaben sind mit \alph und \Alph möglich.
% \renewcommand{\theequation}{\roman{mycount}}

%\newcommand{\uproman}[1]{\uppercase\expandafter{\romannumeral#1}}
%\newcommand{\lowroman}[1]{\romannumeral#1\relax}
%\renewcommand{\theequation}{\roman{equation}}
%\renewcommand{\proofname}{Beweis}




% \section
% \subsection
% \subsubsection 
% \paragraph{Einleitende Worte}
% \subparagraph





 
%---------Dokument-----------

%---------Titel--------------
\title{WunderPass-White-Paper}
\author{G. Fricke, S.Tschurilin}
\date{\today{}, Berlin}
 
%---------Inhalt-------------
\begin{document}

\maketitle
\tableofcontents{}

% !TEX root = paper.tex

\section{Abstract}
\label{sec:abstract}
\todo{TODO: Abstract}    % binde die Datei 'Abstract.tex' ein
% !TEX root = paper.tex

\section{Einleitung}
\label{sec:einleitung}

\input{[02][Einleitung]/[Einleitung][Identität]}    % binde die Datei '[Einleitung][Identität].tex' ein
\input{[02][Einleitung]/[Einleitung][digitale Identität]}    % binde die Datei '[Einleitung][digitale Identität].tex' ein
% !TEX root = paper.tex

\subsection{Missstände der digitalen Identität}
\label{sec:einleitung_probleme_digitaler_identitaet}

\vspace{0.3cm}

Das im letzte Kapitel beleuchtete Verständnis der \textit{digitalen Identität} lässt bereits erahnen, dieses sei alles andere als optimal. Nicht aus technischer Sicht, nicht aus gesetzlicher Sicht und schon gar nicht aus Sicht des Anwenders. Profitierende Akteure des Status quo in diesem Kontext, sind bestenfalls diejenigen, die sich aufgrund einer etwaigen Vormachtstellung an Ineffizienzen des Gesamtsystems bereichern können, weil sie eben weniger Nachteile durch besagte Ineffizienzen erfahren als der restliche Markt. Also Google, Apple, Amazon, Facebook etc. Nur darf der Umstand, die größten Player da draußen, haben gar kein eigenes Interesse daran, das aktuelle Verständnis der \textit{digitalen Identität} (öffentlich) zu hinterfragen, nicht darüber hinwegtäuschen, das dieses tatsächlich alles andere als optimal und sehr wohl zu hinterfragen sei.

Dies liegt in erster Linie daran, dass die Einsicht zur Notwendigkeit einer sauberen Spezifikation der digitalen Identität erst viel später reifte, als ihre praktische Notwendigkeit. Spätestens mit dem massentauglichen Vormarsch des Web 2.0, mussten von so gut wie jedem Online-Dienst Userdaten modelliert werden. Da wären Gedanken, wie wir diese hier anstellen, hellseherisch gewesen. Die heutigen Definitionen \ref{defDigIdentity} und \ref{defAccount} entstanden also aus damaliger Sicht "by doing" und nicht etwa aus (dummen) Überlegungen.

\vspace{0.1cm}

Denn für Anbieter von Online-Diensten ist es schier unabdingbar, Daten des Users - also zumindest einen Teil der \textit{Identität} - zu erfassen: Sei es 

\begin{itemize}
  \item im Falle eines Versandhandels: \textbf{die Lieferadresse}
  \item im Falle der Absicherung gegenüber Jugendlichen: \textbf{die Altersfreigabe}
  \item im Falle von Entgeltforderungen: \textbf{Konto- oder Kreditkartendaten}
\end{itemize}
Auch die für das Marketing Verantwortlichen eines solchen Anbieters sind vielmals an einem \textbf{registrierten und wiedererkennbaren Kunden} und an dessen Kaufverhalten interessiert. [Der letzte Absatz folgte vielen Formulierungen der \href{https://vsis-www.informatik.uni-hamburg.de/getDoc.php/thesis/47/DA_Gordian_Kaulbarsch.pdf}{Diplomarbeit "Identitäten und ihre Schnittstellen auf Basis von Ontologien in einer dezentralen Umgebung"}]. 

\vspace{0.3cm}

Aber die ebenso suboptimale Fortentwicklung der eher ungesteuert geborenen \textit{digitalen Identität} blieb fortan nicht nur dem "Ist-Eben-So-Gewachsen" geschuldet.

Im Gegensatz zum User war es für den Dienstanbieter meist interessanter, \textbf{Informationen über die Nutzer an zentraler Stelle vorzuhalten}, deren Kontrolle ihm selbst oblag. Denn besagte Datenerfassung - gegeben durch freiwilligen oder gar erzwungen durch verpflichtend eingeforderten Daten-Input seitens des Anwenders - ermöglichte dem Dienstanbieter die Wiedererkennung und Verfolgung des Users, bzw. das Speichern und Auslesen von identifizierenden Dateien – sogenannten Cookies – und die Vergabe von zusätzlich identifizierenden Session-IDs. Auf diese Weise ließen und lassen sich heute noch extrem große Mengen an Daten erfassen, verknüpfen und systematisch auswerten. [Der letzte Absatz folgte vielen Formulierungen der \href{https://vsis-www.informatik.uni-hamburg.de/getDoc.php/thesis/47/DA_Gordian_Kaulbarsch.pdf}{Diplomarbeit "Identitäten und ihre Schnittstellen auf Basis von Ontologien in einer dezentralen Umgebung"}].

\vspace{0.3cm}

Ungeachtet dessen, wem oder was die besagte suboptimale "Geburt" und Fortentwicklung der digitalen Identität geschuldet sei, wollen wir im folgenden die konkreten Probleme und Missstände dieser aufarbeiten.

\vspace{0.3cm}

\begin{Problem}[fehlende Eindeutigkeit]

Das Problem der fehlenden Eindeutigkeit der Identität in der digitalen Welt wird am besten deutlich an dem Vergleich des sprachlichen Unterschieds zwischen den beiden Begriffen \textit{"dasselbe"} und \textit{"das Gleiche"}. Während ich im REWE-Supermarkt und am EasyJet-Terminal am Flughafen dieselbe Person darstelle, bin ich beim (online) REWE-Lieferdienst und beim Buchen eines Flugtickets auf der EasyJet-Homepage - aus Sicht der beiden Dienstleister - nur der gleiche Online-Konsument. Bestenfalls ist dies überhaupt erkennbar...

Ich bin mit \textit{denselben} Personen befreundet, mit denen ich auch gleichzeitig auf WhatsApp, Facebook, LinkedIn etc. connectet, ohne dass die sichere - geschweige denn zweifellos logisch implizierte - Gewissheit besteht, dass es sich tatsächlich stets um dieselbe Person handelt. Es könnte theoretisch ja auch ein Fake-Account sein (Facebook) oder längst veraltete Telefonnummer (WhatsApp), die sich hinter der geglaubten Identität verbirgt.

Die Sicherstellung der Eindeutigkeit erfolgt stets analog: Z. B. aus einem  (plausiblen) Chat-Verlauf bei WhatsApp oder einem Foto auf Instagram, wo man selbst drauf ist, was die geglaubte Identität beweist.

\vspace{0.2cm}

Dass diese \textit{analoge Verifizierung} aber nichts taugt, zeigt spätestens das Beispiel, dass ich sowohl eine KFZ-Führerschein- als auch eine Motorboots-Führerschein-Identität habend, bei einem Alkohol-Vergehen - was gesetzlich beide Identitäten beträfe - nur an derjenigen Identität belangt werde, die im direkten Zusammenhang mit dem Vergehen stand. Weil es eben oft bürokratisch und schwierig ist zwei \textit{gleiche} Datensätze aus unterschiedlichen digitalen Systemen zu \textit{derselben} Person zusammenzuführen. Weil eben Gleichheit keine Eindeutigkeit garantiert.

\vspace{0.2cm}

Verkörpert wird das Problem der fehlenden Eindeutigkeit in der digitalen Welt durch den sogenannten "Sign-Up", wo ich mich mal mit meiner Email-Adresse, mal mit meiner Telefonnummer, mal mit einem frei wählbaren Nickname und mal mit Google oder Facebook registrieren kann.

\end{Problem}

\vspace{0.3cm}


\begin{Problem}[Redundanz und fehlerbehaftete Daten]

Kann heutzutage noch irgendeiner zählen, wie oft er schon sein Email-Adresse eingeben musste, um sich irgendwo zu registrieren? Und das trotz sämtlicher Browser-Autovervollständigung. Wie oft seine Adresse bei Versandhandeln? Seine Kreditkarten-Nummer oder zumindest -CVC? Ebenso werden die meisten die Konsequenzen von Umzügen in eine neue Wohnung, den Wechsel der Telefonnummer oder den Verlust oder Ablauf einer Kreditkarte im Hinblick auf die bürokratischen Konsequenzen bei etwaigen Online-Diensten einzuordnen wissen. \textbf{Fuckup pur}.

Und das alles nur, weil unsere Daten abermals und abermals redundant von jedem Online-Service separat gespeichert werden. Ich ziehe nur einmal um, muss diese Info aber zig Mal mit Anderen teilen. ich verliere nur einmal meine Kreditkarte - und bekomme eine neue - muss dies aber an zig Stellen manuell aktualisieren. Ich wechsele meine Telefonnummer und es wird von 100 Kontakten trotzdem 10 geben, die mich deswegen nicht mehr erreichen können werden. Es wird Stellen geben, wo sich Typos in meine persönlichen Daten, meine Email-Adresse oder meine Telefonnummer einschleichen, von denen ich nichts ahne und andere Stellen, von denen ich noch nicht einmal mehr weiß, sie besäßen noch Daten von mir, die zu aktualisieren sind.

Dies ist nicht nur ein Problem beim User (Aufwand) sondern ebenso großes Problem beim Dienstleister (falsche Daten).

\end{Problem}

\vspace{0.3cm}


\begin{Problem}[mangelhafte UX]

Die heutige Existenz von zig, wenn nicht gar hunderten von Online-Accounts (= digitale Identitäten) pro User sehen wir nicht nur aufgrund der beiden eben formulierten Probleme der Uneindeutigkeit und Redundanz, sondern insbesondere auch aus User-Sicht als gänzlich unzeitgemäß und unzumutbar. Weil es eben nicht der Anspruch des digitalen Fortschritts unserer Zeit sein kann, zig und hunderte von Accounts und Passwörtern verwalten zu müssen, und diesen Missstand mit Verweis auf etwaige unterstützende Passwort-Manager auszublenden. \textbf{Was wir brauchen, sind keine Passwort-Manager oder Auto-Completion-Browser-Extensions, sondern ein grundlegendes Neudenken des digitalen Identifizierungs-Managements (Sign-up / Sign-in).}

\vspace{0.2cm}

Um den hiesigen Appell besser nachvollziehen zu können, brauchen wir einen etwas technischeren Blick auf den heute gängigen Sign-up-/Sign-in-Prozess. Für die Nutzung eines Online-Dienstes bedarf es folgender (simpler) Elemente:

\begin{itemize}
  \item Anlegen einer neuen Online-Identität beim zugehörigen Online-Dienst (\textbf{Sign-up}) als Mapping zwischen
  \begin{itemize}
  	\item technischem Identifier beim zugehörigen Online-Dienst (Kundennummer/
  	\newline Nickname/Telefonnummer/Emailadresse).
  	\item Userdaten
  \end{itemize}
  \item Identifizierung mittels Eingabe des technischen Identifier, um dem zugehörigen Online-Dienst mitzuteilen, wer man ist (Teil des \textbf{Sing-in}s).
  \item Autorisierung mittels Eingabe des persönlichen Passworts (oder auch 2FA), um dem zugehörigen Online-Dienst zu beweisen, man sei auch tatsächlich derjenige, als den man sich ausgibt (Teil des \textbf{Sing-in}s).
\end{itemize}

Alle dieser drei Elemente sind fraglos nötig (gleichwohl das erste genau genommen nur einmal universell für alle existierenden Online-Dienste nötig wäre; siehe auch die beiden oben adressierten Probleme), das Problem hier ist nur, dass hier viel zu viel manuelles Zutun vom User eingefordert wird und damit die besagte UX ruiniert. 

Dabei ist der Status quo hinsichtlich der Autorisierung bereits auf ganz gutem (UX-)Weg. Der Stand hinsichtlich der Identifizierung ist dagegen weiterhin katastrophal! Und katastrophal heterogen und uneinheitlich noch dazu.

\end{Problem}

\vspace{0.3cm}



\begin{Problem}[Datenschutz]

\todo{TODO: ausformulieren}
\begin{itemize}
  \item meine Daten liegen an zig/hunderten Stellen gespeichert
  \item Hacks sind an zig/hunderten Stellen möglich
\end{itemize}

\end{Problem}

\vspace{0.3cm}



\begin{Problem}[Daten werden nicht dort erfasst, wo sie gebraucht werden]

\todo{TODO: ausformulieren}
\begin{itemize}
  \item Daten werden an anderer Stelle erhoben als sie gebraucht werden --> Beispiel mit der Supermarkt-Kassiererin bzw. Fluggesellschaften
\end{itemize}

\end{Problem}

\vspace{0.3cm}


\begin{Problem}[Datenmissbrauch/Bereicherung]

\todo{TODO: ausformulieren}
Big Tech nutzt meine Daten, um daran Geld zu verdienen. Und ich werde nicht an der Wertschöpfung beteiligt.

\end{Problem}

\vspace{0.3cm}


\begin{Problem}[Abhängigkeit von Big Tech]

\todo{TODO: ausformulieren}
Derzeit dominieren zentrale ID-Provider wie Google und Facebook die Verwaltung von Identitätsdaten sehr vieler IT-Dienste weltweit, was zu einer großen Abhängigkeit unserer Gesellschaft in Bezug auf den Fortgang der Digitalisierung führt.

\end{Problem}

\vspace{0.3cm}


\begin{Problem}[Ungenutzte Möglichkeiten]

\todo{TODO: ausformulieren}
Daten-Querverweise $\rightarrow$ Beispiel anführen 

(zB aus \href{https://norbert-pohlmann.com/glossar-cyber-sicherheit/self-sovereign-identity-ssi/}{Vorlesung zu SSI})

\end{Problem}

\vspace{0.5cm}
    % binde die Datei '[Einleitung][Probleme].tex' ein

\vspace{0.5cm}
    % binde die Datei 'Einleitung.tex' ein
% !TEX root = paper.tex
\section{Vision}
\label{sec:vision}
\todo{TODO}    % binde die Datei 'Vision.tex' ein
% !TEX root = paper.tex
\section{Unser Ansatz}
\label{sec:ansatz}
\todo{TODO}    % binde die Datei 'Unser Ansatz.tex' ein
% !TEX root = paper.tex
\section{Economics} 
\label{sec:economics}

% !TEX root = paper.tex

\subsection{Einleitung}
\label{sec:eco_einleitung}

Wir beginnen mit einer gewagten Behauptung:

\vspace{0.2cm}

\begin{Hypothese}[Daten haben einen Wert]
\textbf{Digitale Daten besitzen einen realen} (nicht leicht zu beziffernden) \textbf{Value} - zumindest für die von ihnen direkt oder indirekt adressierten digitalen Individuen (User und Service-Provider). Dieser Value existiert bereits in Isolation des einzelnen Datensatzes, wird aber mit Zunahme der verfügbaren Gesamtdaten innerhalb eines Netzwerks durch entstehende Synergien nicht nur in Summe sondern zudem ebenfalls pro einzelnem Datensatz stets größer (Netzwerkeffekt). Eine einzelne Information - ein Datensatz - besitzt also bereits einen isolierten Mehrwert - anfangs vielleicht nur für sehr wenige Teilnehmer des Netzwerks - und gewinnt zudem zunehmend weiter an Wert mit Wachstum des "Wissens" des Gesamtnetzwerks und verhilft auch anderen Dateninformationen zu deren Wertsteigerung.

Wir behaupten damit also, jeder digitale Datensatz habe sogar einen Value über die von ihm adressierten digitalen Individuen hinaus. Und zwar für die Gesamtheit des Netzwerks und all seiner Teilnehmer. Dies jedoch natürlich nicht im gleichen Maße für alle.

\vspace{0.1cm}

Damit ist \textbf{digitale Datenerfassung und -auswertung wertschöpfend} für die gesamte digitale Welt und wünschenswert. Lediglich die \textbf{Verteilung des geschöpften Values muss hinterfragt werden}. Wir möchten ein Okösystem definieren, der genau dies gerecht und transparent tut.
 
\end{Hypothese}

\vspace{0.3cm}
Rein formal mathematisch betrachtet, ist die formulierte Behauptung ziemlich einleuchtend - zumindest wenn man auf die Forderung, dieser "Value" (sei er mit $v_{data}$ bezeichnet) habe ein positives Vorzeichen, verzichtet. Als "Value" also zunächst lediglich abstrakt einen "Impact" annimmt (der auch einen "Schaden" mit $v_{data} < 0$ darstellen könnte, wenn die Daten in irgendeiner Weise missbraucht werden). Von 

\begin{equation*}
\vert v_{data} \vert > 0
\end{equation*}

können wir also ziemlich bedenkenlos ausgehen.

\vspace{0.2cm}

Auf der gesellschaftlich/sozialen Ebene wird man dagegen deutlich mehr Widerspruch zur getätigten Hypothese ernten (bzw. von abstrakt-denken-könnenden Menschen zumindest das negative Vorzeichen von $v_{data}$ vorgehalten bekommen). Denn leider ist die elektronische Datenverarbeitung - in ihrem wahren Sinne des Wortes - ziemlich in Verruf geraten. Man hört den Tadel von mit Datenschutz in Verbindung gebrachten Missständen deutlich lauter als die Anerkennung des Nutzens von Datenerfassung und ihrer Verarbeitung - auf die im Übrigen so gut wie niemand mehr verzichten können würde. Weil Menschen eben schnell vergessen und zudem in der Regel kein ausreichendes technisches Verständnis besitzen. 

Kein Mensch würde doch heute wieder bei Taxizentralen anrufen wollen und seine Abholadresse durchgeben, weil GPS-Lokalisierungen unterbunden werden sollen. Gleiches bei Food-Lieferanten. Und auch nur die Wenigsten auf die Intelligenz von Google, weil Google nur das für uns tun kann, was sie für uns tun, weil sie das über uns wissen, was sie eben über uns wissen. Der Zug in diesem Kontext ist bereits unumkehrbar abgefahren. Weil eben selbst so gut wie in jeder Lebenssituation von dem besagten Daten-Value $v_{data}$ profitieren. Und zwar mit unbestreitbarem positiven Vorzeichen.

Was die besagten Datenschutz-Skeptiker da beanstanden, ist tatsächlich etwas ganz anderes als sie glauben: Es ist nicht die Datenerfassung, -verarbeitung und -monetarisierung, sondern die teils unfaire Verteilung von $v_{data}$ an die Netzwerkteilnehmer (insbesondere die beteiligten). Das ist auch genau das Problem, was wir mit WunderPass zu lösen versuchen.

\vspace{0.2cm}

Aber zunächst einmal zurück zu unserer einleitenden Hypothese. Sie formal zu beweisen ist äußerst schwer - wenn nicht gar unmöglich. Sie ist aber - laut unserer festen Überzeugung - trotzdem wahr, was wir an folgendem vielschichtigem (zugegeben ziemlich konstruiertem) Beispiel - bestehend aus "Journeys" mehrerer User - veranschaulichen möchten.

\vspace{0.3cm}

\begin{Example}[Ökosystem von Datensätzen] 

\vspace{0.2cm}

\underline{\textbf{Setup:}}

\begin{itemize}
  \item User A plant im Zeitraum x eine Reise von Berlin nach London.
  \begin{itemize}
  	\item User A hat bereits seinen Flug bei einem Flug-Provider gebucht (z. B. EasyJet).
  	\item User A hat ebenso ein Hotel gebucht (z. B. über HRS).
  	\item User A besitzt einen Airbnb-Account und hat in dem letzten Jahr bereits häufig Wohnungen im Ausland angemietet.
  \end{itemize}
  \item User B plant in demselben (oder zumindest stark überlappenden) Zeitraum x eine Reise von London nach Berlin.
  \begin{itemize}
  	\item User B hat ebenso seinen Flug gebucht - und zwar bei demselben Flug-Provider  wie User A.
  	\item User B hat für den Zeitraum seiner Abwesenheit seine Wohnung in London bei Airbnb zur Vermietung eingestellt, jedoch bisher kein Angebot erhalten.
  	\item User B hat für seinen Aufenthalt in Berlin ein Auto bei einem 
  	\newline Autovermietung-Provider (z. B. Sixt) reserviert.
  	\item User B scheint keinen Account bei Providern privaten Car-Sharings (wie Drivy) zu besitzen.
  \end{itemize}
  \item User C wohnt in Berlin, besitzt ein Auto, welches er im Zeitraum x (oder einem überlappenden Zeitraum) nicht benötigt, und es deshalb bei einem Provider von privatem Car-Sharing (z. B. Drivy) zur Vermietung angeboten, ohne jedoch bisher ein Angebot erhalten zu haben.
\end{itemize}

\vspace{0.3cm}

\underline{\textbf{Informationsgehalt \& -value:}}

\vspace{0.2cm}

Wir wollen an dieser Stelle den gänzlich offensichtlichen (und tendenziell isolierten) Informationsgehalt/Datenverarbeitung - wie z. B. Reservierungsbestätigungen, Rechnungen oder schlichtweg zusammenfassende "Reminder" - der im obigen Setup-Kontext stehenden Daten ignorieren und diese stattdessen in einem deutlich stärker "rausgezoomten" und übergeordnetem Kontext betrachten und auf mögliche Synergien auswerten.

\vspace{0.1cm} 

Im Folgenden eine punktuelle Zusammenfassung der relevanten Informationen bzw. vorliegenden Datensätzen unseres Beispiel-Szenarios - teils samt erfolgter Interpretation:

\vspace{0.2cm}

\begin{tabular}[h]{|c|c|c|c|c}
\hline
\textbf{insight} & \textbf{Information} & \textbf{time} & \textbf{data owner} \\
\hline
\textbf{info 1} & [Berlin $\rightarrow$ London] zu Zeitraum x & $x$ & EasyJet und User A \\
\hline
\textbf{info 2} & [London $\rightarrow$ Berlin] zu Zeitraum x & $x$ & EasyJet und User B \\
\hline
\textbf{info 3} & User A benötigt Unterkunft in London & $x$ & \parbox{3.5cm}{1st: User A \\ 2nd: EasyJet \& HRS} \\
\hline
\textbf{info 4} & User A hat Unterkunft in London & $x$ & HRS und User A \\
\hline
\textbf{info 5} & \parbox{5.7cm}{User A hätte theoretisch Interesse \\ an Airbnb- Wohnung in London} & $x$ & Airbnb und User A \\
\hline
\textbf{info 6} & \parbox{5.5cm}{User B sucht einen Mieter \\ für Wohnung in London} & $\approx x$ & Airbnb und User B \\
\hline
\textbf{info 7} & User B benötigt ein Auto in Berlin & $\approx x$ & Sixt und User B \\
\hline
\textbf{info 8} & \parbox{5.5cm}{User C möchte sein Auto \\ in Berlin vermieten} & $\approx x$ & Drivy und User C \\
\hline
\end{tabular}\vspace*{0.3cm}\\*

\vspace{0.3cm}

Aus obiger Auflistung wird bereits ersichtlich, worauf wir hier eigentlich hinauswollen: Nämlich die offensichtliche Tatsache, die tatsächliche "Journey" weiche möglicherweise stark von der optimalen "Journey" (optimal im Sinne der Gesamtheit aller betroffenen Teilnehmer unseres Beispiel-Cases) ab, weil kein \textit{vollumfängliches Wissen aller beteiligten Teilnehmer über alle Gegebenheiten besteht}. Das Problem hierbei ist schlichtweg die Tatsache, dass oben aufgezählte \textit{Insights} nur einigen der Teilnehmer bekennt sind, jedoch auch andere Teilnehmer betreffen. Wir können hierbei von \textit{Informations-Vor- und nachteilen} bestimmter Teilnehmer sprechen.

\vspace{0.2cm}

Angenommen obige \textit{Insights} lägen allen Teilnehmern vor. Dann ergäben sich folgende zusätzliche \textit{Insights}:

\vspace{0.2cm}

\begin{tabular}[h]{|c|c|c|c}
\hline
\textbf{insight} & \textbf{Information} & \textbf{owner} \\
\hline
\textbf{info 9} & User A könnte Wohnung von User B in London mieten & "Gott" \\
\hline
\textbf{info 10} & \parbox{10cm}{\textbf{gegeben Info 9:} \\ (1) Stornierungsrisiko der HRS-Buchung seitens User A  \\ (2) HRS könnte u. U. das Zimmer von User A gewinnbringender weitervermieten (bei großer Nachfrage)} & "Gott" \\
\hline
\textbf{info 11} & User B könnte den Wagen von User C in Berlin anmieten & "Gott" \\
\hline
\textbf{info 12} & \parbox{10cm}{\textbf{gegeben Info 11:} \\ (1) Stornierungsrisiko der Sixt-Reservierung seitens User B  \\ (2) Sixt könnte u. U. den reservierten Wagen von User B gewinnbringender vermieten (bei großer Nachfrage)} & "Gott" \\
\hline
\end{tabular}\vspace*{0.3cm}\\*

\vspace{0.2cm}

Zusammenfassend stellen wir den bisher aufgearbeiteten Informationsgehalt pro betroffenen Teilnehmer auf.

\vspace{0.1cm}

Legende: 

User A sei abgekürzt mit \textcolor{red}{\textbf{A}}, User B mit \textcolor{green}{\textbf{B}} und User C mit \textcolor{blue}{\textbf{C}}.

EasyJet sei mit \textcolor{violet}{\textbf{EJ}}, HRS mit \textcolor{dunkelgruen}{\textbf{HRS}}, Airbnb mit \textcolor{cyan}{\textbf{ABN}}, Sixt mit \textcolor{gray}{\textbf{SX}} und Drivy mit \textcolor{orange}{\textbf{DRV}}.

\vspace{0.2cm}

\begin{tabular}[h]{|c|c|c|c|c|c|c|c|c|c|c}
\hline
\textbf{info} & \textbf{owner} & \textcolor{red}{\textbf{A}} & \textcolor{green}{\textbf{B}} & \textcolor{blue}{\textbf{C}} & \textcolor{violet}{\textbf{EJ}} & \textcolor{dunkelgruen}{\textbf{HRS}} & \textcolor{cyan}{\textbf{ABN}} & \textcolor{gray}{\textbf{SX}} & \textcolor{orange}{\textbf{DRV}} \\
\hline
\textbf{1} & \textcolor{red}{\textbf{A}} + \textcolor{violet}{\textbf{EJ}} & $\surd$ & \textcolor{dunkelgruen}{+} & (o) & $\surd$ & \textcolor{dunkelgruen}{+} & \textcolor{dunkelgruen}{+} & \textcolor{dunkelgruen}{+} & \textcolor{dunkelgruen}{+} \\
\hline
\textbf{2} & \textcolor{green}{\textbf{B}} + \textcolor{violet}{\textbf{EJ}} & (o) & $\surd$ & \textcolor{dunkelgruen}{+} & $\surd$ & \textcolor{dunkelgruen}{+} & \textcolor{dunkelgruen}{+} & \textcolor{dunkelgruen}{+} & \textcolor{dunkelgruen}{+} \\
\hline
\textbf{3} & \parbox{1.8cm}{\textcolor{red}{\textbf{A}} + evtl. \\ (\textcolor{violet}{\textbf{EJ}}+\textcolor{dunkelgruen}{\textbf{HRS}})} & $\surd$ & \textcolor{dunkelgruen}{++} & (o) & (o) & \textcolor{dunkelgruen}{++} & \textcolor{dunkelgruen}{++} & (o) & (o) \\
\hline
\textbf{4} & \textcolor{red}{\textbf{A}} + \textcolor{dunkelgruen}{\textbf{HRS}} & $\surd$ & ? & (o) & (o) & $\surd$ & ? & (o) & (o) \\
\hline
\textbf{5} & \textcolor{red}{\textbf{A}} + \textcolor{cyan}{\textbf{ABN}} & $\surd$ & \textcolor{dunkelgruen}{++} & (o) & (o) & ? & $\surd$ & (o) & (o) \\
\hline
\textbf{6} & \textcolor{green}{\textbf{B}} + \textcolor{cyan}{\textbf{ABN}} & \textcolor{dunkelgruen}{++} & $\surd$ & (o) & (o) & (o) & $\surd$ & (o) & (o) \\
\hline
\textbf{7} & \textcolor{green}{\textbf{B}} + \textcolor{gray}{\textbf{SX}} & (o) & $\surd$ & \textcolor{dunkelgruen}{++} & (o) & (o) & (o) & $\surd$ & \textcolor{dunkelgruen}{++} \\
\hline
\textbf{8} & \textcolor{blue}{\textbf{C}} + \textcolor{orange}{\textbf{DRV}} & (o) & \textcolor{dunkelgruen}{++} & $\surd$ & (o) & (o) & (o) & (o) & $\surd$ \\
\hline
\textbf{9} & ----- & \textcolor{dunkelgruen}{+++} & \textcolor{dunkelgruen}{+++} & (o) & (o) & \textcolor{red}{- -} & \textcolor{dunkelgruen}{++} & (o) & (o) \\
\hline
\textbf{10} & ----- & $\surd$ & $\surd$ & (o) & (o) & \textcolor{dunkelgruen}{+++} & (o) & (o) & (o) \\
\hline
\textbf{11} & ----- & (o) & \textcolor{dunkelgruen}{+++} & \textcolor{dunkelgruen}{+++} & (o) & (o) & (o) & \textcolor{red}{- -} & \textcolor{dunkelgruen}{++} \\
\hline
\textbf{12} & ----- & (o) & $\surd$ & $\surd$ & (o) & (o) & (o) & \textcolor{dunkelgruen}{+++} & (o) \\
\hline
\end{tabular}\vspace*{0.3cm}\\*

\vspace{0.2cm}

\todo{Interpretation Verlierer/Gewinner der zusätzlichen Insights}:

\begin{itemize}
  \item Info 9 ist zwar absolut nicht im Sinne von HRS, kann HRS jedoch das Bekanntwerden dieser Info nicht verhindern, bekommt Info 10 für sie an signifikanter Relevanz (Value).
  \item Gleiches gilt für die Infos 11 und 12 aus Sicht von Sixt. 
\end{itemize}

\vspace{0.2cm}
\todo{Gesamt-Value mit zusätzlichen Insights vs. ohne}

\end{Example}

\vspace{0.3cm}

\todo{WIP}

\vspace{0.3cm}

\begin{Fazit}[unser Ökosystem generiert Value]

\begin{itemize}
  \item Wir schöpfen Mehrwert, indem wir Datenerfassung ermöglichen (die ja einen nachgewiesenen Value besitzen. \todo{Beispiele für Value durch Querverweise}
  \item Besitzer der Daten werden entlohnt.
  \item Nutzer der Daten zahlen für Daten, generieren damit aber Value, der wiederum entlohnt wird.
  \item Am Ende haben alle Teilnehmer entweder Value generiert oder aber im Wert des values verkonsumiert
  \item Wir partizipieren am extrinsischen Wert des Tokens (Kurs-Entwicklung durch positive Wertschöpfung des gesamten Ökosystems).
  \item Incentives sind nötig, um das Henne-Ei-Problem zu lösen
  \item Incentives sollten nachträglich mit der dadurch geschaffenen Wertschöpfung verrechtet werden. 
\end{itemize}

\end{Fazit}

\vspace{0.3cm}

\todo{TODO}    % binde die Datei '[Economics][Einleitung].tex' ein
% !TEX root = paper.tex

\subsection{Goals}
\label{sec:eco_goals}
\todo{TODO}    % binde die Datei '[Economics][Goals].tex' ein
% !TEX root = paper.tex
\subsection{Quantifizierung}
\label{sec:eco_zahlen}
\todo{Einleitung - Start}

Wir wollen den Mehrwert von User-Provider-Connections mittels Wunderpass einen bezifferbaren Mehrwert verleihen und diesen fundiert argumentieren. Dazu müssen wir diesen Value messen und beziffern können. Die Ergebnisse dieses Kapitels werden insbesondere für das im Kapitel \ref{sec:wpt_reward_pool} beleuchteten "Reward-Pools" von großer Bedeutung sein. Bzw. sogar im gesamten übergeordneten Kapitel \ref{sec:eco_wpt}.
\todo{Einleitung - Ende}

% !TEX root = [Economics][Quantifizierung].tex

\subsubsection{Grundlegende Definitionen}
\label{sec:eco_zahlen_def}

Sei $t_0$ der initiale Zeitpunkt all unserer Messungen und Betrachtungen (vermutlich der Zeitpunkt des MVP-Launches).

Darauf aufbauend betrachten wir das künftige Zeitintervall $T$, welches einzig an Relevanz für unser Vorhaben und alle in diesem Kapitel getätigten Ausführungen besitzt:

\begin{equation*}
  T = [t_0; \infty[
\end{equation*}
Der Zeitstrahl muss nicht zwingend unendlich sein. Er muss ebenfalls nicht zwingend infinitesimal fortlaufend sein und kann stattdessen je nach Kontext endlich und/oder diskret betrachtet werden. Also z. B. auch wahlweise als 

\begin{equation*}
  T = [t_0; t_{ende}]
\end{equation*}

\begin{equation*}
  T = [t_0; t_1;...; t_{ende}]
\end{equation*}
definiert sein. In letzteren beiden Fällen wird jedoch $t_{ende}$ in aller Regel eine kontextbezogene (unverzichtbare) Bedeutung haben, die eine solche Definition des Zeitstrahls unverzichtbar macht. So könnte $t_{ende}$ z. B. für eine mathematisch quantifizierbare Erreichung unserer Vision stehen. \\

Sei $\mathbf{t \in T}$ fortan stets ein beliebiger Zeitpunkt, zu welchem wir eine Aussage treffen möchten. \\


Wir definieren die Anzahl aller zum Zeitpunkt $t$ potenziellen User $U^{(t)}$ überhaupt und ihre (maximale) Anzahl $n^{(t)}$ als \\

\begin{Def}\label{defU}
\begin{equation*}
  U^{(t)} = \left\{ u^{(t)}_1; u^{(t)}_2;...; u^{(t)}_{n} \right\}
\end{equation*}
\end{Def} 

\vspace{0.3cm}


Und ganz analog dazu ebenfalls die potenziellen Service-Provider $S^{(t)}$ und ihre (maximale) Anzahl $m^{(t)}$ als \\

\begin{Def}\label{defS}
\begin{equation*}
  S^{(t)} = \left\{ s^{(t)}_1; s^{(t)}_2;...; s^{(t)}_{m}\right\}
\end{equation*}
\end{Def}

\vspace{0.3cm}

Man beachte, dass die definierten Mengen $U^{(t)}$ und $S^{(t)}$ bzw. ihre Größe gewissermaßen den Fortschritt der Digitalisierung insgesamt beschreiben (potenzielle User brauchen einen Zugang zum digitalen Ökosystem und potenzielle Provider sind unabhängige Service-Dienstleister, die eigenmächtig darüber entscheiden, zu solchen zu werden) und in keiner Weise im Einfluss Wunderpasses stehen. Viel mehr beschreiben sie die "Umstände der Welt", mit denen WunderPass (wie alle anderen) "arbeiten" müssen.  

\vspace{0.6cm}


Nun definieren den \textbf{\textit{Connection-Koeffizienten}} zwischen den eben definierten potenziellen Usern $\mathbf{U^{(t)}}$ und den Service-Providern $\mathbf{S^{(t)}}$ zum Zeitpunkt $t$ als boolesche Funktion $\mathbf{\alpha^{(t)}}$, die über über die Tatsache \textit{"is connected"} bzw. \textit{"is not connected"} entscheidet: \\

\begin{Def}\label{defKoeff}

\begin{equation*}
  \alpha^{(t)} : U^{(t)} \times S^{(t)} \rightarrow \{0; 1\} 
\end{equation*}

\[
\alpha^{(t)}(u, s):=\left\{%
\begin{array}{ll}
    1, & \hbox{falls User $u \in U^{(t)}$ mit mit Provider $s \in S^{(t)}$ connectet ist} \\
    0, & \hbox{andernfalls} \\
\end{array}%
\right.
\]

\vspace{1cm}

Bzw. wenn man die diskreten Auslegungen der Pools $U^{(t)} = \left\{ u^{(t)}_1; u^{(t)}_2;...; u^{(t)}_{n} \right\}$ und $S^{(t)} = \left\{ s^{(t)}_1; s^{(t)}_2;...; s^{(t)}_{m} \right\}$ heranzieht, alternativ als

\[
\alpha^{(t)}_{ij}:=\left\{%
\begin{array}{ll}
    1, & \hbox{falls User $u^{(t)}_i \in U^{(t)}$ mit mit Provider $s^{(t)}_j \in S^{(t)}$ connectet ist} \\
    0, & \hbox{andernfalls} \\
\end{array}%
\right.
\]

\end{Def}

\vspace{0.6cm}

Man beachte, dass wir bei den diskreten/Aufzählungs-basierten Definitionen oben, der Übersicht halber etwas "geschlampt" haben, indem wir - klar zeitbedingte - Indizes stillschweigend als $n$ und $m$ bezeichnet haben, gleichwohl diese korrekterweise $n^{(t)}$ und $m^{(t)}$ lauten müssten. Nur verwirrt eben ein Ausdruck wie $u^{(t)}_{n^{(t)}}$ mehr, als dieser in seiner pedantischen Korrektheit einen Mehrwert generiert. Wir werden genannte Ungenauigkeit zudem im weiteren Verlauf in gleicher Weise fortführen und gehen davon aus, der Leser wisse damit umzugehen. 

\vspace{0.3cm}


    % binde die Datei '[Economics][Quantifizierung][Definitionen].tex' ein

\subsubsection{Zustandsbeschreibung der digitalen Welt}
\label{sec:eco_zahlen_zustand_digitalisierung}

Mit diesen geschaffenen Formalisierungs-Werkzeugen lassen sich nun einige Dinge formal deutlich besser greifen. Und zwar zum einen im Folgenden die übergeordneten "Umstände der digitalen Welt" (auf die WunderPass bestenfalls sehr geringfügig Einfluss üben kann) aber zum anderen ebenfalls unser gesamtes Vorhaben inklusive der übergeordneten WunderPass-Vision, die in den darauf folgenden Kapitels beleuchtet wird. 

\vspace{0.3cm}

Aufgrund der bereits weiter oben erwähnten nicht möglichen Einflussnahme auf die Mengen $U^{(t)}$ und $S^{(t)}$ benötigen wir noch ein weiteres Hilfsmittel, dessen Existenz wir im Folgenden einfach voraussetzen möchten - und diese mit Möglichkeiten der Markt-Analyse rechtfertigen.

\vspace{0.3cm}

\begin{Assumption}[Digitalisierungs-Orakel]\label{assumptionOrakel}

Sei $t \in T$. Anstatt die (nicht wirklich berechtigte) Kenntnis der Mengen $U^{(t)}$ und $S^{(t)}$ vorzugeben, wollen wir lieber die (realistischere) Existenz einer "Schätzfunktion" $dP^{(t)}$ (digital progress) annehmen. Wir definieren $dP^{(t)}$ als

\begin{align*}
dP &: T \rightarrow \mathbb{N} \times \mathbb{N}  \\
dP^{(t)} &:= \left(n^{(t)}, m^{(t)}\right)
\end{align*}
wobei $n^{(t)} = |U^{(t)}|$ und $m^{(t)} = |S^{(t)}|$ darstellen sollen, ohne dafür zwingend die exakten Mengen $U^{(t)}$ und $S^{(t)}$ kennen zu müssen.

\end{Assumption}

\vspace{0.3cm}

Und auf der letzten Annahme aufbauend der Vollständigkeit halber die aus praktischer Sicht vollkommen alternativlose Annahme ergänzen, laut der Service-Provider stets eine große Anzahl an Users ansprechen/bedienen und damit zahlenmäßig den Usern stark unterlegen sind.

\vspace{0.3cm}

\todo{[TODO1][Annahme 2 ist noch buggy]}

\begin{Assumption}[Verhältnismäßigkeit der Teilnehmer]\label{assumptionRatio}
Für alle $t \in T$ mit $\left(n^{(t)}, m^{(t)}\right) = dP^{(t)}$ gilt:

\begin{equation*}
m^{(t)} << n^{(t)} \tag{i}
\end{equation*}

\vspace{0.3cm}

Diese Aussage mag zahlenmäßig noch etwas "griffiger" formuliert werden. Dafür möchten wir das Verhältnis der Größen $n^{(t)}$ und $m^{(t)}$ abschätzen: Für unseren Zeithorizont, an dessen Ende - einem ausreichend späten, aber auch nicht in unabsehbar fernen Zukunft liegenden Zeitpunkt $t_{end} \in T$ - wir von einer WunderWelt sprechen, sei die Annahme

\begin{align*}
n^{(t_{end})} &\thickapprox 10 Mrd. \textrm{ und } m^{(t_{end})} \thickapprox 10.000 \textrm{ bzw.} \\
dP^{(t_{end})} &\thickapprox (10^{10}, 10^{4}) = 10^{4} * (10^{6}, 1) \tag{ii}
\end{align*}
nicht ganz abwegig. Genauso wenig unvernünftig scheint die Annahme, WunderPass begänne seine Welteroberung mit einem MVP mit lediglich einem einzigen Service-Provider - z. B. dem Guard (siehe Kap. \ref{sec:guard}] - und einer überschaubaren Anzahl an angepeilten Usern, also

\begin{align*}
n^{(t_0)} &\thickapprox 1.000 \textrm{ und } m^{(t_0)} = 1 \textrm{ bzw.} \\
dP^{(t_0)} &\thickapprox (1.000, 1) \tag{iii}
\end{align*}

\vspace{0.3cm}

Mit den beiden zuletzt getroffenen (quantitativen) Annahmen (ii) und (iii) lässt sich auch die initiale (qualitative) Annahme (i) ebenfalls quantifizieren:

\vspace{0.3cm}

Für alle $t \in ]t_0; t_{end}[$ mit $\left(n^{(t)}, m^{(t)}\right) = dP^{(t)}$ gilt:

\begin{equation*}
1.000 = \frac{n^{(t_0)}}{m^{(t_0)}} < \frac{n^{(t)}}{m^{(t)}} < \frac{n^{(t_{end})}}{m^{(t_{end})}} = 1.000.000 \tag{iv}
\end{equation*}

\end{Assumption}

\vspace{0.3cm}

Wir fassen Annahme \ref{assumptionRatio} in einer abschließenden Definition zusammen:

\vspace{0.3cm}

\begin{Def}\label{defRatio}

Seien $t \in T$ und $dP^{(t)} = \left(n^{(t)}, m^{(t)}\right)$ wie in Annahme \ref{assumptionOrakel} beschrieben. Wie definieren die "Verhältnismäßigkeit der Teilnehmer" als

\begin{align*}
\sigma &: T \rightarrow \mathbb{Q} \\
\sigma^{(t)} &= \frac{n^{(t)}}{m^{(t)}}
\end{align*}

\vspace{0.3cm}

Zudem halten wir fest, Annahme \ref{assumptionRatio} lege nahe, man könne in der Praxis stets von

\begin{equation*}
1.000 < \sigma^{(t)} < 1.000.000
\end{equation*}
ausgehen.

\end{Def}

\todo{[ende TODO1]}

\vspace{0.6cm}



\subsubsection{Zustandsbeschreibung WunderPass - simple Betrachtung}
\label{sec:eco_zahlen_zustand_wp}

% !TEX root = paper.tex

\paragraph{Status quo} 
\label{sec:eco_zahlen_zustand_wp_now}
\textrm{ }

\vspace{0.3cm}

Aufbauend auf die bisher erzielten Ergebnisse, wollen wir nun auch dem Stand von WunderPass für einen beliebigen Zeitpunkt $t \in T$ einen formalisierten Charakter verleihen und definieren zunächst einmal mittels der in Def \ref{defKoeff} beschriebenen Koeffizienten $\alpha^{(t)}_{ij}$ die sogenannten "connected Pools" von Usern und Service-Providern zum Zeitpunkt $t \in T$:

\vspace{0.3cm}

\begin{Def}\label{defPools}

Wir definieren den "connected User-Pool" $\widehat{U}^{(t)} \subseteq U^{(t)}$ und den "connected Service-Provider-Pool" $\widehat{S}^{(t)} \subseteq S^{(t)}$ als

\begin{align}
\widehat{U}^{(t)}:&= \left\{u \in U^{(t)} \mid \exists s^{*} \in S^{(t)} \textrm{ mit } \alpha^{(t)}(u, s^{*}) = 1 \right\} \tag{i} \\ 
\widehat{S}^{(t)}:&= \left\{s \in S^{(t)} \mid \exists u^{*} \in U^{(t)} \textrm{ mit } \alpha^{(t)}(u^{*}, s) = 1 \right\} \tag{ii}
\end{align}

\vspace{0.3cm}

Für die diskrete/sortierte Variante ist dies wieder gleichbedeutend mit
\begin{align}
\widehat{U}^{(t)} &= \left\{ \widehat{u}^{(t)}_1; \widehat{u}^{(t)}_2;...; \widehat{u}^{(t)}_{\widehat{n}} \right\} \tag{iii} \\ 
\widehat{S}^{(t)} &= \left\{ \widehat{s}^{(t)}_1; \widehat{s}^{(t)}_2;...; \widehat{s}^{(t)}_{\widehat{m}}\right\} \tag{iv}
\end{align}

\vspace{0.6cm}

Der Wert $\widehat{n} \leq n$ beschreibt die Größe des connecteten User-Pools - also die Anzahl $\widehat{n}$ der tatsächlich mit WunderPass connecteten User unter den $n$ potenziellen Usern. Analog steht $\widehat{m} \leq m$ für die Anzahl der tatsächlich mit WunderPass connecteten Providern. Der Vollständigkeit halber übertragen wir das aus Def \ref{defKoeff} stammende Verständnis der Connection-Koeffizienten auch auf die eben definierten "connected Pools"

\[
\widehat{\alpha}^{(t)}_{ij}:=\left\{%
\begin{array}{ll}
    1, & \hbox{falls User $\widehat{u}^{(t)}_i \in \widehat{U}^{(t)}$ mit mit Provider $\widehat{s}^{(t)}_j \in \widehat{S}^{(t)}$ connectet ist} \\
    0, & \hbox{andernfalls} \\
\end{array}%
\right. \tag{v}
\] 

\end{Def}

\vspace{0.6cm}

Man beachte bei den diskreten/sortierten Schreibweisen der definierten Mengen $U^{(t)}$, $\widehat{U}^{(t)}$, $S^{(t)}$ und $\widehat{S}^{(t)}$, dass in aller Regel $u^{(t)}_i \neq \widehat{u}^{(t)}_i$ und $s^{(t)}_j \neq \widehat{s}^{(t)}_j$ gelten. Die sich teils trivial aus den letzten Definitionen ergebenden Zusammenhänge fallen wir in Form eines Theorems zusammen:

\vspace{0.3cm}

\input{[Economics][Quantifizierung][theoremPools]}    % binde die Datei '[Economics][Quantifizierung][theoremPools].tex' ein


Es ist klar, dass WunderPass sich in gewisser Weise an den definierten numerischen Messgrößen ihrer angebundenen Teilnehmer $\widehat{n}$ und $\widehat{m}$ messen können wird. Zusätzlich dazu möchten wir ein - womöglich deutlich relevanteres - numerisches Maß formalisieren. Nämlich die intuitive und sehr simple KPI "Gesamtzahl bestehender User-to-Provider-Connections" zum Zeitpunkt $t \in T$.

\vspace{0.3cm}

\begin{Def}\label{defGamma}

\begin{equation*}
  \Gamma : T \rightarrow \mathbb{N} 
\end{equation*}

\begin{equation*}
  \Gamma(t):= \sum_{i=1}^n \sum_{j=1}^m \alpha^{(t)}_{ij} \textrm{ mit } (n, m) = \left(n^{(t)}, m^{(t)}\right) = dP^{(t)}
\end{equation*}

\end{Def}

\vspace{0.6cm}

Nun beweisen wir folgende sich ergebende Zusammenhänge:

\begin{Theorem}\label{theremConnectionsCount}

Sei $t \in T$ ein beliebiger Zeitpunkt, $(n, m) = dP^{(t)}$ und $\widehat{U}^{(t)} = \left\{ \widehat{u}^{(t)}_1; \widehat{u}^{(t)}_2;...; \widehat{u}^{(t)}_{\widehat{n}} \right\}$ und $\widehat{S}^{(t)} = \left\{ \widehat{s}^{(t)}_1; \widehat{s}^{(t)}_2;...; \widehat{s}^{(t)}_{\widehat{m}} \right\}$ die connecteten Teilnehmer-Pools mit $\widehat{n} < n$ sowie $\widehat{m} < m$.

Zudem soll in Anlehnung an Annahme \ref{assumptionRatio} $\widehat{m} << \widehat{n}$ gelten. Dann gelten zusätzlich auch folgende Aussagen:

\vspace{0.3cm}

\begin{align*}
\Gamma(t)&= \sum_{i=1}^{\widehat{n}} \sum_{j=1}^{\widehat{m}} \widehat{\alpha}^{(t)}_{ij} \tag{i} \label{theremConnectionsCount_1} \\ 
\widehat{n} &\leq \Gamma(t) \leq \widehat{n} * \widehat{m} \tag{ii} \\
\widehat{n} = \Gamma(t) &\Leftrightarrow \forall i \in \left\{1;...; \widehat{n} \right\}
\textrm{ gilt } \sum_{j=1}^{\widehat{m}} \widehat{\alpha}^{(t)}_{ij} = 1 \tag{iii} \\
\Gamma(t) = \widehat{n} * \widehat{m} &\Leftrightarrow \widehat{\alpha}^{(t)}_{ij} = 1 \textrm{  } \forall i \in \left\{1;...; \widehat{n} \right\} \textrm{ und } \forall j \in \left\{1;...; \widehat{m} \right\} \tag{iv}
\end{align*}

\vspace{0.3cm}

Man beachte, Aussage (ii) impliziert insbesondere

\begin{equation*}
\Gamma(t) = 0 \Leftrightarrow \widehat{n} = 0 \Leftrightarrow \widehat{U}^{(t)} = \emptyset  \Leftrightarrow \widehat{S}^{(t)} = \emptyset
\end{equation*}

\vspace{0.3cm}

Aussage (iv) beschreibt dagegen quasi eine "\textbf{\textit{Voll-Vernetzung}}" der aktuell connecteten Teilnehmer!

\end{Theorem}

\vspace{0.3cm}


\begin{proof}[Beweis] \textrm{ }

\vspace{0.3cm}

Die Aussage (i) ist intuitiv nahezu trivial. Das explizite Vorrechnen dagegen etwas aufwendig, erfolgt aber in Grunde sehr ähnlich wie der Beweis der Aussagen (v) und (vi) des Theorems \ref{theoremPools}.

\vspace{0.4cm}

zu (ii): 

$\Gamma(t) \leq \widehat{n} * \widehat{m}$ ergibt sich aus
\begin{equation*}
  \Gamma(t) = \sum_{i=1}^{\widehat{n}} \sum_{j=1}^{\widehat{m}} \widehat{\alpha}^{(t)}_{ij} \leq \sum_{i=1}^{\widehat{n}} \sum_{j=1}^{\widehat{m}} 1 = \widehat{n} * \widehat{m}
\end{equation*}

\vspace{0.3cm}

Nun zeigen wir $\widehat{n} \leq \Gamma(t)$. Für $n = 0$ ergibt sich die Aussage aus Punkt (iv) aus Theorem \ref{theoremPools}. Sei also $n > 0$. Dann ist

\begin{align*}
\Gamma(t) &\overset{\text{(i)}}{=} \sum_{i=1}^{\widehat{n}} \sum_{j=1}^{\widehat{m}} \widehat{\alpha}^{(t)}_{ij} \\
&\overset{(Def \ref{defPools})}{\geq} \sum_{i=1}^{\widehat{n}} 1 = \widehat{n}
\end{align*}

\vspace{0.4cm}

zu (iii): 
"$\Leftarrow$" ist trivial. 

\vspace{0.4cm}

Zu "$\Rightarrow$": Sei $\widehat{n} = \Gamma(t)$. Angenommen es gäbe ein $i^{*} \in \left\{1;...; \widehat{n} \right\}$ mit $\sum_{j=1}^{\widehat{m}} \widehat{\alpha}^{(t)}_{i^{*}j} > 1$. Dann müsste es aufgrund der Annahme aber auch ein $i^{**} \in \left\{1;...; \widehat{n} \right\}$ mit $\sum_{j=1}^{\widehat{m}} \widehat{\alpha}^{(t)}_{i^{**}j} < 1$ also $\sum_{j=1}^{\widehat{m}} \widehat{\alpha}^{(t)}_{i^{**}j} = 0$ geben. In diesem Fall wäre aber $\widehat{u}^{(t)}_{i^{**}} \notin \widehat{U}^{(t)}$ und somit auch $i^{**} \notin \left\{1;...; \widehat{n} \right\}$. Widerspruch!

\vspace{0.4cm}

zu (iv): 
"$\Leftarrow$" ist wieder trivial.

\vspace{0.4cm} 

Zu "$\Rightarrow$": Es gelte also $\Gamma(t) = \widehat{n} * \widehat{m}$. Angenommen es gäbe ein $i^{*} \in \{1,...,\widehat{n}\}$ und ein $j^{*} \in \{1,...,\widehat{m}\}$, sodass $\alpha^{(t)}_{i^{*}j^{*}} = 0$. Dann wäre unter Gültigkeit der Aussage (i)

\begin{align*}
\widehat{n} * \widehat{m} &= \Gamma(t) = \sum_{i=1}^{\widehat{n}} \sum_{j=1}^{\widehat{m}} \widehat{\alpha}^{(t)}_{ij} \\
&\leq \left(\sum_{i=1}^{\widehat{n}} \sum_{j=1}^{\widehat{m}} 1 \right) - 1 = \widehat{n} * \widehat{m} - 1 < \widehat{n} * \widehat{m} \\
\end{align*}
Widerspruch!
  
\end{proof}
\vspace{0.6cm}


 % binde die Datei '[Economics][Quantifizierung][WP][Status quo].tex' ein

% !TEX root = paper.tex

\paragraph{Messbarkeit Status quo} 
\label{sec:eco_zahlen_zustand_wp_nowVlaue}
\textrm{ }

\vspace{0.3cm}

Kommend von der intuitiven Annahme, die Größe der definierten "connected Pools" $\widehat{U}^{(t)}$ und $\widehat{S}^{(t)}$ sei irgendwie erstrebenswert in unserem Sinne, definierten wir im vorangehenden Abschnitt das - aus unserer Sicht fundierteres und geeigneteres - Maß $\Gamma(t)$, um dem Verständnis von "erstrebenswerter Zustand" besser gerecht zu werden.

In diesem Abschnitt wollen wir die - bisher eher wertfrei/objektiv formulierten -  Ergebnisse des vorigen Abschnitts in den Kontext der "Erstrebenswertigkeit" stellen. Also eine formale und quantifizierbare Vergleichbarkeit unserer - ohnehin beim Lesen des letzten Abschnitts mitschwingender - Intuition schaffen, die Werte 
\begin{itemize}
  \item $\widehat{n} = |\widehat{U}^{(t)}|$, 
  \item $\widehat{m} = |\widehat{S}^{(t)}|$ und vor allem 
  \item $\Gamma(t)$ 
\end{itemize}
seien umso besser je größer sie seien. Alle der eben genannten Größen, denen wir hier eine intuitiv spürbare "Erstrebenswertigkeit" beimessen, besitzen einen klaren Zeitbezug. Daher überrascht es kaum, wir strebten die genannte quantifizierbare Vergleichbarkeit für je zwei beliebige Zeitpunkte $t_1, t_2 \in T$ an. Formale Vergleichbarkeit schreit nur so nach der mathematisch verstandenen "Ordnungsrelation":

\vspace{0.3cm}

\begin{Def}\label{defRelation}

Wir bedienen uns der in Definition \ref{defGamma} beschriebenen Funktion $\Gamma(t)$, um damit eine \href{https://de.wikipedia.org/wiki/Ordnungsrelation}{Ordnungsrelation} 
auf unserem Zeitstrahl $T$ für je zwei beliebige Zeitpunkte $t_1, t_2 \in T$ zu erhalten: 

\vspace{0.3cm}

\begin{equation*}
  R_{\preceq} \subseteq T \times T \textrm{ mit}
\end{equation*}

\begin{equation*}
  R_{\preceq}:= \left\{ (t_1, t_2) \in T \times T \mid \Gamma(t_1) \leq \Gamma(t_2) \right\}
\end{equation*}
\vspace{1cm}
Mittels $R_{\preceq}$ erhalten wir eine Ordnung unseres Zeitstahls $T$ und erklären zudem insbesondere, was "erstrebenswert" bedeutet. Ein beliebiger Zeitpunkt $t_1 \in T$ ist nämlich verbal genau dann "nicht weniger erstrebenswert" in Sinne unserer Vision als ein beliebiger anderer Zeitpunkt $t_2 \in T$, falls $(t_1, t_2) \in R_{\preceq}$ gilt.

\vspace{0.3cm}

\begin{equation*}
  \textrm{Wir schreiben fortan statt } (t_1, t_2) \in R_{\preceq} \textrm{ lieber } t_1 \preceq t_2 
\end{equation*}

\end{Def}

\vspace{1cm}

Man beachte, dass es sich bei der definierten Ordnungsrelation gar um eine \href{https://de.wikipedia.org/wiki/Ordnungsrelation#Totalordnung}{Totalordnung} handelt!
Der Form halber ergänzen wir an der Stelle noch um zwei weitere - schematisch induzierte - Relationen auf unserem Zeitstrahl $T$:

\vspace{0.3cm}

\begin{Def}\label{defRelationen}

Um zusätzlich zur in Def \ref{defRelation} definierten Ordnungsrelation "$\preceq$", auch dem Verständnis von "echt besser" und "gleich gut" Rechnung zu tragen, definieren wir die beiden Relationen "$\prec$" und "$\simeq$"

\vspace{0.3cm}

\begin{equation*}
  R_{\prec}:= \left\{ (t_1, t_2) \in T \times T \mid \Gamma(t_1) < \Gamma(t_2) \right\}
\end{equation*}

\begin{equation*}
  R_{\simeq}:= \left\{ (t_1, t_2) \in T \times T \mid \Gamma(t_1) = \Gamma(t_2) \right\}
\end{equation*}

\vspace{1cm}

Bei $R_{\prec}$ handelt es sich im Übrigen wieder um eine Ordnungsrelation. Bei $R_{\simeq}$ dagegen nicht.

\end{Def}

\vspace{0.3cm}

Auch für die letzten beiden Relationen wollen wir fortan die vereinfachte Schreibweise $t_1 \prec t_2$ und $t_1 \simeq t_2$ nutzen.

\vspace{0.6cm}

 % binde die Datei '[Economics][Quantifizierung][WP][Messbarkeit].tex' ein

% !TEX root = paper.tex

\paragraph{Fazit} 
\label{sec:eco_zahlen_zustand_wp_fazit}
\textrm{ }

\vspace{0.3cm}

Ungeachtet des Werts der bisher erzielten erfolgreichen Ergebnisse hinsichtlich der quantitativen Einordnung des WunderPass-Fortschritts zu einem Zeitpunkt $t \in T$, besitzt der Zusatz "...simple Betrachtung" innerhalb der Überschrift des gegenwärtigen Kapitels durchaus seine Rechtfertigung.

Wir haben zwar die Größe $\Gamma(t)$ als sehr gut geeigneten Gradmesser für den Fortschritt WunderPasses herausgearbeitet und dieses ebenfalls in Abhängigkeit der intuitiven Erfolgsmesser $\widehat{n}$ und $\widehat{m}$ gesetzt sowie nach unten und oben abgeschätzt. Jedoch scheint unser Ökosystem zu komplex und unsere bisherige Betrachtungsweise zu global geprägt, als dass wir guten Gewissens den besagten Zusatz "...simple Betrachtung" in der Überschrift des gegenwärtigen Kapitels weglassen könnten. Den geäußerten Zweifel verdeutlicht folgendes 

\vspace{0.3cm}

\begin{Example*}
Wir nehmen den Zustand zum Zeitpunkt $t \in T$ mit $\widehat{m} = 5$ angebundenen Service-Providern und als durch $\Gamma(t) \approx 50.000$ beschrieben an und schauen uns drei Szenarien an, die allesamt die getroffene Annahme hergeben:

\vspace{0.3cm}

\begin{enumerate}
  \item Wir könnten von $\widehat{n} = 50.000$ angebundenen Usern ausgehen, von denen je 10.000 mit je einem einzigen der $\widehat{m} = 5$ Provider connectet wären und keinem anderem. 
  \item Genauso könnten dieselben $\widehat{n} = 50.000$ angebundene User so verteilt sein, dass 49.996 (quasi alle) mit demselben einzelnen Provider connectet sind, und die restlichen 4 (also quasi niemand) User mit je einem anderen der verbleibenden 4 Provider verbunden sind.
  \item Ein ganz anderes Szenario wäre der Fall von $\widehat{n} \approx 25.000$, von denen jeder mit denselben zwei unserer fünf Service-Providern connectet wäre (und keinem anderen) und zudem ein paar vereinzelte zusätzliche User mit je einem der verbleibenden drei unserer fünf Provider.
\end{enumerate}

\vspace{0.3cm}

Rein an den Größen $\widehat{n}$, $\widehat{m}$, $\Gamma(t)$ gemessen, scheint Fall (3) aufgrund von $\widehat{n} = 25.000$ der schlechteste zu sein. Rein intuitiv scheint genau dieser Fall aber der beste zu sein. Dies ist aber nur ein Gefühl. Es lassen sich ebenso gute Argumente finden, warum Fall (1) oder Fall (2) der beste sein könnten. Es kommt eben darauf an...Gleichwohl für alle der Fälle $\Gamma(t) = 50.000$ gilt, lässt sich zweifelsfrei entscheiden, welcher zwingend der beste sein soll.

Was sich jedoch objektiv beurteilen lässt, ist die Tatsache, dass in Fall (2) vier der fünf Service-Provider quasi "wertlos" sind. Und in Fall (3) immer noch drei von fünf!

\vspace{0.3cm}

Wir könnten also unsere Gegenüberstellung der drei angeführten Cases auch zur folgenden quantitativen Beurteilung stellen:

\vspace{0.3cm}

\begin{enumerate}
  \item $\Gamma_1(t) = 50.000$, $\widehat{n}_1 = 50.000$ und $\widehat{m}_1 = 5$
  \item $\Gamma_2(t) = 50.000$, $\widehat{n}_2 = 50.000$ und $\widehat{m}_2 = 1$
  \item $\Gamma_3(t) = 50.000$, $\widehat{n}_3 = 25.000$ und $\widehat{m}_3 = 2$
\end{enumerate}

\vspace{0.3cm}

Was ist also besser?

\end{Example*}

\vspace{0.6cm}

 % binde die Datei '[Economics][Quantifizierung][WP][Fazit].tex' ein




\subsubsection{Zustandsbeschreibung WunderPass - detaillierte Sicht}
\label{sec:eco_zahlen_zustand_wp_advanced}


\paragraph{Teilnehmer} 
\label{sec:eco_zahlen_zustand_wp_advanced_teilnehmer}
\textrm{ }

\vspace{0.3cm}

\todo{WIP}
\vspace{1cm}

\todo{[TODO4]["individuelle" User- und Provider-Pools]}
\vspace{0.3cm}

Nicht alle Teilnehmer innerhalb der WunderPass-Netzwerks sind gleichbedeutend. Dies ist zweifelsfrei klar hinsichtlich der Unterscheidung zwischen connecteten Usern $\widehat{u} \in \widehat{U}^{(t)}$ und Service-Providern $\widehat{s} \in \widehat{S}^{(t)}$. Jedoch bestehen ebenfalls signifikante Unterschiede jeweils innerhalb der beiden Teilnehmerklassen $\widehat{U}^{(t)}$ und $\widehat{S}^{(t)}$ (siehe \todo{[TODO4]["Sättigung"]}). Um dieser Unterscheidung unserer Teilnehmer gerecht zu werden, definieren wir "connected Pools" pro individuellen Teilnehmer als

\vspace{0.3cm}

\begin{Def}\label{defTeilnehmerPool}

Sei $t \in T$, $\widehat{U}^{(t)}$ und $\widehat{S}^{(t)}$ die übergeordneten "connected" User- und Provider-Pools und $\widehat{u}_{*} \in \widehat{U}^{(t)}$ und $\widehat{s}_{*} \in \widehat{S}^{(t)}$ die entsprechenden Teilnehmer, für deren individuelle "connected Pools" wir uns an dieser Stelle interessieren. Wir definieren die genannten "connected Pools" als

\begin{align*}
&accounts : \widehat{U}^{(t)} \rightarrow \mathcal{P}\left(\widehat{S}^{(t)}\right) \\
&accounts^{(t)}(\widehat{u}_{*}) = \left\{\widehat{s} \in \widehat{S}^{(t)} \textrm{ } | \textrm{ } \alpha^{(t)}(\widehat{u}_{*}, \widehat{s}) = 1 \right\}
\end{align*}
und

\begin{align*}
&userbase : \widehat{S}^{(t)} \rightarrow \mathcal{P}\left(\widehat{U}^{(t)}\right) \\
&userbase^{(t)}(\widehat{s}_{*}) = \left\{\widehat{u} \in \widehat{U}^{(t)} \textrm{ } | \textrm{ } \alpha^{(t)}(\widehat{u}, \widehat{s}_{*}) = 1 \right\}
\end{align*}

\end{Def}

\vspace{0.6cm}

\begin{Lemma}\label{lemmaPools}

\begin{align}
\bigcup_{\widehat{u} \in \widehat{U}^{(t)}} \left(accounts^{(t)} (\widehat{u})\right) = \widehat{S}^{(t)} \tag{i} \\
\bigcup_{\widehat{s} \in \widehat{S}^{(t)}} \left(userbase^{(t)} (\widehat{s})\right) = \widehat{U}^{(t)} \tag{ii}
\end{align}

\end{Lemma}

\vspace{0.3cm}

\begin{proof}[Beweis] \textrm{ }

\vspace{0.3cm}

zu (i): Es ist

\begin{align*}
\bigcup_{\widehat{u} \in \widehat{U}^{(t)}} \left(accounts^{(t)} (\widehat{u})\right) &\overset{\text{Def \ref{defTeilnehmerPool}}}{=} \bigcup_{\widehat{u} \in \widehat{U}^{(t)}} \left\{\widehat{s} \in \widehat{S}^{(t)} \textrm{ } | \textrm{ } \alpha^{(t)}\left(\widehat{u}, \widehat{s}\right) = 1 \right\} \\
&\overset{(*)}{=} \left\{\widehat{s} \in \widehat{S}^{(t)} \mid \exists \widehat{u} \in \widehat{U}^{(t)} \textrm{ mit } \alpha^{(t)}\left(\widehat{u}, \widehat{s}\right) = 1 \right\} \\
&\overset{\text{Th. \ref{theoremPools} (\ref{theoremPools_6})}}{=} \widehat{S}^{(t)}
\end{align*}
Die Gleichheit zu (*) ergibt aus der Tatsache, Mengen-Vereinigungen ignorierten Doppelzählungen! 

\vspace{0.3cm}

Aussage (ii) folgt ganz analog!

\end{proof}

\vspace{0.3cm}


\begin{Theorem}\label{theremPoolsCount}

\begin{equation*}
\sum_{\widehat{u} \in \widehat{U}^{(t)}} |accounts^{(t)} (\widehat{u})| = \sum_{\widehat{s} \in \widehat{S}^{(t)}} |userbase^{(t)} (\widehat{s})| = \Gamma(t)
\end{equation*}

\end{Theorem}

\vspace{0.3cm}

\begin{proof}[Beweis] \textrm{ }

\vspace{0.3cm}

Es ist

\begin{align*}
\Gamma(t)&\overset{\text{Th. \ref{theremConnectionsCount} (\ref{theremConnectionsCount_1})}}{=} \sum_{i=1}^{\widehat{n}} \sum_{j=1}^{\widehat{m}} \widehat{\alpha}^{(t)}_{ij} \\
&= \sum_{\widehat{u} \in \widehat{U}^{(t)}} \textrm{  } \sum_{\widehat{s} \in \widehat{S}^{(t)}} \alpha^{(t)}\left(\widehat{u}, \widehat{s}\right) \\
&= \sum_{\widehat{u} \in \widehat{U}^{(t)}} \textrm{  } \sum_{\widehat{s} \in \widehat{S}^{(t)} \textrm{ mit } \alpha^{(t)}\left(\widehat{u}, \widehat{s}\right) = 1} 1 \\
&\overset{\text{Def \ref{defTeilnehmerPool}}}{=} \sum_{\widehat{u} \in \widehat{U}^{(t)}} \textrm{  } \sum_{s \in accounts^{(t)} (\widehat{u})} 1 \\
&= \sum_{\widehat{u} \in \widehat{U}^{(t)}} |accounts^{(t)} (\widehat{u})|
\end{align*}

\vspace{0.3cm}

Die zweite Gleichheit folgt analog, falls man die Kommutativität der Def \ref{defGamma} beachtet: 

\begin{equation*}
  \Gamma(t)= \sum_{i=1}^n \sum_{j=1}^m \alpha^{(t)}_{ij} = \sum_{j=1}^m \sum_{i=1}^n \alpha^{(t)}_{ij}
\end{equation*}

\end{proof}


\vspace{0.6cm}
\todo{[ende TODO4]}
\vspace{1cm}\todo{[TODO4]["individuelle" User- und Provider-Pools]}
\vspace{0.3cm}

Nicht alle Teilnehmer innerhalb der WunderPass-Netzwerks sind gleichbedeutend. Dies ist zweifelsfrei klar hinsichtlich der Unterscheidung zwischen connecteten Usern $\widehat{u} \in \widehat{U}^{(t)}$ und Service-Providern $\widehat{s} \in \widehat{S}^{(t)}$. Jedoch bestehen ebenfalls signifikante Unterschiede jeweils innerhalb der beiden Teilnehmerklassen $\widehat{U}^{(t)}$ und $\widehat{S}^{(t)}$ (siehe \todo{[TODO4]["Sättigung"]}). Um dieser Unterscheidung unserer Teilnehmer gerecht zu werden, definieren wir "connected Pools" pro individuellen Teilnehmer als

\vspace{0.3cm}

\begin{Def}\label{defTeilnehmerPool}

Sei $t \in T$, $\widehat{U}^{(t)}$ und $\widehat{S}^{(t)}$ die übergeordneten "connected" User- und Provider-Pools und $\widehat{u}_{*} \in \widehat{U}^{(t)}$ und $\widehat{s}_{*} \in \widehat{S}^{(t)}$ die entsprechenden Teilnehmer, für deren individuelle "connected Pools" wir uns an dieser Stelle interessieren. Wir definieren die genannten "connected Pools" als

\begin{align*}
&accounts : \widehat{U}^{(t)} \rightarrow \mathcal{P}\left(\widehat{S}^{(t)}\right) \\
&accounts^{(t)}(\widehat{u}_{*}) = \left\{\widehat{s} \in \widehat{S}^{(t)} \textrm{ } | \textrm{ } \alpha^{(t)}(\widehat{u}_{*}, \widehat{s}) = 1 \right\}
\end{align*}
und

\begin{align*}
&userbase : \widehat{S}^{(t)} \rightarrow \mathcal{P}\left(\widehat{U}^{(t)}\right) \\
&userbase^{(t)}(\widehat{s}_{*}) = \left\{\widehat{u} \in \widehat{U}^{(t)} \textrm{ } | \textrm{ } \alpha^{(t)}(\widehat{u}, \widehat{s}_{*}) = 1 \right\}
\end{align*}

\end{Def}

\vspace{0.6cm}

\begin{Lemma}\label{lemmaPools}

\begin{align}
\bigcup_{\widehat{u} \in \widehat{U}^{(t)}} \left(accounts^{(t)} (\widehat{u})\right) = \widehat{S}^{(t)} \tag{i} \\
\bigcup_{\widehat{s} \in \widehat{S}^{(t)}} \left(userbase^{(t)} (\widehat{s})\right) = \widehat{U}^{(t)} \tag{ii}
\end{align}

\end{Lemma}

\vspace{0.3cm}

\begin{proof}[Beweis] \textrm{ }

\vspace{0.3cm}

zu (i): Es ist

\begin{align*}
\bigcup_{\widehat{u} \in \widehat{U}^{(t)}} \left(accounts^{(t)} (\widehat{u})\right) &\overset{\text{Def \ref{defTeilnehmerPool}}}{=} \bigcup_{\widehat{u} \in \widehat{U}^{(t)}} \left\{\widehat{s} \in \widehat{S}^{(t)} \textrm{ } | \textrm{ } \alpha^{(t)}\left(\widehat{u}, \widehat{s}\right) = 1 \right\} \\
&\overset{(*)}{=} \left\{\widehat{s} \in \widehat{S}^{(t)} \mid \exists \widehat{u} \in \widehat{U}^{(t)} \textrm{ mit } \alpha^{(t)}\left(\widehat{u}, \widehat{s}\right) = 1 \right\} \\
&\overset{\text{Th. \ref{theoremPools} (\ref{theoremPools_6})}}{=} \widehat{S}^{(t)}
\end{align*}
Die Gleichheit zu (*) ergibt aus der Tatsache, Mengen-Vereinigungen ignorierten Doppelzählungen! 

\vspace{0.3cm}

Aussage (ii) folgt ganz analog!

\end{proof}

\vspace{0.3cm}


\begin{Theorem}\label{theremPoolsCount}

\begin{equation*}
\sum_{\widehat{u} \in \widehat{U}^{(t)}} |accounts^{(t)} (\widehat{u})| = \sum_{\widehat{s} \in \widehat{S}^{(t)}} |userbase^{(t)} (\widehat{s})| = \Gamma(t)
\end{equation*}

\end{Theorem}

\vspace{0.3cm}

\begin{proof}[Beweis] \textrm{ }

\vspace{0.3cm}

Es ist

\begin{align*}
\Gamma(t)&\overset{\text{Th. \ref{theremConnectionsCount} (\ref{theremConnectionsCount_1})}}{=} \sum_{i=1}^{\widehat{n}} \sum_{j=1}^{\widehat{m}} \widehat{\alpha}^{(t)}_{ij} \\
&= \sum_{\widehat{u} \in \widehat{U}^{(t)}} \textrm{  } \sum_{\widehat{s} \in \widehat{S}^{(t)}} \alpha^{(t)}\left(\widehat{u}, \widehat{s}\right) \\
&= \sum_{\widehat{u} \in \widehat{U}^{(t)}} \textrm{  } \sum_{\widehat{s} \in \widehat{S}^{(t)} \textrm{ mit } \alpha^{(t)}\left(\widehat{u}, \widehat{s}\right) = 1} 1 \\
&\overset{\text{Def \ref{defTeilnehmerPool}}}{=} \sum_{\widehat{u} \in \widehat{U}^{(t)}} \textrm{  } \sum_{s \in accounts^{(t)} (\widehat{u})} 1 \\
&= \sum_{\widehat{u} \in \widehat{U}^{(t)}} |accounts^{(t)} (\widehat{u})|
\end{align*}

\vspace{0.3cm}

Die zweite Gleichheit folgt analog, falls man die Kommutativität der Def \ref{defGamma} beachtet: 

\begin{equation*}
  \Gamma(t)= \sum_{i=1}^n \sum_{j=1}^m \alpha^{(t)}_{ij} = \sum_{j=1}^m \sum_{i=1}^n \alpha^{(t)}_{ij}
\end{equation*}

\end{proof}


\vspace{0.6cm}
\todo{[ende TODO4]}
\vspace{1cm}





















\subsubsection{Other Stuff}
\label{sec:eco_zahlen_zustand_todo}

\vspace{2cm}
\todo{[TODO2][Abschätzung $\frac{\widehat{n}}{\widehat{m}}$]}
\vspace{0.3cm}

Aussagen aus Annahme \ref{assumptionRatio} und Theorem \ref{theremConnectionsCount} - Aussage (ii) - verwerten und Annahme \ref{assumptionRatio} deutlich verschärfen.

\todo{[ende TODO2]}
\vspace{1cm}


\todo{[TODO3]["Verdichtung"]}
\vspace{0.3cm}

Die Maße $\widehat{n}$, $\widehat{m}$ und $\Gamma(t)$ sind sehr objektiv und teils zielführend. Sie scheinen aber nicht zu reichen. So kann es z. B. User $\widehat{u} \in \widehat{U}^{(t)}$ geben, die im worst case ausschließlich zu einem einzigen Provider $\widehat{s} \in \widehat{S}^{(t)}$ connectet sind (und somit aber trotzdem den Wert von $\widehat{n}$ beeinflussen, oder noch schlimmer analog Provider $\widehat{s} \in \widehat{S}^{(t)}$, die als "angebunden" gelten, weil sie mit marginal wenigen Usern (im worst case mit einem einzigen) connectet sind. Solche Teilnehmer spielen eigentlich für den zahlenmäßigen WunderPass-Fortschritt keinerlei Rolle, beeinflussen jedoch unsere relevanten Messgrößen (KPI).

Von daher benötigen wir noch eine weitere Präzisierung in Form von

\begin{itemize}
  \item "80-20-Regel" heranziehen, indem man die Mengen $\widehat{U}^{(t)}$ und $\widehat{S}^{(t)}$ so modifiziert/verkleinert, dass $\Gamma(t)$ davon kaum einen Einfluss spürt (sich lediglich um einen marginalen Prozentsatz verringert).
  \item Formeln auf die davon modifizierten Größen $\widehat{\widehat{n}}$ und $\widehat{\widehat{m}}$ anpassen.
  \item $\Rightarrow$ Die Grenzen von [Theorem \ref{theremConnectionsCount}][Aussage (ii)] werden damit deutlich schärfer.
  \item $\Rightarrow$ kann sicherlich in Abschnitt \ref{sec:eco_zahlen_business_plan} für den Umgang mit dem Verhältnis $\frac{\widehat{n}}{\widehat{m}}$ genutzt werden.
  \item Wird vermutlich auch Relevanz bei den "individuellen" (Definition erfolgt noch) User- und Provider-Pools zum Einsatz kommen.
\end{itemize}

\todo{[ende TODO3]}
\vspace{1cm}

\todo{[TODO4.1]["Exklusive Connections"]}
\vspace{0.3cm}

\begin{itemize}
  \item Eine Connection zu einem Service-Provider ist exklusiv, wenn der zugehörige User zu keinem anderen Service-Provider connectet ist.
    \item Es gibt mindestens $n_{nexcl} \geq \Gamma(t) - \widehat{n}$ nicht exklusive Connections.
  
\end{itemize}

\todo{[ende TODO3.1]}
\vspace{1cm}











% !TEX root = [Economics][Quantifizierung].tex

\todo{[TODO6][deprecated Inhalt verarbeiten]}
\vspace{0.3cm}

Mit diesen geschaffenen Formalisierungs-Werkzeugen lässt sich nun auch die übergeordnete WunderPass-Vision formal erfassen - und zwar indem man den Zeitpunkt $t_{*} \in T$ ihrer Erreichung benennt:

\begin{Def}\label{defVision}

Wir betrachten die WunderPass-Vision zu einem Zeitpunkt $t_{*} \in T$ als erreicht, falls

\vspace{0.3cm}

\begin{equation}
\label{eq:1}
  \alpha^{(t_{*})}_{ij} = 1 \textrm{ für alle } i \in \{1,...,n\} \textrm{ und } j \in \{1,...,m\}
\end{equation}\\
erfüllt ist. Darüber hinaus ist es noch nicht ganz klar, welche Aussage für die Zeitpunkte $t > t_{*}$ hinsichtlich der Visions-Erreichung zu treffen sei. Grundsätzlich ist es ja durchaus denkbar, die obige Voraussetzung gelte für $t > t_{*}$ nicht mehr. Bleibt die Vision in diesem Fall trotzdem als 'erreicht' zu betrachten?

\end{Def}

\vspace{1cm}

Die gelungene Formalisierung unserer Vision mittels Definition \ref{defVision} mag einen Fortschritt hinsichtlich unserer "Business-Mathematics" darstellen, bleibt jedoch losgelöst zunächst einmal ziemlich wertlos. Zum einen ist das Erreichen der Vision im formellen Sinne der Definition \ref{defVision} weder praxistauglich noch akribisch erforderlich. Zudem bleibt zum anderen der resultierende (intrinsische) Business-Value der Visions-Erreichung bisher weiterhin nicht ohne Weiteres erkennbar.
Vielmehr sollten wir die Anforderung von Gleichung \eqref{eq:1} als eine Messlatte unseres Fortschritts heranziehen, und eher als (unerreichbare) 100\%-Zielerreichungs-Marke betrachten. Zudem müssen wir zeitnah - obgleich die vollständige oder nur fortschreitend partielle - Zielerreichung unserer Vision in klaren, quantifizierbaren Business-Value übersetzen.

Dazu definieren wir als erstes ein intuitives Maß der Zielerreichung:

\vspace{0.3cm}

\begin{Def}\label{defGamma2}

\begin{equation*}
  \Gamma : T \rightarrow \mathbb{N} 
\end{equation*}

\begin{equation*}
  \Gamma(t):= \sum_{i=1}^n \sum_{j=1}^m \alpha^{(t)}_{ij} 
\end{equation*}

\end{Def}

\vspace{1cm}

Damit liefert uns die definierte $\Gamma$-Funktion aber auch ein extrem greifbares und intuitiv nachvollziehbares Fortschrittsmaß unseres Vorhabens. Zudem fügt sich dieses perfekt in unsere mittels Definition \ref{defVision} quantifizierte Unternehmens-Vision und unterliegt einer fundamentalen (bezifferbaren) Obergrenze. Dies zeigt folgendes Lemma:

\vspace{0.3cm}

\begin{Lemma}

Es gelten folgende Aussagen:

\begin{align}
\Gamma(t) &\leq n^{(t)} * m^{(t)} \textrm{ für alle } t \in T \tag{i} \label{eq:l1_erste} \\ 
  \text{es gilt Gleichheit bei }  \eqref{eq:l1_erste} &\Leftrightarrow \text{ es gilt Gleichung } \eqref{eq:1} \text{ aus Def } \ref{defVision} \tag{ii} \label{eq:l1_zweite}
\end{align}

\end{Lemma}

\vspace{0.3cm}

Gleichung \eqref{eq:l1_zweite} ermöglicht uns die Definition \ref{defGamma} auf ein relatives Zielereichungs-Maß auszuweiten:

\vspace{0.3cm}

\begin{Def}\label{defKleinGamma}
\begin{equation*}
  \gamma : T \rightarrow [0; 1] 
\end{equation*}

\begin{equation*}
  \gamma(t):= \frac{\Gamma(t)}{n^{(t)} * m^{(t)}}
\end{equation*}

\end{Def}

\todo{[ende TODO6]}
\vspace{1.5cm}    % binde die Datei '[Economics][Quantifizierung][deprecated].tex' ein






\todo{Ab hier WIP}
\vspace{1cm}


\subsubsection{Business-Plan in Mathematics}
\label{sec:eco_zahlen_business_plan}

Diese letzten Werkzeuge lassen und Begriffe wie "Zielsetzung" bzw. "Milestone" einführen.

\vspace{0.3cm}

\begin{Def}\label{defZiel}

Seien $t \in T$ und zudem entsprechend $(n^{(t)}, m^{(t)}) = dP^{(t)}$ der angenommene Zustand der digitalen Welt zu einem beliebig gewählten Zeitpunkt. Wir definieren die - allein durch WunderPass zu bestimmende - Zielfunktion als

\end{Def}


\subsubsection{Quantifizierung des Status quo}
\label{sec:eco_zahlen_status_quo}



\paragraph{Vernetzung \& Netzwerk-Effekt}
\label{sec:zahlen_status_quo_netzwerk_effekt}

\textrm{ }
\vspace{0.3cm}

Die WunderPass-Vision steht in ihrer Formulierung ganz klar im Sinne einer gewissen "Vernetzung". Wir möchten, dass möglichst viele User sich mit möglichst vielen Service-Providern "connecten" (bzw. connectet sind/bleiben). Schränkt man seine Sichtweise alleinig auf diese Vision (ohne diese zunächst zu hinterfragen), liefern uns die zuletzt eingeführten Größen $\alpha^{(t)}_{ij}$, $\Gamma(t)$ und $\gamma(t)$ ziemlich gute Gradmesser, um zweifelsfreie Aussagen hinsichtlich der Vergleichbarkeit zweier Zeitpunkte $t_1, t_2 \in T$ treffen zu können. Es ist irgendwie klar, $\alpha^{(t)}_{ij} = 1$ sei im Sinne unserer Vision irgendwie besser als $\alpha^{(t)}_{ij} = 0$.

Aus diesem Blickwinkel (in dem die Vision zunächst ein Selbstzweck bleibt) erscheint die folgende Definition mehr als intuitiv einleuchtend, um die obige Formulierung "irgendwie besser" zu formalisieren und vor allem zu quantifizieren. 



















\vspace{0.3cm}

\todo{WIP:}
Hier stand vorher Definition \ref{defRelation}

\vspace{1cm}

Man beachte, dass es sich bei der definierten Ordnungsrelation gar um eine \href{https://de.wikipedia.org/wiki/Ordnungsrelation#Totalordnung}{Totalordnung} handelt!
Der Form halber ergänzen wir an der Stelle noch um zwei weitere - schematisch induzierte - Relationen auf unserem Zeitstrahl $T$:

\vspace{0.3cm}

\todo{WIP:}
Hier stand vorher Definition \ref{defRelationen}

\vspace{0.3cm}

Auch für die letzten beiden Relationen wollen wir fortan die vereinfachte Schreibweise $t_1 \prec t_2$ und $t_1 \simeq t_2$ nutzen.
























 

\vspace{1cm}

Diese Netzwerk-Bewertungs-Modell besitzt jedoch im aktuellen Zustand drei wesentliche Schwachstellen:

\begin{itemize}
  \item Es beschreibt uns misst weiterhin ausschließlich den intrinsischen Wert der Vernetzung innerhalb unserer kleinen "Visions-Welt", dem es noch an Bezug zur "Außenwelt" und dem Business-Case fehlt. Diesen Umstand wollen wir weiterhin zunächst einmal ignorieren.
  \item Es bewertet in der aktuellen Form ausschließlich "unsere Welt" bzw. unseren Fortschritt als Ganzes. Die definierte "besser"-Relation misst das "Besser" aus Sicht der Allgemeinheit. Der einzelne Teilnehmer bleibt individuell unberücksichtigt. Es ist schwer vorstellbar, ein Ökosystem zu designen, welches intrinsisch nach dem Wohl/Optimum Aller strebt (und damit eben einmal einen formalen Beweis für das Funktionieren des Kommunismus zu liefern.) 
  \item Es lässt den sogenannten \href{https://de.wikipedia.org/wiki/Netzwerkeffekt}{Netzwerkeffekt} außer Acht! Denn selbst wenn man eben einmal das Problem des Bullet 1 aus der Welt schafft, und ein Preisschild an den Mehrwert einer Connection zwischen User und Provider bekommt. Die Literatur zum besagten Netzwerkeffekt liefert gute Argumente für die Annahme, eine von uns anvisierte User-Provider-Connection kann nur sehr selten alleinstehend in ihrem Mehrwert bewertet werden. Vielmehr bemisst sich dieser etwaige Mehrwert in dem Zusammenspiel und den Synergien mit anderen User-Provider-Connections. Es lassen sich viele Beispiele finden, um diesen Umstand zu begründen. So kann es z. B. sein, dass ein Finance-Aggregator-Service für einen User um so wertvoller wird, je mehr Finance-Provider der User selbst mit seiner WunderIdentity connectet. Hierbei wird es kaum einen Unterschied für ihn machen, ob die genannten Finance-Provider mit 100 anderen WunderPass-Usern connectet seien oder mit 10 Mio. Im Case einer Splitwise-Connection (oder auch einer etwaigen EventsWithFriends-App) dagegen entsteht der Mehrwert erst dann, wenn auch ganz viele Freunde des Users diese Splitwise-Connection mit WunderPass besitzen. Andernfalls beläuft sich der Mehrwert seiner eigenen Connection so ziemlich gen Null.
\end{itemize}

\vspace{1cm}

Insbesondere der letzte Punkt wirft einige interessante Fragen auf, zu denen wir eine Antwort finden werden müssen. Oder zumindest Hypothesen und Annahmen treffen.
Was bedeutet eigentlich

\begin{equation*}
  \alpha^{(t)}_{kj} * \alpha^{(t)}_{lj} = 1 \textrm{ für zwei User } u^{(t)}_k, u^{(t)}_l \in U^{(t)} \textrm{ die beide mit Privider } s^{(t)}_j \in S^{(t)} \textrm{ connectet sind?}
\end{equation*}
Sind diese dann damit gleichbedeutend in irgendeiner Weise ebenfalls "\textit{miteinander connectet}"? Und was würde eine solche Implikation für unser bisheriges Modell bedeuten? Wie (un)abhängig ist eine solche "indirekte Connection" von ihrer "Brücke" - dem Service-Provider? All diese Fragen lassen sich zudem analog auf "indirekte Connections" zwischen Providern übertragen - die dann etwaige User als "Brücke" nützten. Zu guter Letzt ließe sich diese neue Komplexität beliebig potenzieren, indem man mittels Rekursion indirekte Connections "zweitens, drittens,... Grades" definiert.

Um der aufkommenden Komplexität Herr zu werden, wollen wir uns zunächst einmal dem zweiten der oben genannten Schwachstellen unseres bisherigen Modells zuwenden, und dieses idealerweise dahingehend erweitern, auch individuelle Bewertungen unserer Teilnehmer $u \in U^{(t)}$ und $s \in S^{(t)}$ zu erfassen.


\subsubsection{Individuelle Wertschöpfung der Teilneher}
\label{sec:eco_zahlen_teilnehmer}

Hallo    % binde die Datei '[Economics][Quantifizierung].tex' ein
% !TEX root = paper.tex

\subsection{Token-Economics (WPT)}
\label{sec:eco_wpt}

\todo{WIP}

\vspace{0.3cm}

\begin{Praemisse}[generelle Anforderungen an den Token]

\begin{itemize}
  \item Der Token soll ein \textbf{echter} Utility-Token sein. Er braucht also zwingend einen \textbf{intrinsischen Wert}.
\end{itemize}

\vspace{0.2cm}

Die Teilnehmer (User und Provider) müssen einen intrinsischen Vorteil am Besitz von Tokens innerhalb des Ökosystems erfahren. Sie müssen quasi "irgendwas mit dem Token machen können" - und zwar innerhalb des Ökosystems und nicht mittels "Verkaufs nach außen". Wenn man als Teilnehmer die Möglichkeit besitzt, Tokens für/durch irgendetwas zu erwerben, muss auch die Möglichkeit bestehen, diesen für irgendetwas (anderes; "nützliches") innerhalb des Ökosystems auszugeben. Idealerweise verhält sich unser Token zur Euro, wie sich der Euro zum nicht existenten "Weltall-Taler" verhält - also ohne Rechtfertigung zu besitzen, das Ökosystem verlassen zu müssen.

Falls die Schaffung einer solchen Ökonomie nicht gelingen sollte - weil z.B. die Service-Provider mehr Value generieren, als sie innerhalb des Ökosystems "konsumieren" können -  sollte diese Forderung zumindest für den Teilnehmer "User" sichergestellt werden. Denn der User partizipiert in seinem Dasein eher als Konsument innerhalb des Digital-Universums, als als Wertschöpfer, weshalb seinem intrinsischen Vorteil am Besitz von Tokens mit dem damit ermöglichten Konsum von digitalen Dienstleistungen Genüge getan sein sollte.

\vspace{0.3cm}

\begin{itemize}
  \item Der Token sollte natürlich auch einen \textbf{extrinsischen} Wert besitzen.
\end{itemize}

\vspace{0.2cm}

Nicht all zu laut (der Community ggü.) kommuniziert, wäre unsere ganze Unternehmung im Falle des Fehlen des extrinsischen Werts nichts anderes als ein kommunistischer Akt. Nur diese Beschaffenheit des Tokens liefert uns ein Monetarisierungs-Modell. Und auch deutet zudem vieles darauf hin, die Service-Provider-Teilnehmer kämen ohne einen extrinsischen Wert nicht aus.

\vspace{0.3cm}

\begin{itemize}
  \item Der Token soll \textbf{nicht inflationär} sein - also einen definierten Cap besitzen.
\end{itemize}

\vspace{0.2cm}

Mit voriger Forderung - laut der man "etwas mit dem Token innerhalb des Systems machen kann", verleiht die gegenständige Forderung das dieses "Etwas", was mittels des Tokens ermöglicht wird, einem gewissen Qualitätsanspruch genügen muss. Je größer die Qualität dieses besagten "Etwas" - also z. B. einer Dienstleistung, die mit ausschließlich mit dem Token bezahlt werden kann - ist, desto \textit{wertvoller} wird auch der Besitz des Tokens. Und damit auch sowohl sein intrinsischer als auch extrinsischer Wert. Schlichtweg deshalb, weil der Token und somit der mögliche Konsum besagter Dienstleistung gecappt ist.

\vspace{0.3cm}

\begin{itemize}
  \item \todo{Kreislauf}.
\end{itemize}

\vspace{0.2cm}

\todo{Kreislauf-Beschreibung}

\end{Praemisse}

\vspace{0.3cm}


\begin{Praemisse}[Daten haben einen Wert]

\vspace{0.2cm}

\todo{TODO: Evaluierung extrem schwierig. Folgende Aussagen/Antworten sind zu beweisen.}

\vspace{0.2cm}

\begin{itemize}
  \item Wer besitzt Daten/Informationen?
  \item Für wen sind diese Daten von "Wert" (Geld verdienen)?
  \item Wie kann der Wert der Daten maximiert werden? Wer profitiert im welchen Maße davon?
  \item Wer würde für diese Daten bezahlen und wie viel?
  \item Wie ist die (maximale) Wertschöpfung zu verteilen? Wer wird beteiligt? Wie wird die maximierende Rolle der Wertschöpfung belohnt?
  \item Wer trägt etwaige Risiken und in welchem Verhältnis?
  \item Wie ist das alles in die Token-Economics zu integrieren? 
\end{itemize}

\end{Praemisse}


\vspace{0.3cm}

\begin{Fazit}[unser Ökosystem generiert Value]

\begin{itemize}
  \item Wir schöpfen Mehrwert, indem wir Datenerfassung ermöglichen (die ja einen nachgewiesenen Value besitzen?
  \item Besitzer der Daten werden entlohnt
  \item Nutzer der Daten zahlen für Daten, generieren damit aber Value, der wiederum entlohnt wird.
  \item Am Ende haben alle Teilnehmer entweder Value generiert oder aber im Wert des values verkonsumiert
  \item Wir partizipieren am extrinsischen Wert des Tokens (Kurs-Entwicklung durch positive Wertschöpfung des gesamten Ökosystems).
  \item Incentives sind nötig, um das Henne-Ei-Problem zu lösen
  \item Incentives sollten nachträglich mit der dadurch geschaffenen Wertschöpfung verrechtet werden. 
\end{itemize}

\end{Fazit}

\vspace{0.3cm}

\begin{Solution}[möglicher Token-Flow]

\begin{itemize}
  \item Ein User nutzt einen Service-Provider A, der WunderPass unterstützt, und ist auch mit seinem WunderPass bei Provider A eingeloggt.
  \begin{itemize}
  	\item Beispiel 1: Der Service-Provider A ist ein Identity-Data-Management-Service, der die persönlichen Daten des Users verwaltet und bei Bedarf Dritten zur Verfügung stellen kann.
  	\item Beispiel 2: Der Service-Provider A ist EasyJet.
  \end{itemize}
  \item Der User und der Service-Provider A erzielen - wie auch immer - eine Übereinkunft darüber, dass die von Provider A verwalteten - den User betreffenden Daten - theoretisch mittels des WunderPass-Lookups mit Dritten geteilt werden können sollen. 
  \begin{itemize}
  	\item Beispiel 1: Die Daseinsberechtigung des Identity-Data-Management-Service beschränkt sich eigentlich ausschließlich auf das Teilen von Daten mit Dritten. Hierbei ist die obige Anforderung also trivialerweise unabdingbar.
  	\item Beispiel 2: Beim Beispiel von EasyJet könnten die besagten Daten z. B. gebuchte Flugtickets sein, die man mit Drittdiensten teilt, um daran ausgerichtet gezielte Werbeangebote im zugehörigen Ausland zu ermöglichen.
  \end{itemize}
  User und Provider einigen sich auf einen Preis/Preisformel für das Teilen dieser Daten - und zwar auf den konkreten Preis von \textbf{x WPT} (WunderPass-Utility-Token).
  \item Service-Provider B (der ebenfalls WunderPass unterstützt) möchte Userdaten des Service-Provider A nutzen, falls solche vorliegen.
  \begin{itemize}
  	\item Beispiel 1: Hierbei könnte Provider B so ziemlich jeder denkbare Online-Dienst sein, der irgendwelche persönlichen Userdaten benötigt (z. B. Adresse, Email, Kreditkarte etc.).
  	\item Beispiel 2: Hierbei könnte es sich z. B. um (schlecht ausgelastete) Hotels handeln, die anhand der EasyJet-Flugdaten über die Destination des Users wissend, besondere Angebote an ihn ausspielen wollen.
  \end{itemize}
  \item Service-Provider B callt der WunderPass-Lookup-Service, um die Existenz etwaiger Daten und deren \textbf{Preis x WPT} in Erfahrung zu bringen.
  \item Liegen Lookup-Daten vor, kann Provider B entscheiden, ob er diese zum angegebenen Preis beziehen möchte. 
  \item Möchte Service-Provider B Gebrauch vom Lookup machen, muss er in Vorleistung gehen und den Betrag von \textbf{2 * x WPT} in den Lookup-Contract einzahlen.
  \item Die eingebrachten \textbf{2 * x WPT} werden - abzüglich einer kleinen WunderPass-Fee - im Lookup-Contract gelockt. Service-Provider B erhält im Gegenzug einen \textit{Berechtigungs-Token} für den Abruf von entsprechenden Daten von Provider A (hierbei ist eher ein technischer Security-Token und kein Crypto-Token gemeint).
  \item Die Zugriffsberechtigung für das Abrufen der Daten von Provider A soll dabei einer \textbf{zeitlichen Beschränkung z} unterliegen (z. B. "eine Woche"). \textbf{z} ist hierbei ebenso individuell (Teilnehmer- und Daten-abhängig) zu sehen wie \textbf{x}.
  \item Service-Provider B fragt unter Vorlage des Berechtigungs-Token die gewünschten Daten beim Service-Provider A an.
  \begin{itemize}
  	\item Provider A muss den Berechtigungs-Token validieren (beim Lookup-Service).
  	\item A muss unter Umständen die Freigabe beim User einfordern (ggf. sollte der User in irgendeiner Weise "bestraft" werden, falls er den Datenzugriff verwehrt).
  	\item Provider A und B müssen einen gewissen "Handshake" implementieren, der A bescheinigt, wie vereinbart die korrekten Daten an B ausgeliefert zu haben. 
  \end{itemize}
  \item Provider A liefert die Daten an Provider B aus und erhält im Gegenzug ein Bestätigungszertifikat von B.
  \item Mit dem Bestätigungszertifikat kann Provider A seine Vergütung beim Lookup-Contract einlösen. Dabei wird die Hälfte der gelockten Einlage von Provider B (also an dieser Stelle die Hälfte von \textbf{2 * x WPT} - also \textbf{x WPT}) ausgeschüttet. Und zwar zur Hälfte an Provider A und zur andern Hälfte an den User.
  \item \textbf{x WPT} des ursprünglich eingezahlten Deposits von B bleiben weiterhin im Lookup-Contract gelockt. 
  \item Jede künftige Anfrage von B an A (bezüglich desselben Datensatzes) innerhalb des definierten Zeitraums \textbf{z} releast immer wieder die Hälfte des verbliebenen gelockten Deposits.
  \item Nach Ablauf des definierten \textbf{Zeitraums z}
  \begin{itemize}
  	\item bekommt B den verbliebenen (nicht ausgeschütteten) Teil seines Deposits zurückerstattet.
  	\item wird der \textit{Berechtigungs-Token} ungültig.
  	\item hat A keinen Anspruch mehr, für die Datenauslieferung an B vergütet zu werden (auch dann, falls er Daten ausgeliefert, ohne vorher die abgelaufene Gültigkeit des Berechtigungs-Tokens zu validieren).
  \end{itemize}
  \item Es ist denkbar, die an der User ausgezahlten Rewards in irgendeiner Weise (zeitlich) zu locken und deren Release an bestimmte Bedingungen zu knüpfen ($\rightarrow$ um den User zu incentivieren irgendetwas zu tun).
\end{itemize}

\vspace{0.5cm}

\underline{\textbf{Cashflow}}:

\begin{itemize}
  \item Provider B zahlt für den Lookup. Aber auch nur dann, falls er den Lookup nutzt. Andernfalls erhält er seinen getätigten Deposit (abzüglich einer kleinen Fee an WunderPass) zurück. Er zahlt in gleichem Teil an Provider A und den User. Aus der Verwertung der bezogenen Daten kreiert er einen Value (im Sinne seiner Dienstleistung). Einen Value, der auch durchaus im Sinne des Users sein könnte. Es ist also gut denkbar, dass Provider B eine Rechtfertigung besitzt, den User an seinen Kosten zu beteiligen (z. B. mittels einer Fee für die erbrachte Dienstleitung, die den gekauften Datensatz erforderte; idealerweise ebenfalls in \textbf{WPT} vom User zu erbringen).
  \item Provider A ist der klare Nutznießer des Datenaustauschs. Der "Daten-Trade" hat - direkt betrachtet - erst einmal gar nichts mit seinem Kerngeschäfts zu tun (es sei denn, A sei wie in Beispiel 1 ein Identity-Data-Management-Service, dessen Kerngeschäft ausschließlich darin besteht, Daten zur Verfügung zu stellen). In der perfekten WunderWelt kann Provider A in einem anderen Case, analog als Provider B auftreten, um seine erhaltenen Token-Rewards für für ihn relevante Daten auszugeben.
  \item Der User scheint hierbei auch der Nutznießer von etwaigen "Daten-Deals" zu sein. Seine Stellung als solcher ist aber weniger klar als diejenige von Provider A, da er von dem stattgefundenen Datenaustausch indirekt ebenso profitieren könnte, indem er z. B. auf Basis der Datennutzung eine bessere Dienstleistung von Provider B erhält. Der Pitch "der User monetarisiert seine Daten" kling zwar sehr attraktiv, muss man hierbei jedoch sehr aufpassen, den Bogen nicht zu überspannen. Denn - während die Rolle von Provider A als Profiteur unbestreitbar ist - wird die Zahlungsbereitschaft von Provider B von Fall zu Fall ganz unterschiedlich und nur bedingt vorhanden sein. Denn schließlich ist es alles andere als selbsterklärend, ein Online-Shop solle für Adressdaten des Users bezahlen, um seine Bestellung zustellen zu können, während der User davon profitiert. In diesem Fall wäre es eher nachvollziehbar, Provider B und der User würden sich die an Provider A zu entrichtenden Fees für die Bereitstellung der Adressdaten teilen. Hierbei ist die \textbf{Verteilung der Fees leider extrem heterogen}.
\end{itemize} 

\end{Solution}


\vspace{0.3cm}

\todo{TODO}

\newpage
\subsubsection{Ideen}
% !TEX root = paper.tex

\begin{itemize}
  \item Staken von Sub-Projects. 
  \begin{itemize}
  	\item Teilprojekt wird als \textbf{Curation Market} implementiert und bekommt damit seinen eigenen Token.
  	\item Die Projekteinlage erfolge in WUNDER-Tokens (Staking).
  	\item Investoren von WUNDER hätten damit die Möglichkeit, die für sie besonders interessanten Projekte stärker zu unterstützen als lediglich das übergeordnete WunderPass-Projekt.
  	\item Der WUNDER bekäme damit einen intrinsischen Wert: Man braucht ihn, um sich an den Teilprojekten zu beteiligen.
  \end{itemize}
\end{itemize}

\vspace{0.5cm}
 % binde die Datei '[Economics][Token-Economics][Ideen].tex' ein


\subsubsection{Einleitung}
\label{sec:wpt_einleitung}
\todo{TODO}

\subsubsection{Kreislauf}
\label{sec:wpt_kreislauf}
\todo{TODO}

\subsubsection{Token-Design}
\label{sec:wpt_design}
\todo{TODO}

\subsubsection{Incentivierung}
\label{sec:wpt_incent}
\todo{TODO}

\subsubsection{Milestones-Reward-Pool}
\label{sec:wpt_reward_pool}
\todo{TODO}

\subsubsection{WPT in Zahlen}
\label{sec:wpt_zahlen}
\todo{TODO}

\subsubsection{Fazit}
\label{sec:wpt_fazit}
\todo{TODO}    % binde die Datei '[Economics][Token-Economics].tex' ein
% !TEX root = C:/Users/Slava/White-Paper/[05][Economics]/Economics.tex

\subsection{Fazit}
\label{sec:eco_fazit}
\todo{TODO}    % binde die Datei '[Economics][Fazit].tex' ein    % binde die Datei 'Economics.tex' ein
% !TEX root = paper.tex

\section{NFT-Pass}
\label{sec:nft-pass}

Ein exzellentes Mittel, um \textit{WunderPass} als Geschäftsmodell, Unternehmung und Unternehmen ein symbolisches - gewissermaßen plastisches - Sinnbild einzuverleiben, ist die Repräsentation von \textit{WunderPass} als Service/Protokoll mittels eines - eigens dafür kreierten - NFTs: \textbf{"Des WunderPass"} (im Folgenden auch \textit{NFT-Pass} genannt)

\vspace{0.3cm}

\begin{Fazit}[\textit{WunderPass} deabstrahiert durch \textbf{"den WunderPass"} als NFT]

"Ich nutze \textit{WunderPass}" wird symbolisiert durch "Ich besitze \textbf{meinen WunderPass}"!

\end{Fazit}

\vspace{0.3cm}

% !TEX root = paper.tex

\subsection{Konzeption}

\vspace{0.3cm}

Unser Anspruch an den zu modellierenden \textit{NFT-Pass} ist grob der folgende:

\vspace{0.2cm}

\begin{itemize}
  \item Der \textit{NFT-Pass} muss sich ganz klar von dem Großteil der heutigen - in größter Regel als Sammlerstück verstandenen - den Markt überflutenden NFTs abgrenzen. Er braucht einen klar ersichtlichen \textbf{intrinsischen Wert}. Man muss also "etwas mit dem \textit{NFT-Pass} anfangen/machen können" und diesen nicht "lediglich besitzen", um ihn ausschließlich mit einer gewissen Wahrscheinlichkeit gewinnbringend weiterverkaufen zu können ("Hot Potato"). Der Token bedarf also gewisser Eigenschaften eines \textit{Governance-Tokens} (DAO) oder Ähnlichem.
  \item 
  \begin{sloppypar}  
  Der \textit{NFT-Pass} braucht ungeachtet des vorigen Bullet-Points jedoch trotzdem zusätzlich ebenso eine ähnliche Beschaffenheit - wie solche der aktuell üblichen marktbeherrschenden NFTs - als Sammlerstück - gleichwohl nicht erstrangig. 
  \end{sloppypar}
  \item Anders als die aktuell gängigen NFTs soll unser \textit{NFT-Pass} \textbf{nicht begrenzt} in der Anzahl seiner Stücke sein. Stattdessen sollen theoretisch beliebig viele \textit{NFT-Pässe} existieren können. Nichtsdestotrotz soll unser \textit{NFT-Pass} ebenso die Eigenschaft der "nicht inflationären Begehrtheit" einverleibt bekommen. Dies möchten wir mittels einer ausgeklügelten Minting-Logik abbilden, die ein \textbf{endliches Sub-Set} an raren und begehrten \textit{NFT-Pässen} innerhalb des \textbf{unendlichen Gesamt-Sets} der \textit{NFT-Pässe} sicherstellt. Soll heißen: Es werden einerseits \textit{NFT-Pässe} exis\-tieren, die den heutigen NFTs - im Sinne ihres Sammlerwertes - gleichkommen, während die restlichen andererseits mit ihrer steigenden Gesamtanzahl zunehmend entwerten, bis sie irgendwann (als NFT betrachtet) nahezu wertlos und lediglich "funktional" werden.
  \item Die Rarität und Begehrtheit unseres \textit{NFT-Pass} soll Gamification-Mechanismen folgen:
  \begin{itemize}
    \item Wir brauchen an etwaigen Stellen ein (wertbestimmendes) \textit{first-come-first-serve-Prinzip}.
    \item Wir brauchen an anderen Stellen ein (ebenso wertbestimmendes) Zufallsprinzip.
    \item Wir brauchen irgendwo ebenso ein (geringes) Maß an persönlicher Individua\-lisierung des \textit{NFT-Pass} - ausschließlich durch den User gesteuert.
    \item Abrundend könnte ein \textbf{gemeinnützig wertbestimmendes} (randomisiertes) Merkmal wirken. (Beispiel: Wenn die \textit{NFT-Pässe} irgendwann inflationär geworden sind, könnte der zehn-millionste plötzlich wieder richtig krass sein.)
  \end{itemize}
  \item Der \textit{NFT-Pass} muss gänzlich transparent und vor allem verständlich für den interessierten - gleichwohl vielleicht technisch nicht bewandertsten - User sein.
\end{itemize}

\vspace{0.3cm}

In den kommenden Abschnitten folgt ein initialer Abriss unserer Vorstellung des \textit{NFT-Pass}:

\vspace{0.3cm}


% !TEX root = paper.tex

\subsubsection{Status-Property}

\vspace{0.2cm}

Diese NFT-Property - die wir gleichzeitig als die Main-Property unseres \textit{NFT-Pass} ansehen - soll der oben formulierten Anforderung nach einem first-come-first-serve-Prinzip Rechnung tragen. Zeitlich früher ausgestellte NFT-Pässe sollen einen rareren und begehrteren \textit{Pass-Status} inne haben als die späteren. Und vor allem sollen die \textit{NFT-Pässe} eines bestimmten ausgestellten Status in ihrer Anzahl begrenzt sein und nach Erreichen einer zu definierenden Höchstgrenze fortan nie wieder ausgestellt (ge\-mintet) werden können.

\vspace{0.3cm}

\begin{NFT-Prop}[Pass-Status]

Wir definieren folgende \textit{NFT-Pass-Status} mit den dazugehörenden Eigenschaften:

\begin{itemize}
    \item Status A (\textbf{Diamond})
    \begin{itemize}
    	\item Anzahl Pässe: 200
    	\item Gemintet an Nummer: 1 bis 200
    \end{itemize}
    \item Status B (\textbf{Black})
    \begin{itemize}
    	\item Anzahl Pässe: 1.600
    	\item Gemintet an Nummer: 201 bis 1800
    \end{itemize}
    \item Status C (\textbf{Pearl})
    \begin{itemize}
    	\item Anzahl Pässe: 12.800
    	\item Gemintet an Nummer: 1801 bis 14.600
    \end{itemize}
    \item Status D (\textbf{Platin})
    \begin{itemize}
    	\item Anzahl Pässe: 102.400
    	\item Gemintet an Nummer: 14.601 bis 117.000
    \end{itemize}
    \item Status E (\textbf{Ruby})
    \begin{itemize}
    	\item Anzahl Pässe: 819.200
    	\item Gemintet an Nummer: 117.001 bis 936.200
    \end{itemize}
    \item Status F (\textbf{Gold})
    \begin{itemize}
    	\item Anzahl Pässe: 6.553.600
    	\item Gemintet an Nummer: 936.201 bis 7.489.800
    \end{itemize}
    \item Status G (\textbf{Silver})
    \begin{itemize}
    	\item Anzahl Pässe: 52.428.800
    	\item Gemintet an Nummer: 7.489.801 bis 59.918.600
    \end{itemize}
    \item Status H (\textbf{Bronze})
    \begin{itemize}
    	\item Anzahl Pässe: 419.430.400
    	\item Gemintet an Nummer: 59.918.601 bis 479.349.000
    \end{itemize}
    \item Status I (\textbf{White})
    \begin{itemize}
    	\item Anzahl Pässe: $\infty$
    	\item Gemintet an Nummer: 479.349.001 bis $\infty$
    \end{itemize}
\end{itemize}

\end{NFT-Prop}

\vspace{0.3cm}

Diese NFT-Property ist per Definition trivialerweise \textbf{deterministisch}: Es ist stets zweifellos klar, welchen Status ein an x-ter Stelle geminteter \textit{NFT-Pass} haben wird. Die hinzugezogene "Reverse-Halving-Logik" \textbf{belohnt die Early-Adopter} mit einem begehrten NFT, dessen Rarität per Protokoll mit der Zeit stets abnimmt.

\vspace{0.1cm}


    % binde die Datei '[NFT-Pass][Konzept][Status].tex' ein

Die Beschaffenheit dieser \textit{first-come-first-serve-Property} soll jedoch einzigartig bleiben. Die folgenden Properties werden nicht mehr deterministisch sein, um unserem \textit{NFT-Pass} ein \textbf{unvorherbestimmbares "Eigenleben"} einzuverleiben. 

% !TEX root = paper.tex

\subsubsection{Wunder-Property}

\vspace{0.3cm}

Diese NFT-Property soll zwar einem ähnlichen abstufenden Raritätsprinzip zu Grunde liegen wie die Main-Property, dies jedoch nicht mehr einem first-come-first-serve- sondern stattdessen einem Zufallsprinzip folgend.

Ebenfalls abweichend von der Beschaffenheit der Main-Property soll bei dieser Pro\-perty die Rarität nicht mittels einer absoluten Obergrenze abgebildet werden, sondern mittels einer relativen. (Dies zahlt auf die oben formulierte Anforderung nach einem \textbf{gemeinnützig gewinnbringendem Value} unseres \textit{NFT-Pass} ein.

\vspace{0.3cm}

\begin{NFT-Prop}[Hologramm (Welt-Wunder)]

Wir definieren folgende \textit{NFT-Pass-Hologramme} mit den dazugehörenden Eigenschaften:

\begin{itemize}
    \item WW1
    \begin{itemize}
    	\item Mögliche Ausprägung: \textbf{Pyramids of Giza}
    	\item Anteil Pässe: 0,390625\% $\left( \frac{1}{256} \right)$
    \end{itemize}
    \item WW2
    \begin{itemize}
    	\item Mögliche Ausprägung: \textbf{Great Wall of China}
    	\item Anteil Pässe: 0,78125\% $\left( \frac{1}{128} \right)$
    \end{itemize}
    \item WW3
    \begin{itemize}
    	\item Mögliche Ausprägung: \textbf{Petra} 
    	\item Anteil Pässe: 1,5625\% $\left( \frac{1}{64} \right)$
    \end{itemize}
    \item WW4
    \begin{itemize}
    	\item Mögliche Ausprägung: \textbf{Colosseum} 
    	\item Anteil Pässe: 3,125\% $\left( \frac{1}{32} \right)$
    \end{itemize}
    \item WW5
    \begin{itemize}
    	\item Mögliche Ausprägung: \textbf{Chichén Itzá} 
    	\item Anteil Pässe: 6,25\% $\left( \frac{1}{16} \right)$
    \end{itemize}
    \item WW6
    \begin{itemize}
    	\item Mögliche Ausprägung: \textbf{Machu Picchu} 
    	\item Anteil Pässe: 12,5\% $\left( \frac{1}{8} \right)$
    \end{itemize}
    \item WW7
    \begin{itemize}
    	\item Mögliche Ausprägung: \textbf{Taj Mahal} 
    	\item Anteil Pässe: 25\% $\left( \frac{1}{4} \right)$
    \end{itemize}
    \item WW8
    \begin{itemize}
    	\item Mögliche Ausprägung: \textbf{Christ the Redeemer} 
    	\item Anteil Pässe: 50\% + x $\left( \frac{1}{2} + \frac{1}{256} \right)$
    \end{itemize}
\end{itemize}

\end{NFT-Prop}

\vspace{0.3cm}

Das Besondere an dieser Property spiegelt sich in der Tatsache wider, gewisse rar beschaffene Ausprägungen seien nur "zeitweise" ausgeschöpft, da sich ihre (rare) Anzahl lediglich \textbf{relativ} an der Gesamtzahl der aktuell \textit{ausgestellten NFT-Pässe} bemisst und nicht wie die Main-Property einer absoluten Obergrenze obliegt, deren Erreichung unumkehrbar ist. Soll heißen: Ist die prozentuale Obergrenze an Pässen mit einer beChrist the Redeemerstimmten Ausprägung der gegenwärtigen Property zu einem be\-stimmten Zeitpunkt erreicht, kann zwar für einen gewissen Zeitraum kein Pass mit dieser Ausprägung mehr ausgestellt werden. Sobald jedoch die Gesamtanzahl der \textit{ausgestellten NFT-Pässe} wieder groß genug ist - sodass die Anzahl der vorhandenen \textit{NFT-Pässe} mit der betroffenen Ausprägung wieder die prozentuale Obergrenze unterschreitet - werden Pässe der besagten Ausprägung "wieder verfügbar".

\vspace{0.3cm}

\begin{Algo}[Verlosungs-Mechanismus für Hologramm-Property]

\begin{itemize}
    \item Zunächst bestimme man die Gesamtanzahl aller bisher geminteter Pässe $n$.
    \item Gleiches tue man nun für die Counts der geminteten Pässe pro Ausprägung der Hologramm-Property WW1 bis WW8 als entsprechende Größen $n_1, n_2,...,n_8$.
    \item Und damit anschließend die aktuelle prozentuale Verteilung der Ausprägung auf die aktuell geminteten Pässe als $\sigma_i:= \frac{n_i}{n}$ für $i \in \lbrace 1,...,8 \rbrace$ berechnen.
    \item Seien $\Theta_i$ für $i \in \lbrace 1,...,8 \rbrace$ die oben definierten \textbf{relativen} Obergrenzen der \newline Ausprägungen der Hologramm-Property WW1 bis WW8.
    \item Alle Ausprägungen mit $\sigma_i \geq \Theta_i$ können zum aktuellen Zeitpunkt nicht vergeben werden und damit auch nicht beim Minting eines neuen Pass berücksichtigt werden.
    \item Für die Ausprägungen mit $\sigma_i < \Theta_i$ berechnen wir den Normierungsfaktor
\end{itemize} 

\begin{equation*}
\omega := \sum_{\sigma_i < \Theta_i} \Theta_i \textrm{ } \leq 1
\end{equation*} 

\begin{itemize}
    \item Damit errechnen wir die aktuell vorliegenden Wahrscheinlichkeiten $\rho_i$ für unsere Hologramm-Ausprägungen als
\end{itemize} 

\[
\rho_i:=\left\{%
\begin{array}{ll}
    0, & \hbox{falls $\sigma_i \geq \Theta_i$} \\[0,3cm]
    \hbox{\LARGE $\frac{\Theta_i}{\omega}$,} & \hbox{falls $\sigma_i < \Theta_i$}. \\
\end{array}%
\right.
\] 

Man vergewissere sich an dieser Stelle gedanklich, auch für die neuen \newline Wahrscheinlichkeiten gelte \[\sum_{i = 1}^7 \rho_i \textrm{ } = 1.\]

\begin{itemize}
    \item Am Ende bestimme man mittels Randomisierung anhand der Wahrscheinlichkeiten $\rho_i$ für $i \in \lbrace 1,...7 \rbrace$ die zu vergebende Hologramm-Ausprägung. 
\end{itemize}

\end{Algo}

\vspace{0.3cm}

Was hier so kompliziert klingt, lässt sich aber super simpel veranschaulichen:

Die \textit{Verlosung} der Wunder erfolgt in einem periodischen 256er-Turnus ($256 = 2^{n}$ mit $n=8$ für die acht bereitgestellten Hologramme). Nach jedem 256. geminteten Pass schmeißt man 256 Lose in eine Lostrommel: Ein Los für die \textit{Pyramiden}, zwei für die \textit{Chinesische Mauer}, vier für \textit{Petra} etc. Die \textit{Jesus-Statue} kommt letztendlich mit 129 Losen in die Trommel.

Nun ziehen wir blind ein Los und vergeben das gezogenen Hologramm an den nächsten zu mintenden NFT-Pass. Wir tun dies solange, bis die Trommel leer ist. Anschließend fangen wir wieder von Vorne an und befüllen die Trommel erneut mit denselben 256 Losen.

\textbf{Achtung:} Wir befüllen die Trommel ausschließlich nachdem sie komplett leer geworden ist und nicht etwa zwischendurch mal.

\vspace{0.3cm}

    % binde die Datei '[NFT-Pass][Konzept][Wunder].tex' ein
%% !TEX root = paper.tex

\subsubsection{Wunder-Property}

\vspace{0.3cm}

Diese NFT-Property soll zwar einem ähnlichen abstufenden Raritätsprinzip zu Grunde liegen wie die Main-Property, dies jedoch nicht mehr einem first-come-first-serve- sondern stattdessen einem Zufallsprinzip folgend.

Ebenfalls abweichend von der Beschaffenheit der Main-Property soll bei dieser Pro\-perty die Rarität nicht mittels einer absoluten Obergrenze abgebildet werden, sondern mittels einer relativen. (Dies zahlt auf die oben formulierte Anforderung nach einem \textbf{gemeinnützig gewinnbringendem Value} unseres \textit{NFT-Pass} ein.

\vspace{0.3cm}

\begin{NFT-Prop}[Hologramm (Welt-Wunder)]

Wir definieren folgende \textit{NFT-Pass-Hologramme} mit den dazugehörenden Eigenschaften:

\begin{itemize}
    \item WW1
    \begin{itemize}
    	\item Mögliche Ausprägung: \textbf{Pyramids of Giza}
    	\item Anteil Pässe: 0,390625\% $\left( \frac{1}{256} \right)$
    \end{itemize}
    \item WW2
    \begin{itemize}
    	\item Mögliche Ausprägung: \textbf{Great Wall of China}
    	\item Anteil Pässe: 0,78125\% $\left( \frac{1}{128} \right)$
    \end{itemize}
    \item WW3
    \begin{itemize}
    	\item Mögliche Ausprägung: \textbf{Petra} 
    	\item Anteil Pässe: 1,5625\% $\left( \frac{1}{64} \right)$
    \end{itemize}
    \item WW4
    \begin{itemize}
    	\item Mögliche Ausprägung: \textbf{Colosseum} 
    	\item Anteil Pässe: 3,125\% $\left( \frac{1}{32} \right)$
    \end{itemize}
    \item WW5
    \begin{itemize}
    	\item Mögliche Ausprägung: \textbf{Chichén Itzá} 
    	\item Anteil Pässe: 6,25\% $\left( \frac{1}{16} \right)$
    \end{itemize}
    \item WW6
    \begin{itemize}
    	\item Mögliche Ausprägung: \textbf{Machu Picchu} 
    	\item Anteil Pässe: 12,5\% $\left( \frac{1}{8} \right)$
    \end{itemize}
    \item WW7
    \begin{itemize}
    	\item Mögliche Ausprägung: \textbf{Taj Mahal} 
    	\item Anteil Pässe: 25\% $\left( \frac{1}{4} \right)$
    \end{itemize}
    \item WW8
    \begin{itemize}
    	\item Mögliche Ausprägung: \textbf{Christ the Redeemer} 
    	\item Anteil Pässe: 50\% + x $\left( \frac{1}{2} + \frac{1}{256} \right)$
    \end{itemize}
\end{itemize}

\end{NFT-Prop}

\vspace{0.3cm}

Das Besondere an dieser Property spiegelt sich in der Tatsache wider, gewisse rar beschaffene Ausprägungen seien nur "zeitweise" ausgeschöpft, da sich ihre (rare) Anzahl lediglich \textbf{relativ} an der Gesamtzahl der aktuell \textit{ausgestellten NFT-Pässe} bemisst und nicht wie die Main-Property einer absoluten Obergrenze obliegt, deren Erreichung unumkehrbar ist. Soll heißen: Ist die prozentuale Obergrenze an Pässen mit einer beChrist the Redeemerstimmten Ausprägung der gegenwärtigen Property zu einem be\-stimmten Zeitpunkt erreicht, kann zwar für einen gewissen Zeitraum kein Pass mit dieser Ausprägung mehr ausgestellt werden. Sobald jedoch die Gesamtanzahl der \textit{ausgestellten NFT-Pässe} wieder groß genug ist - sodass die Anzahl der vorhandenen \textit{NFT-Pässe} mit der betroffenen Ausprägung wieder die prozentuale Obergrenze unterschreitet - werden Pässe der besagten Ausprägung "wieder verfügbar".

\vspace{0.3cm}

\begin{Algo}[Verlosungs-Mechanismus für Hologramm-Property]

\begin{itemize}
    \item Zunächst bestimme man die Gesamtanzahl aller bisher geminteter Pässe $n$.
    \item Gleiches tue man nun für die Counts der geminteten Pässe pro Ausprägung der Hologramm-Property WW1 bis WW8 als entsprechende Größen $n_1, n_2,...,n_8$.
    \item Und damit anschließend die aktuelle prozentuale Verteilung der Ausprägung auf die aktuell geminteten Pässe als $\sigma_i:= \frac{n_i}{n}$ für $i \in \lbrace 1,...,8 \rbrace$ berechnen.
    \item Seien $\Theta_i$ für $i \in \lbrace 1,...,8 \rbrace$ die oben definierten \textbf{relativen} Obergrenzen der \newline Ausprägungen der Hologramm-Property WW1 bis WW8.
    \item Alle Ausprägungen mit $\sigma_i \geq \Theta_i$ können zum aktuellen Zeitpunkt nicht vergeben werden und damit auch nicht beim Minting eines neuen Pass berücksichtigt werden.
    \item Für die Ausprägungen mit $\sigma_i < \Theta_i$ berechnen wir den Normierungsfaktor
\end{itemize} 

\begin{equation*}
\omega := \sum_{\sigma_i < \Theta_i} \Theta_i \textrm{ } \leq 1
\end{equation*} 

\begin{itemize}
    \item Damit errechnen wir die aktuell vorliegenden Wahrscheinlichkeiten $\rho_i$ für unsere Hologramm-Ausprägungen als
\end{itemize} 

\[
\rho_i:=\left\{%
\begin{array}{ll}
    0, & \hbox{falls $\sigma_i \geq \Theta_i$} \\[0,3cm]
    \hbox{\LARGE $\frac{\Theta_i}{\omega}$,} & \hbox{falls $\sigma_i < \Theta_i$}. \\
\end{array}%
\right.
\] 

Man vergewissere sich an dieser Stelle gedanklich, auch für die neuen \newline Wahrscheinlichkeiten gelte \[\sum_{i = 1}^7 \rho_i \textrm{ } = 1.\]

\begin{itemize}
    \item Am Ende bestimme man mittels Randomisierung anhand der Wahrscheinlichkeiten $\rho_i$ für $i \in \lbrace 1,...7 \rbrace$ die zu vergebende Hologramm-Ausprägung. 
\end{itemize}

\end{Algo}

\vspace{0.3cm}

Was hier so kompliziert klingt, lässt sich aber super simpel veranschaulichen:

Die \textit{Verlosung} der Wunder erfolgt in einem periodischen 256er-Turnus ($256 = 2^{n}$ mit $n=8$ für die acht bereitgestellten Hologramme). Nach jedem 256. geminteten Pass schmeißt man 256 Lose in eine Lostrommel: Ein Los für die \textit{Pyramiden}, zwei für die \textit{Chinesische Mauer}, vier für \textit{Petra} etc. Die \textit{Jesus-Statue} kommt letztendlich mit 129 Losen in die Trommel.

Nun ziehen wir blind ein Los und vergeben das gezogenen Hologramm an den nächsten zu mintenden NFT-Pass. Wir tun dies solange, bis die Trommel leer ist. Anschließend fangen wir wieder von Vorne an und befüllen die Trommel erneut mit denselben 256 Losen.

\textbf{Achtung:} Wir befüllen die Trommel ausschließlich nachdem sie komplett leer geworden ist und nicht etwa zwischendurch mal.

\vspace{0.3cm}



% !TEX root = paper.tex

\begin{sloppypar}
Diese NFT-Property soll ebenso wie die beiden vorigen einem abstufenden Raritätsprinzip zu Grunde liegen - und zwar ausschließlich dem Zufall folgend.
\end{sloppypar}

Im Gegensatz zu den beiden vorigen Properties obliegt die \textit{Pattern-Property} keiner absoluten Obergrenze - insbesondere auch dann nicht, falls einige Pattern zu einem Zeitpunkt verhältnismäßig unter- oder überrepräsentiert sind.

\vspace{0.3cm}

\begin{NFT-Prop}[Background (Pattern)]
\label{pattern}

Wir definieren folgende \textit{NFT-Pass-Background-Muster} mit den dazugehörenden Eigenschaften:

\begin{itemize}
    \item P1
    \begin{itemize}
    	\item Mögliche Ausprägung: \textbf{Safari Fun} 
    	\item Wahrscheinlichkeit: 0,1953125\% $\left( \frac{1}{512} \right)$
    \end{itemize}
    \item P2
    \begin{itemize}
    	\item Mögliche Ausprägung: \textbf{Triangular Bars} 
    	\item Wahrscheinlichkeit: 0,390625\% $\left( \frac{1}{256} \right)$
    \end{itemize}
    \item P3
    \begin{itemize}
    	\item Mögliche Ausprägung: \textbf{Pointillism} 
    	\item Wahrscheinlichkeit: 0,78125\% $\left( \frac{1}{128} \right)$
    \end{itemize}
    \item P4
    \begin{itemize}
    	\item Mögliche Ausprägung: \textbf{Wavy waves} 
    	\item Wahrscheinlichkeit: 1,5625\% $\left( \frac{1}{64} \right)$
    \end{itemize}
    \item P5
    \begin{itemize}
    	\item Mögliche Ausprägung: \textbf{Stony desert} 
    	\item Wahrscheinlichkeit: 3,125\% $\left( \frac{1}{32} \right)$
    \end{itemize}
    \item P6
    \begin{itemize}
    	\item Mögliche Ausprägung: \textbf{WunderPass} 
    	\item Wahrscheinlichkeit: 6,25\% $\left( \frac{1}{16} \right)$
    \end{itemize}
    \item P7
    \begin{itemize}
    	\item Mögliche Ausprägung: \textbf{Zigzag} 
    		\item Wahrscheinlichkeit: 12,5\% $\left( \frac{1}{8} \right)$
    \end{itemize}
    \item P8
    \begin{itemize}
    	\item Mögliche Ausprägung: \textbf{Linear}  
    	\item Wahrscheinlichkeit: 25\% $\left( \frac{1}{4} \right)$
    \end{itemize}
    \item P9
    \begin{itemize}
    	\item Mögliche Ausprägung: \textbf{Curves}
    	\item Wahrscheinlichkeit: 50,1953125\% $\left( \frac{257}{512} \right)$
    \end{itemize}
\end{itemize}

\end{NFT-Prop}

\vspace{0.3cm}

    % binde die Datei '[NFT-Pass][Konzept][Pattern].tex' ein

% !TEX root = paper.tex

\subsubsection{Edition}

\vspace{0.3cm}

Die Edition unseres WunderPasses soll als Property auf die anfangs geforderte Möglichkeit einer gewissen Individualisierung des WunderPasses durch seinen Besitzer einzahlen. Zu individuell darf eine solche NFT-Property aber auch nicht sein, da der NFT zwingend seinen Eigentümer wechseln können soll, da das ganze Unterfangen mit dem NFT-Pass andernfalls ad absurdum führte.

Um die Edition-Property noch etwas interessanter zu gestalten, sollen Exemplare jeder Edition nicht endlos verfügbar sein, sondern stattdessen irgendwann einmal \textit{aufgebraucht}. In solch einem Fall soll sich der User aber nicht einfach irgendeine andere Edition auswählen müssen, sondern erhält die \textit{"Oberedition"} (Parent) seiner ursprünglich gewünschten Edition. Und falls auch diese \textit{aufgebraucht} sein sollte, die \textit{"Oberedition"} der \textit{"Oberedition"} usw. 

\vspace{0.2cm}

\begin{NFT-Prop}[Edition]

Als Ausprägung der WunderPass-NFT-\textbf{Edition} haben wir uns für \textbf{Städte} der Welt entschieden. Die \textbf{Parent-Edition} einer Stadt ist das dazugehörige \textbf{Land}, dessen 
Parent-Edition wiederum \textbf{Kontinent} und die \textbf{oberste Editions-Ebene} dann die \textbf{Welt-Edition}. Letztere unterliegt folglich auch keiner stückweisen Obergrenze mehr.

\vspace{0.2cm}

\underline{\textbf{\textit{Beispiel einer Edition-Kette:}}}

\vspace{0.2cm}

\begin{equation*}
\textrm{Berlin } \rightarrow \textrm{ Germany } \rightarrow \textrm{ Europe } \rightarrow \textrm{ World }
\end{equation*} 

\vspace{0.2cm}

Es gilt folgendes grobe Regel-Set, was jedoch explizit auch nach Launch modifizierbar bleiben soll:

\begin{itemize}
    \item Die möglichen Editionen werden von uns bestimmt. Diese müssen nicht zwingend beim Launch des NFT vollständig benannt werden, sondern können stattdessen auch nachträglich eingepflegt werden.
    \item Jede berücksichtigte \textit{Städte-Edition} ist genau \textbf{100} Mal verfügbar. Sind alle 100 Exemplare einer \textit{Städte-Edition} bereits gemintet (verbraucht), erhält die nächste Mint-Anfrage nach einem WunderPass derselben Edition automatisch die zu dieser Städte-Edition gehörende \textit{Landes-Edition}.
    \item Die \textit{Landes-Editionen} sind in einer maximalen Stückzahl von je \textbf{10.000} pro berücksichtigtem Land verfügbar. Sind auch diese aufgebraucht, wird die durch den User ausgewählte Stadt auf die ihrem Land übergeordnete \textit{Kontinent-Edition} gemappt.
    \item Die \textit{Kontinent-Editionen} sind in einer maximalen Stückzahl von je \textbf{1.000.000} für jeden Kontinent (außer der Antarktis) vorgesehen. Sollte auch diese Menge irgendwann erschöpfen, greifen wir zu der übergeordneten \textit{Welt-Edition}
\end{itemize} 

\end{NFT-Prop}

\vspace{0.24cm}

\underline{\textbf{Quantitative Daten zu den Editionen:}}

\begin{itemize}
    \item Nach aktuellem Stand sind mindestens 693 \textit{Städte-Edition} vorgesehen.
    \item Die genannten \textit{Städte-Edition} verteilen sich dabei aktuell auf 179 \textit{Landes-Edition}.
    \item Die unterschiedlichen \textit{Kontinent-Edition} belaufen sich auf 6 (Nord- und Südamerika, Europa, Afrika, Asien und Australien).
    \item Die übergeordnete \textit{Welt-Edition} ist in ihrer Stückzahl unbegrenzt. 
    \item Die Auswahl der angebotenen \textit{Städte-Editionen} folgt (mit Augenmaß) in etwa folgender Logik:
    \begin{itemize}
    	\item Die Hauptstadt eines jeden mit einer \textit{Landes-Edition} versehenen Landes ist gleichzeitig auch eine verfügbare \textit{Städte-Edition}.
    	\item Mit Ausnahme der Hauptstädte erfordert die Größe einer Stadt (nach Einwohnern) ein Mindestmaß $m_1$, um als \textit{Städte-Edition} aufgenommen zu werden.
    	\item Sofern es das vorige Kriterium hergibt, sollen nach Möglichkeit für jedes Land mit einer eigenen \textit{Landes-Edition} mindestens seine 5 größten Städte mit einer eigenen \textit{Städte-Edition} versehen werden.
    	\item Überschreiten die $n$ größten Städte eines in die \textit{Landes-Editionen} aufgenommenen Landes eine bestimmte Mindestgröße (nach Einwohnern) $m_2$, werden alle $n$ Städte in die verfügbaren \textit{Städte-Editionen} aufgenommen. Dieses Kriterium wird aufgrund des vorigen ausschließlich für $n > 5$ relevant.
    	\item Städte der G7-Länder werden (ungeachtet etwaiger Mindestgröße) vermehrt in die \textit{Städte-Editionen} aufgenommen (bis zu 25 \textit{Städte-Editionen} pro Land).
    \end{itemize} 
    \item Einzelne Städte können bei Bedarf auch bei Missachtung aller vorigen Kriterien aufgenommen werden.
\end{itemize}

\vspace{0.5cm}




    % binde die Datei '[NFT-Pass][Konzept][Edition].tex' ein

% !TEX root = paper.tex

% https://de.wikibooks.org/wiki/LaTeX-Kompendium:_Schnellkurs:_Grafiken

Dem teils trockenen Text der vorigen Kapitel sollen hier einfach wortlos einige denkbare Ausprägungen unseres WunderPasses in Bild folgen:

\begin{figure}[h]
  \centering
  \subfloat[][]{\includegraphics[width=0.4\linewidth]{{"[06][NFT-Pass]/images/diamand 1"}}}%
  \qquad
  \subfloat[][]{\includegraphics[width=0.4\linewidth]{{"[06][NFT-Pass]/images/diamand 2"}}}%
  \caption{zwei \textit{diamond} Pässe mit je unterschiedlichen Hologrammen und Pattern}%
\end{figure}

\begin{figure}[h]
  \centering
  \subfloat[][]{\includegraphics[width=0.4\linewidth]{{"[06][NFT-Pass]/images/black"}}}%
  \qquad
  \subfloat[][]{\includegraphics[width=0.4\linewidth]{{"[06][NFT-Pass]/images/pearl"}}}%
  \caption{Pässe des Status \textit{black} und \textit{pearl}}%
\end{figure}

\begin{figure}[h]
  \centering
  \subfloat[][]{\includegraphics[width=0.4\linewidth]{{"[06][NFT-Pass]/images/gold"}}}%
  \qquad
  \subfloat[][]{\includegraphics[width=0.4\linewidth]{{"[06][NFT-Pass]/images/bronze"}}}%
  \caption{rechts ein bronzener Pass mit den sehr sehr seltenen \textit{Pyramiden von Gizeh} als Hologramm}%
\end{figure}
    % binde die Datei '[NFT-Pass][Konzept][Design].tex' ein
%% !TEX root = paper.tex

% https://de.wikibooks.org/wiki/LaTeX-Kompendium:_Schnellkurs:_Grafiken

Dem teils trockenen Text der vorigen Kapitel sollen hier einfach wortlos einige denkbare Ausprägungen unseres WunderPasses in Bild folgen:

\begin{figure}[h]
  \centering
  \subfloat[][]{\includegraphics[width=0.4\linewidth]{{"[06][NFT-Pass]/images/diamand 1"}}}%
  \qquad
  \subfloat[][]{\includegraphics[width=0.4\linewidth]{{"[06][NFT-Pass]/images/diamand 2"}}}%
  \caption{zwei \textit{diamond} Pässe mit je unterschiedlichen Hologrammen und Pattern}%
\end{figure}

\begin{figure}[h]
  \centering
  \subfloat[][]{\includegraphics[width=0.4\linewidth]{{"[06][NFT-Pass]/images/black"}}}%
  \qquad
  \subfloat[][]{\includegraphics[width=0.4\linewidth]{{"[06][NFT-Pass]/images/pearl"}}}%
  \caption{Pässe des Status \textit{black} und \textit{pearl}}%
\end{figure}

\begin{figure}[h]
  \centering
  \subfloat[][]{\includegraphics[width=0.4\linewidth]{{"[06][NFT-Pass]/images/gold"}}}%
  \qquad
  \subfloat[][]{\includegraphics[width=0.4\linewidth]{{"[06][NFT-Pass]/images/bronze"}}}%
  \caption{rechts ein bronzener Pass mit den sehr sehr seltenen \textit{Pyramiden von Gizeh} als Hologramm}%
\end{figure}


% !TEX root = paper.tex

\subsubsection{Beispiel}

\vspace{0.2cm}

\todo{TODO: Beispielrechnung für geminteten NFT-Pass mit der Nummer x}

Angenommen x sei 1.005.965.

\begin{itemize}
  \item vorrechnet, welche ersten 1.005.964 NFT-Pässe schon weggemintet sein könnten und Wahrscheinlichkeiten für den neu zu mintenden NFT-Pass erklären.
  \item neuen NFT-Pass unter Einbindung der Wahrscheinlichkeiten und vorgegaukelten Zufalls errechnet.
  \item geminteten neuen NFT-Pass als exakte Grafik in unserem Design hier abbilden.
\end{itemize}

\vspace{0.3cm}    % binde die Datei '[NFT-Pass][Konzept][Beispiel].tex' ein
%% !TEX root = paper.tex

\subsubsection{Beispielhafte Analyse der Collection}

\vspace{0.2cm}

Um ein besseres Gefühl über die formulierte Logik unseres NFTs zu bekommen, wollen wir ein Beispiel mit konkreten Zahlen rechnen und begeben uns dazu eine gute Weile in die Zukunft - zu einem Zeitpunkt, zu dem bereits genau 316.157 NFT-Pässe gemintet wurden. Ich als potenzieller Interessent an einem Pass-NFT möchte verstehen, welchen Pass ich als nächsten in etwa zu erwarten hätte.

Wir analysieren die 316.157 bereits geminteten Pässe.

\vspace{0.2cm} 

\underline{\textbf{Status:}}

\begin{itemize}
  \item Es wurden 200 Pässe des Status \textit{diamond} gemintet.
  \item Es wurden 1.600 Pässe des Status \textit{black} gemintet.
  \item Es wurden 12.800 Pässe des Status \textit{pearl} gemintet.
  \item Es wurden 102.400 Pässe des Status \textit{platin} gemintet.
  \item Es wurden 199.157 Pässe des Status \textit{ruby} gemintet.
  \item Von den insgesamt 819.200 vorgesehenen \textit{ruby} Pässen sind demnach noch 620.043 noch verfügbar.
\end{itemize}

\vspace{0.2cm}

\textit{\textbf{Unser Pass wird also definitiv den Status 'Ruby' haben!}}

\vspace{0.3cm}


\underline{\textbf{Hologramm:}}

\vspace{0.2cm}

Hinsichtlich der Hologramme können wir nur über die ersten 315.904 der 316.157 bisher geminteten Pässe eine definitive Aussage treffen. Die übrigen 253 folgen einer gewissen Wahrscheinlichkeitsverteilung. Zunächst zu den ersten 315.904:

\begin{itemize}
  \item Es wurden 159.186 Pässe mit dem Hologramm der \textit{Jesus-Statue} gemintet.
  \item Es wurden 78.976 Pässe mit dem Hologramm des \textit{Maj Mahal} gemintet.
  \item Es wurden 39.488 Pässe mit dem Hologramm des \textit{Machu Picchu} gemintet.
  \item Es wurden 19.744 Pässe mit dem Hologramm der \textit{Chichén Itzá} gemintet.
  \item Es wurden 9.872 Pässe mit dem Hologramm des \textit{Kolosseum} gemintet.
  \item Es wurden 4.936 Pässe mit dem Hologramm der \textit{Petra} gemintet.
  \item Es wurden 2.468 Pässe mit dem Hologramm der \textit{Chinesischen Mauer} gemintet.
  \item Es wurden 1.234 Pässe mit dem Hologramm den \textit{Pyramiden von Gizeh} gemintet. noch verfügbar.
\end{itemize}

\vspace{0.3cm}

Die Evaluierung der übrigen 253 ist insofern recht dankbar, als dass die 253 schon sehr nah an der zyklischen 256 liegt ($= 2^{n}$, wobei $n=8$ für die acht verfügbaren Hologramme steht). Damit beschränkt sich die Analyse eigentlich lediglich auf die nächsten drei Pässe, von denen der erste unserer ist. 











\vspace{0.5cm}

\begin{itemize}
  \item vorrechnet, welche ersten 316.157 NFT-Pässe schon weggemintet sein könnten und Wahrscheinlichkeiten für den neu zu mintenden NFT-Pass erklären.
  \item neuen NFT-Pass unter Einbindung der Wahrscheinlichkeiten und vorgegaukelten Zufalls errechnet.
  \item geminteten neuen NFT-Pass als exakte Grafik in unserem Design hier abbilden.
\end{itemize}

\vspace{0.3cm} 

% !TEX root = paper.tex

Abschließend möchten wir an dieser Stelle an die anfangs formulierte zentrale Forderung nach einem \textbf{intrinsischen Wert unseres Pass-NFTs} anknüpfen und im folgenden einen Abriss zu denkbaren Einsatzmöglichkeiten des WunderPass-NFTs und seinen potenziellen Vorteilen für 
seinen Besitzer darlegen.  

\vspace{0.1cm}

Konsequenterweise folgen wir bei der Erarbeitung solcher User-Nutzen \& -Vorteile der Devise, ein seltenerer NFT-Pass solle als \textit{gut} gelten und damit auch mit einem größeren \textbf{intrinsischen Wert} einhergehen. Die Seltenheit ist bei unserem NFT einerseits durch Zufall (\textit{Hologramm} \& \textit{Pattern}) aber auch durch First-Mover-Sein (\textit{Status}) gesteuert. 

Wenn es um das Beimessung von Vorzügen und Einsatzmöglichkeiten eines \textit{guten} Pass-NFTs geht, erscheint es irgendwie sinnvoll, eher First-Mover zu begünstigen als etwaige Glückspilze, weshalb wir in der folgenden \textit{"Vorteile-Tabelle"} das Augenmerk tendenziell auf den \textit{Status} des Pass-NFTs legen möchten. Bei einigen der gleich \mbox
zu listenden Vorteilen erscheint jedoch auch der ausschließliche oder zusätzliche Glücksfaktor als ebenfalls sehr charmant, weshalb an solchen Stellen das \textit{Hologramm} des Pass-NFTs zur Beimessung der Vorzüge hinzugezogen wird. Insbesondere möchten wir dem \textit{Pyramiden-Hologramm} explizit und per default dasselbe Statussymbol im Sinne der zu definierenden Vorzüge bei\-messen wie dem \textit{Diamond-Status}.

\textbf{Das \textit{Pattern} findet nach aktuellem Stand jedoch nirgends Berücksichtigung hinsichtlich des intrinsischen Werts eines Pass-NFTs}. Dieses bleibt also zunächst pure Spielerei im Sinne des Pass-NFTs als reines Sammlerstück.

\vspace{0.2cm}

Bei der folgenden Auflistung der Vorzüge und Einsatzmöglichkeiten kategorisieren wir nach 
\textcolor{orange}{\textbf{Status-Vorteilen}}, \textcolor{brown}{\textbf{finanziellen Anreizen}} sowie \textcolor{purple}{\textbf{Mitgestaltungsspielraum}} innerhalb der WunderPass-Company, die an den zugehörigen Farben zu erkennen sind.



\vspace{1.0cm}


\begin{tabular}[c]{|c|c|c|c|c|c|c|c|}
\hline
 & \textbf{\textit{diamond}} & \textbf{\textit{black}} & \textbf{\textit{pearl}} & & \textbf{\textit{pyramid}} & \textbf{\textit{wall}} & \textbf{\textit{petra}} \\
\hline
\textcolor{orange}{Weihnachtsfeier} & $\color{green} \surd\surd\surd$ &  &  &  & $\color{green} \surd\surd\surd$ &  & \\
\hline
\textcolor{orange}{\parbox{2cm}{Workshops \\ Hackathons}} & $\color{green} \surd\surd\surd$ & $\color{green} \surd$ &  &  & $\color{green} \surd\surd\surd$ &  & \\
\hline
\textcolor{orange}{private Discord} & $\color{green} \surd\surd\surd$ & $\color{green} \surd\surd\surd$ &  &  & $\color{green} \surd\surd\surd$ &  & \\
\hline
\textcolor{orange}{Metallkarte} & $\color{green} \surd\surd\surd$ & $\color{green} \surd\surd$ &   $\color{green} \surd$ &  & $\color{green} \surd\surd\surd$ &  & \\
\hline
\textcolor{orange}{Goodies} &  &  &  &  & $\color{green} \surd\surd\surd$ & $\color{green} \surd\surd$ & $\color{green} \surd$ \\
\hline
\textcolor{orange}{Priorisierung} & $\color{green} \surd\surd\surd$ & $\color{green} \surd\surd$ &   $\color{green} \surd$ &  & $\color{green} \surd\surd\surd$ &  & \\
\hline
 &  &  &  &  &  &  & \\
\hline
\textcolor{brown}{\parbox{2.8cm}{Airdrops \\ (Utility Token)}} &  &  &  &  & $\color{green} \surd\surd\surd$ & $\color{green} \surd\surd$ & $\color{green} \surd$ \\
\hline
\textcolor{brown}{Rewards} & $\color{green} \surd\surd\surd$ & $\color{green} \surd\surd$ &   $\color{green} \surd$ &  & $\color{green} \surd\surd\surd$ &  & \\
\hline
\textcolor{brown}{Staking-Zinsen} & $\color{green} +++$ & $\color{green} ++$ &   $\color{green} +$ &  & $\color{green} +++$ &  & \\
\hline
\textcolor{brown}{\parbox{2.8cm}{Beteiligung an \\ NFT-Verkäufen}} & $\color{green} \surd\surd\surd$ &  &  &  & $\color{green} \surd\surd\surd$ &  & \\
\hline
\textcolor{brown}{Dividende} & $\color{green} \surd\surd\surd$ &  &  &  & $\color{green} \surd\surd\surd$ &  & \\
\hline
 &  &  &  &  &  &  & \\
\hline
\textcolor{purple}{early Access} & $\color{green} \surd\surd\surd$ & $\color{green} \surd\surd\surd$ &  &  & $\color{green} \surd\surd\surd$ &  & \\
\hline
\textcolor{purple}{\parbox{2.8cm}{Voting for \\ Product/Features}} & $\color{green} \surd$ & $\color{green} \surd$ & $\color{green} \surd$ &  & $\color{green} \surd$ & $\color{green} \surd$ & $\color{green} \surd$ \\
\hline
\textcolor{purple}{\parbox{2.8cm}{Govern. Tokens \\ (DAO-Membership)}} & $\color{green} \surd\surd\surd$ & $\color{green} \surd\surd$ &   $\color{green} \surd$ &  & $\color{green} \surd\surd\surd$ &  & \\
\hline
 &  &  &  &  &  &  & \\
\hline
\textcolor{red}{\textbf{Backlog}} &  &  &  &  &  &  & \\
\hline
exklusiveres Naming &  &  &  &  &  &  & \\
\hline
Zugang zu Interna &  &  &  &  &  &  & \\
\hline
\parbox{3.0cm}{Vergünstigung für \\ Wunder-Dienste} &  &  &  &  &  &  & \\
\hline
\end{tabular}\vspace*{0.3cm}\\

\vspace{0.5cm}    % binde die Datei '[NFT-Pass][Konzept][Intrinsischer Wert].tex' ein


\vspace{0.5cm}




    % binde die Datei '[NFT-Pass][Konzept].tex' ein
% !TEX root = C:/Users/Slava/White-Paper/[06][NFT-Pass]/[NFT-Pass].tex

\subsection{Technische Umsetzung}

\vspace{0.3cm}

\todo{TODO: technische Implementierung}

\vspace{0.3cm}

\begin{itemize}
  \item Abwandlung des ERC721-Standard, um unsere Metadaten-Logik zu bändigen.
  \item Die Metadaten werden wohl auch einem ähnlichen Konstrukt wie IPFS (off-chain) gespeichert werden und lediglich deren Hash als Datenfeld im Smart-Contract (on-chain), damit die Metadaten nicht nachträglich verändern werden können (dieses Vorgehen wird der absolute Standard sein).
  \item Unsere Metadaten sind jedoch so komplex, das deren Erzeugung (beim Minten) wohl einen zweiten Smart-Contract erfordern wird. Wir haben also quasi einen "Metadaten-Hybriden":
  \begin{itemize}
  	\item Erzeugung on-chain
  	\item Storing off-chain
  \end{itemize}
  \item Der Metadaten-Smart-Contract wird die oben skizzierte Logik implementieren
  \begin{itemize}
  	\item Wie viele Pässe gibts es bereits und welche (hinsichtlich Properties)?
  	\item Wie sind die aktuellen Verteilungen der Properties und deren Contstraints
  	\item Einbindung von Randomisierungs-Orakeln
  	\item Sicherstellung, dass die erzeugten Metadaten auch tatsächlich vom Caller (ERC721-Contract) verwendet wurden und keine nachträgliche Manipulation stattgefunden hat.
  \end{itemize}
  \item Es muss geklärt werden, ob hinsichtlich des Gedanken an den besagten "zweiten Smart-Contract" Standards/Best-Practices existieren, damit wir hier nicht das Rad neu erfinden.
  \item Es bleibt noch nicht ganz klar, wie die Metadaten nach ihrer Erzeugung nach IPFS gelangen, da dies laut meinem Verständnis ein Smart-Contract nicht selbst gewährleisten kann. Moritz Idee war grob die Folgende 
  \begin{itemize}
    \item Der Minting-Contract erzeugt den NFT, lässt seine Metadaten-Referenz jedoch zunächst ungesetzt (der NFT ist damit in gewisser Weise noch "unfertig"; kann in dem Zustand auch noch Constraints unterstellt sein).
    \item Der Minting-Contract callt den Metadaten-Contract mit dem Anliegen, Metadaten zu dem "unfertigen" NFT mit der zugehörigen ID zu erzeugen.
  	\item Der Metadaten-Contract erzeugt die Metadaten, hasht diese und gibt den Hash zurück an den Minting-Contract. Gleichzeitig publisht er ein Create-Event mit der Token-ID und den zugehörigen erzeugten Metadaten.
  	\item Der Minting-Contract speichert den erhaltenen Metadaten-Hash und wartet auf "approvement".
  	\item Das forcierte Event wird von einem dafür bestimmten (off-chain) Web3-Service vernommen und weiterverarbeitet: Die Metadaten werden geparst und nach IPFS gepusht. Als Ergebnis bekommen wir eine entsprechende IPFS-URI.
  	\item Unser Web3-Service stößt anschließend eine "Set-URI"-Transaktion mit den entsprechenden Input-Daten (Token-ID; IPFS-URI) beim Minting-Contract an, um den gesamten Minting-Prozess für den neuen Token abzuschließen.
  	\item Der Minting-Contract verifiziert die Metadaten mittels des gespeicherten Meta-Daten-Hashs (\todo{Hier ist nicht nicht ganz klar, wie. Ich weiß nicht, ob der Contract einfach die Daten von IPFS laden kann, um den Hash abzugleichen oder ob er vorher die URI implizit vorgeben muss, die irgendwie im Hash berücksichtigt werden muss, oder wie auch immer hier die Best-Practise aussieht}) und updatet die NFT-URI auf den Wert der übergebenen IPFS-URI. 
  	\item Hiermit ist der Minting-Prozess abgeschlossen, der NFT "fertig" gemintet und kann von etwaigen "Temporary-Locked-Constraint" entbunden werden und vom neuen Besitzer frei verfügt werden.
  \end{itemize}
  \item \textbf{Ein etwaiger Crypto-Freelancer muss auf die skizzierten Herausforderungen gechallenget werden.}
\end{itemize}    % binde die Datei '[NFT-Pass][Tech].tex' ein

\vspace{0.5cm}    % binde die Datei 'NFT-Pass.tex' ein
% !TEX root = paper.tex

\section{Abgrenzung zu SSI}
\label{sec:ssi}

\vspace{0.3cm}

\todo{TODO}

\vspace{0.3cm}

\todo{TODO: DID scheint für uns eine zentralere Rolle zu spielen als SSI. DID sollten wir also eher in WunderPass einbinden, als uns davon distanzieren zu versuchen.}

\vspace{0.3cm}    % binde die Datei 'SSI.tex' ein
% !TEX root = paper.tex
\section{Dinge}
\label{sec:dinge}
\todo{TODO}    % binde die Datei 'Dinge.tex' ein
% !TEX root = paper.tex

\section{Project 'Guard'}
\label{sec:guard}
\todo{TODO}    % binde die Datei 'Project Guard.tex' ein
% !TEX root = paper.tex
\section{Community}
\label{sec:community}
\todo{TODO}    % binde die Datei 'Community.tex' ein
% !TEX root = paper.tex
\section{Zusammenfassung}
\label{sec:fazit}
\todo{TODO}    % binde die Datei 'Zusammenfassung.tex' ein
% !TEX root = paper.tex

\section{Anhang}
\label{sec:anhang}

\vspace{0.3cm}

Eine schöne Definition der Identität laut \href{https://link.springer.com/article/10.1007/s11612-001-0022-y}{Döring, N. (1999). Sozialpsychologie des Internet.}

\vspace{0.3cm}

\begin{Business-Def}[Identität laut Döring, N. (1999). Sozialpsychologie des Internet.]

Identität wird heute als komplexe Struktur aufgefasst, die aus einer Vielzahl einzelner Elemente besteht (Multiplizität), von denen in konkreten Situationen jeweils Teilmengen aktiviert sind oder aktiviert werden (Flexibilität). Eine Person hat aus dieser Perspektive nicht nur eine "wahre" Identität, sondern verfugt über eine Vielzahl an gruppen-, rollen-, raum-, körper- oder tätigkeitsbezogenen Teil-Identitäten.

\end{Business-Def}

\vspace{0.3cm}

Folgende hilfreiche Zitate, Aussagen und Formulierungen entstammen der \href{https://vsis-www.informatik.uni-hamburg.de/getDoc.php/thesis/47/DA_Gordian_Kaulbarsch.pdf}{Diplomarbeit "Identitäten und ihre Schnittstellen auf Basis von Ontologien in einer dezentralen Umgebung"}. 

\vspace{0.3cm}

\begin{Zitat}[Betrachtungsweise der digitalen Identität]

In der Informatik finden sowohl der rein mathematische Identitätsbegriff Verwendung
– ein Standardkonzept in den meisten Programmiersprachen –, als auch der sozialpsychologische Aspekt dieses Begriffs. In dieser Arbeit ist ein Identitätsbegriff der Betrachtungsgegenstand, der von beiden Seiten inspiriert ist. Der mathematische Identitätsbegriff bildet die Grundlage: Anhand eines Identifikators ist eine Identität eindeutig
bestimmbar. Dieser Identifikator wird angereichert durch eine beliebige Vielfalt an ergänzenden Attributen und ihre situationsbedingt eingeschränkte Verwendung.

\end{Zitat}

\vspace{0.3cm}


\begin{Zitat}[Avatar]

Diesem Vorbild der Newsgroup-Nutzer folgend unterstützen viele Foren-Systeme im ¨
World Wide Web von vornherein das Anlegen eines Identitäts-Profils. Neben diversen
Identitäts- und Nutzungsdaten kann hier oft ein \textbf{Bild als Stellvertreter und Wiedererkennungsmerkmal und emotionale Botschaft eingesetzt} werden, für welches der Begriff \textbf{"Avatar"} geprägt wurde. [...]

Einen Nachteil neben der mangelnden
Standardisierbarkeit und dem demzufolge bestehenden Mangel an automatischer Auswertbarkeit weist die Identitätsdarstellung im World Wide Web ebenfalls noch auf: Es
gibt keine praktikable Möglichkeit, zu bestimmen, wer auf diese Daten zugreifen kann
und wer nicht. Daten, die sich im World Wide Web befinden, sind im Allgemeinen für jeden einsehbar. 

\end{Zitat}

\vspace{0.3cm}


\begin{Zitat}[Web-Visitenkarte]

Im Rahmen des World Wide Webs hat sich eine Variante privater Homepages herausgebildet, die als Hauptmerkmal die Darstellung der eigenen Person aufweist. Die
starke Verbreitung dieser persönlichen Homepages oder Web-Visitenkarten ist [...] ein Indiz fur den starken Bedarf nach individueller Darstellung der eigenen Identität im virtuellen Raum.

\end{Zitat}

\vspace{0.3cm}


\begin{Zitat}[Skepsis hinsichtlich Datenerfassung]

Als Reaktion auf solche - in Abschnitt \ref{sec:einleitung_probleme_digitaler_identitaet} zitierte - oftmals ungefragt oder aber vom Anwender ungewünscht erfolgenden Datenerfassungsmethoden werden Gegenmaßnahmen eingesetzt: Bei der Dateneingabe werden \textbf{bewusst Falschangaben vorgenommen} oder Cookies und die Quellen von Web-Bugs werden blockiert. Dies geschieht insbesondere bei \textbf{Anbietern, bei denen die Daten nicht zwingend benötigt werden} oder deren Notwendigkeit zur Erfassung dem Nutzer nicht einsichtig ist. Insgesamt hat das Vorgehen vieler Anbieter zumindest bei kritischen Nutzern des Internets ein starkes Misstrauen gegenüber diesen Techniken geweckt. So finden die Gegenmaßnahmen – zum Beispiel die Blockade von Cookies – auch dann leicht statt, wenn sie unbegründet wäre und vielmehr ein echter Vorteil dadurch
ermöglicht wurde. Ein Beispiel für einen solchen Vorteil ist die Vereinfachung und Individualisierung von Informationsangeboten durch personalisierte Darstellung.

\end{Zitat}

\vspace{0.3cm}


\begin{Zitat}[Misstrauen vernichtet Value]

Insgesamt hat das Vorgehen vieler Anbieter zumindest bei kritischen Nutzern des Internets ein starkes Misstrauen gegenüber diesen Techniken geweckt. So finden die Gegenmaßnahmen – zum Beispiel die Blockade von Cookies – auch dann leicht statt, wenn sie unbegründet wäre und vielmehr ein echter Vorteil dadurch
ermöglicht wurde. Ein Beispiel für einen solchen Vorteil ist die Vereinfachung und Individualisierung von Informationsangeboten durch personalisierte Darstellung.

Weniger kritische Nutzer und solche, die sich ein differenziertes Bild über die Vor- und Nachteile dieser Techniken verschafft haben, erhalten für sie speziell zusammengestellte Inhalte, bekommen relevante Angebote unterbreitet oder haben die Möglichkeit, mit
anderen Nutzern mit ähnlichen Interessen oder mit entsprechend ähnlichen Fähigkeiten
in Kontakt zu treten. Dieser Nutzen gilt allerdings immer nur im eingeschränkten Bereich
innerhalb eines Angebotes.

\end{Zitat}

\vspace{0.3cm}


\begin{Zitat}[Synonymisierung]

[...] Dabei geht es nicht immer um eine der Wirklichkeit entsprechende Darstellung, sondern oftmals auch um \textbf{spielerische} oder die reale Identität \textbf{verschleiernde Pseudonyme} und Rollen-Repräsentationen. Manche Dienste – Online-Spiele beispielsweise – fordern dies sogar explizit ein, während andere – zum Beispiel Instant Messenger – dies problemlos ermöglichen. Wesentlich ist bei beiden die Kontinuität der Identifizierbarkeit. \textbf{Auch hier gilt die Beschränkung der Nutzbarkeit der Identitätsdaten auf einen Dienst}. Dies ist beim Beispiel des Online-Spiels wohl auch grundsätzlich sinnvoll – die erschaffene Spiel-Identität hat schließlich oftmals wenig mit der realen Identität gemein –, beim Instant Messaging aber schon \textbf{weniger gewünscht}.

\end{Zitat}

\vspace{0.3cm}


\begin{Zitat}[Identitätsdaten]

Identitätsdaten sind variantenreich und individuell und beschränken sich nicht auf einen
Kundendatensatz oder Anmeldedaten für Online-Dienste. Dies sind allerdings bisher die ¨
Hauptbereiche, in denen Identitätsdaten heute zum Einsatz kommen. Die Daten einer
Identität müssen aber alle Aspekte einer solchen abbilden können. Diese Vielzahl an persönlichen und auch personengebundenen Daten kann viele \textbf{Erleichterungen und Automatisierungen} mit sich bringen, birgt aber \textbf{auch Risiken} und \textbf{erschwert die Handhabung}. So ist bei einer Betrachtung von Konzepten zu einem Identitätsmanagement immer auch der
Blick zu richten auf die \textbf{Frage nach der Kontrolle der Daten} durch den Anwender, nach den \textbf{Verwendungsmöglichkeiten durch zur Nutzung dieser Daten} berechtigte Personen und nach Möglichkeiten des \textbf{unvorhergesehenen Missbrauchs}. Als noch entscheidenderes Kriterium für die \textbf{Akzeptanz durch die Anwender} ist aber sicherlich die Frage nach dem \textbf{Mehraufwand}: Kann ein Konzept, beziehungsweise seine Umsetzung in einer Anwendung gewisse Kriterien erfüllen, \textbf{dass es der Anwender als vorteilhaft und nicht als belastend wertet}? [...]

\end{Zitat}

\vspace{0.3cm}

Wichtige Aspekte aus dem letzten Zitat:

\begin{itemize}
  \item Pros
  \begin{itemize}
    \item Erleichterungen und Automatisierungen
    \item Kontrolle der Daten beim User
    \item Verwendungsmöglichkeiten durch Nutzung der Daten
  \end{itemize}
  \item Kontras
  \begin{itemize}
    \item Risiken
    \item erschwerte Handhabe
    \item Kontrolle der Daten beim Provider
    \item unvorhergesehener Missbrauch
  \end{itemize}
  \item Akzeptanz $\rightarrow$ Rechtfertigt der Nutzen den Mehraufwand?
\end{itemize}

\vspace{0.3cm}


\begin{Zitat}[E-Mail (auch als Identifier)]

Der E-Mail-Standard ist sicher kein Standard für ein Identiätsmanagement. Er sei hier
aber erwähnt, da es sich um den ältesten und am weitesten verbreiteten digitalen Standard handelt, der sich primär auf Individuen und somit Identitäten bezieht. [...]

Neben dem reinen Aspekt der gegenseitigen Erreichbarkeit weist eine E-Mail-Adresse
nur durch die – heute oft freie – Wahl der Kennung Individualität auf. Die meisten E-Mail-Systeme interpretieren ebenfalls zusätzliche Angaben des vollen Namens und der
Organisation. Neben der eher "seriösen" Variante, den eigenen Namen in voller oder teilweise abgekürzter Form zu verwenden, versuchen viele Personen eine bestimmte Geisteshaltung, Zuneigung oder Gruppenzugehörigkeit durch die Wahl der richtigen Kennung
auszudrücken. Genau hier ist aber auch schon die Grenze des E-Mail-Standards als Identitätskonzept erreicht: Weder lässt sich eine Namenswahl klar deuten – es sei denn, man ist mit dem weiteren Kontext des Anwenders vertraut –, noch lässt sich dieses durch automatische Prozesse sinnvoll auswerten. \textbf{Eine E-Mail-Adresse bleibt als Identitätskonzept das, was sie von Anfang an auch nur sein sollte: Ein eindeutiges  Identifizierungszeichen, um der damit verknüpften Identität Daten zukommen lassen zu können.}

\end{Zitat}

\vspace{0.3cm}


\begin{Zitat}[Workaround im Status quo]

Die wenigen bisherigen Standards und auch andere Konzepte ermöglichen nicht mehr
als die Speicherung der eigenen Kennung, Kontaktdaten oder Daten zum Bezahlen. [...]

Viele Anwender verwalten schon heute eine Reihe von Daten, die auf Basis einer
digitalen Identitäten-Infrastruktur zusammengefasst betrachtet werden könnten: Dazu
zählen solche Dinge wie das digitale Adressbuch, ein Kalender, die Lesezeichen, Verwaltungsdaten von Sammlungen (beispielsweise Fotos, Bücher oder Musik), ¨
Wunschlisten (beispielsweise bei Onlineshops), Lebensläufe, \textbf{Ergebnisstände von Computerspielen} und vieles mehr. All diese Daten liegen bisher in verschiedenen Strukturen vor, ohne Gesamtstruktur und \textbf{ohne, dass sich automatisierte Querbezüge bei Bedarf herstellen ließen, obwohl es alles Daten sind, die sich der Identität des Anwenders zuordnen ließen}. Diese fehlende Gesamtstruktur kann zur Folge unerwünschte \textbf{Redundanzen und auch Inkonsistenzen} mit sich bringen.

\end{Zitat}

\vspace{0.3cm}


\begin{Zitat}[Übergeordnete Struktur $\rightarrow$ "Querverweise"]

[...] Notwendig ist hierbei eine zusätzliche Struktur, die – ohne die bestehenden Daten und ihren eventuell aktuellen Bezug zueinander zu verändern – diesen Daten eine Gesamtstruktur verleiht: eine Identitätsdatengesamtstruktur. Auf diese Weise ließen sich Daten wie bisher speichern. Zusätzlich ließen sich mittels dieser Struktur aber auch Zusammenhänge herstellen, die unabhängig von der ursprünglichen Gebundenheit der Daten bestünden.

Derart ließen sich auch Identitätsdaten auf automatisierte Weise kontrolliert weitergeben. \textbf{"Kontrolliert" in diesem Zusammenhang bedeutet die Möglichkeit für den Anwender, selbst zu entscheiden, an wen er welche Daten wann und zu welchen Bedingungen übermittelt.} Dies ließe sich leicht bewerkstelligen, indem er Teile der Struktur mit entsprechenden Freigaben oder Einschränkungen versähe. Einen geeigneten Kommunikations- oder Kooperationsdienst vorausgesetzt, könnte der Anwender so bestimmten anderen Anwendern gezielt Daten über sich zukommen lassen, ohne sich selbst um die Zusammenstellung dieser Daten oder deren Übertragung kümmern zu müssen.

\end{Zitat}

\vspace{0.3cm}


\begin{Fazit}[Link zu WunderPass]

Insbesondere das letzte Zitat schreit förmlich nach WunderPass. Die "Struktur", von der dort abstrakt die Rede ist, heißt "WunderPass" (zumindest auf den Aspekt der "Querverweise" bezogen).

\end{Fazit}

\vspace{0.3cm}


\begin{Zitat}[Herausforderung]

Der Ansatz, einen Großteil der persönlichen Daten strukturell der Identität zuzuordnen,
bietet ein großes Potenzial für die Personalisierung und Individualisierung in Datennetzen. Es bestehen allerdings auch grunds¨atzliche Probleme, die ein solches Konzept überwinden muss: Wenn individuelle und umfassende Strukturen die Identitätsdaten in einen Gesamtzusammenhang bringen sollen, so müssen diese Strukturen erstellt werden. Wenn die Strukturen die Kommunikation unterstützen sollen, muss die individuelle ¨
Struktur auf Empfängerseite bekannt sein, um dort von Vorteil sein zu können.

\end{Zitat}

\vspace{0.3cm}


\begin{Fazit}[Link zu WunderPass]

Das letzte Zitat beschreibt nicht anderes als unser "Henne-Ei-Problem" hinsichtlich der Anbindung WunderPasses an Drittanbieter.

\end{Fazit}

\vspace{0.3cm}


\begin{Fazit}[Aufwand beim User]

Die Forderung nach oben zitierter Struktur (aka WunderPass) erfordert aber das Zutun des Users, welches mit nicht unerheblichem Aufwand einhergeht. Die Vorteile genannter Struktur werden dabei nicht zwangsläufig von Anfang an für den User ersichtlich sein. Das bedeutet im Umkehrschluss, er müsse zu einem Aufwand gedrängt werden, dessen Mehrwert sich für ihn kaum erschließt.

Es erfordert als eines Incentivierungs-Mechanismus (z. B. als Bestandteil etwaiger Token-Economics). Gleichzeitig ist es aus dem Blickwinkel des gesamten Ökosystems nicht zu rechtfertigen, der User werde ausschließlich aufgrund seiner Ignoranz - nämlich seine eigenen Vorteile aus obiger Struktur nicht erkennen zu können - Nutznießer von (vom Ökosystem gemeinschaftlich getragenen) Incentives. Daher wäre ein Hebel innerhalb der Token-Economics wünschenswert, der den User - ab Eintreten persönlicher Vorteile durch die "Querverweise" - die ausgeschütteten Incentives wieder zurückzahlen lässt.
 

\end{Fazit}

\vspace{0.3cm}    % binde die Datei 'Anhang.tex' ein


 
\end{document}